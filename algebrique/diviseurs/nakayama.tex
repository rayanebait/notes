\documentclass[a4paper,12pt]{book}
\usepackage{amsmath,  amsthm,enumerate}
\usepackage{csquotes}
\usepackage[provide=*,french]{babel}
\usepackage[dvipsnames]{xcolor}
\usepackage{quiver, tikz}

%symbole caligraphique
\usepackage{mathrsfs}

%hyperliens
\usepackage{hyperref}

%pseudo-code
\usepackage{algorithm}
\usepackage{algpseudocode}

\usepackage{fancyhdr}

\pagestyle{fancy}
\addtolength{\headwidth}{\marginparsep}
\addtolength{\headwidth}{\marginparwidth}
\renewcommand{\chaptermark}[1]{\markboth{#1}{}}
\renewcommand{\sectionmark}[1]{\markright{\thesection\ #1}}
\fancyhf{}
\fancyfoot[C]{\thepage}
\fancyhead[LO]{\textit \leftmark}
\fancyhead[RE]{\textit \rightmark}
\renewcommand{\headrulewidth}{0pt} % and the line
\fancypagestyle{plain}{%
    \fancyhead{} % get rid of headers
}

%bibliographie
\usepackage[
backend=biber,
style=alphabetic,
sorting=ynt
]{biblatex}

\addbibresource{bib.bib}

\usepackage{appendix}
\renewcommand{\appendixpagename}{Annexe}

\definecolor{wgrey}{RGB}{148, 38, 55}

\setlength\parindent{24pt}

\newcommand{\Z}{\mathbb{Z}}
\newcommand{\R}{\mathbb{R}}
\newcommand{\rel}{\omathcal{R}}
\newcommand{\Q}{\mathbb{Q}}
\newcommand{\C}{\mathbb{C}}
\newcommand{\N}{\mathbb{N}}
\newcommand{\K}{\mathbb{K}}
\newcommand{\A}{\mathbb{A}}
\newcommand{\B}{\mathcal{B}}
\newcommand{\Or}{\mathcal{O}}
\newcommand{\F}{\mathscr F}
\newcommand{\Hom}{\textrm{Hom}}
\newcommand{\disc}{\textrm{disc}}
\newcommand{\Pic}{\textrm{Pic}}
\newcommand{\End}{\textrm{End}}
\newcommand{\Spec}{\textrm{Spec}}
\newcommand{\Supp}{\textrm{Supp}}
\renewcommand{\Im}{\textrm{Im}}
\newcommand{\im}{\textrm{im}}


\newcommand{\m}{\mathfrak{m}}
\newcommand{\n}{\mathfrak{n}}
\newcommand{\p}{\mathfrak{p}}


\newcommand{\cL}{\mathscr{L}}
\newcommand{\G}{\mathscr{G}}
\newcommand{\D}{\mathscr{D}}
\newcommand{\E}{\mathscr{E}}
\renewcommand{\Pr}{\mathbb{P}}
\renewcommand{\P}{\mathscr{P}}
\renewcommand{\H}{\mathscr{H}}

\makeatletter
\newcommand{\colim@}[2]{%
  \vtop{\m@th\ialign{##\cr
    \hfil$#1\operator@font colim$\hfil\cr
    \noalign{\nointerlineskip\kern1.5\ex@}#2\cr
    \noalign{\nointerlineskip\kern-\ex@}\cr}}%
}
\newcommand{\colim}{%
  \mathop{\mathpalette\colim@{\rightarrowfill@\scriptscriptstyle}}\nmlimits@
}
\renewcommand{\varprojlim}{%
  \mathop{\mathpalette\varlim@{\leftarrowfill@\scriptscriptstyle}}\nmlimits@
}
\renewcommand{\varinjlim}{%
  \mathop{\mathpalette\varlim@{\rightarrowfill@\scriptscriptstyle}}\nmlimits@
}
\makeatother

\theoremstyle{plain}
\newtheorem{thm}[subsection]{Théoreme}
\newtheorem{lem}[subsection]{Lemme}
\newtheorem{prop}[subsection]{Proposition}
\newtheorem{cor}[subsection]{Corollaire}
\newtheorem{heur}{Heuristique}
\newtheorem{rem}{Remarque}
\newtheorem{note}{Note}

\theoremstyle{definition}
\newtheorem{conj}{Conjecture}
\newtheorem{prob}{Problème}
\newtheorem{quest}{Question}
\newtheorem{prot}{Protocole}
\newtheorem{algo}{Algorithme}
\newtheorem{defn}[subsection]{Définition}
\newtheorem{exmp}[subsection]{Exemples}
\newtheorem{exo}[subsection]{Exercices}
\newtheorem{ex}[subsection]{Exemple}
\newtheorem{rep}{Réponse}
\newtheorem{concl}{Conclusion}
\newtheorem{exs}[subsection]{Exemples}

\theoremstyle{remark}

\definecolor{wgrey}{RGB}{148, 38, 55}
\definecolor{wgreen}{RGB}{100, 200,0} 
\hypersetup{
    colorlinks=true,
    linkcolor=wgreen,
    urlcolor=wgrey,
    filecolor=wgrey
}

\title{Courbes géometriquement intègres propres lisses?}
\date{}


\begin{document}
\maketitle
\tableofcontents
\[\ldots\]   

Je regarde quasi toujours $f\colon X\to Y$ entre courbes 
géometriquement intègres propres lisses. 

\section{Cadre}
Donc un tel $X\to Y$ est surjectif fini. Et on peut supposer
$X,Y$ projectives.

\chapter{Ramification sur les courbes}
De $f^\sharp\colon \Or_Y\to f_*\Or_X$ on a
\[\Or_{Y,f(x)}\to \Or_{X,x}\]
défini par 
\[(g,V_{f(x)})\mapsto (f^\sharp(V_{f(x)})(g), f^{-1}V_{f(x)})\]
\begin{note}
  Juste un petit rappel de définitions.
\end{note}
\section{La facilité du cadre !}
Donc y se passe un truc fun, si $B/A$ est une extension finie
d'anneaux de Dedekind on sait que 
\[[Frac(B):Frac(A)]=\sum e_if_i\]
en particulier,
\begin{enumerate}
  \item Pour toutes courbes intègres propres lisses,
    $X\to Y$ non constant est fini ! 
  \item En particulier, on prends un affine qui contient $f^{-1}y$
    par exemple $f^{-1}U$ avec $y\in U$ affine! 
  \item D'où $\sum e_{x/y}=[k(X):k(Y)]$! (On est sur $k$ 
    algébriquement clos)
\end{enumerate}

\section{La différence}
On a $\Or_X(X)=\Or_Y(Y)=k$. Donc pas d'arguments globaux.

\section{Nombre fini de points ramifiés.}
On peut se ramener a un ouvert affine le complémentaire a qu'un
nombre fini de points. Ensuite, on regarde le discriminant d'un
$\theta$ tel que $k(Y)(\theta)=k(X)$ et $\theta$ entier sur 
$A(U)$ avec $U$ l'ouvert de $Y$. Ensuite c'est l'argument usuel.

\begin{rem}
  À noter qu'on peut donc pas faire systématiquement tout 
  globalement.
\end{rem}


\printbibliography
\end{document}

