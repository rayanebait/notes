\documentclass[a4paper,12pt]{book}
\usepackage{amsmath,  amsthm,enumerate}
\usepackage{csquotes}
\usepackage[provide=*,french]{babel}
\usepackage[dvipsnames]{xcolor}
\usepackage{quiver, tikz}

%symbole caligraphique
\usepackage{mathrsfs}

%hyperliens
\usepackage{hyperref}

%pseudo-code
\usepackage{algorithm}
\usepackage{algpseudocode}

\usepackage{fancyhdr}

\pagestyle{fancy}
\addtolength{\headwidth}{\marginparsep}
\addtolength{\headwidth}{\marginparwidth}
\renewcommand{\chaptermark}[1]{\markboth{#1}{}}
\renewcommand{\sectionmark}[1]{\markright{\thesection\ #1}}
\fancyhf{}
\fancyfoot[C]{\thepage}
\fancyhead[LO]{\textit \leftmark}
\fancyhead[RE]{\textit \rightmark}
\renewcommand{\headrulewidth}{0pt} % and the line
\fancypagestyle{plain}{%
    \fancyhead{} % get rid of headers
}

%bibliographie
\usepackage[
backend=biber,
style=alphabetic,
sorting=ynt
]{biblatex}

\addbibresource{bib.bib}

\usepackage{appendix}
\renewcommand{\appendixpagename}{Annexe}

\definecolor{wgrey}{RGB}{148, 38, 55}

\setlength\parindent{24pt}

\newcommand{\Z}{\mathbb{Z}}
\newcommand{\R}{\mathbb{R}}
\newcommand{\rel}{\omathcal{R}}
\newcommand{\Q}{\mathbb{Q}}
\newcommand{\C}{\mathbb{C}}
\newcommand{\N}{\mathbb{N}}
\newcommand{\K}{\mathbb{K}}
\newcommand{\A}{\mathbb{A}}
\newcommand{\B}{\mathcal{B}}
\newcommand{\Or}{\mathcal{O}}
\newcommand{\F}{\mathscr F}
\newcommand{\Hom}{\textrm{Hom}}
\newcommand{\disc}{\textrm{disc}}
\newcommand{\Pic}{\textrm{Pic}}
\newcommand{\End}{\textrm{End}}
\newcommand{\Spec}{\textrm{Spec}}
\newcommand{\Supp}{\textrm{Supp}}
\renewcommand{\Im}{\textrm{Im}}


\newcommand{\m}{\mathfrak{m}}
\newcommand{\p}{\mathfrak{p}}


\newcommand{\cL}{\mathscr{L}}
\newcommand{\G}{\mathscr{G}}
\newcommand{\D}{\mathscr{D}}
\newcommand{\E}{\mathscr{E}}
\renewcommand{\Pr}{\mathbb{P}}
\renewcommand{\P}{\mathscr{P}}
\renewcommand{\H}{\mathscr{H}}

\makeatletter
\newcommand{\colim@}[2]{%
  \vtop{\m@th\ialign{##\cr
    \hfil$#1\operator@font colim$\hfil\cr
    \noalign{\nointerlineskip\kern1.5\ex@}#2\cr
    \noalign{\nointerlineskip\kern-\ex@}\cr}}%
}
\newcommand{\colim}{%
  \mathop{\mathpalette\colim@{\rightarrowfill@\scriptscriptstyle}}\nmlimits@
}
\renewcommand{\varprojlim}{%
  \mathop{\mathpalette\varlim@{\leftarrowfill@\scriptscriptstyle}}\nmlimits@
}
\renewcommand{\varinjlim}{%
  \mathop{\mathpalette\varlim@{\rightarrowfill@\scriptscriptstyle}}\nmlimits@
}
\makeatother

\theoremstyle{plain}
\newtheorem{thm}[subsection]{Théoreme}
\newtheorem{lem}[subsection]{Lemme}
\newtheorem{prop}[subsection]{Proposition}
\newtheorem{cor}[subsection]{Corollaire}
\newtheorem{heur}{Heuristique}
\newtheorem{rem}{Remarque}
\newtheorem{note}{Note}

\theoremstyle{definition}
\newtheorem{conj}{Conjecture}
\newtheorem{prob}{Problème}
\newtheorem{quest}{Question}
\newtheorem{prot}{Protocole}
\newtheorem{algo}{Algorithme}
\newtheorem{defn}[subsection]{Définition}
\newtheorem{exmp}[subsection]{Exemples}
\newtheorem{exo}[subsection]{Exercices}
\newtheorem{ex}[subsection]{Exemple}
\newtheorem{exs}[subsection]{Exemples}

\theoremstyle{remark}

\definecolor{wgrey}{RGB}{148, 38, 55}
\definecolor{wgreen}{RGB}{100, 200,0} 
\hypersetup{
    colorlinks=true,
    linkcolor=wgreen,
    urlcolor=wgrey,
    filecolor=wgrey
}

\title{Notes perso : Géométrie algébrique}
\date{}

\begin{document}
\maketitle
\tableofcontents

\chapter{Produits fibrés et foncteur de points}
Petite intuition du foncteur de points : en fait sur un $R$-schéma $T$
on a des $R$-flèches $T\to \Spec(R[T_1,\ldots,T_n])$, essentiellement 
c'est donné par 
\[\Hom_R(R[T_1,\ldots,T_n],\Gamma(T,\Or_T))\simeq \Gamma(T,\Or_T)^n\]
là où c'est marrant c'est que si on regarde le noyau donné par des 
polynômes (c'est une $R$-flèche) $f_1,\ldots, f_s$ on obtient une 
flèche dans 
\[\Hom_R(R[T_1,\ldots, T_n]/(f_1,\ldots, f_s), \Or_T(T))\]
en particulier des équations pour des sections globales de $T$.

\begin{rem}
    En toute généralité je sais pas si ça dit grand chose de $T$. Mais
    ça doit être intéressant de creuser.
\end{rem}


\section{Construction}




\chapter{Morphismes d'espaces localement annelés}
Le but là c'est de revoir pourquoi dans le cas localement annelé 
$(f,f^\sharp)\colon X\to Y$ la flèche $f^\sharp$ correspond bien au
pullback $g\mapsto g\circ f$ (c'est pas tjr défini sur un Schéma affine)
et l'intuition est exacte dans le cas des variétés. Ensuite de voir
pourquoi les schémas affines généralisent bien les variétés et ensembles
algébriques initiales.

\section{Le cas des $k$-algèbre de type fini}
Dans le cas des variétés abstraites, on prends deux variétés affines
$X\simeq (Z(I)\subset \A^n, \Or_X)$ et $(Y,\Or_Y)$. Étant donné un
morphisme d'espaces localement annelés $(f,f^\sharp)\colon X\to Y$ on 
a
% https://q.uiver.app/#q=WzAsMTAsWzEsMCwiZl57XFwjfVxcY29sb25cXG1hdGhjYWwgT19ZKFUpIl0sWzIsMCwiXFxtYXRoY2FsIE9fWChmXnstMX1VKSJdLFsxLDEsIlxcbWF0aGNhbCBPX3tZLGYoeCl9Il0sWzIsMSwiXFxtYXRoY2FsIE9fe1gseH0iXSxbMSwyLCJcXGthcHBhKGYoeCkpIl0sWzIsMiwiXFxrYXBwYSh4KSJdLFswLDAsInMiXSxbMCwyLCJzKGYoeCkpIl0sWzMsMCwiZl57XFwjfShzKSJdLFszLDIsImZee1xcI30ocykoeCkiXSxbMCwxXSxbMCwyXSxbMSwzXSxbMiwzXSxbMiw0XSxbMyw1XSxbNCw1LCJpZCJdLFs2LDcsIiIsMSx7InN0eWxlIjp7InRhaWwiOnsibmFtZSI6Im1hcHMgdG8ifX19XSxbOCw5LCIiLDAseyJzdHlsZSI6eyJ0YWlsIjp7Im5hbWUiOiJtYXBzIHRvIn19fV1d
\[\begin{tikzcd}
	s & {f^{\#}\colon\mathcal O_Y(U)} & {\mathcal O_X(f^{-1}U)} & {f^{\#}(s)} \\
	& {\mathcal O_{Y,f(x)}} & {\mathcal O_{X,x}} \\
	{s(f(x))} & {\kappa(f(x))=k} & {\kappa(x)=k} & {f^{\#}(s)(x)}
	\arrow[maps to, from=1-1, to=3-1]
	\arrow[from=1-2, to=1-3]
	\arrow[from=1-2, to=2-2]
	\arrow[from=1-3, to=2-3]
	\arrow[maps to, from=1-4, to=3-4]
	\arrow[from=2-2, to=2-3]
	\arrow[from=2-2, to=3-2]
	\arrow[from=2-3, to=3-3]
	\arrow["id", from=3-2, to=3-3]
\end{tikzcd}\]
Un truc qui fait tiquer c'est que ça semble pas utiliser le fait que
c'est localement annelé, déjà c'est pas clair que c'est un morphisme
de $k$-algèbres les morphismes de corps résiduels. En fait l'endroit
où ça l'utilise c'est que l'existence de ce morphisme vient du fait que
\[\Or_{Y,f(x)}\to \Or_{X,x}\to \kappa(x)=\Or_{X,x}/\p_x\]
passe au quotient parce que $f^\sharp_x(\p_{f(x)})\subseteq \p_x$.

\section{L'équivalence de catégorie avec les variétés abstraites affines}
Essentiellement, d'un côté on a une flèche topologique continue 
\[|f|\colon X\to Y\]
et une flèche de faisceau localement annelée
\[f^\sharp\colon \Or_Y\to f_*\Or_X\]
on a pas forcément une flèche polynomiale encore de $\A^n\to \A^m$.
On sait quand même que $\Or_Y(Y)\to \Or_X(X)$ correspond à 
$f_*\colon A(Y)\to A(X)$, faut juste relever en $A(\A^m)\to A(X)$
et obtenir $X\to \A^m$ comme une restriction $\A^n\to \A^m$! À l'inverse,
si on a $f\colon X\to Y$ polynomiale, on obtient $A(Y)\to A(X)$. Faut
juste obtenir $\Or_Y(U)\to \Or_X(f^{-1} U)$, mais c'est clair que le 
pullback de fonctions marche!




\section{L'équivalence de catégorie avec schémas affines}
\subsection{L'équivalence}
On regarde $\Spec\colon Ring\to Aff$, c'est essentiellement surjectif
par définition et $\Gamma\circ \Spec=id$. Ducoup faut montrer que c'est
pleinement fidèle. En reprenant les définitions, de 
\[(f,f^\sharp)\colon \Spec(B)\to \Spec(A)\]
un morphisme de schémas affines faut montrer que si $\varphi:=\Gamma(f)$
alors $\Spec(\varphi)$, la flèche de pullback, est égale à $f$. On peut 
montrer que $\Spec(\varphi)=f$ et $\varphi_x=f^\sharp_x$ pour tout $x$.
Dire que $f^\sharp(\p_x)=\varphi^{-1}(\p_x)$ on peut le déduire de 
l'existence de 
% https://q.uiver.app/#q=WzAsNixbMSwwLCJCIl0sWzAsMCwiQSJdLFswLDEsIkFfe1xccF97Zih4KX19Il0sWzEsMSwiQl97XFxwX3h9Il0sWzAsMiwiXFxrYXBwYShmKHgpKSJdLFsxLDIsIlxca2FwcGEoeCkiXSxbMSwwLCJcXHZhcnBoaSJdLFsxLDIsImlfeCIsMl0sWzIsMywiZl5cXHNoYXJwX3giLDJdLFswLDMsImpfeCJdLFsyLDRdLFszLDVdLFs0LDVdXQ==
\[\begin{tikzcd}
	A & B \\
	{A_{\p_{f(x)}}} & {B_{\p_x}} \\
	{\kappa(f(x))} & {\kappa(x)}
	\arrow["\varphi", from=1-1, to=1-2]
	\arrow["{i_x}"', from=1-1, to=2-1]
	\arrow["{j_x}", from=1-2, to=2-2]
	\arrow["{f^\sharp_x}"', from=2-1, to=2-2]
	\arrow[from=2-1, to=3-1]
	\arrow[from=2-2, to=3-2]
	\arrow[from=3-1, to=3-2]
\end{tikzcd}\]
non trivial, on peut voir ça en regardant le carré du haut, on a 
\begin{enumerate}
    \item $j_x^{-1}(\p_xB_{\p_x})=\p_x$.
    \item $i_{f(x)}^{-1}(\p_{f(x)}A_{\p_{f(x)}})=\p_{f(x)}$
\end{enumerate}
et $(f^\sharp_x)^{-1}(\p_x)\subset \p_{f(x)}$, comme $j_x\circ\varphi=
f_x^\sharp\circ i_x$. On obtient 
\[\varphi^{-1}(\p_x)\subset \p_{f(x)}\]
l'inclusion inverse est conséquence directe du fait que c'est
localement annelé. Faut quand même montrer que $\Spec(\varphi)^\sharp$
donné par les flèches induites par $\varphi$ est égale à $f^\sharp$.
Mais ça c'est clair sur les fibres par unicité donc partout.



\subsection{Le morphisme sur les fibres}
Retour sur l'adjonction entre $f^{-1}$ et $f_*$ pour un morphisme de 
faisceaux $f\colon (X,\F)\to (Y,\G)$. On remarque que des deux carrés 
% https://q.uiver.app/#q=WzAsOSxbMSwwLCJcXE9yX1goZl57LTF9VSkiXSxbMCwwLCJcXE9yX1koVSkiXSxbMCwxLCJcXE9yX3tZLGYoeCl9Il0sWzEsMSwiXFxPcl97WCx4fSJdLFs0LDAsIlxcT3JfWChWKSJdLFs0LDEsIlxcT3Jfe1gseH0iXSxbMywxLCIoZl57LTF9XFxPcl9ZKV97Zih4KX0iXSxbMywwLCIoZl57LTF9XFxPcl9ZKShWKSJdLFszLDIsIlxcT3Jfe1ksZih4KX0iXSxbMSwwXSxbMSwyXSxbMiwzXSxbMCwzXSxbNiw1XSxbNyw2XSxbNiw4LCJcXHNpbWVxIiwxLHsic3R5bGUiOnsiYm9keSI6eyJuYW1lIjoibm9uZSJ9LCJoZWFkIjp7Im5hbWUiOiJub25lIn19fV0sWzcsNF0sWzQsNV1d
\[\begin{tikzcd}
	{\Or_Y(U)} & {\Or_X(f^{-1}U)} && {(f^{-1}\Or_Y)(V)} & {\Or_X(V)} \\
	{\Or_{Y,f(x)}} & {\Or_{X,x}} && {(f^{-1}\Or_Y)_{f(x)}} & {\Or_{X,x}} \\
	&&& {\Or_{Y,f(x)}}
	\arrow[from=1-1, to=1-2]
	\arrow[from=1-1, to=2-1]
	\arrow[from=1-2, to=2-2]
	\arrow[from=1-4, to=1-5]
	\arrow[from=1-4, to=2-4]
	\arrow[from=1-5, to=2-5]
	\arrow[from=2-1, to=2-2]
	\arrow[from=2-4, to=2-5]
	\arrow["\simeq"{description}, draw=none, from=2-4, to=3-4]
\end{tikzcd}\]
C'est étonnamment celui de droite qui est intuitif à prouver. On
regarde la famille des $f^{-1}\Or_Y(V)\to \Or_X(V)\to \Or_{X,x}$ ça 
donne une flèche de la limite des $x\in V$ tels que $f(V)\subseteq U$
d'où la limite des $f(x)\in U$!





\chapter{Notes de lecture du Shafarevich}
\section{Variété produit}
Ce qui est assez clair c'est le produit de variétés affines. L'idée
en général c'est de construire 
\[X\times Y\to \Pr^n(k)\]
un isomorphisme sur son image qu'on peut reconstruire localement de 
sorte à ce qu'il soit unique. 
\begin{note}
    Page 54.
\end{note}
Concrètement dans le cas affine, les fermés de $\A^{n+m}$
qui sont fermés de $\A^n\times \A^m$ c'est les fermés du type
$V(F(X_1,\ldots, X_n)), G(X_{n+1},\ldots, X_m))$. Pour le projectif
c'est assez similaire. Le truc c'est que brutalement définir
\[\Pr^n\times \Pr^m\to \Pr^{n+1+m+1-1}\]
ca marche pas. Un tuple du produit $(u_0,\ldots, u_n, v_0,\ldots, v_m)$
est doublement homogène donc définit pas un point de $\Pr^{n+m+1}$. 
Ducoup faut déf plutôt
\begin{align*}
    \phi\colon\Pr^n\times \Pr^m&\to \Pr^{(n+1)(m+1)-1}\\
([u_0,\ldots, u_n],[v_0,\ldots, v_m])&\mapsto (u_iv_j)_{i,j}
\end{align*}
Et là on a clairement 
\[\phi(\Pr^n\times \Pr^m)=\bigcap_{i,j}V\left(\begin{vmatrix}
X_{ij}& X_{il}\\ X_{kj}&X_{kl}\end{vmatrix}\right)\]
En fait ce qui est cool c'est la remarque suivante.
\begin{rem}
    Si on regarde les mineures $2x2$ de $(X_{ij})$ et le lieu de leur
    zéros communs. On peut prouver que $\phi$ se surjecte dedans. En
    particulier la matrice $(w_{ij})=(u_iv_j)$ est de rang $1$ et en plus
    c'est un produit de matrices $1\times (n+1)$ par $(m+1)\times 1$.
\end{rem}
En fait c'est non trivial le produit projectif mdr. Déjà à faire :
\begin{exo}
    Les fermés de $\Pr^n\times \Pr^m$ sont les zéros de polynômes 
    bihomogènes. Réponse : L'idée c'est que si on prend un polynôme 
    homogènes en $(X_{ij})$ et qu'on regarde \[X_{ij}\mapsto X_iY_j\]. 
    On obtient un polynôme bihomogène en les $X_i$ et les $Y_j$ de 
    mêmes degrés. En fait inversement si on a un polynôme bihomogène
    $G$ de degré $r$ en $X_i$ et $s$ en $Y_j$ avec $r>s$, on peut 
    remarquer que $V(G)$ dans $\A^{n+1}\times \A^{m+1}$ est égal à 
    à \[\bigcap V(Y_j^{r-s} G)\]
    (à vérifier, ça a l'air de marcher justement parce qu'on regarde
    dans $\A^{n+1}\times \A^{m+1}$ et pas $\A^{n+1+m+1}$.)
    \textbf{Meilleure réponse} : Si on regarde $Z(G)\subset \varphi(
    \Pr^n\times\Pr^m)$ on peut montrer que $G$ est bihomogène de mêmes
    degré. À l'inverse pour def des fermés de $\Pr^n$ on doit utiliser
    des générateurs homogènes et donc $\bigcap_i V(Y_j^{r-s}G)$ c'est des
    générateurs homogènes.
\end{exo}
Pour $\Pr^n\times \A^n$ on peut prendre les homogènes pour les premières.


\section{Graphes}
En gros étant donné la structure d'une variété projective $X$ on peut
se demander si la diagonale dans $X\times X$ est fermée. C'est assez
clair si on remarque que dire que deux points projectifs 
$u=[u_0:\ldots:_n]$ et $v=[v_0:\ldots:v_n]$ sont égaux ssi ils sont
proportionnels. Autrement dit si \[u_i/u_j=v_i/v_j\]
ou \[u_iv_j=u_jv_i.\] Ensuite étant donné une fonction 
\[f\colon Y\to X\]
on peut regarder \[id\times f\colon X\times Y\to X\times X\]
d'où le graphe est fermé.

\section{Les variétés projectives sont complètes}
Y'a la preuve de Shafarevich et la preuve de Dat qui a l'air plus 
parlante et qui dit qu'on peut "compléter" les courbes ponctionnées d'un
point sur une variété complète.

Pour la preuve de Shafarevich, on peut
regarder $Z(G)\subset X\times Y\to Y$ et donner des conditions pour que
$y\in p_2(Z(G))$ pour $G$ un polynôme bihomogène. Déjà on peut se ramener
à $X=\Pr^n$ car $X$ est projective et $Y=\A^n$. En fait, 
$y\in p_2(Z(G))$ ça veut juste dire que $G(\bar U, y)$ a un zéro 
projectif. Autrement dit, $I_s\nsubseteq (G(\bar U,y))$. Par cette 
condition, en notant $(M_{a})_a$ les monômes de degrés $s$ on obtient
\[M_a(\bar U)=F_a(\bar U)G(\bar U, y)\]
avec $\deg(F_a)=s-\deg(G)$. Enfin si on écrit $F_a$ en somme monomes
de degré $s-\deg(G)$ disons en $N_b$. Alors les $M_a$ sont combinaisons
linéaires des $N_bG$. Et on obtient une matrices entre la base canonique
des monômes de degré $s$ et les $N_bG$ à coefficients des polynomes en
$y$ à coefficient dans $k$! La condition $y\in p_2(Z(G))$ équivaut alors
au fait que le rang de la matrice soit $<$ au nombre de monôme de 
degré $s$ en $n$ variables. On prend $y$ qui annule les mineures de 
taille cette dimension!

\begin{cor}
    L'image d'une variété projective est fermée.
\end{cor}

Suffit de factoriser $X\to Y$ en $X\to X\times Y\to Y$, la première 
flèche est $x\mapsto (x, f(x))$ et envoie $X$ sur son graphe.


\section{Un peu de birationnalité (exo 5 (1.3)}
Si on prend $f_d,f_{d-1}\in k[X_1,\ldots, X_n]$ de degrés $d$ et $d-1$
tels que $f_d+f_{d-1}$ est irréductible, alors $Z(f_d+f_{d-1})$ est 
birationnelle à $\A^{n-1}$. 

Mon idée c'est que $f_d/f_{d-1}$ est de degré $1$, d'où 
$f_d/f_{d-1} + 1$ définit un hyperplan. Ducoup faut trouver un 
isomorphisme de $k(\bar X_1, \ldots, \bar X_n)$ vers 
$k(X_1, \ldots, X_{n-1})$. Pour ça deux manières de faire, si $f_d$ 
a un monôme qui étend $f_{d-1}$ on peut juste le factoriser en haut
et en haut de la fraction. On obtient une expression de la forme 
\[F_0.(X_k+F_1)\]
où $F_0$ est de degré $0$ et $F_1$ de degré $1$ qui a seulement pour 
dénominateur le monôme. On peut faire le changement de variable
$X_k\mapsto X_k/F_0 $, on obtient $X_k$ + $F_1(F_0, X_1,\ldots, X_n)$.

\textbf{Meilleure manière :} En fait on peut écrire $f_d/f_{d-1}$ comme
$1/F_0(f_d/M)$ où $F_0$ est une fraction rationnelle de degré $0$ et
$M$ un monôme. En gros on a factorisé en bas et en haut par $M$. Ensuite,
en notant $M(i)=\{M\in S_{d_1}| X_i\mid M\}$ et \[f_{d,i}= f_{d,i-1}-
X_{i-1}\sum_{M\in M(f_{d,i-1},i-1)} a_{M, i-1}M\] et $M(f_{d,i},i)=\{M\in M(i)| M\in Monome(f_{d,i})\}$
on peut écrire
\[f_d/M=\left(\sum_i X_i\sum_{N\in M(f_{d,i}, i)} a_{N, i} N\right)/M\]
en particulier les $N/M$ ont degré $0$. Enfin si on note $I$ le support 
de la somme de gauche, on peut déf 
\[\varphi\colon K[T_1,\ldots,T_n]\to K[Z(f_d+f_{d-1})]_{f_{d-1}, M}\]
par $T_i\mapsto X_i\sum_{N\in M(f_{d,i}, i)} a_{N, i} N/M$ si $i\in I$
et $T_i\mapsto X_i$ sinon. On peut remarquer que la flèche préserve la 
graduation, i.e. $\ker(\varphi)$ est engendré par des polynômes de degré
$1$. On obtient une flèche rationnelle dominante injective
\[Z(f_d+f_{d-1})\to Z(\ker(\varphi))\]
où $\ker(\varphi)$ définit un hyperplan de $\A^n$ d'où le résultat.

\textbf{Bien meilleure manière!} En fait, déjà pour rappel la petite 
section \ref{trad}. Puis on obtient la preuve suivante.

\subsection{Une nouvelle preuve}
On peut regarder $\A^n\to \A^n$ la flèche 
\[(t_1,\ldots,t_n)\mapsto (t_1, t_1t_2,\ldots, t_1t_n)\]
on obtient un isomorphisme 
\[k[T_1,\ldots, T_n]_{T_1}\simeq k[T_1,\ldots, T_n]\]
d'inverse $T_1\mapsto T_1;~T_i\mapsto T_i/T_1$. Maintenant on peut 
regarder \[f_d(T_1, T_2/T_1, \ldots, T_n/T_1)+f_{d-1}(\ldots)=0\] qu'on
obtient via l'isomorphisme induit 
\[k[Z(f_d(T_i,i)+f_{d-1}(T_i,i)]_{T_1}\simeq k[Z(f_d(T_1,T_1T_2,\ldots,T_1T_n)+f_{d-1}(T_1,T_1lT_2,\ldots, T_1T_n))]_{T_1}\]
Mais le membre de droite est égal à $k[Z(T_1f_d+f_{d-1})_{T_1}]$ en
particulier $Z(f_d+f_{d-1})$ est birationnel à $Z(T_1f_d+f_{d-1})$.
Maintenant si $f_{d-1}\ne 0$ on peut définir
\[k[T_1,\ldots, T_n]\to k[Z(T_1f_d+f_{d-1})]_{f_{d-1}}\]
via $T_1\mapsto T_1f_d/f_{d-1}$ et $T_i\mapsto T_i$. Le noyau a l'air 
d'être clairement $T_1+1$ d'où on a fini.

\section{Surjectivité des morphismes finis}
Ducoup si on regarde la condition $x\in f^{-1}(y)$, ça se traduit par
$f_*\m_y\subset \m_x$. Inversement, si $f^{-1}(y)$ est vide alors
$f_*\m_yk[X]=k[X]$ et on applique le lemme de Nakayama car 
$\m_y\ne k[Y]$. \textbf{Faudrait essayer de voir plus concrètement
comment trouver une racine.}
\section{Trouver un ouvert dans $f(X)$}
L'idée c'est de décomposer $f$ : $f\colon X\to Y\times \A^r\to Y$ 
défini par $k[Y]\subset k[Y][u_1,\ldots, u_r]\subset k[X]$ où 
$k(X)=k(Y)(u_1,\ldots, u_r, z)$. La première flèche faut juste 
choisir $u_i\in k[X]$, rendre l'extension entière avec la 
technique habituelle puis tout dérouler en utilisant le fait 
que les morphismes finis sont surjectifs. La deuxième c'est
juste que la première projection on peut trouver un ouvert dans
l'image d'un ouvert facilement.

\section{Projections}
Ça c'était cool, et plus deep que prévu. Ducoup, 
on considère des "e.v" de $\Pr^n$ (juste les regarder
dans $\Pr^{n+1}$). Donnés par $E=Z(L_1,\ldots,L_{n-d})$,
et on déf 
\[\pi\colon \Pr^n-E\to \Pr^{n-d-1};(\bar x)\mapsto (L_i(\bar x))_i\]
L'idée géometrique c'est que si on prend 
$\Pr^{n-d-1}\simeq H\subset \Pr^n-E$, alors $H\cap E=0$ et $H\oplus E=
\A^{n+1}$ ce qui définit un projecteur. En particulier, si
$x\in \Pr^{n}-E$ alors 
\[E\oplus <x>\cap H\textrm{ est de dimension }1.\]
Donc on a un point d'intersection dans $\Pr^{n}$, et il est donné par 
$\pi$. C'est pas entièrement clair, mais c'est clair que c'est une 
projection au sens où $H\oplus E=\A^{n+1}$ a des projecteurs associés.

Maintenant pour prouver que c'est une flèche finie, c'est plutôt cool,
déjà on regarde $\pi\colon D(L_j)\cap X\to \A^{n-d-1}\cap \pi(X)$ et 
on veut montrer que $\pi^*\colon k[\A^{n-d-1}]\to k[D(L_j)\cap X]$
est finie, et $g\in \Or_X(D(L_j))$ est de la forme $g=G/L_i^m$
($\Pr^n-E=\cup D(L_i)$). Maintenant $\pi^*(T_i)=L_i/L_j$ donc on doit 
trouver une relation algébrique unitaire entre $g$ est les $\pi^*(T_i)$.
C'est là que c'est fort. On remarque que l'image de
\[\pi_m\colon X\to \Pr^{n-d};(\bar x)\mapsto (L_1^m,\ldots,
L_{n-d}^m,G)\]
est fermée. On obtient $F_1,\ldots, F_s$ des polynômes qui annulent 
l'image ! Mais ça suffit pas encore, j'ai l'impression qu'on peut
montrer au plus qu'on a seulement une relation du type 
$X\circ(L_i)^{\alpha}G^k -\ldots$ pas unitaire et qu'on devra localiser 
à trop de $L_i$. Le deuxième point important c'est que
\[[0:\ldots:0:1]\notin \pi_m(X)\]
parce que les $L_i$ s'annulent pas simultanément, en particulier, 
\[V(F_1,\ldots,F_s, z_0,\ldots, z_{n-d-1})=\emptyset\]
D'où ... il existe $k$ t.q. 
$z_{n-d}^k\in (F_1(\bar z),\ldots, F_s(\bar z))+(z_i, i) \subset 
k[z_0,\ldots, z_{n-d}]$. Tu vois le truc venir ? En particulier, 
si on regarde dans $\pi_m(X)$ on obtient 
\[\psi(z_1,\ldots,z_{n-d})=z_{n-d}-\sum z_j H_j = 0\mod I(\pi_m(X))\]
En regardant dans $z_i\ne 0$ on obtient 
\[k[\A^{n-d}\cap D(z_i)]=k[z_0,\ldots, z_{n-d}]_{z_i}\to 
\Or_X(D(z_i\circ \pi_m)\cap X)\]
la flèche $z_j\mapsto L_j^m/L_i^m$ et $z_{n-d}\mapsto G/L_i^m$. De sorte
que dans $\Or_X(D(z_i\circ \pi_m)\cap X)$ on ait 
\[\psi(G_i/L_i^m)=0;
    ~\psi\in k[\pi(X)\cap D(T_{i})][T]\textrm{ unitaire.}
\]




\chapter{Traductions variétés vers k-algèbres}
\section{L'équivalence de catégorie de base}
Quand on a $\varphi\colon k[Y]\to k[X];~(\overline{Y_i})_i\mapsto 
(g_i(\overline{X_j},j)_i$. Les $Y_i$ vérifient les équations de $Y$
donc par définition les $g_i(\overline{X}_j,j)$ aussi! D'où, la flèche 
$\varphi^a\colon X\to Y$ telle que 
\[(x_j)_{j=1,\ldots,n}\mapsto (g_i(x_j,j))_{j=1,\ldots, m}\]
est bien définie et régulière! Maintenant, si on regarde $\ker(\varphi)$,
c'est un idéal qui contient $I(Y)$ et dont les $g_i$ vérifient les 
équations ! D'où l'image de $\varphi^a$ est contenue dans 
\[Z(\ker(\varphi))\subset Y\]
en particulier, $\varphi^a$ \textbf{est dominante} si et seulement si 
$\varphi$ \textbf{est injective} !

À l'inverse, si $\varphi$ \textbf{est surjective} on obtient un
isomorphisme et donc une injection 
\[k[Y]/\ker(\varphi)\simeq k[X]\]
mais $k[Y]\to k[Y]/\ker(\varphi)$ est par définition l'injection 
$Z(\ker(\varphi))\subset Y$. En particulier, $X$ \textbf{s'injecte dans}
$Y$.
\section{Parler des fibres}
En fait la fibre $f^{-1}(y)$ vérifie des équations $f^*\m_y$ et 
$f(x)=y$ veut dire que $f^*\m_y\subset \m_x$. On peut obtenir 
des conditions de surjectivité.

\section{Nakayama}
Bon encore et toujours, le Nakayama, j'ai vu une preuve un peu en détail.
En gros, si $M$ est un $A$-module de type fini. Et si $IM=M$, on peut 
dire que plusieurs choses : soit $(m_0,\ldots, m_n)$ une base de $M$,
on a 

\[m_j = \sum_{i=1}^n \alpha_{ij}m_i\]
avec $\alpha_{ij}\in I$. D'où la matrice $A=(\alpha_{ij})$ vérifie,
$\ker(A-I_n)=M$ en tant qu'endomorphisme du $A$-module $M$. Ensuite, 
par la formule de Cramer, on a $\det(A-I_n).M=d.M=0$. Avec $d\in 1+I$, 
ça ca se vérifie en développant la diagonale. 

On a des objets maintenant il reste juste à trouver des critères sur
$1+I$ et $d$.

Si $A\subset B=M$ est un sur-anneau de $A$, on a $1_A\in M=B$,
d'où $aB=0$ seulement si $a=0$. On peut en déduire via 
$0\in1+I$ ssi $I=(1)$ que 
\[I\subset A\implies IB\subset B\]
où les inclusions sont strictes.

Maintenant pareil, si $1+I$ ne contient que des inversibles, on obtient
le critère avec $I=\bigcap_{\m\in Specm(A)} \m$ que $I.M=M$ implique
$M=0$, ici car $d$ est inversible. Plus généralement ducoup si $1+I$
ne contient que des inversible, si $M'+IM=M$ pour un $M'$ quelconque 
alors $M=M'$, y suffit d'appliquer la méthode à $M/M'$.

\textbf{À faire, le dernier critère : espace vectoriel sur le quotient
implique générateur d'idéal.}




\printbibliography
\end{document}

