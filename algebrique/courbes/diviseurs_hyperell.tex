\documentclass[12pt]{article}
\usepackage{amsmath, latexsym, amsfonts,amssymb, amsthm,
    amscd, geometry, xspace, enumerate,mathtools}

\usepackage[dvipsnames]{xcolor}
\usepackage{quiver, tikz}
\usepackage{mathrsfs}
\usepackage{hyperref}

\definecolor{wgrey}{RGB}{148, 38, 55}

\setlength{\oddsidemargin}{-10mm}
\setlength{\evensidemargin}{5mm}
\setlength{\textwidth}{175mm}
\setlength{\headsep}{0mm}
\setlength{\topmargin}{0mm}
\setlength{\textheight}{220mm}
\setlength\parindent{24pt}

\newcommand{\Z}{\mathbb{Z}}
\newcommand{\R}{\mathbb{R}}
\newcommand{\rel}{\omathcal{R}}
\newcommand{\Q}{\mathbb{Q}}
\newcommand{\C}{\mathbb{C}}
\newcommand{\N}{\mathbb{N}}
\newcommand{\K}{\mathbb{K}}
\newcommand{\A}{\mathbb{A}}
\newcommand{\B}{\mathcal{B}}
\newcommand{\stSheaf}{\mathcal{O}}

\newcommand{\cL}{\mathscr{L}}
\newcommand{\F}{\mathscr{F}}
\newcommand{\G}{\mathscr{G}}
\newcommand{\D}{\mathscr{D}}
\newcommand{\E}{\mathscr{E}}

\theoremstyle{plain}
\newtheorem{thm}[subsubsection]{Theorem}
\newtheorem{lem}[subsubsection]{Lemma}
\newtheorem{prop}[subsubsection]{Proposition}
\newtheorem{cor}[subsubsection]{Corollary}

\theoremstyle{definition}
\newtheorem{defn}[subsubsection]{Definition}
\newtheorem{rmq}[subsubsection]{Remark}
\newtheorem{conj}[subsubsection]{Conjecture}
\newtheorem{exmp}[subsubsection]{Examples}
\newtheorem{quest}[subsubsection]{Exercises}

\theoremstyle{remark}
\newtheorem{rem}{Remark}
\newtheorem{note}{Note}

\hypersetup{
    colorlinks=true,
    linkcolor=blue,
    urlcolor=wgrey,
    filecolor=wgrey
}

\definecolor{wgrey}{RGB}{148, 38, 55}


\begin{document}
Là le but c'est donner mon avancement sur ce que j'ai bossé. Et d'avoir
un avant après pour me rendre compte.
\section{Curves}
\subsection{07/07/2024, 22:31}
Ducoup aujourd'hui j'ai regardé les \textbf{courbes hyperelliptiques en
genre 2} et pourquoi on peut regarder les diviseurs non nuls de degré
$0$ comme:
\begin{itemize}
    \item $\sum_{i=1}^r (P_i) - r.(\infty)$.
    \item $r\leq g$.
\end{itemize}
L'idée c'était de partir de l'involution $y\mapsto -y$ pour étudier:
\begin{itemize}
    \item $(X-a)$ et voir que $div(X-a) = P_a+\iota P_a - 2(\infty)$.
    \item Ducoup fallait regarder $l((P_a)+(\iota P_a))$.
\end{itemize}
Au début c'était vraiment flou rien que la tête de la partie de degré $0$
d'un diviseur quelconque. Pareil la localisation dans l'anneau de 
fonction. Je suis tjr un peu dubitatif de ce que je fais. Bon là ce que
j'ai éclairci:
\begin{itemize}
    \item Pour calculer l'ordre exact en $P_a$ algébriquement fallait
        regarder dans les localisés: 
        \[(\F_q[X,Y]/(y^2-G(X)))_{\mathfrak m_{P_a}}\]
où $\mathfrak_{P_a} = (X-a, Y-y_a)$. J'arrivais pas à voir pourquoi 
$Y-y_a\in (X-a)$, me suis rendu compte que le point clé c'était de 
distinguer $y_a\ne 0$ et $=0$ (c'était clair en fait). Dans le premier
cas, l'idée c'est que $Y+y_a$ est inversible! On pouvait ensuite écrire
$Y^2-y_a = G(X)-G(a)$ et voir que $X-a\mid G(X)-G(a)$!! Cool non?
    \item Géometriquement on peut regarder la fibre de 
        $[X-a,1] : C\to \mathbb P^1$. Et là c'est clair les points
        ramifiés ou non. On peut calculer le degré en regardant la fibre
        en $[0,1]$.
\end{itemize}
Ce qui est marrant c'est que là c'était beaucoup plus clair 
géometriquement. On a juste à travailler dans $\F_q[X]$. Alors que
si on travaille sur la courbe elle même faut utiliser pleins de tricks.
En fait le théorème:
\[\sum_{\phi(P)= Q} e(P) = deg(\phi)\]
permet de se ramener à une étude sur le pullback de l'anneau de fonctions
! 

\textbf{Pour la suite}:
\begin{itemize}
    \item Il s'avère que $\F_q[X]$ est de Dedekind car principal. Donc
        on a toute la théorie habituelle des extensions d'anneaux de 
        Dedekind! Explorer ça. 
    \item La différence c'est que le groupe de classe a pas l'air fini.
        Vu qu'on a pas d'analyse et de Minkowski.
        On a aussi \textbf{pas de place archimédienne}! C'est quoi l'impact?
    \item Y'a certaines correspondance entre groupe de classe et groupe
        de diviseurs, faut voir p.q!
    \item Regarder un peu plus la ramification comme dans la remarque du
        dessus.
\end{itemize}


\end{document}
