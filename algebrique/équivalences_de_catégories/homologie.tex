\documentclass[a4paper,12pt]{book}
\usepackage{amsmath,  amsthm,enumerate}
\usepackage{csquotes}
\usepackage[provide=*,french]{babel}
\usepackage[dvipsnames]{xcolor}
\usepackage{quiver, tikz}

%symbole caligraphique
\usepackage{mathrsfs}

%hyperliens
\usepackage{hyperref}

%pseudo-code
\usepackage{algorithm}
\usepackage{algpseudocode}

\usepackage{fancyhdr}

\pagestyle{fancy}
\addtolength{\headwidth}{\marginparsep}
\addtolength{\headwidth}{\marginparwidth}
\renewcommand{\chaptermark}[1]{\markboth{#1}{}}
\renewcommand{\sectionmark}[1]{\markright{\thesection\ #1}}
\fancyhf{}
\fancyfoot[C]{\thepage}
\fancyhead[LO]{\textit \leftmark}
\fancyhead[RE]{\textit \rightmark}
\renewcommand{\headrulewidth}{0pt} % and the line
\fancypagestyle{plain}{%
    \fancyhead{} % get rid of headers
}

%bibliographie
\usepackage[
backend=biber,
style=alphabetic,
sorting=ynt
]{biblatex}

\addbibresource{bib.bib}

\usepackage{appendix}
\renewcommand{\appendixpagename}{Annexe}

\definecolor{wgrey}{RGB}{148, 38, 55}

\setlength\parindent{24pt}

\newcommand{\Z}{\mathbb{Z}}
\newcommand{\R}{\mathbb{R}}
\newcommand{\rel}{\omathcal{R}}
\newcommand{\Q}{\mathbb{Q}}
\newcommand{\C}{\mathbb{C}}
\newcommand{\N}{\mathbb{N}}
\newcommand{\K}{\mathbb{K}}
\newcommand{\A}{\mathbb{A}}
\newcommand{\B}{\mathcal{B}}
\newcommand{\Or}{\mathcal{O}}
\newcommand{\F}{\mathscr F}
\newcommand{\Hom}{\textrm{Hom}}
\newcommand{\disc}{\textrm{disc}}
\newcommand{\Pic}{\textrm{Pic}}
\newcommand{\End}{\textrm{End}}
\newcommand{\Spec}{\textrm{Spec}}
\newcommand{\Supp}{\textrm{Supp}}
\renewcommand{\Im}{\textrm{Im}}


\newcommand{\m}{\mathfrak{m}}
\newcommand{\p}{\mathfrak{p}}


\newcommand{\cL}{\mathscr{L}}
\newcommand{\G}{\mathscr{G}}
\newcommand{\D}{\mathscr{D}}
\newcommand{\E}{\mathscr{E}}
\renewcommand{\Pr}{\mathbb{P}}
\renewcommand{\P}{\mathscr{P}}
\renewcommand{\H}{\mathscr{H}}

\makeatletter
\newcommand{\colim@}[2]{%
  \vtop{\m@th\ialign{##\cr
    \hfil$#1\operator@font colim$\hfil\cr
    \noalign{\nointerlineskip\kern1.5\ex@}#2\cr
    \noalign{\nointerlineskip\kern-\ex@}\cr}}%
}
\newcommand{\colim}{%
  \mathop{\mathpalette\colim@{\rightarrowfill@\scriptscriptstyle}}\nmlimits@
}
\renewcommand{\varprojlim}{%
  \mathop{\mathpalette\varlim@{\leftarrowfill@\scriptscriptstyle}}\nmlimits@
}
\renewcommand{\varinjlim}{%
  \mathop{\mathpalette\varlim@{\rightarrowfill@\scriptscriptstyle}}\nmlimits@
}
\makeatother

\theoremstyle{plain}
\newtheorem{thm}[subsection]{Théoreme}
\newtheorem{lem}[subsection]{Lemme}
\newtheorem{prop}[subsection]{Proposition}
\newtheorem{cor}[subsection]{Corollaire}
\newtheorem{heur}{Heuristique}
\newtheorem{rem}{Remarque}
\newtheorem{note}{Note}

\theoremstyle{definition}
\newtheorem{conj}{Conjecture}
\newtheorem{prob}{Problème}
\newtheorem{quest}{Question}
\newtheorem{prot}{Protocole}
\newtheorem{algo}{Algorithme}
\newtheorem{defn}[subsection]{Définition}
\newtheorem{exmp}[subsection]{Exemples}
\newtheorem{exo}[subsection]{Exercices}
\newtheorem{ex}[subsection]{Exemple}
\newtheorem{exs}[subsection]{Exemples}

\theoremstyle{remark}

\definecolor{wgrey}{RGB}{148, 38, 55}
\definecolor{wgreen}{RGB}{100, 200,0} 
\hypersetup{
    colorlinks=true,
    linkcolor=wgreen,
    urlcolor=wgrey,
    filecolor=wgrey
}

\title{Équivalences de catégories en géometrie algébrique}
\date{}

\begin{document}
\maketitle
\tableofcontents



\section{L'équivalence de catégorie avec les variétés abstraites affines}
Essentiellement, d'un côté on a une flèche topologique continue 
\[|f|\colon X\to Y\]
et une flèche de faisceau localement annelée
\[f^\sharp\colon \Or_Y\to f_*\Or_X\]
on a pas forcément une flèche polynomiale encore de $\A^n\to \A^m$.
On sait quand même que $\Or_Y(Y)\to \Or_X(X)$ correspond à 
$f_*\colon A(Y)\to A(X)$, faut juste relever en $A(\A^m)\to A(X)$
et obtenir $X\to \A^m$ comme une restriction $\A^n\to \A^m$! À l'inverse,
si on a $f\colon X\to Y$ polynomiale, on obtient $A(Y)\to A(X)$. Faut
juste obtenir $\Or_Y(U)\to \Or_X(f^{-1} U)$, mais c'est clair que le 
pullback de fonctions marche! Parce que si on a 
\[g\in \Or_Y(U)\]
alors $g\circ f\in \Or_X(f^{-1}U)$ est régulière, car localement
les fractions sont données $f_*P/f_*Q$.




\section{L'équivalence de catégorie avec schémas affines}
\subsection{L'équivalence}
On regarde $\Spec\colon Ring\to Aff$, c'est essentiellement surjectif
par définition et $\Gamma\circ \Spec=id$. Ducoup faut montrer que c'est
pleinement fidèle. En reprenant les définitions, de 
\[(f,f^\sharp)\colon \Spec(B)\to \Spec(A)\]
un morphisme de schémas affines faut montrer que si $\varphi:=\Gamma(f)$
alors $\Spec(\varphi)$, la flèche de pullback, est égale à $f$. On peut 
montrer que $\Spec(\varphi)=f$ et $\varphi_x=f^\sharp_x$ pour tout $x$.
Dire que $f^\sharp(\p_x)=\varphi^{-1}(\p_x)$ on peut le déduire de 
l'existence de 
% https://q.uiver.app/#q=WzAsNixbMSwwLCJCIl0sWzAsMCwiQSJdLFswLDEsIkFfe1xccF97Zih4KX19Il0sWzEsMSwiQl97XFxwX3h9Il0sWzAsMiwiXFxrYXBwYShmKHgpKSJdLFsxLDIsIlxca2FwcGEoeCkiXSxbMSwwLCJcXHZhcnBoaSJdLFsxLDIsImlfeCIsMl0sWzIsMywiZl5cXHNoYXJwX3giLDJdLFswLDMsImpfeCJdLFsyLDRdLFszLDVdLFs0LDVdXQ==
\[\begin{tikzcd}
	A & B \\
	{A_{\p_{f(x)}}} & {B_{\p_x}} \\
	{\kappa(f(x))} & {\kappa(x)}
	\arrow["\varphi", from=1-1, to=1-2]
	\arrow["{i_x}"', from=1-1, to=2-1]
	\arrow["{j_x}", from=1-2, to=2-2]
	\arrow["{f^\sharp_x}"', from=2-1, to=2-2]
	\arrow[from=2-1, to=3-1]
	\arrow[from=2-2, to=3-2]
	\arrow[from=3-1, to=3-2]
\end{tikzcd}\]
non trivial, on peut voir ça en regardant le carré du haut, on a 
\begin{enumerate}
    \item $j_x^{-1}(\p_xB_{\p_x})=\p_x$.
    \item $i_{f(x)}^{-1}(\p_{f(x)}A_{\p_{f(x)}})=\p_{f(x)}$
\end{enumerate}
et $(f^\sharp_x)^{-1}(\p_x)\subset \p_{f(x)}$, comme $j_x\circ\varphi=
f_x^\sharp\circ i_x$. On obtient 
\[\varphi^{-1}(\p_x)\subset \p_{f(x)}\]
l'inclusion inverse est conséquence directe du fait que c'est
localement annelé. Faut quand même montrer que $\Spec(\varphi)^\sharp$
donné par les flèches induites par $\varphi$ est égale à $f^\sharp$.
Mais ça c'est clair sur les fibres par unicité donc partout.






\chapter{Traductions variétés vers k-algèbres}
\section{L'équivalence de catégorie de base}
Quand on a $\varphi\colon k[Y]\to k[X];~(\overline{Y_i})_i\mapsto 
(g_i(\overline{X_j},j)_i$. Les $Y_i$ vérifient les équations de $Y$
donc par définition les $g_i(\overline{X}_j,j)$ aussi! D'où, la flèche 
$\varphi^a\colon X\to Y$ telle que 
\[(x_j)_{j=1,\ldots,n}\mapsto (g_i(x_j,j))_{j=1,\ldots, m}\]
est bien définie et régulière! Maintenant, si on regarde $\ker(\varphi)$,
c'est un idéal qui contient $I(Y)$ et dont les $g_i$ vérifient les 
équations ! D'où l'image de $\varphi^a$ est contenue dans 
\[Z(\ker(\varphi))\subset Y\]
en particulier, $\varphi^a$ \textbf{est dominante} si et seulement si 
$\varphi$ \textbf{est injective} !

À l'inverse, si $\varphi$ \textbf{est surjective} on obtient un
isomorphisme et donc une injection 
\[k[Y]/\ker(\varphi)\simeq k[X]\]
mais $k[Y]\to k[Y]/\ker(\varphi)$ est par définition l'injection 
$Z(\ker(\varphi))\subset Y$. En particulier, $X$ \textbf{s'injecte dans}
$Y$.
\section{Parler des fibres}
En fait la fibre $f^{-1}(y)$ vérifie des équations $f^*\m_y$ et 
$f(x)=y$ veut dire que $f^*\m_y\subset \m_x$. On peut obtenir 
des conditions de surjectivité.





\printbibliography
\end{document}

