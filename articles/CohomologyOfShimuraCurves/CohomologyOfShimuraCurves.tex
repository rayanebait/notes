\documentclass[amsart,10pt]{article}
\usepackage{amsmath,  amsthm, enumerate}
\usepackage{csquotes}
\usepackage[provide=*,french]{babel}
\usepackage[dvipsnames]{xcolor}
\usepackage{quiver, tikz}

%symbole caligraphique
\usepackage{mathrsfs}

%hyperliens
\usepackage{hyperref}

%pseudo-code
\usepackage{algpseudocode}
\usepackage{algorithm}
\makeatletter
  \renewcommand{\ALG@name}{Algorithme}
  \makeatother


%bibliographie
\usepackage[
backend=biber,
style=alphabetic,
sorting=ynt
]{biblatex}

\addbibresource{bib.bib}

\usepackage{appendix}
\renewcommand{\appendixpagename}{Annexe}

\definecolor{wgrey}{RGB}{148, 38, 55}

\setlength\parindent{24pt}

\newcommand{\Z}{\mathbb{Z}}
\newcommand{\R}{\mathbb{R}}
\newcommand{\rel}{\omathcal{R}}
\newcommand{\Q}{\mathbb{Q}}
\newcommand{\C}{\mathbb{C}}
\newcommand{\N}{\mathbb{N}}
\newcommand{\K}{\mathbb{K}}
\newcommand{\A}{\mathbb{A}}
\newcommand{\B}{\mathcal{B}}
\newcommand{\Or}{\mathcal{O}}
\newcommand{\F}{\mathbb F}
\newcommand{\m}{\mathfrak m}
\renewcommand{\b}{\mathfrak b}
\renewcommand{\a}{\mathfrak a}
\newcommand{\p}{\mathfrak p}
\newcommand{\I}{\mathfrak I}
\newcommand{\Hom}{\textrm{Hom}}
\newcommand{\disc}{\textrm{disc}}
\newcommand{\Pic}{\textrm{Pic}}
\newcommand{\End}{\textrm{End}}
\newcommand{\Spec}{\textrm{Spec}}
\newcommand{\Frac}{\textrm{Frac}}


\newcommand{\PSL}{\textrm{PSL}}
\newcommand{\SL}{\textrm{SL}}
\newcommand{\h}{\mathfrak h}

\newcommand{\cL}{\mathscr{L}}
\newcommand{\G}{\mathscr{G}}
\newcommand{\D}{\mathscr{D}}
\newcommand{\E}{\mathscr{E}}
\newcommand{\U}{\mathscr{U}}

\theoremstyle{plain}
\newtheorem{thm}{Théoreme}
\newtheorem{lem}{Lemme}
\newtheorem{prop}{Proposition}
\newtheorem{cor}{Corollaire}
\newtheorem{heur}{Heuristique}
\newtheorem{rem}{Remarque}
\newtheorem{rembis}{Remarque}
\newtheorem{note}{Note}

\theoremstyle{definition}
\newtheorem{conj}{Conjecture}
\newtheorem*{eq}{Équivalences}
\newtheorem{prob}{Problème}
\newtheorem{quest}{Question}
\newtheorem{prot}{Protocole}
\newtheorem{algo}{Algorithme}
\newtheorem{defn}{Définition}
\newtheorem{defnbis}{Définition}
\newtheorem{ex}{Exemple}
\newtheorem{exo}{Exercices}

\theoremstyle{remark}

\definecolor{wgrey}{RGB}{148, 38, 55}
\definecolor{wgreen}{RGB}{100, 200,0} 
\hypersetup{
    colorlinks=true,
    linkcolor=wgreen,
    urlcolor=wgrey,
    filecolor=wgrey
}

\title{Cohomology of Shimura Curves in quasi-linear time}
\date{}

\begin{document}
\maketitle


Intérêt courbes de Shimura ?  Courbes modulaires -> modularité courbes ells. 
Courbes Shimura -> langlands surfaces abéliennes? ($SP_2(\Z)\subset GL_4$)

%Tentative 1
As can be seen by the existence of databases such as the LMFDB \cite{}, the Langlands
program and its algorithmic counterpart have been the subject of extensive efforts,
both in the theoritical and algorithmic number theory communities. 

The modularity of elliptic curves, can be seen as a part
of Langlands programme for $GL_2(\Q)$.
The $GL_2(F)$ case, where $F$ is a totally real number field
with a single split place at infinity.

%Tentative 2
Tabulating families of modular forms has remained a popular subject among 
the algorithmic number theory community. Partly fueled by various aspects of 
the Langlands programme. The study of classical modular forms 
can be seen as part of $GL_2(\Q)$-Langlands, 



Following the success of modular symbols in computing spaces of classical modular
forms, various algorithms have been made to generalize the idea to Shimura Curves
\cite{databasehilbert} (Citer le survey). In the classical setting as well as in
the Hilbert setting 


Tn : Trace formula -> modular symbols $\equiv$$ O(n....+d^3)->\tilde O(dn)$ 
En fait y font une petite remarque dans le cas classique pour dire que c'est de
l'algèbre linéaire sparse une fois qu'on a la base de Manin.


Manin's trick -> word problem

Amélioration : calcul de $H^1(\Gamma, W_k(\C))$ en temps quasi-linéaire pour tout
poids $(k_1,...,k_n)$ (ou au moins $(k,2,...,2)$. Si 
$\Gamma=\Gamma_0^D(N)$, calcul de $\Gamma_0^D(N)$ par revêtement de $\Gamma_0^D(1)$
en temps quasi-linéaire. 



Assumed $\h$, $\PSL_2(\R)$. 
\section{Preliminaries}
Let $G$ be a topological group and $X$ be a topological space.
We say a left action of $G$ on $X$ is 
\begin{itemize}
		\item \textbf{Discontinuous} if $G$ is endowed with the discrete
				topology.
		\item \textbf{Proper} if for any compact
sets $K$ and $K'$ in $X$, the set of $g\in G$ such that the intersection $K\cap g.K'$ is nonempty is compact.
\item\textbf{Properly discontinuous} if it is both discontinous
				and proper.
		\item\textbf{Totally discontinuous} if it is properly discontinous
				and free.
\end{itemize}
\begin{rem}
		One can also define a proper action of $G$ on $X$ to be 
		a $G$-action on $X$ such that the map 
		$G\times X\to X\times X$ defined by $(g,x)\mapsto (x,g.x)$
		is proper as a continuous map.
\end{rem}

La prop clé ici c'est que $\h$ est haussdorf localement compact.

An idiomatic example of a proper action is given by the action of
$\PSL_2(\R):=\SL_2(\R)/\{\pm I_2\}$ on the upper-half plane
$\h\subset \C$ by homographies. Further, discrete subgroups of 
$\PSL_2(\R)$ or so-called Fuchsian groups yield properly
discontinuous on $\h$.

\subsection{Fuchsian groups}
$\textrm{Isom}(\h)=\textrm{Isom}^+(\h)\cup \tau\textrm{Isom}^+(\h)$
avec $\tau\colon z\mapsto -\bar z$.
\begin{prop}
		$\PSL_2(\R)$ acts properly on $\h$.
\end{prop}
As a corollary, discrete subgroups of $\PSL_2(\R)$ act properly
discontinuously on $\h$.

\subsection{Arithmetic fuchsian groups}



\subsection{Fundamental groups of surfaces}







\end{document}
