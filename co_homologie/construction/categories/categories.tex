\documentclass[a4paper,12pt]{book}
\usepackage{amsmath,  amsthm,enumerate}
\usepackage{csquotes}
\usepackage[provide=*,french]{babel}
\usepackage[dvipsnames]{xcolor}
\usepackage{quiver, tikz}

%symbole caligraphique
\usepackage{mathrsfs}

%hyperliens
\usepackage{hyperref}

%pseudo-code
\usepackage{algorithm}
\usepackage{algpseudocode}

\usepackage{fancyhdr}

\pagestyle{fancy}
\addtolength{\headwidth}{\marginparsep}
\addtolength{\headwidth}{\marginparwidth}
\renewcommand{\chaptermark}[1]{\markboth{#1}{}}
\renewcommand{\sectionmark}[1]{\markright{\thesection\ #1}}
\fancyhf{}
\fancyfoot[C]{\thepage}
\fancyhead[LO]{\textit \leftmark}
\fancyhead[RE]{\textit \rightmark}
\renewcommand{\headrulewidth}{0pt} % and the line
\fancypagestyle{plain}{%
    \fancyhead{} % get rid of headers
}

%bibliographie
\usepackage[
backend=biber,
style=alphabetic,
sorting=ynt
]{biblatex}

\addbibresource{bib.bib}

\usepackage{appendix}
\renewcommand{\appendixpagename}{Annexe}

\definecolor{wgrey}{RGB}{148, 38, 55}

\setlength\parindent{24pt}

\newcommand{\Z}{\mathbb{Z}}
\newcommand{\R}{\mathbb{R}}
\newcommand{\rel}{\omathcal{R}}
\newcommand{\Q}{\mathbb{Q}}
\newcommand{\C}{\mathbb{C}}
\newcommand{\Cat}{\mathcal{C}}
\newcommand{\Dat}{\mathcal{D}}
\newcommand{\Aat}{\mathcal{A}}
\newcommand{\N}{\mathbb{N}}
\newcommand{\K}{\mathbb{K}}
\newcommand{\A}{\mathbb{A}}
\newcommand{\B}{\mathcal{B}}
\newcommand{\Or}{\mathcal{O}}
\newcommand{\F}{\mathscr F}
\newcommand{\Hom}{\textrm{Hom}}
\newcommand{\disc}{\textrm{disc}}
\newcommand{\Pic}{\textrm{Pic}}
\newcommand{\End}{\textrm{End}}
\newcommand{\Spec}{\textrm{Spec}}
\newcommand{\Supp}{\textrm{Supp}}
\newcommand{\Ouv}{\textrm{Ouv}}
\newcommand{\im}{\textrm{im}}
\newcommand{\coker}{\textrm{coker}}
\newcommand{\coim}{\textrm{coim}}


\newcommand{\cL}{\mathscr{L}}
\newcommand{\G}{\mathscr{G}}
\newcommand{\D}{\mathscr{D}}
\newcommand{\E}{\mathscr{E}}
\renewcommand{\P}{\mathscr{P}}
\renewcommand{\H}{\mathscr{H}}

\makeatletter
\newcommand{\colim@}[2]{%
  \vtop{\m@th\ialign{##\cr
    \hfil$#1\operator@font colim$\hfil\cr
    \noalign{\nointerlineskip\kern1.5\ex@}#2\cr
    \noalign{\nointerlineskip\kern-\ex@}\cr}}%
}
\newcommand{\colim}{%
  \mathop{\mathpalette\colim@{\rightarrowfill@\scriptscriptstyle}}\nmlimits@
}
\renewcommand{\varprojlim}{%
  \mathop{\mathpalette\varlim@{\leftarrowfill@\scriptscriptstyle}}\nmlimits@
}
\renewcommand{\varinjlim}{%
  \mathop{\mathpalette\varlim@{\rightarrowfill@\scriptscriptstyle}}\nmlimits@
}
\makeatother

\theoremstyle{plain}
\newtheorem{thm}{Théoreme}
\newtheorem{lem}{Lemme}
\newtheorem{prop}{Proposition}
\newtheorem{cor}{Corollaire}
\newtheorem{heur}{Heuristique}
\newtheorem{rem}{Remarque}
\newtheorem{note}{Note}

\theoremstyle{definition}
\newtheorem{conj}{Conjecture}
\newtheorem{prob}{Problème}
\newtheorem{quest}{Question}
\newtheorem{prot}{Protocole}
\newtheorem{algo}{Algorithme}
\newtheorem{defn}{Définition}
\newtheorem{exmp}{Exemples}
\newtheorem{exo}{Exercices}
\newtheorem{ex}{Exemple}
\newtheorem{exs}{Exemples}

\theoremstyle{remark}

\definecolor{wgrey}{RGB}{148, 38, 55}
\definecolor{wgreen}{RGB}{100, 200,0} 
\hypersetup{
    colorlinks=true,
    linkcolor=wgreen,
    urlcolor=wgrey,
    filecolor=wgrey
}

\title{Penser avec les catégories}
\date{2024-2025}

\begin{document}
\maketitle
\tableofcontents

\section{Yoneda et foncteurs représentables}


\chapter{Objets universels dans les catégories abéliennes}

\section{Remarques sur la variance}
Le bi-foncteur $h_{\_}(\_)$ est celui qu'on a besoin pour
la cohomologie. Et on a $h_M(A)=h^A(M)$.
\subsection{Le plongement de Yoneda}
Le foncteur $I\mapsto h_I$ est covariant et pleinement fidèle,
via $I\to I'$ et $M\to I$ fournit $M\to I\to I'$. Mais ca c'est Yoneda.
À l'inverse $P\mapsto h^P$ est contravariant vu que étendre 
naturellement $P'\to M$ c'est par pullback. 

\subsection{Spécialisation, foncteur représentable}
Donc une fois $I$ fixé, $h_I$ est contravariant, vu que
$M\to I$ s'étend par pullback. Et $h^P$ est covariant
vu que $P\to A$ s'étend par pushforward. Et surtout que 
$h_I(M)\to h_I(M')=h^M(I)\to h^{M'}(I)$ donc on a la variance
de $h^{\_}$.

\section{Exactitude du foncteur Hom}
Ducoup, l'exactitude de $A\to B\to C\to 0$
via $h_M$ est toujours vraie. On obtient
\[0\to h_M(C)\to h_M(B)\to h_M(A)\]
et pour obtenir l'exactitude à droite, il
faut que $M=I$ soit un injectif.

On a deux suites exactes à traiter
\[0\to A\to B\to C\]
et 
\[A\to B\to C\to 0\]
c'est la deuxième qui m'interesse pour
la cohomologie via $h_I$. Je regarde que
\[A\to B\to C\to 0\]
On en déduit deux
suites exactes
\[0\to h_I(C)\to h_I(B)\to h_I(A)\]
et
\[0\to h^C\to h^B\to h^A\]
la deuxième donne pas la propriété des injectifs.
L'exactitude de la deuxième se vérifie terme à terme
donc c'est équivalent l'autre.

\subsection{Preuves du deuxième cas}

\subsubsection{Injectivité et surjectivité sur les
côtés}

Le cas $0\to h^C\to h^B$ est clair l'injectivité c'est le fait que
un épi $B\to C\to 0$ vient de $(C\to M)\mapsto (B\to C\to M)$
est injectif.
Et ça se traduit en mono dans la catégorie opposée où
cette fois un mono $0\to A\to B$ fournit un mono
\[0\to h_A\to h_B\]
par la variance et $M\to A\mapsto M\to A\to B$ uniquement.
\begin{rem}
  Marrant y'a ptet un truc philosophique sur cette
  asymétrie ?
\end{rem}
\subsubsection{Exactitude au milieu où $\ker$ et $\coker$}

\begin{rem}
Pour rappel un noyau $K$ est donné par une
injection exacte :
\[0\to h_K\to h_B\to h_C\]
et un conoyau par
\[0\to h^{coK}\to (h^B\to h^A)=\ker(h^B\to h^A)\]
l'image elle c'est
\[0\to h_{Im}\to h_{B}\to h_{coK(A\to B)}=\ker(h_B\to h_{coK(A\to B)})\]
\end{rem}
Maintenant on veut pas exactement montrer que $h_K=h_{\im}$
sachant $K=\im$ vu que c'est immédiat. On veut montrer
que $\ker(h_I(B)\to h_I(A))=\im(h_I(C)\to h_I(B))$ pour tout $I$
mais le problème c'est que ça se traduit en
\[\ker(h^B\to h^A)=\im(h^C\to h^B)\]
et donc on a pas a priori la propriété universelle direct. Le fait
que ce soit pas intuitif c'est que on demande que de $B\to M$
on ait 
$A\to B\to M=0$ ssi $A\to B\to M$ se factorise par $A\to B\to C\to
M$ i.e.
% https://q.uiver.app/#q=WzAsNCxbMSwxLCJNIl0sWzEsMCwiQiJdLFswLDAsIkEiXSxbMiwwLCJDIl0sWzEsMF0sWzIsMV0sWzEsM10sWzMsMCwiIiwxLHsic3R5bGUiOnsiYm9keSI6eyJuYW1lIjoiZGFzaGVkIn19fV1d
\[\begin{tikzcd}
  A & B & C \\
	& M
	\arrow[from=1-1, to=1-2]
	\arrow[from=1-2, to=1-3]
	\arrow[from=1-2, to=2-2]
	\arrow[dashed, from=1-3, to=2-2]
\end{tikzcd}\]
On dirait que $M$ est un projectif. Ça s'éclaircit en rappelant que
% https://q.uiver.app/#q=WzAsNCxbMSwxLCJNIl0sWzEsMCwiQiJdLFswLDAsIkEiXSxbMiwwLCJDIl0sWzEsMF0sWzIsMV0sWzEsM10sWzMsMCwiIiwxLHsic3R5bGUiOnsiYm9keSI6eyJuYW1lIjoiZGFzaGVkIn19fV1d
\[\begin{tikzcd}
  A & B & C &0\\
   & M & 
	\arrow[from=1-1, to=1-2]
	\arrow[from=1-2, to=1-3]
	\arrow[from=1-3, to=1-4]
	\arrow[from=1-2, to=2-2]
	\arrow[dashed, from=1-3, to=2-2]
\end{tikzcd}\]
et là on peut définir $C\to M$ parce que $B\to C$ est un épi 
intuitivement. Plus concrètement l'idée c'est une extension de
celle du théorème d'iso.
Étant donné $coim$ comment je déf $coim\to im$! Là c'est l'inverse!
En fait on prouve que \[C=\coker(A\to B)\] via
$coim(B\to C)\simeq C$ dans la catégorie abélienne et
\[\coker(A\to B)=\coker(\im(A\to B)\to B)=\coker(\ker(B\to C)\to B)=\coim(B\to C)\] par exactitude au milieu. 


\subsubsection{Exactitude au milieu en bref}
On veut que $\ker(h^B\to h^A)=\im(h^C\to h^B)$, ça revient à dire
que $C=\coker(A\to B)$. Et on utilise l'exactitude pour voir que
$\coker(A\to B)=\coim(B\to C)=\im(B\to C)=C$. En termes de 
foncteurs. La coimage vérifie 
\[0\to h^{\coim(B\to C)}\to h^B\to h^{\ker(B\to C)}\]
\[h^{\coim(B\to C)}\]
et on a un pont $h^{\coim(B\to C)}=h_{\im(B\to C)}$.

\subsubsection{Le théorème d'iso est un pont entre $h^{\_}$ et $h_{\_}$}


\subsection{Preuves du premier cas}
Pour l'exactitude au milieu, on utilise seulement que 
\[0\to A \to B\]
exact implique $A\simeq \im(A\to B)$ en particulier, $M\to B$
tel que $M\to B\to C = 0_{M,C}$ implique 
$M\to B=M\to \im(A\to B)\to B$ et via l'isomorphisme on obtient
naturellement $M\to A$ tel que $M\to B$ et $M\to A\to B$.

\section{Injectifs et projectifs}
Pour rappel $h_I(\_):=\Hom(\_,I)$ et 
$h^{P}(\_):=\Hom(P,\_)$. Être injectif $I$ ça
revient à ce que l'exactitude de 
\[0\to M'\to M\]
devienne l'exactitude de $h_I(M)\to h_I(M')\to 0$.
C'est un peu bizarre à intuiter au sens où le
diagramme est
% https://q.uiver.app/#q=WzAsMyxbMCwwLCJNJyJdLFsxLDAsIk0iXSxbMCwxLCJJIl0sWzAsMSwiIiwwLHsic3R5bGUiOnsidGFpbCI6eyJuYW1lIjoiaG9vayIsInNpZGUiOiJ0b3AifX19XSxbMCwyXSxbMSwyLCIiLDAseyJzdHlsZSI6eyJib2R5Ijp7Im5hbWUiOiJkYXNoZWQifX19XV0=
\[\begin{tikzcd}
	{M'} & M \\
	I
	\arrow[hook, from=1-1, to=1-2]
	\arrow[from=1-1, to=2-1]
	\arrow[dashed, from=1-2, to=2-1]
\end{tikzcd}\]
et que de $M'\to I$ on a une flèche qui induit l'existence
de $M\to M$. C'est une condition de surjectivité! I.e.
$h_I$ est exacte à droite (en passant à la catégorie opposée). 
À l'inverse pour les projectifs
% https://q.uiver.app/#q=WzAsMyxbMCwwLCJNJyJdLFsxLDAsIk0iXSxbMCwxLCJQIl0sWzEsMCwiIiwyLHsic3R5bGUiOnsiaGVhZCI6eyJuYW1lIjoiZXBpIn19fV0sWzIsMF0sWzIsMSwiIiwyLHsic3R5bGUiOnsiYm9keSI6eyJuYW1lIjoiZGFzaGVkIn19fV1d
\[\begin{tikzcd}
	{M'} & M \\
	P
	\arrow[two heads, from=1-2, to=1-1]
	\arrow[from=2-1, to=1-1]
	\arrow[dashed, from=2-1, to=1-2]
\end{tikzcd}\]

\section{Monomorphismes et épimorphismes}

En résumé $A\to B$ est un mono veut
dire que 
\[h_A\to h_B\]
est injectif via $i_*=h_i:=(M\to A)\mapsto (M\to A\to B)$.
(ca se vérif bien terme à terme) et pour la remarque, 
$h_A$ est un foncteur de $\Aat$ dans $Ab$
dans le cadre des catégories abéliennes donc
c'est une catégorie abélienne et l'injectivité 
se traduit en $0\to h_A\to h_B$.
\subsection{Détail}
On a $0\to A\to B$ est exacte c'est pareil que
\[0\to \Hom_{\Cat}(M,A)\to \Hom_{\Cat}(M,B)\]
est injective via le pushforward : 
$i\circ f =i\circ g\implies f=g$. En passant à
la catégorie opposée dans $\Cat$, on obtient les
épis par $B\to A\to 0$ et les flèches c'est :
\[\Hom_{\Cat^op}(B,M)\to \Hom_{\Cat^op}(A,M)\to 0\]
est surjective via $i_*^{op}$. D'où
\[\Hom_{\Cat}(M,B)\to \Hom_{\Cat}(M,A)\to 0\]
est surjective.
\begin{rem}
  C'est comme ça que les flèches s'inversent!
  La condition
\[\Hom_{\Cat}(B,M)\to \Hom_{\Cat}(A,M)\to 0\]
  via le pullback est bizarre. Ça veut dire que
  toutes les flèches $A\to M$ proviennent de 
  $B\to M$ sachant que $A\hookrightarrow B$ 
  s'injecte dans $B$. C'est pas exactement tout
  de suite en lien avec les injectifs.
\end{rem}
En résumé, 
\section{Ker et coker}
Un $\ker$ c'est ça :
% https://q.uiver.app/#q=WzAsNCxbMCwwLCJLIl0sWzEsMCwiQSJdLFsyLDAsIkIiXSxbMCwxLCJNIl0sWzAsMV0sWzEsMl0sWzMsMV0sWzMsMCwiIiwxLHsic3R5bGUiOnsiYm9keSI6eyJuYW1lIjoiZGFzaGVkIn19fV0sWzMsMiwiT197TSxCfSIsMSx7ImN1cnZlIjoxfV1d
\[\begin{tikzcd}
	K & A & B \\
	M
	\arrow[from=1-1, to=1-2]
	\arrow[from=1-2, to=1-3]
	\arrow[dashed, from=2-1, to=1-1]
	\arrow[from=2-1, to=1-2]
	\arrow["{O_{M,B}}"{description}, curve={height=6pt}, from=2-1, to=1-3]
\end{tikzcd}\]
Maintenant, on peut le traduire en 
\[h_{K}\to h_A\to h_B\]
est exacte. En plus, $K\to A$ est un mono par
le propriété universelle. D'où
\[0\to h_K\to h_A\to h_B\]
est exacte. 



\section{Snake Lemma}
Avec élément c'est assez clair comment on construit $\delta$ la
flèche de connexion. Sans élément ça l'es moins. J'ai eu une
idée et je me suis spoil la suite. On peut faire comme ça, donc
on regarde $\ker(w)\to C\to B\to B'\to A\to \coker(u)$. Et l'idée
c'est que on regarde des éléments dans $B$ tels que $B\to C\to C'$
est nulle, et pour qu'y soient bien définis, on les regarde modulo
$A$. En résumé, on regarde \[\ker(B\to C\to C')=B''\] on sait que
$(B''\to B\to B')\to C'=0$ donc on obtient $B''\to A'$. Pour 
conclure on a clairement $A\to B''$ et ça fournit
\[B''/A\to A'/A=\coker(u)\] en plus comme
$B\to C$ est un épi, $B''\to B$ est un mono, et $A\to B\to C$ est
exacte, on obtient que 
\[B''/A=B''/\ker(B\to C)\simeq\im( B''\to B\to C)=\ker(w)\] d'où
\[\ker(w)\to \coker(u)\]

\section{Obtenir des flèches}
Quelques tricks pour obtenir des flèches. Étant donné
$A\to B\to C$, on a $\ker(A\to B)\to \ker(A\to B\to C)$.
Parce que \[\ker( A\to B)\to A\to B\to C=(\ker(A\to B)\to A\to B)\to C\]
est nulle d'où la flèche. À l'inverse on a 
\[\coker(A\to B\to C)\to \coker(B\to C)\]
via 
\[A\to B\to C\to \coker(B\to C)=A\to (B\to C\to \coker(B\to C))=0\]
d'où la flèche. Aussi, on a 
\[\coker(A\to B)\to \coker(A\to B\to C)\]
via \[A\to B\to (C\to \coker(A\to B\to C))=(A\to B \to C\to \coker(A\to B\to C))=0\]
d'où la flèche. En particulier on obtient la flèche du théorème
d'isomorphisme.
\section{Obtenir des flèches, version 2}
En fait on peut regarder 
% https://q.uiver.app/#q=WzAsOSxbMSwwLCJYIl0sWzIsMCwiWSJdLFsyLDEsIloiXSxbMSwxLCJaIl0sWzMsMCwiXFxjb2tlcihmKSJdLFs0LDAsIjAiXSxbMywxLCIwIl0sWzAsMSwiMCJdLFswLDAsIjAiXSxbMCwxLCJmIl0sWzEsMiwiZyJdLFswLDMsImdcXGNpcmMgZiIsMl0sWzMsMiwiaWQiLDJdLFsxLDRdLFs0LDVdLFsyLDZdLFs0LDZdLFs3LDNdLFs4LDBdXQ==
\[\begin{tikzcd}
	0 & X & Y & {\coker(f)} & 0 \\
	0 & Z & Z & 0
	\arrow[from=1-1, to=1-2]
	\arrow["f", from=1-2, to=1-3]
	\arrow["{g\circ f}"', from=1-2, to=2-2]
	\arrow[from=1-3, to=1-4]
	\arrow["g", from=1-3, to=2-3]
	\arrow[from=1-4, to=1-5]
	\arrow[from=1-4, to=2-4]
	\arrow[from=2-1, to=2-2]
	\arrow["id"', from=2-2, to=2-3]
	\arrow[from=2-3, to=2-4]
\end{tikzcd}\]
et on obtient d'un coup
\[0\to \ker(g\circ f)\to \ker(g)\to \coker(f)\to \coker(g\circ f)\to \coker(g)\to 0\]

\section{Théorème d'isomorphisme}
On regarde $\ker(A\to B)\to A\to B$, alors
\[\coim(A\to B)=\coker(\ker(A\to B)\to A)\to \coker((\ker(A\to B)\to A)\to B)=\coker(0\to B)=B\]
et la flèche est bien induite pas $A\to B$.

\chapter{Catégories abéliennes}
On demande à avoir un objet zéro, les
produits finis et que les $\Hom$ soient des
groupes. Ça donne une catégorie additive. Ensuite
on ajoute les noyaux. Grâce

\section{Codiagonale, produit = coproduit}
Ducoup juste un point, étant donné les produits
à deux éléments on a tout les produits finis. Et
si on des flèches nulles on peut voir $A\times B$
comme un coproduit via $i_A$ et $i_B$. Puis
\[\begin{tikzcd}
	& X \\
	A & {A\times B} & B
	\arrow["f"{description}, from=2-1, to=1-2]
	\arrow["{i_A}"{description}, from=2-1, to=2-2]
	\arrow["h"{description}, from=2-2, to=1-2]
	\arrow["g"{description}, from=2-3, to=1-2]
	\arrow["{i_B}"{description}, from=2-3, to=2-2]
\end{tikzcd}\]
où $h=f\circ p_A+g\circ p_B$ fait de $A\times B$ un coproduit.
On définit ensuite $\delta_B$ la codiagonale
% https://q.uiver.app/#q=WzAsNCxbMSwxLCJCXFx0aW1lcyBCIl0sWzAsMSwiQiJdLFsxLDAsIkIiXSxbMiwxLCJCIl0sWzEsMCwiaV97QiwxfSIsMl0sWzAsMiwiIiwyLHsic3R5bGUiOnsiYm9keSI6eyJuYW1lIjoiZGFzaGVkIn19fV0sWzMsMCwiaV97QiwyfSJdLFsxLDIsImlkX0IiLDFdLFszLDIsImlkX0IiLDFdXQ==
\[\begin{tikzcd}
	& B \\
	B & {B\times B} & B
	\arrow["{id_B}"{description}, from=2-1, to=1-2]
	\arrow["{i_{B,1}}"', from=2-1, to=2-2]
	\arrow[dashed, from=2-2, to=1-2]
	\arrow["{id_B}"{description}, from=2-3, to=1-2]
	\arrow["{i_{B,2}}", from=2-3, to=2-2]
\end{tikzcd}\]
ça donne une somme de $f\colon A\to B$ via
% https://q.uiver.app/#q=WzAsNSxbMCwxLCJBIl0sWzIsMSwiQlxcdGltZXMgQiJdLFsyLDAsIkIiXSxbMiwyLCJCIl0sWzQsMSwiQiJdLFswLDEsIihmLGcpIiwxLHsic3R5bGUiOnsiYm9keSI6eyJuYW1lIjoiZGFzaGVkIn19fV0sWzAsMywiZyIsMV0sWzAsMiwiZiIsMV0sWzEsMywicF97QiwyfSIsMV0sWzEsMiwicF97QiwxfSIsMV0sWzEsNCwiXFxkZWx0YV9CIiwxXSxbMiw0XSxbMyw0XV0=
\[\begin{tikzcd}
	&& B \\
	A && {B\times B} && B \\
	&& B
	\arrow[from=1-3, to=2-5]
	\arrow["f"{description}, from=2-1, to=1-3]
	\arrow["{(f,g)}"{description}, dashed, from=2-1, to=2-3]
	\arrow["g"{description}, from=2-1, to=3-3]
	\arrow["{p_{B,1}}"{description}, from=2-3, to=1-3]
	\arrow["{\delta_B}"{description}, from=2-3, to=2-5]
	\arrow["{p_{B,2}}"{description}, from=2-3, to=3-3]
	\arrow[from=3-3, to=2-5]
\end{tikzcd}\]
Bon ducoup, si on a des produits on a des coproduits. 
\begin{defn}
  Une catégorie additive c'est produits finis, coproduits finis,
  objet $O$, d'où les $\Hom$ sont des monoides abéliens puis 
  on dit qu'en fait des groupes.
\end{defn}
\section{Catégories abéliennes}
On ajoute les noyaux et conoyaux à une catégorie additive puis
on dit que $\coim(A\to B)\to \im(A\to B)$ est un isomorphisme.
\begin{rem}
  Attention c'est pas évident que $A\to \im(A\to B)$ soit un
  épi. Mais c'est vrai je crois.
\end{rem}

\subsection{Splitting}
Dans une catégorie additive, on a un splitting, ssi 
$B\simeq A\oplus C$. L'idée c'est que $p_C$ et $p\circ f$ sont
des flèches universelles $A\oplus C\to C$ donc par unicité puis
par le five lemma avoir une section implique splitter.

\section{Foncteurs entre catégories abéliennes}
On veut montrer que préserver les produits à deux éléments
implique être additif. En gros faut montrer que c'est équivalent
à préserver les coproduits pour la codiagonale. L'idée c'est
de montrer que pour $F\colon \Cat\to \Dat$ on a
\begin{enumerate}
  \item $F(O_{\Cat}\times A)=F(0_{\Cat})\times F(A)$ d'où 
    $h_{F(O_{\Cat})}(D)$ est réduit à un point via la prop
    universelle 
    $h_{F(O_{\Cat})}\times h_{F(A)}=h_{O_{\Cat}\times F(A)}$.
  \item Ensuite faut montrer que $i_A\colon A\to A\sqcup B$ est
    préservée. Parce que préserver un coproduit c'est préserver
    les flèches avec. Suffit d'utiliser que $id_A$ et $O_{A,B}$
    sont préservés ! On sait déjà que 
    $F(A)\times F(B)\simeq F(A)\sqcup F(B)$.
  \item Maintenant on peut montrer que c'est additif simplement
    via la codiagonale.
  \item Additif implique produit c'est juste de montrer directement
    que c'est un produit.
\end{enumerate}
À noter sur additif implique préserve les produits, on cherche
$D\to F(A\times B)$ et on peut définir $i_A\circ f + i_B\circ g$!
Tout commute et l'unicité est donnée par 
\[h=id_{F(A\times B)}h=F(i_A)F(p_A)h+F(i_B)F(p_B)h\]

\section{Foncteurs exacts}
Y se passe un truc marrant.
\subsection{Suites exactes splittées}
C'est clair que $0\to A\to A\oplus B\to B\to 0$ est exacte. Pour à 
droite on peut regarder $0\to h^C \to h^A\times h^B\to h^B$ 
qui est exacte dans $Ab$ et 
$0\to h_A\to h_{A\sqcup B}=h_A\sqcup h_B=h_A\oplus h_B\to h_B$.

\subsection{Foncteurs exacts sont additifs.}
En gros $F$ exact à gauche implique
\[0\to F(A)\to F(A\oplus B)\to F(B)\]
exacte et $F$ préserve la section $F(i_B)$ d'où c'est un épi
à droite ! Maintenant 
\[0\to F(A)\to F(A\oplus B)\to F(B)\to 0\]
split d'où $F(A\oplus B)\simeq F(A)\oplus F(B)$. En particulier,
la codiagonale est préservée.


\section{Injectifs et projectifs}
Ducoup via la section d'avant on peut montrer que
$I\oplus J$ est injectif si $I$ et $J$ le sont via
\[0\to I\to I\oplus J\to J\to 0\]
en utilisant $h_{I\oplus J}=h_I\oplus h_J$ grâce à la section
d'avant. Ça devient clair parce que ça traduit la prop universelle
du produit cet iso.
\subsection{Adjonctions, exactitude et injectifs/projectifs}
On regarde $F\colon \Cat\rightleftarrows \Dat \colon G$ des 
adjoints, pour connaître le sens on regarde
\[h_{\_}(F(\_))=\Hom_{\Dat}(F(\_),\_)\simeq\Hom_{\Cat}(\_,G(\_))=h_{G(\_)}(\_)\]
Là, $F$ est adjoint à gauche et $G$ à droite. On a 
\begin{enumerate}
  \item $F$ exact à droite implique $G$ exact à gauche. Et
    inversement.
  \item $F$ exact implique $G$ préserve les injectifs.
\end{enumerate}
On regarde 
\[0\to G(A)\to G(B)\to G(C)\]
alors 
\[0\to h_{G(A)}\to h_{G(B)}\to h_{G(C)}\]
est exacte car isomorphe à
\[0\to h_{A}(F(\_))\to h_{B}(F(\_))\to h_{C}(F(\_))\]
qui exact pour pleins de raisons mdr. Pour préserver
les injectifs c'est juste que 
\[h_{G(I)}=h_{I}\circ F\]
qui sont deux foncteurs exacts.

\subsection{Dans $Mod_R$}
Je serai bref, en gros on veut que $0\to A\to B$ implique
$h_I(B)\to h_I(A)\to 0$. Y suffit de le vérifier pour tout 
les idéaux $A=I\subset R=B$! Maintenant, on montre que
\begin{itemize}
  \item Injectif implique divisible.
  \item $R$ est principal implique l'inverse.
\end{itemize}
En particulier dans $\Z$.
\subsubsection{Ab a assez d'injectifs}
On prends $A\in Ab$ et $S\subset A$ une famille génératrice.
Alors $p\colon \bigoplus_{s\in S}s\Z\to A\to 0$ via $s\Z\to A$
donnée par $s\mapsto s.1_A$. Puis 
\[(\bigoplus_{s\in S}s\Z)/K\simeq A\]
et maintenant on regarde
\[(\bigoplus_{s\in S}s\Z)\hookrightarrow (\bigoplus_{s\in S}s\Q)\]
puis
\[(\bigoplus_{s\in S}s\Z)\to (\bigoplus_{s\in S}s\Q)\to (\bigoplus_{s\in S}s\Q)/K\]
qui passe au quotient et le truc c'est que 
$K\to (\bigoplus_{s\in S}s\Q)$ c'est bien déf c'est juste
\[\coker(K\to (\bigoplus_{s\in S}s\Z)\to (\bigoplus_{s\in S}s\Q))\]
d'où \[A\hookrightarrow (\bigoplus_{s\in S}s\Q)/K\]
et le truc de droite est divisible donc injectif.

\subsubsection{$Mod_R$ a assez d'injectifs}
L'idée c'est que 
\[Res\colon Mod_R\rightleftarrows Ab\colon \Hom_{\Z}(R,\_)\]
sont adjoints et $Res$ est exact d'où 
\[\Hom_R(\_,G(I))=h_I\circ Res\] est la
composition de deux foncteurs exacts. Où $Res$ c'est juste le
foncteur d'oubli.

\begin{rem}
  Comme $f^*$ est exact de $Sh(Y)\to Sh(X)$, $f_*$ préserve
  les injectifs.
\end{rem}

\chapter{Cohomologie}
J'écris $A^{n-1}\to A^n\to A^{n+1}$ de manière croissante 
(cohomologie).
On déf pour $A^\bullet\in Ch(\Cat)$, $Z^n(A):=\ker(A^n\to A^{n+1})$
et $I^n(A)=\im(A^{n-1}\to A^n)$. Et on note $d^n=A^n\to A^{n+1}$.
Comme $d^nd^{n-1}=0$ on obtient $I^n\to Z^n$. Puis on déf
\[H^n(A^\bullet):=\coker(I^n(A)\to Z^n(A))\]
\section{Fonctorialité et additivité de $H^n(\_)$}
\subsection{Construction}
Ducoup faut commencer par définir $H^n(A^\bullet)$. C'est juste
qu'on a un carré
% https://q.uiver.app/#q=WzAsNixbMCwwLCJJXm4oQSkiXSxbMSwwLCJJXm4oQikiXSxbMSwxLCJaXm4oQikiXSxbMCwxLCJaXm4oQSkiXSxbMSwyLCJIXm4oQikiXSxbMCwyLCJIXm4oQSkiXSxbMCwxXSxbMywyXSxbMSwyXSxbMCwzXSxbMiw0XSxbMyw1XSxbNSw0LCIiLDEseyJzdHlsZSI6eyJib2R5Ijp7Im5hbWUiOiJkYXNoZWQifX19XV0=
\[\begin{tikzcd}
	{I^n(A)} & {I^n(B)} \\
	{Z^n(A)} & {Z^n(B)} \\
	{H^n(A)} & {H^n(B)}
	\arrow[from=1-1, to=1-2]
	\arrow[from=1-1, to=2-1]
	\arrow[from=1-2, to=2-2]
	\arrow[from=2-1, to=2-2]
	\arrow[from=2-1, to=3-1]
	\arrow[from=2-2, to=3-2]
	\arrow[dashed, from=3-1, to=3-2]
\end{tikzcd}\]
d'où les flèches de $\coker$ en bas. Et ce carré c'est comme
d'hab eux mêmes via
% https://q.uiver.app/#q=WzAsOCxbMCwyLCJBXm4iXSxbMCwxLCJBXntuLTF9Il0sWzEsMSwiQl57bi0xfSJdLFsxLDIsIkJebiJdLFswLDMsIlxcY29rZXIoZF57bi0xfSkiXSxbMSwzLCJcXGNva2VyKGRee24tMX0pIl0sWzAsMCwiXFxrZXIoZF57bi0xfSkiXSxbMSwwLCJcXGtlcihkXntuLTF9KSJdLFswLDNdLFsxLDBdLFsyLDNdLFsxLDJdLFswLDRdLFszLDVdLFs0LDUsIiIsMSx7InN0eWxlIjp7ImJvZHkiOnsibmFtZSI6ImRhc2hlZCJ9fX1dLFs2LDFdLFs2LDcsIiIsMSx7InN0eWxlIjp7ImJvZHkiOnsibmFtZSI6ImRhc2hlZCJ9fX1dLFs3LDJdXQ==
\[\begin{tikzcd}
	{\ker(d^{n-1})} & {\ker(d^{n-1})} \\
	{A^{n-1}} & {B^{n-1}} \\
	{A^n} & {B^n} \\
	{\coker(d^{n-1})} & {\coker(d^{n-1})}
	\arrow[dashed, from=1-1, to=1-2]
	\arrow[from=1-1, to=2-1]
	\arrow[from=1-2, to=2-2]
	\arrow[from=2-1, to=2-2]
	\arrow[from=2-1, to=3-1]
	\arrow[from=2-2, to=3-2]
	\arrow[from=3-1, to=3-2]
	\arrow[from=3-1, to=4-1]
	\arrow[from=3-2, to=4-2]
	\arrow[dashed, from=4-1, to=4-2]
\end{tikzcd}\]
pour les $Z^n$, et le $\coker$ on en a besoin pour
% https://q.uiver.app/#q=WzAsOCxbMCwxLCJJXm4oQSkiXSxbMSwxLCJJXm4oQikiXSxbMSwzLCJcXGNva2VyKEJee24tMX1cXHRvIEJee259KSJdLFswLDMsIlxcY29rZXIoQV57bi0xfVxcdG8gQV57bn0pIl0sWzAsMiwiQV5uIl0sWzEsMiwiQl5uIl0sWzAsMCwiQV57bi0xfSJdLFsxLDAsIkJee24tMX0iXSxbMCwxLCIiLDEseyJzdHlsZSI6eyJib2R5Ijp7Im5hbWUiOiJkYXNoZWQifX19XSxbMywyXSxbMCw0XSxbMSw1XSxbNSwyXSxbNCwzXSxbNCw1XSxbNiwwXSxbNywxXSxbNiw3XV0=
\[\begin{tikzcd}
	{A^{n-1}} & {B^{n-1}} \\
	{I^n(A)} & {I^n(B)} \\
	{A^n} & {B^n} \\
	{\coker(A^{n-1}\to A^{n})} & {\coker(B^{n-1}\to B^{n})}
	\arrow[from=1-1, to=1-2]
	\arrow[from=1-1, to=2-1]
	\arrow[from=1-2, to=2-2]
	\arrow[dashed, from=2-1, to=2-2]
	\arrow[from=2-1, to=3-1]
	\arrow[from=2-2, to=3-2]
	\arrow[from=3-1, to=3-2]
	\arrow[from=3-1, to=4-1]
	\arrow[from=3-2, to=4-2]
	\arrow[from=4-1, to=4-2]
\end{tikzcd}\]
\subsection{Fonctorialité}
La fonctorialité c'est juste que la construction est induite
par $A^\bullet\to B^\bullet\to C^\bullet$.
\subsection{Additivité}
On montre que $Z^n$ et $I^n$ sont additifs parce qu'y préservent
les sommes directes! (limites commutent avec les limites et 
inversement) Ça se fait bien à la main en plus. Maintenant on
peut montrer que
\[I^n(A)\oplus I^n(B)=I^n(A\oplus B)\to Z^n(A\oplus B)=Z^n(A)\oplus Z^n(B)\]
a pour conoyau $(I^n(A)\oplus I^n(B))/(Z^n(A)\oplus Z^n(B))$ et
on peut montrer la propriété universelle dessus facilement. En
particulier, $H^n(A\oplus B)\simeq H^n(A)\oplus H^n(B)$ est donnée
par $Z^n(A)\oplus Z^n(B)\to H^n(A)\oplus H^n(B)$.

\section{Suite exacte longue}
L'idée c'est d'appliquer le lemme du serpent sur
% https://q.uiver.app/#q=WzAsOCxbMSwwLCJBXntuKzF9L0lee24rMX0oQSkiXSxbMiwwLCJCXntuKzF9L0lee24rMX0oQikiXSxbMywwLCJDXntuKzF9L0lee24rMX0oQykiXSxbNCwwLCIwIl0sWzAsMSwiMCJdLFsxLDEsIlpee24rMX0oQSkiXSxbMiwxLCJaXntuKzF9KEIpIl0sWzMsMSwiWl57bisxfShDKSJdLFswLDFdLFsyLDNdLFs0LDVdLFs1LDZdLFs2LDddLFswLDVdLFsxLDZdLFsyLDddXQ==
\[\begin{tikzcd}
	& {A^{n+1}/I^{n+1}(A)} & {B^{n+1}/I^{n+1}(B)} & {C^{n+1}/I^{n+1}(C)} & 0 \\
	0 & {Z^{n+1}(A)} & {Z^{n+1}(B)} & {Z^{n+1}(C)}
	\arrow[from=1-2, to=1-3]
	\arrow[from=1-2, to=2-2]
	\arrow[from=1-3, to=2-3]
	\arrow[from=1-4, to=1-5]
	\arrow[from=1-4, to=2-4]
	\arrow[from=2-1, to=2-2]
	\arrow[from=2-2, to=2-3]
	\arrow[from=2-3, to=2-4]
\end{tikzcd}\]
et pour montrer que les lignes sont exactes on peut utiliser le
lemme du serpent sur la rangée du dessus. Les flèches verticales
c'est pas dur. Ensuite c'est juste que le noyau c'est un $H^n$ 
et le conoyau c'est un $H^{n+1}$ (ça parait clair).


\section{Homotopies}
Un morphisme $f^\bullet$ tel que on ait $h^n\colon A^n
\to B^{n-1}$ pour tout $n$ tel que
\[f^n=d^{n-1}h^n+h^{n+1}d^n\]
est dit contractile. En gros homotope à zéro. Et deux flèches
sont homotopes si $f^\bullet -g^\bullet $ est contractile.
La flèche induite sur les $H^n$ est nulle ! Parce que
\[d^{n-1}h^n(x)\in I^n.\]
et 
\[h^{n+1}d^n(x)\in h^{n+1}d^n((Z^{n})(A))=0.\]
Le deuxième truc $d^n(x)$ est déjà nul. En termes de catégories
abéliennes, c'est que on regarde
% https://q.uiver.app/#q=WzAsMyxbMCwwLCJIXm4oQSkiXSxbMSwwLCJIXm4oQilcXG9wbHVzIEhebihCKSkiXSxbMiwwLCJIXm4oQikiXSxbMCwxLCIoZF57bi0xfWhebixoXntuKzF9ZF5uKSJdLFsxLDIsIlxcZGVsdGFfQiJdXQ==
\[\begin{tikzcd}
	{H^n(A)} & {H^n(B)\oplus H^n(B))} & {H^n(B)}
	\arrow["{(d^{n-1}h^n,h^{n+1}d^n)}", from=1-1, to=1-2]
	\arrow["{\delta_B}", from=1-2, to=1-3]
\end{tikzcd}\]
d'où onpeut regarder termes à termes. Puis le premier termes
se factorise en 
% https://q.uiver.app/#q=WzAsNSxbMCwwLCJIXm4oQSkiXSxbMiwwLCJIXm4oQikiXSxbMCwxLCJaXm4oQSkiXSxbMSwxLCJJXm4oQikiXSxbMiwxLCJaXm4oQikiXSxbMCwxXSxbMiwzLCJkXntuLTF9aF5uIiwwLHsic3R5bGUiOnsiYm9keSI6eyJuYW1lIjoiZGFzaGVkIn19fV0sWzMsNF0sWzQsMV0sWzIsMF1d
\[\begin{tikzcd}
	{H^n(A)} && {H^n(B)} \\
	{Z^n(A)} & {I^n(B)} & {Z^n(B)}
	\arrow[from=1-1, to=1-3]
	\arrow[from=2-1, to=1-1]
	\arrow["{d^{n-1}h^n}", dashed, from=2-1, to=2-2]
	\arrow[from=2-2, to=2-3]
	\arrow[from=2-3, to=1-3]
\end{tikzcd}\]
et le deuxième en 
% https://q.uiver.app/#q=WzAsNSxbMCwwLCJIXm4oQSkiXSxbMiwwLCJIXm4oQikiXSxbMCwxLCJaXm4oQSkiXSxbMiwxLCJaXm4oQikiXSxbMSwxLCJBXntuKzF9Il0sWzAsMV0sWzMsMV0sWzIsMF0sWzQsMywiaF57bisxfSJdLFsyLDQsImRebiJdLFsyLDMsIk9fe1pebihBKSxaXm4oQil9IiwyLHsiY3VydmUiOjN9XV0=
\[\begin{tikzcd}
	{H^n(A)} && {H^n(B)} \\
	{Z^n(A)} & {A^{n+1}} & {Z^n(B)}
	\arrow[from=1-1, to=1-3]
	\arrow[from=2-1, to=1-1]
	\arrow["{d^n}", from=2-1, to=2-2]
	\arrow["{O_{Z^n(A),Z^n(B)}}"', curve={height=18pt}, from=2-1, to=2-3]
	\arrow["{h^{n+1}}", from=2-2, to=2-3]
	\arrow[from=2-3, to=1-3]
\end{tikzcd}\]
d'où le résultat.

\section{Résolutions}
\subsection{Assez d'injectifs et résolutions}
Avoir assez d'injectif ça fait que on peut construire 
$0\to A\to I^0$ puis $0\to I^0/A\ I^1$ d'où 
$\ker(A\to I^0/A\to I^1)=A$. Et on continue par récurrence. 

\subsection{$H^n(A)$ ne dépend pas de la résolution.}
En gros le plan c'est de 
\begin{enumerate}
  \item $f\colon A\to B$, on obtient une extension $f^\bullet\colon
    I^\bullet\to J^\bullet$.
\end{enumerate}
Pour la construire on utilise la même idée que pour construire
des résolutions injectives via les conoyaux. 
\begin{enumerate}
  \item Elle est unique à homotopie près car $O$ s'étend en
    un contractile. Ça c'est un poil compliqué, mais intéressant
    y'a un décalage dans les flèches de la résolution.
  \item En particulier étant donné deux résolutions $I^\bullet$
    et $J^\bullet$ et deux extensions $f^\bullet,g^\bullet\colon 
    I^\bullet\to J^\bullet$ comme $F$ est exact il est additif
    et donc préserve les homotopies! Pareil $H^n$ est 
    additif d'où $H^n\circ F$ est additif et le résultat.
\end{enumerate}
Enfin, on étend l'identité $id_A$ de deux manières
\[f\colon I^\bullet \to J^\bullet\]
et
\[g\colon J^\bullet \to I^\bullet\]
en particulier $(g\circ f)^\bullet$ est homotope à $id_{I^\bullet}$
et $(f\circ g)^\bullet$ est homotope à $id_{J^\bullet}$. D'où 
l'isomorphisme canonique!


\section{Foncteurs dérivés}
Là on définit $R^nF(A):=H^n(F(I^\bullet))$ pour n'importe quelle
résolution de $A$. C'est un foncteur additif parce que $H^n$ l'est.
$R^nF(I)=0$ pour $n>0$ via la résolution triviale. On a une
suite exacte longue qui est fonctorielle.

\begin{rem}
  Apparamment dans la preuve pour la suite exacte longue, y faut
  prendre une résolution somme directe sur $B$. AH, c'est parce
  que si on a juste une suite exacte courte c'est pas clair
  que $B$ est un produit d'où $0\to F(I^\bullet)\to F(J^\bullet)\to
  F(K^\bullet)\to 0$ est exacte!!
\end{rem}

\chapter{Cohomologie des faisceaux}
\section{$Sh(X)$ est abélienne.}
Faut montrer que $Sh(X)$ est abélienne. On peut montrer que
$PSh(X)$ l'est facilement. Puis si $i$ est le foncteur d'oubli
on a l'adjonction
\[\Hom_{Sh(X)}((\F)^\sharp,\G)\simeq \Hom_{PSh(X)}(\F,i(\G))\]
pas la propriété universelle. Maintenant étant donné un 
diagramme $(F_i)_i$ dans $Sh(X)$ on le pousse dans $PSh(X)$ et
là 
\[0\to i(\F_i)(U\cup V)\to i(\F_i(U))\oplus i(\F_i(V))\to i(\F_i(U\cap V))\]
passe à la limite ! Vu que les limites commutent avec les
limites. En particulier la limite est un faisceau. Et on récupère
tout par faisceautisation en fait. 

\section{Résolution de Godemont 1}
L'idée c'est que les faisceaux flasques sont acycliques. Pour deux
raisons, si
\[0\to \F'\to\F\to \F''\to 0\]
est exacte alors, si $\F'$ est flasque, $\Gamma_U$ est exact à 
droite sur cette s.e.s. On en déduit que si $\F'$ et $\F$ sont
flasques, alors $\F''$ aussi en utilisant la surjection
\[\F(U)(\to \F(V)\to \F''(V)\to 0\]
et la surjection $\F(U)\to \F''(U)\to 0$. Maintenant ça permet
de dire que si on pose
\[0\to \F\to C^0(\F)\to (Z^1(\F)\to C^0(Z^1(\F))=:C^1(\F))\]
etc.. $Z^i(\F)=C^{i-1}(\F)/Z^{i-1}(\F)$ et $C^i(\F):=C^0(Z^i(\F))$.
Alors si $\F$ est flasque $Z^n(\F)$ aussi par récurrence puis 
$C^n(\F)$. Ensuite la résolution est $\Gamma_U$ acyclique ! Puis
elle calcule la bonne cohomologie.

\chapter{Hypercohomologie}
Donc là ça se corse y'a plusieurs défs.
\section{Résolutions de complexes bornés par le bas}
Étant donné $f\colon A^\bullet\to B^\bullet $ et deux résolutions
$A^\bullet\to I^\bullet$, $B^\bullet\to J^\bullet$ on peut
étendre $f$ en $I^\bullet \to J^\bullet$ de manière unique
à homotopie près. L'idée c'est de regarder $f$ sur les $H^0$ puis
de l'étendre comme d'habitude. La commutativité des petits
carrés est claire et celle des grands est simplement parce
que celle sur les $H^0(I^\bullet)\to H^0(J^\bullet)$ est celle 
induite par $f$. En particulier, on a 
\[[A^\bullet, J^\bullet]\simeq [I^\bullet, J^\bullet]\]
on a aussi
\[[A^\bullet, I^\bullet]\simeq [B^\bullet, I^\bullet]\]
pour tout complexe d'injectif.

\section{Catégories dérivée }
On remarque que les flèches contractiles forment un sous-groupe!
En particulier on peut regarder 
\[K^+(\Cat):= (Ch^+(\Cat),\textrm{flèches modulo homotopies}) \]
et 
\[D^+(\Cat):= (Ch^+(\mathbb I),\textrm{flèches modulo homotopies})\]
où le deuxième c'est des complexes bornés par le bas d'injectifs.
\section{Et adjonctions dans le cas des faisceaux}
\subsection{Cadre}
On définit $Sh(X)\to D^+(X)$ par $\F$ associe une résolution
injective $I^\bullet$, c'est pleinement fidèle.

Maintenant étant donné $f\colon X\to Y$ et
$f_*\colon Sh(X)\rightleftarrows Sh(Y)\colon f^*$, on construit
\[Lf^*\colon D^+(Y)\to D^+(X)\colon Rf_*\]
la deuxième est facile vu que $f_*$ préserve les injectifs. 
L'autre faut juste avoir $\rho\colon K^+(X)\to D^+(X)$ ou on
associe une résolution injective. 
\begin{rem}
  C'est bien déf à unique isomorphisme près! Suffit d'étendre 
  l'identité des deux côtés.
\end{rem}
En plus $R(g\circ f)_*=Rg_*Rf_*$. Pareil pour $L$ par adjonction.


\section{Adjonctions encore?}
Étant donné $f_X\colon X\to \{*\}$ on a une adjonction
\[(f_X)_*\colon Sh(X)\to Sh(\{*\})\colon f_X^*\]
où dans ce cas particulier $(f_X)_*=\Gamma_X$ et $f_X^*=(A\mapsto
(A_X))$ le faisceau constant!

\section{Hypercohomologie et complexe total.}
Étant donné un complexe de faisceaux $\F^\bullet\in Ch^+(Sh(X))$,
on peut avoir une double résolution
% https://q.uiver.app/#q=WzAsMTUsWzAsMiwiMCJdLFsxLDIsIkNeMChcXEZeMCkiXSxbMiwyLCJDXjAoXFxGXjEpIl0sWzEsMSwiQ14xKFxcRl4wKSJdLFsxLDAsIkNeMihcXEZeMCkiXSxbMiwxLCJDXjEoXFxGXjEpIl0sWzMsMiwiQ14wKFxcRl4yKSJdLFsxLDMsIjAiXSxbMiwzLCIwIl0sWzMsMywiMCJdLFszLDEsIkNeMShcXEZeMikiXSxbMiwwLCJDXjIoXFxGXjEpIl0sWzMsMCwiQ14yKFxcRl4yKSJdLFswLDEsIjAiXSxbMCwwLCIwIl0sWzcsMV0sWzEsMywiZF97MCwwfSIsMl0sWzMsNCwiZF97MCwxfSIsMl0sWzAsMV0sWzEsMl0sWzIsNl0sWzMsNV0sWzIsNSwiZF97MSwwfSIsMl0sWzgsMl0sWzksNl0sWzUsMTEsImRfezEsMX0iLDJdLFs0LDExXSxbMTEsMTJdLFs1LDEwXSxbMTAsMTIsImRfezIsMX0iLDFdLFs2LDEwLCJkX3syLDB9IiwxXSxbMTQsNF0sWzEzLDNdXQ==
\[\begin{tikzcd}
	0 & {C^2(\F^0)} & {C^2(\F^1)} & {C^2(\F^2)} \\
	0 & {C^1(\F^0)} & {C^1(\F^1)} & {C^1(\F^2)} \\
	0 & {C^0(\F^0)} & {C^0(\F^1)} & {C^0(\F^2)} \\
	& 0 & 0 & 0
	\arrow[from=1-1, to=1-2]
	\arrow[from=1-2, to=1-3]
	\arrow[from=1-3, to=1-4]
	\arrow[from=2-1, to=2-2]
	\arrow["{d_{0,1}}"', from=2-2, to=1-2]
	\arrow[from=2-2, to=2-3]
	\arrow["{d_{1,1}}"', from=2-3, to=1-3]
	\arrow[from=2-3, to=2-4]
	\arrow["{d_{2,1}}"{description}, from=2-4, to=1-4]
	\arrow[from=3-1, to=3-2]
	\arrow["{d_{0,0}}"', from=3-2, to=2-2]
	\arrow[from=3-2, to=3-3]
	\arrow["{d_{1,0}}"', from=3-3, to=2-3]
	\arrow[from=3-3, to=3-4]
	\arrow["{d_{2,0}}"{description}, from=3-4, to=2-4]
	\arrow[from=4-2, to=3-2]
	\arrow[from=4-3, to=3-3]
	\arrow[from=4-4, to=3-4]
\end{tikzcd}\]
où les résolutions verticales c'est les résolutions de Godement
en dimension $1$. Et les flèches horizontales c'est celles
\textbf{presque} induites par la fonctorialité de la résolution de
Godement. On les notes 
\[\delta^{i,j}:=(-1)^jC^i(\delta^j)\]
où $\delta^j\colon \F^j\to \F^{j+1}$. Et on considère
le complexe
\[Tot^n(C^\bullet(\F^\bullet)):=\bigoplus_{i+j=n}C^i(\F^j)\]
muni de la différentielle 
$d_n:=\oplus_{i+j=n}\delta^{i,j}+d^{i,j}$. C'est une différentielle
grâce à l'anticommutativité de $\delta^{i,j}$. Enfin, on a
\[\F^j\to Tot^j(C^\bullet(\F^\bullet))\]
en envoyant dans le premier terme! 
\begin{thm}
\[\F^\bullet\to Tot^\bullet(C^\bullet(\F^\bullet))\]
est une résolution flasque quasi-isomorphe!
\end{thm}
\subsection{Définition}
Maintenant étant donné $\F^\bullet\to I^\bullet$ une résolution
injective, ou flasque comme avant on déf 
\[H^n(X,\F^\bullet):=H^n(I^\bullet(X))\]

\begin{rem}
  C'est bien une cohomologie sur un complexe simple.
\end{rem}

\section{Application et récap}
On a
\[Rf_*\colon D^+(X)\to D^+(Y)\]
On remplace $\Gamma_X$ par $(f_X)_*$ avec $f_X\colon X\to \{*\}$.
D'où 
$R(f_X)_*\F=(f_X)_*I^\bullet=I^\bullet(X)=\Gamma(X,I^\bullet)$.
\begin{rem}
  La cohomologie de $I^\bullet(X)$ est pas triviale hein.
\end{rem}
Maintenant $R^if_*\F=H^i(Rf_*\F)=H^i(f_*I^\bullet)$.


\section{Catégorie dérivées et adjonctions}
\section{Mayer-Vietoris}



%\printbibliography
\end{document}

