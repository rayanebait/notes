\documentclass[a4paper,12pt]{book}
\usepackage{amsmath,  amsthm,enumerate}
\usepackage{csquotes}
\usepackage[provide=*,french]{babel}
\usepackage[dvipsnames]{xcolor}
\usepackage{quiver, tikz}

%symbole caligraphique
\usepackage{mathrsfs}

%hyperliens
\usepackage{hyperref}

%pseudo-code
\usepackage{algorithm}
\usepackage{algpseudocode}

\usepackage{fancyhdr}

\pagestyle{fancy}
\addtolength{\headwidth}{\marginparsep}
\addtolength{\headwidth}{\marginparwidth}
\renewcommand{\chaptermark}[1]{\markboth{#1}{}}
\renewcommand{\sectionmark}[1]{\markright{\thesection\ #1}}
\fancyhf{}
\fancyfoot[C]{\thepage}
\fancyhead[LO]{\textit \leftmark}
\fancyhead[RE]{\textit \rightmark}
\renewcommand{\headrulewidth}{0pt} % and the line
\fancypagestyle{plain}{%
    \fancyhead{} % get rid of headers
}

%bibliographie
\usepackage[
backend=biber,
style=alphabetic,
sorting=ynt
]{biblatex}

\addbibresource{bib.bib}

\usepackage{appendix}
\renewcommand{\appendixpagename}{Annexe}

\definecolor{wgrey}{RGB}{148, 38, 55}

\setlength\parindent{24pt}

\newcommand{\Z}{\mathbb{Z}}
\newcommand{\R}{\mathbb{R}}
\newcommand{\rel}{\omathcal{R}}
\newcommand{\Q}{\mathbb{Q}}
\newcommand{\C}{\mathbb{C}}
\newcommand{\Cat}{\mathcal{C}}
\newcommand{\Dat}{\mathcal{D}}
\newcommand{\Aat}{\mathcal{A}}
\newcommand{\N}{\mathbb{N}}
\newcommand{\K}{\mathbb{K}}
\newcommand{\A}{\mathbb{A}}
\newcommand{\B}{\mathcal{B}}
\newcommand{\Or}{\mathcal{O}}
\newcommand{\F}{\mathscr F}
\newcommand{\Hom}{\textrm{Hom}}
\newcommand{\disc}{\textrm{disc}}
\newcommand{\Pic}{\textrm{Pic}}
\newcommand{\End}{\textrm{End}}
\newcommand{\Spec}{\textrm{Spec}}
\newcommand{\Supp}{\textrm{Supp}}
\newcommand{\Ouv}{\textrm{Ouv}}
\newcommand{\im}{\textrm{im}}
\newcommand{\coker}{\textrm{coker}}
\newcommand{\coim}{\textrm{coim}}

\newcommand{\m}{\mathfrak{m}}
\renewcommand{\a}{\mathfrak{a}}
\newcommand{\p}{\mathfrak{p}}

\newcommand{\cL}{\mathscr{L}}
\newcommand{\G}{\mathscr{G}}
\newcommand{\D}{\mathscr{D}}
\newcommand{\E}{\mathscr{E}}
\renewcommand{\P}{\mathscr{P}}
\renewcommand{\H}{\mathscr{H}}

\makeatletter
\newcommand{\colim@}[2]{%
  \vtop{\m@th\ialign{##\cr
    \hfil$#1\operator@font colim$\hfil\cr
    \noalign{\nointerlineskip\kern1.5\ex@}#2\cr
    \noalign{\nointerlineskip\kern-\ex@}\cr}}%
}
\newcommand{\colim}{%
  \mathop{\mathpalette\colim@{\rightarrowfill@\scriptscriptstyle}}\nmlimits@
}
\renewcommand{\varprojlim}{%
  \mathop{\mathpalette\varlim@{\leftarrowfill@\scriptscriptstyle}}\nmlimits@
}
\renewcommand{\varinjlim}{%
  \mathop{\mathpalette\varlim@{\rightarrowfill@\scriptscriptstyle}}\nmlimits@
}
\makeatother

\theoremstyle{plain}
\newtheorem{thm}{Théoreme}
\newtheorem{lem}{Lemme}
\newtheorem{prop}{Proposition}
\newtheorem{cor}{Corollaire}
\newtheorem{heur}{Heuristique}
\newtheorem{rem}{Remarque}
\newtheorem{note}{Note}

\theoremstyle{definition}
\newtheorem{conj}{Conjecture}
\newtheorem{prob}{Problème}
\newtheorem{quest}{Question}
\newtheorem{prot}{Protocole}
\newtheorem{algo}{Algorithme}
\newtheorem{defn}{Définition}
\newtheorem{exmp}{Exemples}
\newtheorem{exo}{Exercices}
\newtheorem{ex}{Exemple}
\newtheorem{exs}{Exemples}

\theoremstyle{remark}

\definecolor{wgrey}{RGB}{148, 38, 55}
\definecolor{wgreen}{RGB}{100, 200,0} 
\hypersetup{
    colorlinks=true,
    linkcolor=wgreen,
    urlcolor=wgrey,
    filecolor=wgrey
}

\title{Cohomologie des groupes}
\date{2024-2025}

\begin{document}
\maketitle
\tableofcontents

\chapter{Catégories et catégories abéliennes.}
J'veux un petit inventaire de choses cool.
\section{Yoneda, théorème d'isomorphisme et exactitude}
On applique $h\colon \Aat\to Ab$.

De $0\to A\to B\to C\to 0$ exacte on tire
\[0\to h_A\to h_B\to h_C\]
pour l'exactitude à gauche et
\[0\to h^C\to h^B\to h^A\]
pour l'exactitude à droite. On voit ça comme ça :
% https://q.uiver.app/#q=WzAsNSxbMCwwLCIwIl0sWzEsMCwiQSJdLFsyLDAsIkIiXSxbMywwLCJDIl0sWzIsMSwiTSJdLFswLDFdLFsxLDJdLFsyLDNdLFs0LDEsIiIsMCx7InN0eWxlIjp7ImJvZHkiOnsibmFtZSI6ImRhc2hlZCJ9fX1dLFs0LDJdXQ==
\[\begin{tikzcd}
	0 & A & B & C \\
	&& M
	\arrow[from=1-1, to=1-2]
	\arrow[from=1-2, to=1-3]
	\arrow[from=1-3, to=1-4]
	\arrow[dashed, from=2-3, to=1-2]
	\arrow[from=2-3, to=1-3]
\end{tikzcd}\]
pour le premier, l'exactitude de $0\to A\to B\to C$ montre que
$M$ va dans le noyau si $(M\to B\to C )= 0$. Et on utilise
$A\simeq \im(A\to B)=\ker(B\to C)$ et la PU du
noyau pour montrer qu'on a l'exactitude sur les hom. 
Pour le deuxième, on utilise :
% https://q.uiver.app/#q=WzAsNSxbMCwwLCJBIl0sWzEsMCwiQiJdLFsyLDAsIkMiXSxbMSwxLCJNIl0sWzMsMCwiMCJdLFswLDFdLFsxLDJdLFsyLDRdLFsxLDNdLFsyLDMsIiIsMCx7InN0eWxlIjp7ImJvZHkiOnsibmFtZSI6ImRhc2hlZCJ9fX1dXQ==
\[\begin{tikzcd}
	A & B & C & 0 \\
	& M
	\arrow[from=1-1, to=1-2]
	\arrow[from=1-2, to=1-3]
	\arrow[from=1-2, to=2-2]
	\arrow[from=1-3, to=1-4]
	\arrow[dashed, from=1-3, to=2-2]
\end{tikzcd}\]
et là de l'exactitude de $A\to B\to C\to 0$ et du théorème 
d'isomorphisme on a $coker(A\to B)\simeq coim(B\to C)\simeq C$
d'où $(A\to B\to M)=0$ implique l'existence de la flèche vers
$\coker(A\to B)$ est la factorisation par $C$.

Montrer l'inverse consiste juste à spécialiser et utiliser les PU.


\section{Adjonctions}
Étant donné $F\colon \Cat\to\Dat\colon G$ si
\[\Hom_\Dat(F(\_),\_)\simeq \Hom_\Cat(\_,G(\_))\]
naturellement alors $F$ est adjoint à gauche de $G$
qui lui est adjoint à droite de $F$. Via Yoneda ça dit que 
$h^{F(\_)}(\_)\simeq h^{\_}(G(\_))$
dans \[Nat(\widehat{\Cat\times \Dat}, \widehat{\Dat\times\Cat})\]
y'a aussi la formulation $h^{FG(\_)}(\_)\simeq h^{G(\_)}(G(\_))$.


\subsection{Propriétés d'exactitude}
On déduit tout du premier isomorphisme.
Un adjoint à gauche est exact à droite et inversement. Plus
généralement un adjoint à gauche préserve les colimites et les 
limites finies (La preuve est non triviale).

Si $F$ est adjoint à gauche de $G$ on a
\[0\to h^{F(C)}\to h^{F(B)}\to h^{F(A)}\]
qui est iso à 
\[0\to h^{C}\circ G\to h^{B}\circ G\to h^{A}\circ G\]
en particulier si $A\to B\to C\to 0$ est exacte à droite alors
$F(A\to B\to C\to 0)$ est exacte à droite. 

Maintenant si $G$ est exact, alors $h^{F(P)}\simeq h^{P}\circ G$
d'où $h^{F(P)}$ est exact si $P$ est projectif.

À l'inverse si $F$ est exact, $h_{G(I)}\simeq h_I(F(\_))$ est
exact dès que $I$ est injectif.

\subsection{Limites, adjoints et l'axiome du choix.}
Quand j'ai une catégorie $I$-cocomplète pour $I$ filtrante,
genre $Mod_R$, bah y'a une adjonction cool pour les colimites 
\[\varinjlim\colon\Aat^I\leftrightarrow\Aat\colon \Delta\]
où $\Delta$ c'est le foncteur constant. MAIS, bah pour définir
le foncteur $\varinjlim$ faut faire un choix de colimite pour
chaque foncteur! Penser le point dans Set.

\begin{rem}
  Pour $I$ filtrante, $\Aat^I$ est abélienne si $\Aat$ l'est.
  Pour prouver que y'a
  les noyaux, via $C,D\colon I\to \Aat$ et $C\to D$, on prend
  $K^0\colon I^0\to \Aat$ la donnée des noyaux termes
  à termes et on montre que ça s'étend naturellement en 
  $K\colon I\to \Aat$ (on trouve juste un foncteur)
  tel que les flèches de noyaux termes à termes ça fait une
  transformation naturelle $K\to C$. C'est bien le noyau parce
  que ça l'est terme à terme. 
\end{rem}

\subsection{Fidélité d'un foncteur ayant un adjoint}
On déduit tout du deuxième isomorphisme, i.e. la counité. Et c'est
clair que $G$ est pleinement fidèle ssi la counité est un 
isomorphisme.

\section{Exemples d'adjonctions}
Les foncteurs $Free$ qu'ont peut définir comme les adjoints
à gauche de foncteurs d'oublis $C\to Set$. Et les foncteurs
d'oublis plus généraux $(\_)^\times\colon Ring\to Grp$ par exemple.


\chapter{(co)homologie}
\section{(co)Homologie sur les résolutions acycliques}
Le Weibel dit qu'en tronquant une résolution projective et
de proche en proche on montre qu'une résolution acyclique calcule
la cohomologie. I.e. via
\[0\to \ker(d_{-1})\to P\to A\to 0\]
puis
\[0\to \ker(d_{m-1})\to \to P_m\to \ldots\to P_0\to A\to 0\]
avec les $P_i$ acycliques et la suite exacte courte de proche
en proche que 
$L_{m+1}F(A)\simeq \ker(F(\ker(d_{m-1}))\to F(P_m))$. Pour faire
de proche en proche faut utiliser que 
$L_iF(A)\simeq L_{i-(m-1)}F(\ker(d_{m-1}))$ et ça c'est la
suite exacte longue et l'acyclicité.

\section{Foncteurs dérivés et limites}
Sur une catégorie AB5 ($\varinjlim\colon \Aat^I\to \Aat$ est
exact et $\Aat$ est cocomplète.) on a 
\[\varinjlim_j L_iF(C(j))\simeq L_iF(\varinjlim_j C(j))\]

\chapter{Cohomologies}
J'prends que des bi-modules pour les preuves.

\section{Tenseur et Hom}
Y'a une adjonction :
\[h^{\_\otimes_R A}(\_)\simeq h^A(h^{\_}(\_))\]
c'est juste qu'à une application $R$-linéaire $B\otimes_RA\to C$ à
gauche on associe l'application vers le dual $A\to (\Hom(B, C))$
via la propriété universelle des produits tensoriels.

\subsection{Exactitudes}
On a $\otimes_R A$ qui est adjoint à gauche d'où est exact à 
droite. 

\section{Ext}
On déf $Ext^n(A,\_)=R^nh^A(\_)$ et $Ext^n(\_,B)=R^nh_B(\_)$.

\subsection{Localisation}

\section{Tor}
On déf $Tor_n^R(A,B)=L_nA\otimes_R(B)$. La notation est symétrique!
\subsection{Groupes abéliens, $Mod_\Z$}
On sé réduit à $A=\Z/n\Z$ où on a la résolution
\[0\to\Z\to\Z\to\Z/n\Z\to 0\]
alors en appliquant $\otimes_R B$, $Tor_*^\Z(\Z/n\Z,B)$ est 
l'homologie
\[0\to B\to B\to 0\]
où on multiplie par $n$, d'où $Tor_0^\Z(\Z/n\Z, B)=B/nB$, 
$Tor_1^\Z(\Z/n\Z,B)=_nB$ et le reste est nul.

Ensuite page 66 du weibel mais ca consiste tjr à écrire 
$A=\varinjlim A_i$ pour les sous groupes de types finis puis
dans ce cas $A_i\simeq \Z^m\oplus_i\Z/n_i\Z$ et de tjr faire
tout commuter avec Tor.
\subsection{Modules plats}
C'est les modules $\otimes_R B$ acycliques. Une liste :
\begin{itemize}
  \item Les modules libres. Parce qu'alors l'injectivité se
    vérifie terme à terme. (C'est direct que $R^{(I)}\otimes N\simeq\oplus_{I}N$).
  \item Si $R$ est un pid, les modules sans torsion. 
  \item $S^{-1}R$. En particulier $Frac(R)$.
\end{itemize}

\subsection{Localisation}
La localisation marche. Si $R$ est commutatif et $A,B$ des
$R$-modules quelconques, $Tor_n^R(A,B)=0$ ssi pour tout
idéal maximal $\m$ on a
\[Tor_n^{R_\m}(A_\m,B_\m)\]


%\printbibliography
\end{document}

