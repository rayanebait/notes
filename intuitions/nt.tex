\documentclass[a4paper,12pt]{book}
\usepackage{amsmath,  amsthm,enumerate}
\usepackage{csquotes}
\usepackage[provide=*,french]{babel}
\usepackage[dvipsnames]{xcolor}
\usepackage{quiver, tikz}

%symbole caligraphique
\usepackage{mathrsfs}

%hyperliens
\usepackage{hyperref}

%pseudo-code
\usepackage{algpseudocode}
\usepackage{algorithm}
\makeatletter
  \renewcommand{\ALG@name}{Algorithme}
  \makeatother
\usepackage{fancyhdr}

\pagestyle{fancy}
\addtolength{\headwidth}{\marginparsep}
\addtolength{\headwidth}{\marginparwidth}
\renewcommand{\chaptermark}[1]{\markboth{#1}{}}
\renewcommand{\sectionmark}[1]{\markright{\thesection\ #1}}
\fancyhf{}
\fancyfoot[C]{\thepage}
\fancyhead[LO]{\textit \leftmark}
\fancyhead[RE]{\textit \rightmark}
\renewcommand{\headrulewidth}{0pt} % and the line
\fancypagestyle{plain}{%
    \fancyhead{} % get rid of headers
}

%bibliographie
\usepackage[
backend=biber,
style=alphabetic,
sorting=ynt
]{biblatex}

\addbibresource{bib.bib}

\usepackage{appendix}
\renewcommand{\appendixpagename}{Annexe}

\definecolor{wgrey}{RGB}{148, 38, 55}

\setlength\parindent{24pt}

\newcommand{\Z}{\mathbb{Z}}
\newcommand{\R}{\mathbb{R}}
\newcommand{\rel}{\omathcal{R}}
\newcommand{\Q}{\mathbb{Q}}
\newcommand{\C}{\mathbb{C}}
\newcommand{\N}{\mathbb{N}}
\newcommand{\K}{\mathbb{K}}
\newcommand{\A}{\mathbb{A}}
\newcommand{\B}{\mathcal{B}}
\newcommand{\Or}{\mathcal{O}}
\newcommand{\F}{\mathbb F}
\newcommand{\m}{\mathfrak m}
\renewcommand{\b}{\mathfrak b}
\renewcommand{\a}{\mathfrak a}
\newcommand{\p}{\mathfrak p}
\newcommand{\I}{\mathfrak I}
\newcommand{\Hom}{\textrm{Hom}}
\newcommand{\disc}{\textrm{disc}}
\newcommand{\Pic}{\textrm{Pic}}
\newcommand{\End}{\textrm{End}}
\newcommand{\Spec}{\textrm{Spec}}

\newcommand{\cL}{\mathscr{L}}
\newcommand{\G}{\mathscr{G}}
\newcommand{\D}{\mathscr{D}}
\newcommand{\E}{\mathscr{E}}

\theoremstyle{plain}
\newtheorem{thm}{Théoreme}
\newtheorem{lem}{Lemme}
\newtheorem{prop}{Proposition}
\newtheorem{cor}{Corollaire}
\newtheorem{heur}{Heuristique}
\newtheorem{rem}{Remarque}
\newtheorem{rembis}{Remarque}
\newtheorem{note}{Note}

\theoremstyle{definition}
\newtheorem{conj}{Conjecture}
\newtheorem*{eq}{Équivalences}
\newtheorem{prob}{Problème}
\newtheorem{quest}{Question}
\newtheorem{prot}{Protocole}
\newtheorem{algo}{Algorithme}
\newtheorem{defn}{Définition}
\newtheorem{defnbis}{Définition}
\newtheorem{ex}{Exemple}
\newtheorem{exo}{Exercices}

\theoremstyle{remark}

\definecolor{wgrey}{RGB}{148, 38, 55}
\definecolor{wgreen}{RGB}{100, 200,0} 
\hypersetup{
    colorlinks=true,
    linkcolor=wgreen,
    urlcolor=wgrey,
    filecolor=wgrey
}

\title{Intuitions}
\date{}

\begin{document}
\maketitle

\section{Sections, Yoneda, faisceaux et topos}
Dans le Serre pour calculer $*_AG_i$ il utilise localement 
\[G_i/A=\langle s_{ij}|j\rangle\]
et $S_i\times A\to G_i$. Ensuite il note $X$ les mots réduits (découpés) et montre 
qu'on peut agir avec G sur les mots réduits ET LÀ, il utilise que $G$ se découpe en des
$G_i\times Y_i$ avec lesquels on sait agir.  Puis que l'action se recolle car on
a une section $G\to X$ !! En gros 1), découpage de l'action $G$ sur $X$, puis
section $G\to X$ via yoneda ($g$ associe $g(1;)$)!
Ça suggérait $X$ comme la descente et les $G_i\times Y_i$ comme le découpage de $G$.
La section montre que le découpage se recolle. Et le crux est qu'on agit via Yoneda.


\section{Exemples d'analogies très (trop) concrète}
\subsection{Suites convergentes et limites projectives}
Une suite $\overline \N\to \R^2$ qui est cauchy et une limite projective $(A_i)$ dans
une catégorie. Le lien ? J'aimerai supposer que si $|y_i-y_{i+1}|>|y_{i+1}-y_{i+2}|$
et si $|y_i-y_{i+1}|<\epsilon$ et $|y_{i+1}-y_{i+2}|<\epsilon$ alors
$|y_i-y_{i+2}|<\epsilon$ implique cauchy (c'est clair). Maintenant d'une
suite de Cauchy je forme la catégorie sur $\overline\N$ formée de flèches entre
$i$ et $j$ si $|y_i-y_j|<\epsilon$, on peut tronquer et là on obtient juste
l'ordre totale sur $\overline \N$. À l'inverse j'aimerai dire que si
je peux construire $\N$ et une colimite via les cônes alors en fait 
j'obtiens une suite de Cauchy.

Peut-être que ce que je peux dire c'est que pour une suite dans $\R^2$
je met une flèche entre $y_i$ et $y_j$ ssi $|y_i-y_j|<\epsilon$. Alors
ma suite est Cauchy ssi éventuellement j'ai un "graphe orienté complet"
qui correspond à $\overline N$. Ou plutôt, si je peux vérifier que
tout les triangles commutent (la condition du haut), alors c'est
Cauchy et donc converge. OUI voilà.

\subsection{Traduire "la suite converge"}
Dans le cas des fonctions, ca veut dire que la définition locale implique la
définition globale.

\subsection{Limites projectives utilisées}
limites de fonctions trivialement.

\subsection{Limites projectives utiles}
Une fonction du disque $=$ limite de fonction des disques fermés et autre exemples$=$
système cohérent de fonction sur n'importe quelle chaine qui remplie le disque.

Cribles ? faisceaux ?


\subsection{Ca suggère les faisceaux!}




\end{document}

