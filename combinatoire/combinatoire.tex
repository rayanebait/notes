\documentclass[a4paper,12pt]{book}
\usepackage{amsmath,  amsthm,enumerate}
\usepackage{csquotes}
\usepackage[provide=*,french]{babel}
\usepackage[dvipsnames]{xcolor}
\usepackage{quiver, tikz}

%symbole caligraphique
\usepackage{mathrsfs}

%hyperliens
\usepackage{hyperref}

%pseudo-code
\usepackage{algorithm}
\usepackage{algpseudocode}

\usepackage{fancyhdr}

\pagestyle{fancy}
\addtolength{\headwidth}{\marginparsep}
\addtolength{\headwidth}{\marginparwidth}
\renewcommand{\chaptermark}[1]{\markboth{#1}{}}
\renewcommand{\sectionmark}[1]{\markright{\thesection\ #1}}
\fancyhf{}
\fancyfoot[C]{\thepage}
\fancyhead[LO]{\textit \leftmark}
\fancyhead[RE]{\textit \rightmark}
\renewcommand{\headrulewidth}{0pt} % and the line
\fancypagestyle{plain}{%
    \fancyhead{} % get rid of headers
}

%bibliographie
\usepackage[
backend=biber,
style=alphabetic,
sorting=ynt
]{biblatex}

\addbibresource{bib.bib}

\usepackage{appendix}
\renewcommand{\appendixpagename}{Annexe}

\definecolor{wgrey}{RGB}{148, 38, 55}

\setlength\parindent{24pt}

\newcommand{\Z}{\mathbb{Z}}
\newcommand{\R}{\mathbb{R}}
\newcommand{\rel}{\omathcal{R}}
\newcommand{\Q}{\mathbb{Q}}
\newcommand{\C}{\mathbb{C}}
\newcommand{\N}{\mathbb{N}}
\newcommand{\K}{\mathbb{K}}
\newcommand{\A}{\mathbb{A}}
\newcommand{\B}{\mathcal{B}}
\newcommand{\Or}{\mathcal{O}}
\newcommand{\F}{\mathscr F}
\newcommand{\Hom}{\textrm{Hom}}
\newcommand{\disc}{\textrm{disc}}
\newcommand{\Pic}{\textrm{Pic}}
\newcommand{\End}{\textrm{End}}
\newcommand{\Spec}{\textrm{Spec}}
\newcommand{\Supp}{\textrm{Supp}}
\renewcommand{\Im}{\textrm{Im}}
\newcommand{\m}{\mathfrak{m}}
\renewcommand{\P}{\mathbb{P}}
\newcommand{\p}{\mathfrak{p}}


\newcommand{\cL}{\mathscr{L}}
\newcommand{\G}{\mathscr{G}}
\newcommand{\D}{\mathscr{D}}
\newcommand{\E}{\mathscr{E}}
\newcommand{\Po}{\mathscr{P}}
\renewcommand{\H}{\mathscr{H}}

\makeatletter
\newcommand{\colim@}[2]{%
  \vtop{\m@th\ialign{##\cr
    \hfil$#1\operator@font colim$\hfil\cr
    \noalign{\nointerlineskip\kern1.5\ex@}#2\cr
    \noalign{\nointerlineskip\kern-\ex@}\cr}}%
}
\newcommand{\colim}{%
  \mathop{\mathpalette\colim@{\rightarrowfill@\scriptscriptstyle}}\nmlimits@
}
\renewcommand{\varprojlim}{%
  \mathop{\mathpalette\varlim@{\leftarrowfill@\scriptscriptstyle}}\nmlimits@
}
\renewcommand{\varinjlim}{%
  \mathop{\mathpalette\varlim@{\rightarrowfill@\scriptscriptstyle}}\nmlimits@
}
\makeatother

\theoremstyle{plain}
\newtheorem{thm}[subsection]{Théoreme}
\newtheorem{lem}[subsection]{Lemme}
\newtheorem{prop}[subsection]{Proposition}
\newtheorem{cor}[subsection]{Corollaire}
\newtheorem{heur}{Heuristique}
\newtheorem{rem}{Remarque}
\newtheorem{note}{Note}

\theoremstyle{definition}
\newtheorem{conj}{Conjecture}
\newtheorem{prob}{Problème}
\newtheorem{quest}{Question}
\newtheorem{prot}{Protocole}
\newtheorem{algo}{Algorithme}
\newtheorem{defn}[subsection]{Définition}
\newtheorem{exmp}[subsection]{Exemples}
\newtheorem{exo}[subsection]{Exercices}
\newtheorem{ex}[subsection]{Exemple}
\newtheorem{exs}[subsection]{Exemples}
\newtheorem{res}{Résumé}
\newtheorem{rep}{Réponse}
\newtheorem{cons}{Conséquence}

\theoremstyle{remark}

\definecolor{wgrey}{RGB}{148, 38, 55}
\definecolor{wgreen}{RGB}{100, 200,0} 
\hypersetup{
    colorlinks=true,
    linkcolor=wgreen,
    urlcolor=wgrey,
    filecolor=wgrey
}

\title{Combinatoire}
\date{}

\begin{document}
\maketitle
\tableofcontents
\chapter{Coefficients binomiaux}
Je note $n$ pour l'entier et/ou l'ensemble
à $\{0,1,\ldots,n-1\}$ à $n$ éléments. Et
$(n)_k$ pour le nombre de combinaisons de 
taille $k$ parmi $n$. I.e. 
\[(n)_k = \frac{n!}{(n-k)!}\]
Y'a
\begin{enumerate}
  \item $n^k$ flèches de $k\to n$.
  \item Y'a $(n)_k$ flèches injectives de $k\to n$.
    Ça correspond aux suites du Knuth.
  \item Y'a $k!$ ordres totaux sur $k$ (les précompositions
    par une permutation!).
  \item On déduit que y'a $\frac{(n)_k}{k!}$
    sous-ensembles de taille $k$ dans $n$.
\end{enumerate}
\begin{rem}
  La distinction entre fonction et sous-ensemble est cool. Quand
  on dit sous-ensemble on s'en fout de l'ordre ! Quand on dit
  fonction on peut précomposer par une permutation
\end{rem}

\section{Absorption et choix imbriqués}
On a les formules d'absorptions 
\[k\begin{pmatrix}n\\k\end{pmatrix}=n\begin{pmatrix}n-1\\k-1\end{pmatrix}\]
via $(n-k)=(n-1)-(k-1)$ et
\[(n-k)\begin{pmatrix}n\\k\end{pmatrix}=n\begin{pmatrix}n-1\\k\end{pmatrix}\]
on peut en déduire
\[\begin{pmatrix}n\\k\end{pmatrix}=\begin{pmatrix}n-1\\k\end{pmatrix}+\begin{pmatrix}n-1\\k-1\end{pmatrix}\]
via l'identité 
\[(n-k)\begin{pmatrix}n\\k\end{pmatrix}+k\begin{pmatrix}n\\k\end{pmatrix}=n\begin{pmatrix}n-1\\k\end{pmatrix}+n\begin{pmatrix}n\\k\end{pmatrix}\]
c'est abusé mdr.

Sinon on peut se rendre compte de la formule via 
\begin{enumerate}
  \item On fixe $i\in n$.
  \item Pour choisir $k$ éléments, soit on
    prends $i$ et on a $k-1$ parmi $n-1$ à
    choisir.
  \item Soit on prends pas $i$ et on a $k$
    parmi $n-1$ à choisir.
\end{enumerate}
\begin{rem}
  Pour une preuve on peut prendre une suite et mettre
  $i$ en premier, puis $(n-1)_{k-1}$.
\end{rem}

\section{Binôme de Newton}
Pour calculer le produit $\prod_{k=1}^n(a+b)=(a+b)^n$ on 
peut choisir un chemin dans la distributivité. Y'a $2^n$ choix
à faire mais ça consiste à dire dire on prend ou $a$ ou $b$
dans le $k$-ème monôme. Le nombre de manière d'avoir $a^kb^{n-k}$
c'est faire $k$ choix parmi $n$ en version désordonnée. D'où
$\begin{pmatrix}n\\k\end{pmatrix}$.

\section{Sommes de binômes}
La première c'est 
\[\sum_{k=0}^n \begin{pmatrix}n\\ k\end{pmatrix}=k^n+1\]
et on la trouve en remarquant que $k^n+1=\#P(n)$ les parties
de $n$. L'autre c'est
\[\sum_{i=k}^n\begin{pmatrix}i\\ k\end{pmatrix}=\begin{pmatrix}n+1\\ k+1\end{pmatrix}\]
et on peut la trouver soit via l'identité
\[(X+1)^n-1=X(\sum_{i=0}^{n-1} (X+1)^i)\]
soit par absorption successives, i.e. 
\[(k+2~k+1)=(k+1~k+1)+(k+1~k)\]
d'où en commençant par le début de la somme et via
\[(k+1~k+1)=(k~k)\]
on peut itérer jusqu'au résultat.

\section{Multinômes}
Y'a l'identité
\[(X_1+\ldots+X^m)^n=\sum_{k_1+\ldots+k_m=n}\begin{pmatrix}n\\k_1,\ldots,k_n\end{pmatrix}\prod_{i=1}^m X_i^{k_i}\]
où $\begin{pmatrix}n\\k_1,\ldots,k_m\end{pmatrix}=\frac{n!}{\prod_i k_i!}$ est le nombre de manière de
partitionner $n$ en $m$ sous-ensembles disjoints de tailles $k_1,\ldots, k_m$. Par rapport
à l'identité du coefficient multinomial la preuve c'est que 
\[(n~k_1).(n-k_1~k_2).(n-(k_1+k_2)~k_3).\ldots=\frac{n!}{\prod_i k_i!}\]
via un choix d'une suite décroissante $k_1,\ldots, k_m$ (y'en a tjr une).
La preuve du multinôme c'est la même idée.

\section{Congruences}
Pour $n|(n~k)$ quand $n\wedge k=1$ on peut
écrire $n(n-1~k-1)=k(n~k)$. Si $n\wedge k=1$ on
a fini. En particulier, $p^e|(p^e~k)$ si $p\nmid k$.
En général, $p|(p^e~k)$ pour $1\leq k\leq p^e$.

Via le calcul du multinôme on déduit aussi que
$n\mid \begin{pmatrix}n\\k_1,\ldots,k_m\end{pmatrix}$
dès que il existe $i$ tel que $k_i\wedge n=1$.




\printbibliography
\end{document}

