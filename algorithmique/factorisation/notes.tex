\documentclass[a4paper,12pt]{book}
\usepackage{amsmath,  amsthm,enumerate}
\usepackage{csquotes}
\usepackage[provide=*,french]{babel}
\usepackage[dvipsnames]{xcolor}
\usepackage{quiver, tikz}

%symbole caligraphique
\usepackage{mathrsfs}

%hyperliens
\usepackage{hyperref}

%pseudo-code
\usepackage{algorithm}
\usepackage{algpseudocode}

\usepackage{fancyhdr}

\pagestyle{fancy}
\addtolength{\headwidth}{\marginparsep}
\addtolength{\headwidth}{\marginparwidth}
\renewcommand{\chaptermark}[1]{\markboth{#1}{}}
\renewcommand{\sectionmark}[1]{\markright{\thesection\ #1}}
\fancyhf{}
\fancyfoot[C]{\thepage}
\fancyhead[LO]{\textit \leftmark}
\fancyhead[RE]{\textit \rightmark}
\renewcommand{\headrulewidth}{0pt} % and the line
\fancypagestyle{plain}{%
    \fancyhead{} % get rid of headers
}

%bibliographie
\usepackage[
backend=biber,
style=alphabetic,
sorting=ynt
]{biblatex}

\addbibresource{bib.bib}

\usepackage{appendix}
\renewcommand{\appendixpagename}{Annexe}

\definecolor{wgrey}{RGB}{148, 38, 55}

\setlength\parindent{24pt}

\newcommand{\Z}{\mathbb{Z}}
\newcommand{\R}{\mathbb{R}}
\newcommand{\rel}{\omathcal{R}}
\newcommand{\Q}{\mathbb{Q}}
\newcommand{\C}{\mathbb{C}}
\newcommand{\N}{\mathbb{N}}
\newcommand{\K}{\mathbb{K}}
\newcommand{\A}{\mathbb{A}}
\newcommand{\B}{\mathcal{B}}
\newcommand{\Or}{\mathcal{O}}
\newcommand{\F}{\mathbb F}
\newcommand{\Hom}{\textrm{Hom}}
\newcommand{\disc}{\textrm{disc}}
\newcommand{\Pic}{\textrm{Pic}}
\newcommand{\End}{\textrm{End}}
\newcommand{\Spec}{\textrm{Spec}}

\newcommand{\cL}{\mathscr{L}}
\newcommand{\G}{\mathscr{G}}
\newcommand{\D}{\mathscr{D}}
\newcommand{\E}{\mathscr{E}}
\renewcommand{\H}{\mathscr{H}}

\theoremstyle{plain}
\newtheorem{thm}[subsection]{Théoreme}
\newtheorem{lem}[subsection]{Lemme}
\newtheorem{prop}[subsection]{Proposition}
\newtheorem{cor}[subsection]{Corollaire}
\newtheorem{heur}{Heuristique}
\newtheorem{rem}{Remarque}
\newtheorem{note}{Note}

\theoremstyle{definition}
\newtheorem{conj}{Conjecture}
\newtheorem{prob}{Problème}
\newtheorem{quest}{Question}
\newtheorem{prot}{Protocole}
\newtheorem{algo}{Algorithme}
\newtheorem{defn}[subsection]{Définition}
\newtheorem{exmp}[subsection]{Exemples}
\newtheorem{exo}[subsection]{Exercices}
\newtheorem{ex}[subsection]{Exemple}
\newtheorem{exs}[subsection]{Exemples}
\newtheorem{avanc}{Avancée(s)}

\theoremstyle{remark}

\definecolor{wgrey}{RGB}{148, 38, 55}
\definecolor{wgreen}{RGB}{100, 200,0} 
\hypersetup{
    colorlinks=true,
    linkcolor=wgreen,
    urlcolor=wgrey,
    filecolor=wgrey
}

\title{Factorisation classique}
\date{}

\begin{document}
\maketitle
\tableofcontents
\chapter{Petits critères}
\section{Méthode $p-1$}
En gros, si $n=\prod p_i^{e_i}$ et que l'un des $p_i$
est petit alors $p_i-1$ est probablement lisse. En 
particulier, si on prends $a\mod n$ aléatoire et
\[a_k:=a^{k!}\]
on peut tester si $a_k-1\wedge n\ne 1$ car $k!$
contiendra les facteurs de $p_i-1$ a un moment donné.

\begin{rem}
    $a$ aléatoire sans choix précis est optimal. 
    $k!$ et pas autre chose a l'air optimal. Peut-être
    qu'il y a d'autres cribles de nombres plus 
    intéressants?
\end{rem}


\chapter{La méthode de fermat et extensions}
\section{Méthode de Fermat}
Une idée, calcul successif des : 
\[issquare(n+k^2)?\]
si oui, alors \[n=(\sqrt{n+k^2}-k)(\sqrt{n+k^2}+k)\]
et on a une factorisation :
\begin{avanc}
    On a un algorithme en prenant $k$ petit, l'algo est
    naif et marche que si $k$ petit.
\end{avanc}
\begin{rem}
    L'ordre de l'autre carré est de $\sqrt n$.
\end{rem}

En fait, si on obtient un multiple de $n$ comme différence de 
deux carrés, i.e. :
\[x^2\equiv y^2\mod n\]
alors faut calculer en plus $n\wedge x-y$. Grande probabilité
que y'ai un facteur commun.
\begin{avanc}
    On peut faire la méthode d'avant, en travaillant $\mod n$.
    Marche toujours que si $k$ est petit.
\end{avanc}

\section{Méthode de Dixon}
Maintenant, on peut utiliser la technique d'avant de la manière
suivante :
\begin{center}
    Choisir une base de premiers $P_B:=\{p\in\mathbb P|p\leq B\}$
    et ajouter $-1$.
\end{center}
Ensuite on choisit aléatoirement $a\mod n$ et on calcule
\[a^2\equiv b\mod n\]
puis on regarde si il est $B$-lisse. Si oui on stocke 
\[a^2=\prod_i p_i^{e_i}\mod n\]
On obtient un ensemble de congruences $C$, on peut en chopper
$\# P_B$ et on forme 

\section{Crible quadratique}







\printbibliography

\end{document}

