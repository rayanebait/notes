\documentclass[a4paper,12pt]{book}
\usepackage{amsmath,  amsthm,enumerate}
\usepackage{csquotes}
\usepackage[provide=*,french]{babel}
\usepackage[dvipsnames]{xcolor}
\usepackage{quiver, tikz}

%symbole caligraphique
\usepackage{mathrsfs}

%hyperliens
\usepackage{hyperref}

%pseudo-code
\usepackage{algorithm}
\usepackage{algpseudocode}

\usepackage{fancyhdr}

\pagestyle{fancy}
\addtolength{\headwidth}{\marginparsep}
\addtolength{\headwidth}{\marginparwidth}
\renewcommand{\chaptermark}[1]{\markboth{#1}{}}
\renewcommand{\sectionmark}[1]{\markright{\thesection\ #1}}
\fancyhf{}
\fancyfoot[C]{\thepage}
\fancyhead[LO]{\textit \leftmark}
\fancyhead[RE]{\textit \rightmark}
\renewcommand{\headrulewidth}{0pt} % and the line
\fancypagestyle{plain}{%
    \fancyhead{} % get rid of headers
}

%bibliographie
\usepackage[
backend=biber,
style=alphabetic,
sorting=ynt
]{biblatex}

\addbibresource{bib.bib}

\usepackage{appendix}
\renewcommand{\appendixpagename}{Annexe}

\definecolor{wgrey}{RGB}{148, 38, 55}

\setlength\parindent{24pt}

\newcommand{\Z}{\mathbb{Z}}
\newcommand{\R}{\mathbb{R}}
\newcommand{\rel}{\omathcal{R}}
\newcommand{\Q}{\mathbb{Q}}
\newcommand{\C}{\mathbb{C}}
\newcommand{\N}{\mathbb{N}}
\newcommand{\K}{\mathbb{K}}
\newcommand{\A}{\mathbb{A}}
\newcommand{\B}{\mathcal{B}}
\newcommand{\Or}{\mathcal{O}}
\newcommand{\F}{\mathscr F}
\newcommand{\Hom}{\textrm{Hom}}
\newcommand{\disc}{\textrm{disc}}
\newcommand{\Pic}{\textrm{Pic}}
\newcommand{\End}{\textrm{End}}
\newcommand{\Spec}{\textrm{Spec}}
\newcommand{\Supp}{\textrm{Supp}}
\renewcommand{\Im}{\textrm{Im}}
\newcommand{\m}{\mathfrak{m}}
\renewcommand{\P}{\mathbb{P}}
\newcommand{\p}{\mathfrak{p}}


\newcommand{\cL}{\mathscr{L}}
\newcommand{\G}{\mathscr{G}}
\newcommand{\D}{\mathscr{D}}
\newcommand{\E}{\mathscr{E}}
\newcommand{\Po}{\mathscr{P}}
\renewcommand{\H}{\mathscr{H}}

\makeatletter
\newcommand{\colim@}[2]{%
  \vtop{\m@th\ialign{##\cr
    \hfil$#1\operator@font colim$\hfil\cr
    \noalign{\nointerlineskip\kern1.5\ex@}#2\cr
    \noalign{\nointerlineskip\kern-\ex@}\cr}}%
}
\newcommand{\colim}{%
  \mathop{\mathpalette\colim@{\rightarrowfill@\scriptscriptstyle}}\nmlimits@
}
\renewcommand{\varprojlim}{%
  \mathop{\mathpalette\varlim@{\leftarrowfill@\scriptscriptstyle}}\nmlimits@
}
\renewcommand{\varinjlim}{%
  \mathop{\mathpalette\varlim@{\rightarrowfill@\scriptscriptstyle}}\nmlimits@
}
\makeatother

\theoremstyle{plain}
\newtheorem{thm}[subsection]{Théoreme}
\newtheorem{lem}[subsection]{Lemme}
\newtheorem{prop}[subsection]{Proposition}
\newtheorem{cor}[subsection]{Corollaire}
\newtheorem{heur}{Heuristique}
\newtheorem{rem}{Remarque}
\newtheorem{note}{Note}

\theoremstyle{definition}
\newtheorem{conj}{Conjecture}
\newtheorem{prob}{Problème}
\newtheorem{quest}{Question}
\newtheorem{prot}{Protocole}
\newtheorem{algo}{Algorithme}
\newtheorem{defn}[subsection]{Définition}
\newtheorem{exmp}[subsection]{Exemples}
\newtheorem{exo}[subsection]{Exercices}
\newtheorem{ex}[subsection]{Exemple}
\newtheorem{exs}[subsection]{Exemples}
\newtheorem{res}{Résumé}
\newtheorem{rep}{Réponse}
\newtheorem{cons}{Conséquence}

\theoremstyle{remark}

\definecolor{wgrey}{RGB}{148, 38, 55}
\definecolor{wgreen}{RGB}{100, 200,0} 
\hypersetup{
    colorlinks=true,
    linkcolor=wgreen,
    urlcolor=wgrey,
    filecolor=wgrey
}

\title{Géométrie algébrique}
\date{}

\begin{document}
\maketitle
\tableofcontents

Ici on reprends les bases mdr.

\chapter{Relations d'équivalences}
Étant donné un ensemble $X$, on définit
$R\subset X\times X$ une relation. C'est
une relation d'éq si on rajoute les conds
habituelles. 

\section{Groupe quotient}
\subsection{Définition directe}
En fait un groupe quotient $G/H$ par déf donc
ce sera un quotient par une relation bien 
spéciale. On pose $G/H$ l'ensemble des
cosets à gauche (qui sera pareil que les
cosets à droite) :
\[gH\subset G\]
on définit $(xH)(yH):=(xy)H$. C'est bien
défini ssi $H$ est normal. En gros
faut que ce soit indépendant du choix
du représentant. Une manière de voir 
qu'être normal permet de bien la définir
c'est que si $(xy)H=(ab)H$ avec $xH=aH$ 
et $yH=bH$ alors
\[xyH=x(yH)=xHy=aHy=a(yH)=abH.\]
À l'inverse l'existence d'une loi de 
groupe du type $(xH)(yH)=(xy)H$ force
pour $h\in H$, d'où $H=hH$, que
\[gH=(eg)H=eHgH=(hg)H\]
d'où $g^{-1}hg\in H$.

\begin{rem}
  Composer des ensembles c'est marrant.
\end{rem}

\subsection{Autre définition}
On peut quotienter $G$ par la relation
d'équivalence $g\sim g'$ si $g^{-1}g'\in H$.
On remarque que pour tout $x\in G$,
\[g\sim g' \equiv xg\sim xg'\]
alors la projection $\pi(g)=gH$ qui
à $g$ associe son orbite est
un morphisme de groupe si et seulement
si $H$ est normal.

\subsection{Propriété}
 Il doit avoir une propriété universelle.
 Notamment celle de conoyau.

\section{Anneau quotient}
Cette fois on veut préserver la multiplication
et l'addition. On définit à nouveau
$x\sim y \equiv x-y\in I$. C'est une relation
d'équivalence parce que $I$ est un idéal bilatère.
(même juste un groupe additif abélien). Maintenant,
la loi est donnée par \[(x+I)+(y+I):=(x+y)+I\] et
\[(x+I)(y+I):=(xy+I)\]
et faut vérifier que l'existence de cette loi
d'anneau force $I$ un sous-groupe additif à
être un idéal.

\subsection{Idéal}
C'est marrant ducoup, on veut montrer
que si $x\in I$ et $a\in I$ alors $ax\in I$.
On peut regarder
\[ax+I=(a+I)(x+I)=(a+I).I\]
mais $I$ est l'élement nul d'où 
\[(a+I).I=I\]
en particulier $ax\in I$.





\section{Théorèmes d'isomorphismes}
Deux choses à mentionner : dans un
groupe $G$, si $N$ est normal et 
$S$ quelquoncque. On a 
\begin{enumerate}
  \item $N\subset SN$
est normal.
  \item $S\cap N\subset S$ est
normal. 
  \item Puis $S/S\cap N \simeq SN/N$.
\end{enumerate}
de même pour les anneaux dans $R$ 
un anneau si $S$ est un sous-anneau 
et $I$ un idéal. On a

\begin{enumerate}
  \item $S\cap I$ est un idéal de $S$.
  \item $S+I$ est un anneau et $I$
    un idéal de $S+I$.
  \item Puis $(S+I)/I\simeq S/S\cap I$.
\end{enumerate}
Pour les modules c'est pareil.

Ça règle mes questionnements d'y a 
quelques temps du genre ça veut dire quoi
$"S/I"$. À refaire dans ma tête
desfois c'est facile.

\section{Lemme chinois et coprimalité d'idéaux}
Ducoup la coprimalité c'est $I+J=(1)$. On déduit
de $i+j=1$ que si $d\in I\cap J$ alors $d=di+dj\in IJ$.
De la même manière pour une famille d'idéaux. 
Maintenant 
\[\varphi\colon A\to \prod A/\a_i\]
est surjective ssi les $\a_i$ sont copremiers deux
à deux via une relation de Bézout. Et injective si 
$\cap \a_i=(0)$ c'est pas clair tel quel mais faut
quotienter ducoup. I.e. remplacer $\a_i$ par
$\a_i/\prod \a_i$.



\printbibliography
\end{document}

