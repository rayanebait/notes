\documentclass[a4paper,12pt]{book}
\usepackage{amsmath,  amsthm,enumerate}
\usepackage{csquotes}
\usepackage[provide=*,french]{babel}
\usepackage[dvipsnames]{xcolor}
\usepackage{quiver, tikz}

%symbole caligraphique
\usepackage{mathrsfs}

%hyperliens
\usepackage{hyperref}

%pseudo-code
\usepackage{algpseudocode}
\usepackage{algorithm}
\makeatletter
  \renewcommand{\ALG@name}{Algorithme}
  \makeatother
\usepackage{fancyhdr}

\pagestyle{fancy}
\addtolength{\headwidth}{\marginparsep}
\addtolength{\headwidth}{\marginparwidth}
\renewcommand{\chaptermark}[1]{\markboth{#1}{}}
\renewcommand{\sectionmark}[1]{\markright{\thesection\ #1}}
\fancyhf{}
\fancyfoot[C]{\thepage}
\fancyhead[LO]{\textit \leftmark}
\fancyhead[RE]{\textit \rightmark}
\renewcommand{\headrulewidth}{0pt} % and the line
\fancypagestyle{plain}{%
    \fancyhead{} % get rid of headers
}

%bibliographie
\usepackage[
backend=biber,
style=alphabetic,
sorting=ynt
]{biblatex}

\addbibresource{bib.bib}

\usepackage{appendix}
\renewcommand{\appendixpagename}{Annexe}

\definecolor{wgrey}{RGB}{148, 38, 55}

\setlength\parindent{24pt}

\newcommand{\Z}{\mathbb{Z}}
\newcommand{\R}{\mathbb{R}}
\newcommand{\rel}{\omathcal{R}}
\newcommand{\Q}{\mathbb{Q}}
\newcommand{\C}{\mathbb{C}}
\newcommand{\N}{\mathbb{N}}
\newcommand{\K}{\mathbb{K}}
\newcommand{\A}{\mathbb{A}}
\newcommand{\B}{\mathcal{B}}
\newcommand{\Or}{\mathcal{O}}
\newcommand{\F}{\mathbb F}
\newcommand{\m}{\mathfrak m}
\renewcommand{\b}{\mathfrak b}
\renewcommand{\a}{\mathfrak a}
\newcommand{\p}{\mathfrak p}
\newcommand{\I}{\mathfrak I}
\newcommand{\Hom}{\textrm{Hom}}
\newcommand{\disc}{\textrm{disc}}
\newcommand{\Pic}{\textrm{Pic}}
\newcommand{\End}{\textrm{End}}
\newcommand{\Spec}{\textrm{Spec}}
\newcommand{\Frac}{\textrm{Frac}}

\newcommand{\cL}{\mathscr{L}}
\newcommand{\G}{\mathscr{G}}
\newcommand{\D}{\mathscr{D}}
\newcommand{\E}{\mathscr{E}}
\newcommand{\U}{\mathscr{U}}

\theoremstyle{plain}
\newtheorem{thm}{Théoreme}
\newtheorem{lem}{Lemme}
\newtheorem{prop}{Proposition}
\newtheorem{cor}{Corollaire}
\newtheorem{heur}{Heuristique}
\newtheorem{rem}{Remarque}
\newtheorem{rembis}{Remarque}
\newtheorem{note}{Note}

\theoremstyle{definition}
\newtheorem{conj}{Conjecture}
\newtheorem*{eq}{Équivalences}
\newtheorem{prob}{Problème}
\newtheorem{quest}{Question}
\newtheorem{prot}{Protocole}
\newtheorem{algo}{Algorithme}
\newtheorem{defn}{Définition}
\newtheorem{defnbis}{Définition}
\newtheorem{ex}{Exemple}
\newtheorem{exo}{Exercices}

\theoremstyle{remark}

\definecolor{wgrey}{RGB}{148, 38, 55}
\definecolor{wgreen}{RGB}{100, 200,0} 
\hypersetup{
    colorlinks=true,
    linkcolor=wgreen,
    urlcolor=wgrey,
    filecolor=wgrey
}

\title{Actions de groupes et orbifolds}
\date{}

\begin{document}
\maketitle

\section{Revêtements et actions de groupes}


\section{Orbifolds}
34.8 du voight. Petite idée pour la suite : définition de manifolds avec 
faisceaux, comparer avec orbifolds.

L'idée est de traiter des actions de groupes non libres en donnant une
structure au quotient. 
\begin{defn}[Orbivariétés]
		Une orbivariété de dimension $n$ est un espace topologique à base
		dénombrable (second-countable) séparé localement homéomorphe à
		$G\backslash\mathbb R^n $ pour un groupe fini $G$ agissant par
		homéos.
\end{defn}
\begin{rem}
		Y'a plusieurs typo sur le voight, "fij" smooth, ou l'ordre dans
		la composition.
\end{rem}
\begin{defn}[Atlas]
		Recouvrement ouvert $(U_i)_i$ de cartes fermé par intersection tel 
		que pour tout $i$, il existe $G_i\acts V_i\subset \mathbb R^n$ avec
		un homéo
		\[ \phi_i\colon U_i\to G_i\backslash V_i\]
		et dès que $U_i\subset U_j$ des flèches injectives
		$f_{ij}\colon G_i\to G_j$
		et des flèches
		\[ \psi_{ij}\colon V_i\to V_j\]
		$G_i$-équivariantes telles que $\phi_j^{-1}\circ \psi_{ij}=\phi_i^{-1}$.
\end{defn}
Les $\psi_{ij}$ ont le role des $\phi_i^{-1}\circ \phi_j$ habituels. En 
demandant $\psi_{ij}$ lisses ou holomorphes on obtient des orbivariétés
lisses ou complexes. En ajoutant $\psi_{ij}$ préservent une $G_i$-métrique
Riemannienne on obtient une orbivariété riemannienne.
\begin{defn}[Point orbital]
		Un point $p$ est orbital si localement autour on a $G\backslash \R^n$
		avec $G$ non trivial.
\end{defn}
\begin{ex}
		Un exemple avec un ensemble discret de points orbitaux c'est 
		Un bon exemple c'est $Y(1)$. 
\end{ex}












\end{document}
