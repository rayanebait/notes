\documentclass[a4paper,12pt]{book}
\usepackage{amsmath,  amsthm,enumerate}
\usepackage{csquotes}
\usepackage[provide=*,french]{babel}
\usepackage[dvipsnames]{xcolor}
\usepackage{quiver, tikz}

%symbole caligraphique
\usepackage{mathrsfs}

%hyperliens
\usepackage{hyperref}

%pseudo-code
\usepackage{algpseudocode}
\usepackage{algorithm}
\makeatletter
  \renewcommand{\ALG@name}{Algorithme}
  \makeatother
\usepackage{fancyhdr}

\pagestyle{fancy}
\addtolength{\headwidth}{\marginparsep}
\addtolength{\headwidth}{\marginparwidth}
\renewcommand{\chaptermark}[1]{\markboth{#1}{}}
\renewcommand{\sectionmark}[1]{\markright{\thesection\ #1}}
\fancyhf{}
\fancyfoot[C]{\thepage}
\fancyhead[LO]{\textit \leftmark}
\fancyhead[RE]{\textit \rightmark}
\renewcommand{\headrulewidth}{0pt} % and the line
\fancypagestyle{plain}{%
    \fancyhead{} % get rid of headers
}

%bibliographie
\usepackage[
backend=biber,
style=alphabetic,
sorting=ynt
]{biblatex}

\addbibresource{bib.bib}

\usepackage{appendix}
\renewcommand{\appendixpagename}{Annexe}

\definecolor{wgrey}{RGB}{148, 38, 55}

\setlength\parindent{24pt}

\newcommand{\Z}{\mathbb{Z}}
\newcommand{\R}{\mathbb{R}}
\newcommand{\rel}{\omathcal{R}}
\newcommand{\Q}{\mathbb{Q}}
\newcommand{\C}{\mathbb{C}}
\newcommand{\N}{\mathbb{N}}
\newcommand{\K}{\mathbb{K}}
\newcommand{\A}{\mathbb{A}}
\newcommand{\B}{\mathcal{B}}
\newcommand{\Or}{\mathcal{O}}
\newcommand{\F}{\mathbb F}
\newcommand{\m}{\mathfrak m}
\renewcommand{\b}{\mathfrak b}
\renewcommand{\a}{\mathfrak a}
\newcommand{\p}{\mathfrak p}
\newcommand{\I}{\mathfrak I}
\newcommand{\Hom}{\textrm{Hom}}
\newcommand{\disc}{\textrm{disc}}
\newcommand{\Pic}{\textrm{Pic}}
\newcommand{\End}{\textrm{End}}
\newcommand{\Spec}{\textrm{Spec}}
\newcommand{\Frac}{\textrm{Frac}}

\newcommand{\cL}{\mathscr{L}}
\newcommand{\G}{\mathscr{G}}
\newcommand{\D}{\mathscr{D}}
\newcommand{\E}{\mathscr{E}}
\newcommand{\U}{\mathscr{U}}

\theoremstyle{plain}
\newtheorem{thm}{Théoreme}
\newtheorem{lem}{Lemme}
\newtheorem{prop}{Proposition}
\newtheorem{cor}{Corollaire}
\newtheorem{heur}{Heuristique}
\newtheorem{rem}{Remarque}
\newtheorem{rembis}{Remarque}
\newtheorem{note}{Note}

\theoremstyle{definition}
\newtheorem{conj}{Conjecture}
\newtheorem*{eq}{Équivalences}
\newtheorem{prob}{Problème}
\newtheorem{quest}{Question}
\newtheorem{prot}{Protocole}
\newtheorem{algo}{Algorithme}
\newtheorem{defn}{Définition}
\newtheorem{defnbis}{Définition}
\newtheorem{ex}{Exemple}
\newtheorem{exo}{Exercices}

\theoremstyle{remark}

\definecolor{wgrey}{RGB}{148, 38, 55}
\definecolor{wgreen}{RGB}{100, 200,0} 
\hypersetup{
    colorlinks=true,
    linkcolor=wgreen,
    urlcolor=wgrey,
    filecolor=wgrey
}

\title{$\Z_p$-extensions, $L\leq \Z_p^n$-extensions}
\date{}

\begin{document}
\maketitle
\section{Filtration de ramification}
On se met dans le cadre totalement ramifié et $G=Gal(K_\infty/K)=\Z_p$
avec $K$ local. Par exemple $\Q_p(\zeta_{p^\infty})/\Q_p(\zeta_p)$.
\subsection{Cadre}
On a $G=G_K^{ab}/H$ et si $\theta_K(N)=H\cap P_{ab,K}$ via la réciprocité
alors 
\[G^{(\nu)}=(G_K^{ab})^{(\nu)}/(G_K^{ab})^{(\nu)}\cap H=U_K^{(\nu)}/N\cap U_K^{(\nu)}\]
\subsection{Sauts de ramification}
Si les $(\nu_i)$ sont les sauts de ramifications 
c'est des entiers via ceux de $G_K^{ab}$ puis comme $G^{\nu_i}/G^{\nu_{i+1}}$
est $p$-élémentaire (ça se fait bien) bah
\[G^{(\nu_i)}\simeq p^i\Z_p\]
pour chaque $i\geq 1$.

\section{Sauts asymptotiques et différentes}
Si $i_0$ est le premier $i$ tel que $\nu_{i}>e_K/(p-1)$ alors
pour tout $i\geq i_0$ (!)
\[\nu_{i+n}=\nu_i+ne_K\]
et si $K_n=K_\infty^{p^n\Z_p}$ alors $v_K(\D_{K_n/K})=c+ne_K+a_n/p^n$
avec $c$ une constante et $a_n$ bornée. 
\subsection{Sauts asymptotiques}
On a trois identification, $U^{(\nu_i)}\simeq G^{(\nu_i)}\simeq p^i\Z_p$.
Maintenant on a $U_K^{(i+e_k)}=(U_K^{(i)})^p$ quand $i>e_K/(p-1)$ simplement
en appliquant $\exp\circ\log$. Via un générateur $g$ topologique
et $\rho\colon G^{(\nu)}\to \U^{(\nu)}$ $\rho(g)$ engendre $\U$.
Maintenant $g^{p^i}$ est un générateur de $p^i\Z_p$ (additif) d'où $\rho(g^{p^i})$ de
$\U^{(\nu_i)}$. On prend $n=i_0$ tel que $\nu_{i_0}>e_K/(p-1)$.
Alors $(\U^{(\nu_{i_0})})^p=\U^{(\nu_{i_0}+e_K)}$.

Maintenant faut regarder dans $\Z_p$, $(\U^{(\nu_{i_0})})^p$ correspond à
$p^{i_0+1}\Z_p$ qui correspond à $G^{(\nu_{i_0+1})}$.

\subsection{Différente}
En notant $G(n)=G/G^{\nu_n}\cap G=\Z_p/p^n\Z_p$ puis on a 
\[G(n)^{(\nu_i)}=G^{(\nu_i)}/G(n)\cap G^{(\nu_i)}=p^i\Z_p/p^n\Z_p\simeq \Z_p/p^{n-i}\Z_p\]
d'où \[|G(n)^{(\nu)}|=\begin{cases} p^{n-i},~\nu_{i-1}<\nu\leq \nu_i\\ 1,~\nu>\nu_{n-1}\end{cases}\]
Maintenant on coupe 
\[v_K(\D_{K_n/K})=\int_{-1}^\infty\left(1-1/|G(n)^{(\nu)}|\right)d\nu\]
en $A_n=\int_{-1}^{i_0}\ldots$ avec $i_0>e_K/(p-1)$ et
$B_n=\int_{i_0}^\infty \left(1-1/|G(n)^{(\nu)}|\right)d\nu$.
Vu qu'on connait $\nu_{i+1}-\nu_i$ à partir de $i_0$ on a
\[B_n=\sum_{i=i_0}^{n-1} e_K(1-1/p^{n-i})=e_k(n-i_0-1)-e_k(p-1)/p^{n-i_0}\]
d'où le résultat en arreangeant un peu.




\section{Remarques}




\end{document}
