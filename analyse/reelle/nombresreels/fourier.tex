\documentclass[a4paper,12pt]{book}
\usepackage{amsmath,  amsthm,enumerate}
\usepackage{csquotes}
\usepackage[provide=*,french]{babel}
\usepackage[dvipsnames]{xcolor}
\usepackage{quiver, tikz}

%symbole caligraphique
\usepackage{mathrsfs}

%hyperliens
\usepackage{hyperref}

%pseudo-code
\usepackage{algpseudocode}
\usepackage{algorithm}
\makeatletter
  \renewcommand{\ALG@name}{Algorithme}
  \makeatother
\usepackage{fancyhdr}

\pagestyle{fancy}
\addtolength{\headwidth}{\marginparsep}
\addtolength{\headwidth}{\marginparwidth}
\renewcommand{\chaptermark}[1]{\markboth{#1}{}}
\renewcommand{\sectionmark}[1]{\markright{\thesection\ #1}}
\fancyhf{}
\fancyfoot[C]{\thepage}
\fancyhead[LO]{\textit \leftmark}
\fancyhead[RE]{\textit \rightmark}
\renewcommand{\headrulewidth}{0pt} % and the line
\fancypagestyle{plain}{%
    \fancyhead{} % get rid of headers
}

%bibliographie
\usepackage[
backend=biber,
style=alphabetic,
sorting=ynt
]{biblatex}

\addbibresource{bib.bib}

\usepackage{appendix}
\renewcommand{\appendixpagename}{Annexe}

\definecolor{wgrey}{RGB}{148, 38, 55}

\setlength\parindent{24pt}

\newcommand{\Z}{\mathbb{Z}}
\newcommand{\R}{\mathbb{R}}
\newcommand{\rel}{\omathcal{R}}
\newcommand{\Q}{\mathbb{Q}}
\newcommand{\C}{\mathbb{C}}
\newcommand{\N}{\mathbb{N}}
\newcommand{\K}{\mathbb{K}}
\newcommand{\A}{\mathbb{A}}
\newcommand{\B}{\mathcal{B}}
\newcommand{\Or}{\mathcal{O}}
\newcommand{\F}{\mathbb F}
\newcommand{\m}{\mathfrak m}
\renewcommand{\b}{\mathfrak b}
\renewcommand{\a}{\mathfrak a}
\newcommand{\p}{\mathfrak p}
\newcommand{\I}{\mathfrak I}
\newcommand{\Hom}{\textrm{Hom}}
\newcommand{\disc}{\textrm{disc}}
\newcommand{\Pic}{\textrm{Pic}}
\newcommand{\End}{\textrm{End}}
\newcommand{\Spec}{\textrm{Spec}}
\newcommand{\Frac}{\textrm{Frac}}

\newcommand{\cL}{\mathscr{L}}
\newcommand{\G}{\mathscr{G}}
\newcommand{\D}{\mathscr{D}}
\newcommand{\E}{\mathscr{E}}
\newcommand{\U}{\mathscr{U}}

\theoremstyle{plain}
\newtheorem{thm}{Théoreme}
\newtheorem{lem}{Lemme}
\newtheorem{prop}{Proposition}
\newtheorem{cor}{Corollaire}
\newtheorem{heur}{Heuristique}
\newtheorem{rem}{Remarque}
\newtheorem{rembis}{Remarque}
\newtheorem{note}{Note}

\theoremstyle{definition}
\newtheorem{conj}{Conjecture}
\newtheorem*{eq}{Équivalences}
\newtheorem{prob}{Problème}
\newtheorem{quest}{Question}
\newtheorem{prot}{Protocole}
\newtheorem{algo}{Algorithme}
\newtheorem{defn}{Définition}
\newtheorem{defnbis}{Définition}
\newtheorem{ex}{Exemple}
\newtheorem{exo}{Exercices}

\theoremstyle{remark}

\definecolor{wgrey}{RGB}{148, 38, 55}
\definecolor{wgreen}{RGB}{100, 200,0} 
\hypersetup{
    colorlinks=true,
    linkcolor=wgreen,
    urlcolor=wgrey,
    filecolor=wgrey
}

\title{Nombres rééls}
\date{}

\begin{document}
\maketitle
Je suis Analyse I de Tao. Le but c'est juste le sup, et l'ordre des preuves.

\section{Ordre}
On met d'abord un ordre sur $\Q$ via celui de $\Z$. On a alors une métrique via la
séparation de $\Q^-$ et $\Q^+$. Après construction de $\R$ comme classes d'équivalences
de suites de Cauchy, on a un ordre dessus et la densité de $\Q$ dedans tombe 
directement. 

\section{Outils qu'on admet}
\subsection{Sur $\N$}
Essentiellement, l'ordre sur $\N$ et la compatibilité avec les lois. Puis l'extension
à $\Q$. Et enfin l'inf sur les parties de $\N$ et la récurrence.

\subsection{Passage à la limite}
Les inégalités types $(\forall n,~x_n\leq x)$ passent à la limite, même pour LIM.

\subsection{Suites de Cauchy}
La définition consiste à itérer l'idée que si on tronque un nombre fini du début
de la suite, l'infinité suivante est contenue dans un intervalle. Et quitte 
à tronquer plus, on peut rétrécir l'intervalle.

\subsection{Les suites de Cauchy sont bornées}
Ça tombe direct de la phrase d'avant.

\subsection{Convergence des suites de Cauchy dans $\R$}
C'est un argument de diagonalisation direct.

\subsection{LIM avant lim, intérêt.}
Si on déf $LIM_n x_n$ la classe d'équivalence de la suite de Cauchy. Muni de
$\Q\to \R$ et de la métrique, on peut faire les arguments habituels de passage
à la limite.

L'intérêt est que la "limite" en question est un nombre réel.

\subsection{Propriété archimédienne}
Elle dit que pour tout $0<\epsilon, M$ il existe un entier $n$ tel que 
\[M<n\epsilon\]
On sait le faire dans $\Q$, donc par densité dans $\R$.

\begin{rem}
		Ducoup c'est marrant on a la densité de $\Q$ dans $\R$ avant la propriété
		archimédienne.
\end{rem}

\section{Supremum}
C'est défini simplement comme un majorant minimal.

L'idée est de converger par en dessous vers le sup avec une suite de Cauchy. On
prends $E$ une partie non vide de $\R$ qui a un majorant. Ensuite on prends,
\[L/n<x_0<K/n\]
où $L,K$ sont entiers, $x_0\in E$ et $K/n$ majore $E$ en majorant son majorant. 
Comme 
\[\{a\in  \Z|E\leq a/n\}\cap [L,K]\]
a un inf, $(a-1)/n$ majore pas $E$. On peut poser $a=m_n$ et recommencer. Le choix
de $L,K$ importe pas, on les change à chaque étape. Ils sont juste la pour contrôler
de très loin et permettre de choisir l'inf!

Ça donne $(m_n/n)_{n\in \N}$ une suite de Cauchy avec la propriété que
$E\leq m_n/n$ mais pas $(m_n-1)/n$. Les inégalités passent à la limite d'où 
le résultat.

\subsection{Exemple 1 : $\exists x\in\R, x^2$}
Il suffit de prendre le sup de $\{x\in \R, x^2\leq 2\}$ (il faut montrer l'égalité).

\subsection{Exemple 2 : Suites convergentes}
On peut mtn prouver que borné + monotone implique convergent. Grâce à ça on obtient
la limsup, la liminf etc...

\subsection{Exemple 3 : Séries convergentes}
Étant donné une suite $(a_n)$ de réels positifs on peut former $(\sum_{i=0}^n a_n)_n$
une suite monotone. Si cette série est majorée alors elle converge. 

\subsection{Exemple 4 : Séries convergentes, réarrangement}
p.174 du Analysis I de Tao.



















\end{document}
