\documentclass[a4paper,12pt]{article}
\usepackage{amsmath,  amsthm,enumerate}
\usepackage{csquotes}
\usepackage[provide=*,french]{babel}
\usepackage[dvipsnames]{xcolor}
\usepackage{quiver, tikz}

%symbole caligraphique
\usepackage{mathrsfs}

%hyperliens
\usepackage{hyperref}

%pseudo-code
\usepackage{algorithm}
\usepackage{algpseudocode}


%bibliographie
\usepackage[
backend=biber,
style=alphabetic,
sorting=ynt
]{biblatex}


\definecolor{wgrey}{RGB}{148, 38, 55}

\setlength\parindent{24pt}

\newcommand{\Z}{\mathbb{Z}}
\newcommand{\R}{\mathbb{R}}
\newcommand{\rel}{\omathcal{R}}
\newcommand{\Q}{\mathbb{Q}}
\newcommand{\C}{\mathbb{C}}
\newcommand{\Cat}{\mathcal{C}}
\newcommand{\Dat}{\mathcal{D}}
\newcommand{\N}{\mathbb{N}}
\newcommand{\K}{\mathbb{K}}
\newcommand{\A}{\mathbb{A}}
\newcommand{\B}{\mathcal{B}}
\newcommand{\Or}{\mathcal{O}}
\newcommand{\F}{\mathscr F}
\newcommand{\Hom}{\textrm{Hom}}
\newcommand{\disc}{\textrm{disc}}
\newcommand{\Pic}{\textrm{Pic}}
\newcommand{\End}{\textrm{End}}
\newcommand{\Spec}{\textrm{Spec}}
\newcommand{\Supp}{\textrm{Supp}}
\renewcommand{\Im}{\textrm{Im}}
\newcommand{\Ouv}{\textrm{Ouv}}
\newcommand{\im}{\textrm{im}}
\newcommand{\coker}{\textrm{coker}}
\newcommand{\coim}{\textrm{coim}}


\newcommand{\cL}{\mathscr{L}}
\newcommand{\G}{\mathscr{G}}
\newcommand{\D}{\mathscr{D}}
\newcommand{\E}{\mathscr{E}}
\renewcommand{\P}{\mathscr{P}}
\renewcommand{\H}{\mathscr{H}}

\makeatletter
\newcommand{\colim@}[2]{%
  \vtop{\m@th\ialign{##\cr
    \hfil$#1\operator@font colim$\hfil\cr
    \noalign{\nointerlineskip\kern1.5\ex@}#2\cr
    \noalign{\nointerlineskip\kern-\ex@}\cr}}%
}
\newcommand{\colim}{%
  \mathop{\mathpalette\colim@{\rightarrowfill@\scriptscriptstyle}}\nmlimits@
}
\renewcommand{\varprojlim}{%
  \mathop{\mathpalette\varlim@{\leftarrowfill@\scriptscriptstyle}}\nmlimits@
}
\renewcommand{\varinjlim}{%
  \mathop{\mathpalette\varlim@{\rightarrowfill@\scriptscriptstyle}}\nmlimits@
}
\makeatother

\theoremstyle{plain}
\newtheorem{thm}[subsection]{Théoreme}
\newtheorem{lem}[subsection]{Lemme}
\newtheorem{prop}[subsection]{Proposition}
\newtheorem{cor}[subsection]{Corollaire}
\newtheorem{heur}{Heuristique}
\newtheorem{rem}{Remarque}
\newtheorem{note}{Note}

\theoremstyle{definition}
\newtheorem{conj}{Conjecture}
\newtheorem{prob}{Problème}
\newtheorem{quest}{Question}
\newtheorem{prot}{Protocole}
\newtheorem{algo}{Algorithme}
\newtheorem{defn}[subsection]{Définition}
\newtheorem{exmp}[subsection]{Exemples}
\newtheorem{exo}[subsection]{Exercices}
\newtheorem{ex}[subsection]{Exemple}
\newtheorem{exs}[subsection]{Exemples}
\newtheorem{slog}{Slogan}

\theoremstyle{remark}

\definecolor{wgrey}{RGB}{148, 38, 55}
\definecolor{wgreen}{RGB}{100, 200,0} 
\hypersetup{
    colorlinks=true,
    linkcolor=wgreen,
    urlcolor=wgrey,
    filecolor=wgrey
}

\title{Devoir maison de théorie de Hodge p-adique}
\author{Rayane Bait}
\date{ }

\begin{document}
\maketitle

\section*{Exercice 1}
\subsection*{1)}
On note $R:=\{\alpha\zeta_p^i\}_{i=0,\ldots, p-1}$. Alors
$R$ est l'ensemble des racines de $f$. En effet,
\[(\alpha\zeta_p^i)^p=\alpha^p(\zeta_p^p)^i=p\] et
$\alpha\zeta_i\ne\alpha\zeta_p^j$ pour $i\ne j\mod p$ car
dans ce cas $\frac{\alpha\zeta_p^i}{\alpha\zeta_p^j}=\zeta_p^{i-j}
\ne 1$,
d'où $|R|=p=\deg(f)$ et l'assertion sur $R$.

En particulier, on en déduit que
$K\subset Q(\alpha,\zeta_p^i)$. En plus, $\zeta_p=\alpha\zeta_p/
\alpha\in K$
d'où $\Q_p(\zeta_p,\alpha)\subset K$.

\subsection*{2)}
On assume pour l'instant que $\Q_p(\zeta_p)/\Q_p$ est 
galoisienne de degré 
$p-1$ et que $\Q_p(\alpha)/\Q_p$ est de degré $p$. On remarque 
alors que
$\Q_p(\alpha)/\Q_p$ et $\Q_p(\zeta_p)/\Q_p$ sont linéairement
disjointes car $p\wedge p-1=1$. En particulier 
\begin{align*}
	[K:\Q_p]&=[K:\Q_p(\zeta_p)][\Q_p(\zeta_p):\Q_p]\\
		&=[\Q_p(\alpha):\Q_p][\Q_p(\zeta_p):\Q_p]\\
		&=p(p-1)
\end{align*}
et $H=Gal(K/\Q_p(\zeta_p))$ est normal dans $G$ car 
$\Q_p(\zeta_p)/\Q_p$ est galoisienne. Enfin $H$ est d'indice 
$|G/H|=|Gal(\Q_p(\zeta_p)/\Q_p)|=p-1$ qui est le résultat voulu. 

On prouve maintenant les assertions. On note $X^p-1=(X-1)\phi_p(X)$
et on a 
\[(X+1)^p-1=X(\sum_{k=1}^{p-1}\begin{pmatrix}p\\ k+1\end{pmatrix}
X^k +p)=X\phi_p(X+1)\]
d'où on déduit que $\phi_p(X+1)$ est $p\Z_p$-Eisenstein donc
irréductible dans $\Z_p[X]$. Maintenant $\Q_p(\zeta_p)$ est le
corps de décomposition de $\phi_p(X)$ sur $\Q_p$ d'où 
$\Q_p(\zeta_p)/\Q_p$ est galoisienne de degré
$[\Q_p(\zeta_p):\Q_p]=\deg(\phi_p)=p-1$. De même $X^p-p$ est
$p\Z_p$-Eisenstein de degré $p$ et $\Q_p(\alpha)$ en est un
corps de rupture d'où le résultat.

\subsection*{3)}
Dans la partie $2)$ on a montré que $\phi_p(X+1)$ est 
$p\Z_p$-Eisenstein. On en déduit directement que 
$\Q_p(\zeta_p-1)/\Q_p$ est totalement ramifiée et que $\zeta_p-1$
,qui est une racine de $\phi_p(X+1)$, en est une uniformisante.
De la même manière, $\Q_p(\alpha)/\Q_p$ est totalement ramifiée
et $\alpha$ en est une uniformisante. Enfin on a 
\[e_{K/\Q_p}=e_{K/\Q_p(\zeta_p)}e_{\Q_p(\zeta_p)/\Q_p}=e_{\Q_p(\alpha)/\Q_p}e_{\Q_p(\zeta_p)/\Q_p}=p(p-1)\]
ce qui prouve que $K/\Q_p$ est totalement ramifiée et 
\[v_p(\frac{\alpha}{\zeta_p-1})=\frac{1}{p}-\frac{1}{p-1}=\frac{p-(p-1)}{p(p-1)}=\frac{1}{p(p-1)}\]
si on note $v_p$ la valuation sur $K$ qui étend la valuation 
$p$-adique normalisée. On a prouvé que $\frac{\alpha}{\zeta_p-1}$
est une uniformisante de $K/\Q_p$. 


\subsection*{4) A REFAIRE YA ERREUR}
Le groupe de Galois $G$ est formé des automorphismes 
\[\{s_{ij}\}_{i=1,\ldots,p-1;j=0,\ldots,p-1}\]
définis par $s_{ij}(\zeta_p)=\zeta_p^i$ et 
$s_{ij}(\alpha)=\alpha\zeta_p^j$. De la ramification totale de 
$K/\Q_p$ on déduit que $\Z_p[\lambda]=\Or_K$ si l'on pose
$\lambda=\frac{\alpha}{\zeta_p-1}$. Soit maintenant $s_{ij}$ un
élément de $G$. Pour $g\in G$ si $i_G(g)$ désigne le plus
grand entier $i$ tel que $g\in G_{i-1}$, on a
\[i_G(g)=e_{K/\Q_p}v_p(g\lambda -\lambda).\]
Pour $g=s_{ij}$ on distingue deux cas, le cas $i=j$ et $i\ne j$.
En général on a
\begin{align*}
	s_{ij}\lambda-\lambda&=\frac{\alpha\zeta_p^j}{\zeta_p^i-1}
	-\frac{\alpha}{\zeta_p-1}\\
			     &=\frac{\alpha(\zeta_p^j-\zeta_p^i+1)}{(\zeta_p^i-1)(\zeta_p-1)}
\end{align*}
et dans tout les cas, $i>0$ assure que
\[v_p(\frac{\alpha}{(\zeta_p^i-1)(\zeta_p-1)})
=\frac{1}{p(p-1)}-\frac{1}{p-1}=\frac{-(p-1)}{p(p-1)}
=\frac{-1}{p}.\]
Il reste à déterminer $v_p(\zeta_p^j-\zeta_p^i+1)$. Si $i=j$,
alors $v_p(\zeta_p^j-\zeta_p^i+1)=v_p(1)=0$ et si $i\ne j$, alors
$v_p(\zeta_p^j-(1-\zeta_p^i))=\min(v_p(\zeta_p^j),v_p(\zeta_p^i-1))=0$. 
On déduit dans tout les cas $i_G(s_{ij})=p(p-1)


\section*{Exercice 2}



\end{document}


