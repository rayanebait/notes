\documentclass[a4paper,12pt]{article}
\usepackage{amsmath,  amsthm,enumerate}
\usepackage{csquotes}
\usepackage[provide=*,french]{babel}
\usepackage[dvipsnames]{xcolor}
\usepackage{quiver, tikz}

%symbole caligraphique
\usepackage{mathrsfs}

%hyperliens
\usepackage{hyperref}

%pseudo-code
\usepackage{algorithm}
\usepackage{algpseudocode}


%bibliographie
\usepackage[
backend=biber,
style=alphabetic,
sorting=ynt
]{biblatex}


\definecolor{wgrey}{RGB}{148, 38, 55}

\setlength\parindent{24pt}

\newcommand{\Z}{\mathbb{Z}}
\newcommand{\R}{\mathbb{R}}
\newcommand{\rel}{\omathcal{R}}
\newcommand{\Q}{\mathbb{Q}}
\newcommand{\C}{\mathbb{C}}
\newcommand{\Cat}{\mathcal{C}}
\newcommand{\Dat}{\mathcal{D}}
\newcommand{\N}{\mathbb{N}}
\newcommand{\K}{\mathbb{K}}
\newcommand{\A}{\mathbb{A}}
\newcommand{\B}{\mathcal{B}}
\newcommand{\Or}{\mathcal{O}}
\newcommand{\F}{\mathscr F}
\newcommand{\Hom}{\textrm{Hom}}
\newcommand{\disc}{\textrm{disc}}
\newcommand{\Pic}{\textrm{Pic}}
\newcommand{\End}{\textrm{End}}
\newcommand{\Spec}{\textrm{Spec}}
\newcommand{\Supp}{\textrm{Supp}}
\renewcommand{\Im}{\textrm{Im}}
\newcommand{\Ouv}{\textrm{Ouv}}
\newcommand{\im}{\textrm{im}}
\newcommand{\coker}{\textrm{coker}}
\newcommand{\coim}{\textrm{coim}}


\newcommand{\m}{\mathfrak m}
\newcommand{\cL}{\mathscr{L}}
\newcommand{\G}{\mathscr{G}}
\newcommand{\D}{\mathscr{D}}
\newcommand{\E}{\mathscr{E}}
\renewcommand{\P}{\mathscr{P}}
\renewcommand{\H}{\mathscr{H}}

\makeatletter
\newcommand{\colim@}[2]{%
  \vtop{\m@th\ialign{##\cr
    \hfil$#1\operator@font colim$\hfil\cr
    \noalign{\nointerlineskip\kern1.5\ex@}#2\cr
    \noalign{\nointerlineskip\kern-\ex@}\cr}}%
}
\newcommand{\colim}{%
  \mathop{\mathpalette\colim@{\rightarrowfill@\scriptscriptstyle}}\nmlimits@
}
\renewcommand{\varprojlim}{%
  \mathop{\mathpalette\varlim@{\leftarrowfill@\scriptscriptstyle}}\nmlimits@
}
\renewcommand{\varinjlim}{%
  \mathop{\mathpalette\varlim@{\rightarrowfill@\scriptscriptstyle}}\nmlimits@
}
\makeatother

\theoremstyle{plain}
\newtheorem{thm}[subsection]{Théoreme}
\newtheorem{lem}[subsection]{Lemme}
\newtheorem{prop}[subsection]{Proposition}
\newtheorem{cor}[subsection]{Corollaire}
\newtheorem{heur}{Heuristique}
\newtheorem{rem}{Remarque}
\newtheorem{note}{Note}

\theoremstyle{definition}
\newtheorem{conj}{Conjecture}
\newtheorem{prob}{Problème}
\newtheorem{quest}{Question}
\newtheorem{prot}{Protocole}
\newtheorem{algo}{Algorithme}
\newtheorem{defn}[subsection]{Définition}
\newtheorem{exmp}[subsection]{Exemples}
\newtheorem{exo}[subsection]{Exercices}
\newtheorem{ex}[subsection]{Exemple}
\newtheorem{exs}[subsection]{Exemples}
\newtheorem{slog}{Slogan}

\theoremstyle{remark}

\definecolor{wgrey}{RGB}{148, 38, 55}
\definecolor{wgreen}{RGB}{100, 200,0} 
\hypersetup{
    colorlinks=true,
    linkcolor=wgreen,
    urlcolor=wgrey,
    filecolor=wgrey
}

\title{Devoir maison de théorie de Hodge p-adique}
\author{Rayane Bait}
\date{ }

\begin{document}
\maketitle

\section*{Exercice 1}
\subsection*{1)}
On note $R:=\{\alpha\zeta_p^i\}_{i=0,\ldots, p-1}$. Alors
$R$ est l'ensemble des racines de $f$. En effet,
\[(\alpha\zeta_p^i)^p=\alpha^p(\zeta_p^p)^i=p\] et
$\alpha\zeta_i\ne\alpha\zeta_p^j$ pour $i\ne j\mod p$ car
dans ce cas $\frac{\alpha\zeta_p^i}{\alpha\zeta_p^j}=\zeta_p^{i-j}
\ne 1$,
d'où $|R|=p=\deg(f)$ et l'assertion sur $R$.

En particulier, on en déduit que
$K\subset Q(\alpha,\zeta_p^i)$. En plus, $\zeta_p=\alpha\zeta_p/
\alpha\in K$
d'où $\Q_p(\zeta_p,\alpha)\subset K$.

\subsection*{2)}
On assume pour l'instant que $\Q_p(\zeta_p)/\Q_p$ est 
galoisienne de degré 
$p-1$ et que $\Q_p(\alpha)/\Q_p$ est de degré $p$. On remarque 
alors que
$\Q_p(\alpha)/\Q_p$ et $\Q_p(\zeta_p)/\Q_p$ sont linéairement
disjointes car $p\wedge p-1=1$. En particulier 
\begin{align*}
	[K:\Q_p]&=[K:\Q_p(\zeta_p)][\Q_p(\zeta_p):\Q_p]\\
		&=[\Q_p(\alpha):\Q_p][\Q_p(\zeta_p):\Q_p]\\
		&=p(p-1)
\end{align*}
et $H=Gal(K/\Q_p(\zeta_p))$ est normal dans $G$ car 
$\Q_p(\zeta_p)/\Q_p$ est galoisienne. Enfin $H$ est d'indice 
$|G/H|=|Gal(\Q_p(\zeta_p)/\Q_p)|=p-1$ qui est le résultat voulu. 

On prouve maintenant les assertions. On note $X^p-1=(X-1)\phi_p(X)$
et on a 
\[(X+1)^p-1=X(\sum_{k=1}^{p-1}\begin{pmatrix}p\\ k+1\end{pmatrix}
X^k +p)=X\phi_p(X+1)\]
d'où on déduit que $\phi_p(X+1)$ est $p\Z_p$-Eisenstein donc
irréductible dans $\Z_p[X]$. Maintenant $\Q_p(\zeta_p)$ est le
corps de décomposition de $\phi_p(X)$ sur $\Q_p$ d'où 
$\Q_p(\zeta_p)/\Q_p$ est galoisienne de degré
$[\Q_p(\zeta_p):\Q_p]=\deg(\phi_p)=p-1$. De même $X^p-p$ est
$p\Z_p$-Eisenstein de degré $p$ et $\Q_p(\alpha)$ en est un
corps de rupture d'où le résultat.

\subsection*{3)}
Dans la partie $2)$ on a montré que $\phi_p(X+1)$ est 
$p\Z_p$-Eisenstein. On en déduit directement que 
$\Q_p(\zeta_p-1)/\Q_p$ est totalement ramifiée et que $\zeta_p-1$
,qui est une racine de $\phi_p(X+1)$, en est une uniformisante.
De la même manière, $\Q_p(\alpha)/\Q_p$ est totalement ramifiée
et $\alpha$ en est une uniformisante. Via 
\[e_{K/\Q_p}=e_{K/\Q_p(\alpha)}e_{\Q_p(\alpha)/\Q_p}\]
on obtient $p\mid e_{K/\Q_p}$ et via 
\[e_{K/\Q_p}=e_{K/\Q_p(\zeta_p)}e_{\Q_p(\zeta_p)/\Q_p}\]
on obtient $p-1\mid e_{K/\Q_p}$ ce qui prouve que $K/\Q_p$ est
totalement ramifiée car $e_{K/\Q_p}\leq[K:\Q_p]=p(p-1)$. Enfin 
\[v_p(\frac{\zeta_p-1}{\alpha})=\frac{1}{p-1}-\frac{1}{p}=\frac{p-(p-1)}{p(p-1)}=\frac{1}{p(p-1)}\]
si on note $v_p$ la valuation sur $K$ qui étend la valuation 
$p$-adique normalisée. On a prouvé que $\frac{\zeta_p-1}{\alpha}$
est une uniformisante de $K/\Q_p$. 

\subsection*{4)}
Le groupe de Galois $G$ est formé des automorphismes 
\[\{s_{ij}\}_{i=1,\ldots,p-1;j=0,\ldots,p-1}\]
définis par $s_{ij}(\zeta_p)=\zeta_p^i$ et 
$s_{ij}(\alpha)=\alpha\zeta_p^j$. De la ramification totale de 
$K/\Q_p$ on déduit que $\Z_p[\lambda]=\Or_K$ et $G=G_0$ si l'on
pose $\lambda=\frac{\zeta_p-1}{\alpha}$. Soit maintenant $s_{ij}$
un élément de $G$. Pour $g\in G$ si $i_G(g)$ désigne le plus
grand entier $i$ tel que $g\in G_{i-1}$, on a
\[i_G(g)-1=e_{K/\Q_p}v_p(\frac{g\lambda}{\lambda} -1).\]
ou $i_G(g)-1=v_K(\frac{g\lambda}{\lambda}-1)$ si $v_K$ est 
normalisée. Maintenant on calcule
\begin{align*}
  \frac{s_{ij}\lambda}{\lambda}-1&=\frac{\zeta_p^i-1}{\alpha\zeta_p^j}.\frac{\alpha}{\zeta_p-1}-1\\
				 &=\frac{\zeta_p^i-1}{\zeta_p^j(\zeta_p-1)}-1\\
				 &=\zeta_p^{-j}\left(\sum_{k=0}^{k-1}\zeta_p^k\right)-1\\
\end{align*}
On remarque que $ \zeta_p=1\mod \m_K$ de sorte que  
\[\zeta_p^{-j}\left(\sum_{k=0}^{k-1}\zeta_p^k\right)-1=1.i-1\mod \m_K\]
ce qui fait sens car 
$\zeta_p^{-j}\left(\sum_{k=0}^{k-1}\zeta_p^k\right)-1$ est dans 
$\Or_K$. On obtient deux cas : d'abord si $i\ne 1$ alors 
pour tout $0\leq j\leq p-1$  on a 
\[i_G(s_{ij})-1=v_K(\frac{s_{ij}\lambda}{\lambda}-1)=0\]
d'où $i_G(s_{ij})=1$. Ensuite si $i=1$ alors on calcule pour
$1\leq j\leq p-1$,
\begin{align*}
v_K\left(\frac{s_{ij}(\lambda)}{\lambda}-1\right)&=v_K\left(\zeta_p^{-j}-1\right)\\
				      &=v_K\left(\zeta_p-1\right)\\
				      &=p(p-1).v_p\left(\zeta_p-1\right)\\
				      &=p
\end{align*}
d'où $i_G(s_{ij})-1=p$. On remarque que $\{s_{ij}|i=1,0\leq j\leq p-1\}=H$. En particulier, pour $k=-1,0$ on a 
$G_k=G$ puis pour $1\leq k\leq p$ on a $\Z/p\Z\simeq G_k=G_1=H$
et enfin pour $p<k$ on a $G_k=\{id_K\}$.

\section*{Exercice 2}
\subsection*{1)}
Pour tout élément $h$ de $H$, on note $(a_{ij}(h))_{i,j}=A(h)$.
Alors pour $1\leq j\leq n$ et $h_1,h_2\in H$ on a 
\[h_2v_j=\sum_{i=1}^na_{ij}(h_2)v_i\]
puis 
\begin{align*}
  h_1h_2v_j&=\sum_{i=1}^nh_1(a_{ij}(h_2)v_i)\\
	   &=\sum_{i=1}^nh_1(a_{ij}(h_2))h_1v_i\\
\end{align*}
d'où
\begin{align*}
  (v_j)_{j=1,\ldots,n}A(h_1h_2)&=(h_1h_2v_j)_{j=1,\ldots,n}\\
			       &=(h_1v_j)_{j=1,\ldots,n}h_1A(h_2)\\
			       &=(v_j)_{j=1,\ldots,n}A(h_1)h_1A(h_2)
\end{align*}
Puis par unicité de $A(h_1h_2)$ on obtient 
$A(h_1h_2)=A(h_1)h_1A(h_2)$.

\subsection*{2)}
On prouve $a)$ implique $b)$. Soit $(w_j)_j$ une base de $V$
dont les vecteurs sont invariants par $H$. Et soit $P$ la matrice
de changement de base de $(v_j)_j$ à $(w_j)_j$, on montre que $B=P$
convient. On a 
\begin{align*}
  (v_j)_j.P&=(w_j)_j\\
	   &=(h.w_j)_j\\
           &=h.((v_j)_j.P)\\
	   &=h.(\sum_ip_{ij}v_i)_j\\
	   &=(\sum_ihp_{ij}hv_i)_j\\
	   &=(hv_j)_jhP\\
	   &=(v_j)_jA(h)hP\\
\end{align*}
d'où $P=A(h)hP$ puis $A(h)=P.h(P)^{-1}$.

Maintenant si $A(h)=Bh(B)^{-1}$ pour $(b_{ij})_{ij}=B\in GL_n(\C)$
alors 
on pose $(w_j)_j=(v_j)_j.B$. C'est une base de $V$ par hypothèse
sur $B$ et on calcule de la même manière que précédemment
\begin{align*}
  h.(w_j)_j&=h.(\sum_i b_{ij}v_i)_j\\
	   &=(\sum_i h(b_{ij})h(v_i))_j\\
	   &=(v_j)_jA(h)h(B)\\
	   &=(v_j)_jB.h(B)^{-1}h(B)\\
	   &=(v_j)_jB\\
	   &=(w_j)_j
\end{align*}
d'où $(w_j)_j$ est une base formée de vecteurs invariants par
$h$ qui est le résultat voulu.

\subsection*{3)}
Soit $f\colon H\to GL_n(\C)$ un cocycle. On remarque que 
\[f(id)=f(id.id)=f(id)id(f(id))=f(id)^2.\] Comme $f$ est à valeurs
dans des matrices inversibles on obtient que \[f(id)=I_n.\] 
On munit maintenant $GL_n(\C)$ de la norme donnée par 
\[||(a_{ij})_{i,j}||:=\max_{i,j}|a_{ij}|_\C\] où $|.|_\C$ est
l'unique valeur absolue sur $\C$ étendant la valeur
absolue $p$-adique normalisée par $|p|_\C=1/p$. Alors la topologie
de $||.||$ coincide avec la topologie produit de $M_n(\C)$ et on
a 
\[B(I_n,1/p^2)\subsetneq\overline{B(I_n,1/p^2)}=1+p^2M_n(\Or_\C)\]
par définition si $B(a,r)$ désigne la boule ouverte centrée en
$a$ et de rayon $r$. Puis $f^{-1}(1+p^2M_n(\Or_C))$ contient 
$f^{-1}(B(I_n,1/p^2))$ qui est ouvert par continuité de $f$ et
non vide car il contient $id$ par la première remarque. Maintenant
une base de voisinage de $id$ est donnée par les groupes de galois 
$Gal(\bar K, M)$ tels que $M$ est de dimension finie sur $L$. Il
existe donc $F_1$ tel que 
\[id\in Gal(\bar K,F_1)\subset f^{-1}(B(I_n,1/p^2))\]
et $[F_1:L]<+\infty$. On pose alors $F$ la clôture galoisienne de
$F_1$ dans $\bar K$ puis $H'=Gal(\bar K, F)$, $H'$ est 
d'indice fini dans $H$ car $F/L$ est de dimension finie et on a
\[f(H')\subset f(Gal(\bar K,F_1))\subset f(f^{-1}(B(I_n,1/p^2)))\subset 1+p^2M_n(\Or_C)\]
qui est bien le résultat voulu.

\subsection*{4)}
\subsection*{4a)}
Par le même argument que dans 3) en remplaçant $1+p^2M_n(\Or_C)$
par $1+p^{m+2}M_n(\Or_\C)$ et $f$ par $f|_{H'}$, on remarque que 
$f|_{H'}$ est bien un cocycle et on obtient $H_1=Gal(\bar K,F_1)$
un sous-groupe d'indice fini de $H'$ tel que \[f|_{H'}(H_1)\subset
1+p^{m+2}M_n(\Or_C).\]
On pose $E$ la clôture galoisienne
de $F_1$ dans $\bar K$. Alors $N=Gal(\bar K,E)$ est d'indice fini
dans $H'$ car $(H':H_1)=[E:F]=[E:F_1][F_1:F]<+\infty$. En
plus \[f|_{H'}(N)=f(N)\subset f(H_1)\subset 1+p^{m+2}M_n(\Or_\C)\]
qui est le premier résultat voulu.

Pour le second on remarque que $E/F$ est presque étale. Pour le
voir on remarque que $F/K$ n'est pas de conducteur fini, par le
théorème de Coates-Greenberg $F/K$ est alors profondément ramifiée
d'où $E/F$ est presque étale car $[E:F]<+\infty$ par construction.
Si $F/K$ était de conducteur fini
on aurait $L\subset F\subset \bar K^{(\nu)}$ pour un $\nu\geq 0$
d'où $L/K$ serait de conducteur fini ce qui contredit le fait
que $L/K$ est profondément ramifiée.

En particulier, $Tr_{E/F} \colon \m_E\to \m_F$ est surjective et 
on peut prendre $y\in \m_E$ tel que 
\[p=Tr_{E/F}(y)=\sum_{\sigma\in Gal(E/F)}\sigma(y)\]
qui est le résultat voulu.

\subsection*{4b)}
Si l'on note $f(\hat\sigma)=1+p^mM_\sigma$ alors on a 
\begin{align*}
  B_m&=\frac{1}{p}\left(\sum_{\sigma\in Gal(E/F)}f(\hat\sigma)\hat\sigma(y)\right)\\
     &=\frac{1}{p}\left(\sum_{\sigma\in Gal(E/F)}(1+p^mM_\sigma)\hat\sigma(y)\right)\\
     &=\frac{1}{p}\left(\sum_{\sigma\in Gal(E/F)}\hat\sigma(y)+p^m\left(\sum_{\sigma\in Gal(E/F)}M_\sigma\hat\sigma(y)\right)\right)\\
     &=1+p^{m-1}\left(\sum_{\sigma\in Gal(E/F)}M_\sigma\hat\sigma(y)\right)
\end{align*}
et le résultat car $\hat\sigma(y)=\sigma(y)$ est dans $\Or_E$.

\subsection*{4c)}
On prouve que pour tout $h\in H'$ on a 
\[f(h)h(B_m)\equiv B_m\mod p^{m+1}.\]
Soit $h\in H'$, pour tout $\sigma\in Gal(E/F)$, il existe
$\sigma'\in Gal(E/F)$ et $h'\in N$ tel que 
$h\hat\sigma=\hat\sigma'h'$.
En effet si $\pi\colon H'\to Gal(E/F)$ est la projection canonique
alors on pose $\sigma'=\pi(h\hat\sigma)$ d'où 
\[(h\hat\sigma)^{-1}\hat\sigma'=h'^{-1}\in \ker(\pi)\]
par construction puis l'assertion car $\ker(\pi)=N$. Maintenant
on calcule
\begin{align*}
  f(h)h(B_m)&=\frac{1}{p}f(h)h.\sum_{\sigma\in Gal(E/F)}f(\hat\sigma)\hat\sigma(y)\\
	    &=\frac{1}{p}\sum_{\sigma\in Gal(E/F)}f(h\hat\sigma)h\hat\sigma(y)\\
	    &=\frac{1}{p}\sum_{\sigma\in Gal(E/F)}f(\hat\sigma'h')\hat\sigma'h'(y)\\
	    &=\frac{1}{p}\sum_{\sigma\in Gal(E/F)}f(\hat\sigma'h')\hat\sigma'(y)\\
\end{align*}
puis en notant $f(h')=I_n+p^{m+2}M_{h'}$ on trouve
\begin{align*}
  f(h)h(B_m)-B_m&=\frac{1}{p}(\sum_{\sigma\in Gal(E/F)}(f(\hat\sigma')\hat\sigma'f(h')).(\hat\sigma')(y)-f(\hat\sigma').(\hat\sigma')(y))\\
		&=\frac{1}{p}(\sum_{\sigma\in Gal(E/F)}\left(f(\hat\sigma')(\hat\sigma'(y))(\hat\sigma'f(h')-I_n)\right)\\
		&=(\sum_{\sigma\in Gal(E/F)}\left(f(\hat\sigma')(\hat\sigma'(y))(p^{m+1}\hat\sigma' M_{h'})\right))\\
		&=p^{m+1}(\sum_{\sigma\in Gal(E/F)}\left(f(\hat\sigma')(\hat\sigma'(y))(\hat\sigma'M_{h'})\right))\\
\end{align*}
où la première égalité provient du fait que $h'$ fixe $y$. 
Maintenant le terme de droite est dans $M_n(\Or_\C)$ d'où le
résultat.

\subsection*{5)}
Soit $f\colon H\to GL_n(\C)$ un cocycle. On construit un 
sous-groupe distingué $H'$ de $H$, des
cocycles $(f_i)_{i\geq 2}$ et des matrices $(B_i)_{i\geq 2}$ tels
que pour $2\leq i$ on ait
\begin{itemize}
  \item $f_i(H')\subset 1+p^mM_n(\Or_C)$,
  \item $B_i\in 1+p^{i-1}M_n(\Or_C)$,
  \item Pour tout $h\in H$, $f_{i+1}(h):=B_{i+1}^{-1}f_i(h)h(B_{i+1})^{-1}$.
  \item Pour tout $h'\in H'$, $f_{i+1}(h')\equiv 1\mod p^{i+1}$.
\end{itemize}
Alors la suite $(\prod_{i=2}^m B_i)_{m\geq 2}$ converge dans 
$1+pM_n(\Or_\C)\subset GL_n(\Or_\C)$ en un $B$ vérifiant pour tout
$h\in H'$ l'identité
\[f(h)\equiv Bh(B)^{-1}\]
qui est le résultat voulu.

\subsection*{Preuve de la convergence}
Sous les hypothèses précédentes on montre que la suite 
$(A_m)_{m\geq 2}:=(\prod_{i=2}^mB_i)_{m\geq 2}$ converge dans
$1+pM_n(\Or_\C)$
muni de la norme présentée en 3). On remarque d'abord que 
$1+pM_n(\Or_\C)=\overline{B(I_n,1/p)}$ est fermé dans 
$M_n(\Or_\C)$ qui est complet pour $||.||$ d'où est lui même
complet. Enfin par l'annexe $0.1$, $A_m$ est dans 
$1+pM_n(\Or_\C)$ d'où sa limite, si elle existe, est dans 
$1+pM_n(\Or_\C)$. Il suffit donc de montrer que $(A_m)_m$ est de
Cauchy.
Soit $\epsilon>0$ et $m\geq 2$ tel que $(1/p)^m<\epsilon$. Alors
si $u\geq v\geq m$ on a 
\begin{align*}
||A_u-A_v||&=||\prod_{i=2}^vB_i(\prod_{k=v+1}^uB_k-1)||\\
	   &\leq ||\prod_{i=2}^vB_i||.||\prod_{k=v+1}^uB_k-1||\\
	   &\leq (1/p)^{v+1}\leq (1/p)^m<\epsilon
\end{align*}
et l'inégalité des normes étant une majoration naive utilisant
l'inégalité ultramétrique pour $|.|_\C$. D'où $(A_m)_m$ converge
en un $B$ et de $B\equiv A_m \mod p^m$ pour tout $h\in H'$
l'identité
\[f_{m+1}(h)\equiv 1\mod p^{m+1}\]
montre que $f(h)\equiv Bh(B)^{-1}\mod p^m$ pour tout $m\geq 2$
d'où $A_mh(A_m)^{-1}$ converge vers $f(h)$
et le résultat en découle.

\subsection*{Initialisation}
On pose $f_1=f$, par
la question $3)$ il existe $H'$ un sous-groupe distingué de $H$
d'indice fini tel que $f(H')\subset 1+p^2M_n(\Or_\C)$ et on peut
construire $B_2$ telle que 
\[f(h)\equiv B_2h(B_2)^{-1}\mod p^3.\]
On pose alors $f_2:h\mapsto B_2^{-1}f_1(h)h(B_2)$. On prouve 
maintenant que $f_2$ est un cocycle. Pour tout $h_1,h_2\in H$
on a 
\begin{align*}
  f_2(h_1h_2)&=B_2^{-1}f(h_1h_2)h_1h_2(B_2)\\
	     &=B_2^{-1}f(h_1).h_1f(h_2)(h_1h_2)(B_2)\\
	     &=B_2^{-1}f(h_1)h_1(B_2)h_1(B_2)^{-1}h_1f(h_2)(h_1h_2)(B_2)\\
	     &=f_2(h_1)h_1(B_2^{-1}f(h_2)h_2(B_2)\\
	     &=f_2(h_1)h_1f_2(h_2)
\end{align*}
où la quatrième égalité est dûe au fait que $h(P^{-1})=h(P)^{-1}$
pour toute matrice $P\in GL_n(\Or_C)$, d'où $f_2$ vérifie la 
condition de cocycle. En plus $f_2$ est 
continue car $h\mapsto h.B_2$ et $h\mapsto B_2f(h)$ sont continues
d'où si on écrit $f_2$ comme la composée
\[H\to GL_n(\Or_C)^2\to GL_n(\Or_C)\]
où la première application est l'application produit et la deuxième
la multiplication on obtient sa continuité.
\subsection*{Hérédité}
On suppose maintenant $(f_i)_{i=2,\ldots, m}$ et 
$(B_i)_{i=2,\ldots,m}$ construits pour $m\geq 2$. Par les questions
$4a)$ et $4b)$ on trouve $B_{m+1}$ vérifiant les hypothèses 
voulues. On pose ensuite 
$f_{m+1}:h\mapsto B_{m+1}^{-1}f_m(h)h(B_{m+1})$. Alors par la
même preuve que pour l'initialisation
$f_{m+1}$ est un cocycle et donc vérifie nos hypothèses ce qui
conclut la preuve.

\subsection*{6)}
On note $(v_j)_{j=1,\ldots, n}$ une base de $V$ quelconque et 
$f\colon h\mapsto A(h)$ comme dans la partie I. Alors $f$ vérifie
la condition de cocycle par $1)$. On assume la continuité. Alors
$f$ est un cocycle.

Maintenant on applique la partie $2$ question $5)$ pour obtenir
$H'$ distingué et d'indice fini dans $H$ et $B\in GL_n(\Or_\C)$
vérifiant pour tout $h\in H'$ 
\[f(h)=Bh(B)^{-1}.\]
Si l'on pose maintenant $(w_j)_j:=(v_j)_j.B$ alors par la question
$2)$, $(w_j)_j$ à ses composantes invariantes par l'action de $H'$.

\subsection*{7)}
Pour chaque $h\in H$, comme $F/K$ est profondément ramifiée par
le même argument que dans 4a), par le théorème d'Ax-Sen-Tate il
suffit de 
montrer que les coefficients de $C(h)$ sont invariants par $H'$.
Autrement dit que pour tout $h'\in H'$ on ait $h'C(h)=C(h)$. 
Soit donc $h\in H$ et $h'\in H'$. Comme $H'$ est distingué dans
$H$, il existe $h''\in H'$ tel que $h'.h=h.h''$. Maintenant
on remarque que $C(h')=I_n$ d'où $C(h'h)=h'C(h)$. Enfin,
$C(h'.h)$ est définie par 
\[(h'.hw_j)_j=(w_j)_jC(h'h)\]
et on a 
\begin{align*}
  (h'.hw_j)_j&=(h.h''w_j)_j\\
	     &=(hw_j)_j\\
	     &=(w_j)_jC(h)
\end{align*}
d'où par unicité $h'C(h)=C(h'h)=C(h)$ et le résultat.

\subsection*{8)}
On considère comme dans $6)$, $(w_j)_j$ une base $H'$-invariante 
de $V$. Alors par $7)$ le cocycle
\begin{align*}
  C(\_)\colon &H\to GL_n(\C)\\
\end{align*}
vérifie $C(H)\subset GL_n(\hat F)$. On considère maintenant 
le $\widehat F$-espace vectoriel $V'=\oplus_{j=1}^nw_j\widehat F$
et on remarque que comme $H'$ fixe $\widehat F$, $H'$ agit
trivialement sur $V'$. En particulier, l'action de $H$ restreinte
à $V'$ se factorise en une action de $H/H'$ qui est fini et agit
sur $\widehat F$ via l'action de $Gal(F/L)$ sur 
$F\otimes_L \widehat L\simeq \widehat F$ où la flèche est un
isomorphisme car $F/L$ est séparable. En particulier on obtient
$H/H'\to Aut(\widehat F)$ une action semi-linéaire, via la 
semi-linéarité de l'action de $H$, d'où par Hilbert 90 on obtient
une base de $V'$, $(u_j)_j$, qui est $H/H'$-invariante et donc à
fortiori $H$-invariante. Comme $(w_j)_j$ est dans $V'$, la famille
$(u_j)_j$ est une $\C$-base de $V$ qui est $H$-invariante ce qui
conclut la preuve.


\section*{Annexe}
\subsection{$1+p^mM_n(\Or_\C)$ est un sous-monoide de $GL_n(\Or_\C)$}
On montre d'abord que $1+p^mM_n(\Or_\C)$ est un sous-ensemble
de $GL_n(\Or_\C)$. Soit $A\in 1+p^mM_n(\Or_\C)=1+M_n(p^m\Or_C)$,
l'application déterminant est polynomiale en les coefficients d'où
$\det(A)\in 1+p^m\Or_\C\subset \Or_\C^\times$ et $A$ est 
inversible dans $GL_n(\Or_\C)$.

Maintenant si $A,B\in 1+p^mM_n(\Or_\C)$ alors 
\begin{align*}
  A.B&=(1+p^mA_1)(1+p^mB_1)\\
     &=1+p^m(A_1+B_1+p^mA_1B_1)\in 1+p^mM_n(\Or_\C)
\end{align*}
d'où la stabilité par produit. Enfin $I_n$ est clairement dans
$1+p^mM_n(\Or_\C)$.

\subsection{Continuité de $h\mapsto A(h)$}
On le montre uniquement dans le cas où $A(H)\subset 1+pM_n(\Or_C)$.
Pour voir la continuité on prend
$h_1,h_2\in H$ et $1>r>0$ tel que $A(h_1)\in B(A(h_2),r)$. Comme la
norme est invariante par $K$-automorphisme on a 
\begin{align*}
  ||A(h_1)-A(h_2)||&=||h_1^{-1}A(h_2)-h_1^{-1}A(h_1)||\\
		   &=||A(h_1^{-1})^{-1}A(h_1^{-1}h_2)-A(h_1^{-1})^{-1}||\\
		   &=||A(h_1^{-1}h_2)-1||<r\\
\end{align*}
car $A(h_1^{-1})\in 1+pM_n(\Or_\C)$ d'où $||A(h_1^{-1})||=1$.
Alors $B(I_n,r)$ contient $A(h_1^{-1}h_2)$ puis
\[id,h_1^{-1}h_2\in f^{-1}(B(I_n,r))\]
d'où il existe $E/L$ une extension finie telle que 
$h_1^{-1}h_2\in Gal(\bar K,E)$.

\end{document}


