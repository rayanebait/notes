\documentclass[a4paper,12pt]{article}
\usepackage{amsmath,  amsthm,enumerate}
\usepackage{csquotes}
\usepackage[provide=*,french]{babel}
\usepackage[dvipsnames]{xcolor}
\usepackage{quiver, tikz}

%symbole caligraphique
\usepackage{mathrsfs}

%hyperliens
\usepackage{hyperref}

%pseudo-code
\usepackage{algpseudocode}
\usepackage{algorithm}
\makeatletter
  \renewcommand{\ALG@name}{Algorithme}
  \makeatother
\usepackage{fancyhdr}

%bibliographie
\usepackage[
backend=biber,
style=alphabetic,
sorting=ynt
]{biblatex}

\addbibresource{bib.bib}


\definecolor{wgrey}{RGB}{148, 38, 55}

\setlength\parindent{24pt}

\newcommand{\Z}{\mathbb{Z}}
\newcommand{\R}{\mathbb{R}}
\newcommand{\rel}{\omathcal{R}}
\newcommand{\Q}{\mathbb{Q}}
\newcommand{\C}{\mathbb{C}}
\newcommand{\N}{\mathbb{N}}
\newcommand{\K}{\mathbb{K}}
\newcommand{\A}{\mathbb{A}}
\newcommand{\B}{\mathcal{B}}
\newcommand{\Or}{\mathcal{O}}
\newcommand{\F}{\mathbb F}
\newcommand{\m}{\mathfrak m}
\renewcommand{\b}{\mathfrak b}
\renewcommand{\a}{\mathfrak a}
\newcommand{\p}{\mathfrak p}
\newcommand{\I}{\mathfrak I}
\newcommand{\Hom}{\textrm{Hom}}
\newcommand{\disc}{\textrm{disc}}
\newcommand{\Pic}{\textrm{Pic}}
\newcommand{\End}{\textrm{End}}
\newcommand{\Spec}{\textrm{Spec}}
\newcommand{\Frac}{\textrm{Frac}}

\newcommand{\cL}{\mathscr{L}}
\newcommand{\G}{\mathscr{G}}
\newcommand{\D}{\mathscr{D}}
\newcommand{\E}{\mathscr{E}}

\theoremstyle{plain}
\newtheorem{thm}{Théoreme}
\newtheorem{lem}{Lemme}
\newtheorem{prop}{Proposition}
\newtheorem{cor}{Corollaire}
\newtheorem{heur}{Heuristique}
\newtheorem{rem}{Remarque}
\newtheorem{rembis}{Remarque}
\newtheorem{note}{Note}

\theoremstyle{definition}
\newtheorem{conj}{Conjecture}
\newtheorem*{eq}{Équivalences}
\newtheorem{prob}{Problème}
\newtheorem{quest}{Question}
\newtheorem{prot}{Protocole}
\newtheorem{algo}{Algorithme}
\newtheorem{defn}{Définition}
\newtheorem{defnbis}{Définition}
\newtheorem{ex}{Exemple}
\newtheorem{exo}{Exercices}

\theoremstyle{remark}

\definecolor{wgrey}{RGB}{148, 38, 55}
\definecolor{wgreen}{RGB}{100, 200,0} 
\hypersetup{
    colorlinks=true,
    linkcolor=wgreen,
    urlcolor=wgrey,
    filecolor=wgrey
}

\title{Groupes de galois de corps locaux}
\date{}

\begin{document}
\maketitle

Je veux étant donné un corps local $K$ décrire les groupes
qui apparaissent dans
% https://q.uiver.app/#q=WzAsNCxbMCwxLCJLXnt0cn0iXSxbMCwyLCJLXnt1bn0iXSxbMCwwLCJcXGJhciBLIl0sWzAsMywiSyJdLFszLDEsIlxcd2lkZWhhdCBcXFoiLDAseyJjdXJ2ZSI6LTJ9XSxbMSwwLCJcXHByb2Rfe2xcXG5lIHB9IFxcWl9cXGVsbCIsMix7ImN1cnZlIjozfV0sWzAsMiwiUF9LIiwwLHsiY3VydmUiOjJ9XSxbMSwyLCJJX0siLDAseyJjdXJ2ZSI6LTN9XV0=
\[\begin{tikzcd}
	{\bar K} \\
	{K^{tr}} \\
	{K^{un}} \\
	K
	\arrow["{P_K}", curve={height=12pt}, from=2-1, to=1-1]
	\arrow["{I_K}", curve={height=-18pt}, from=3-1, to=1-1]
	\arrow["{\prod_{l\ne p} \Z_\ell}"', curve={height=18pt}, from=3-1, to=2-1]
	\arrow["{\widehat \Z}", curve={height=-12pt}, from=4-1, to=3-1]
\end{tikzcd}\]

\section{$Gal(K^{un}/K)\simeq \widehat \Z$}
Y s'agit juste de voir que 
\[Gal(K^{un}/K)\simeq Gal(\bar\F_q/\F_q)\]
et que le deuxième vérifie
\[Gal(\overline \F_q/\F_q)\simeq \varprojlim_{n}Gal(\F_{q^n}/\F_q)\simeq \varprojlim_n\Z/n\Z\]
via Galois.

\section{$Gal(K^{tr}/K^{un})\simeq\prod_{\ell\ne p}\Z_\ell$}
Y suffit de se rappeler que y'a une unique extension
modérément totalement ramifiée de degré $e$ pour 
$e\wedge p=1$. Qu'en plus elle est de la forme 
$X^e-\pi_K$ et que dans $K^{un}$ on a $\mu_e$.

\section{$Gal(K^{tr}/K)\simeq <Fr_K,\tau_K|Fr_K\tau_KFr_K^{-1}=\tau_K^q>$}
Si on regarde $K(\zeta_n)[X]/(X^n-\pi_K)/K$ elle est de
degré au plus $n\varphi(n)$ (y'a tjr le cas bizarre où
$\phi_n$ est split mod $p$) et engendrée par les 
automorphismes 
\[Fr_K(\zeta_n)=\zeta_n^p\]
et 
\[\tau_K(\pi_n)=\pi_n\zeta_n\]
y s'agit ensuite de remarquer qu'on peut choisir les
$\zeta_n$ dans $\varprojlim_{n\wedge p=1}\mu_n$ et
les $\pi_n$ dans 
$\varprojlim_{x\mapsto x^n;n\wedge p=1} \m_{\bar K}$.
On a obtenu $(\pi_n)_n$ et $(\zeta_n)_n$ telles que
$\zeta_n^d=\zeta_{n/d}$ et $\pi_n^d=\pi_{n/d}$.
Ça fournit $\tau_K$ et $Fr_K$ sur $K^{tr}$. La relation
est par calcul direct (oui !).






\end{document}
