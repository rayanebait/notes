\documentclass[a4paper,12pt]{article}
\usepackage{amsmath,  amsthm,enumerate}
\usepackage{csquotes}
\usepackage[provide=*,french]{babel}
\usepackage[dvipsnames]{xcolor}
\usepackage{quiver, tikz}

%symbole caligraphique
\usepackage{mathrsfs}

%hyperliens
\usepackage{hyperref}

%pseudo-code
\usepackage{algpseudocode}
\usepackage{algorithm}
\makeatletter
  \renewcommand{\ALG@name}{Algorithme}
  \makeatother
\usepackage{fancyhdr}





\definecolor{wgrey}{RGB}{148, 38, 55}

\setlength\parindent{24pt}

\newcommand{\Z}{\mathbb{Z}}
\newcommand{\R}{\mathbb{R}}
\newcommand{\rel}{\omathcal{R}}
\newcommand{\Q}{\mathbb{Q}}
\newcommand{\C}{\mathbb{C}}
\newcommand{\N}{\mathbb{N}}
\newcommand{\K}{\mathbb{K}}
\newcommand{\A}{\mathbb{A}}
\newcommand{\B}{\mathcal{B}}
\newcommand{\Or}{\mathcal{O}}
\newcommand{\F}{\mathbb F}
\newcommand{\m}{\mathfrak m}
\renewcommand{\b}{\mathfrak b}
\renewcommand{\a}{\mathfrak a}
\newcommand{\p}{\mathfrak p}
\newcommand{\I}{\mathfrak I}
\newcommand{\Hom}{\textrm{Hom}}
\newcommand{\disc}{\textrm{disc}}
\newcommand{\Pic}{\textrm{Pic}}
\newcommand{\End}{\textrm{End}}
\newcommand{\Spec}{\textrm{Spec}}
\newcommand{\Frac}{\textrm{Frac}}

\newcommand{\cL}{\mathscr{L}}
\newcommand{\G}{\mathscr{G}}
\newcommand{\D}{\mathscr{D}}
\newcommand{\E}{\mathscr{E}}

\theoremstyle{plain}
\newtheorem{thm}{Théoreme}
\newtheorem{lem}{Lemme}
\newtheorem{prop}{Proposition}
\newtheorem{cor}{Corollaire}
\newtheorem{heur}{Heuristique}
\newtheorem{rem}{Remarque}
\newtheorem{rembis}{Remarque}
\newtheorem{note}{Note}

\theoremstyle{definition}
\newtheorem{conj}{Conjecture}
\newtheorem*{eq}{Équivalences}
\newtheorem{prob}{Problème}
\newtheorem{quest}{Question}
\newtheorem{prot}{Protocole}
\newtheorem{algo}{Algorithme}
\newtheorem{defn}{Définition}
\newtheorem{defnbis}{Définition}
\newtheorem{ex}{Exemple}
\newtheorem{exo}{Exercices}

\theoremstyle{remark}

\definecolor{wgrey}{RGB}{148, 38, 55}
\definecolor{wgreen}{RGB}{100, 200,0} 
\hypersetup{
    colorlinks=true,
    linkcolor=wgreen,
    urlcolor=wgrey,
    filecolor=wgrey
}

\title{$\underline{Rep}_{\Q_p,B}(G_K)$}
\date{}

\begin{document}
\maketitle

\section{Trucs sur les modules}
Étant donné $M$ un $R$-module et 
\[\colon R^{(I)}\to M\to 0\]
avec $\pi$ la flèche de gauche on peut regarder
$r\in \pi^{-1}(m)$ pour $m\in M$. Puis $|supp(r)|$
est fini par déf. D'où on peut regarder une relation
de taille minimale pour $m$. Par déf si 
$\sum_{i\in I_n} a_i m_i=m$ pour $n=|I_n|$ minimal
et un des $a_i$ est inversible bah $(\m_i)_i$ est libre ! 
Sinon le $i=i_0$ t.q $a_{i_0}$ est inversible, $m_{i_0}$,
se réécrit en fonction des autres !!

\begin{rem}
  Donc quand y'a un corps en jeu c'est direct une famille
  libre.
\end{rem}
Par contre on voit aussi que passer au corps de fraction peut
réduire la taille des relations logique.

\subsection{Dans un sous-module}
On peut pareil demander pour $N\leq M$ un élément de $N$
d'écriture de taille minimale non nul. Vu que $N$ contient pas
forcément de générateur $m\in M$ c'est intéressant, pour le 
calcul de noyau par exemple.

\subsection{Applications aux produits tensoriels}
On peut donc prendre des écritures de taille minimales
$\sum d_i\otimes c_i$. Ça permet par exemple de faire
le trick d'Artin $gx-x$!


\section{$D_B(\_)$ et $D_C(\_)$}
Pour prouver que \[D_B(\_)\otimes_{E}B\to V\otimes_{\Q_p}B\]
est injective, à noter que $(d_i\otimes 1)_i$ est lin indép
sur $C$ !

\section{À faire}
Comprendre les représentations $\Z_p(1), \Z_p(i):=\Z_p(1)^{\otimes i}$
et leur lien avec les twists de faisceaux. Ducoup comprendre les 
représentation de dimension $1$/abéliennes.

Voir quelques représentations de dimension $2$, genre courbes 
elliptiques. Juste se donner une idée des méthodes.

Capter Ax-Sen-Tate qui est un des plus importants, appliquer à 
la correspondance de galois entre $E$ et $E^\flat$.

Capter Hilbert $90$ et ses applications, accessoirement 





\section{Représentations $p$-adiques}


\end{document}
