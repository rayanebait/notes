\documentclass[a4paper,12pt]{article}
\usepackage{amsmath,  amsthm,enumerate}
\usepackage{csquotes}
\usepackage[provide=*,french]{babel}
\usepackage[dvipsnames]{xcolor}
\usepackage{quiver, tikz}

%symbole caligraphique
\usepackage{mathrsfs}

%hyperliens
\usepackage{hyperref}

%pseudo-code
\usepackage{algpseudocode}
\usepackage{algorithm}
\makeatletter
  \renewcommand{\ALG@name}{Algorithme}
  \makeatother
\usepackage{fancyhdr}





\definecolor{wgrey}{RGB}{148, 38, 55}

\setlength\parindent{24pt}

\newcommand{\Z}{\mathbb{Z}}
\newcommand{\R}{\mathbb{R}}
\newcommand{\rel}{\omathcal{R}}
\newcommand{\Q}{\mathbb{Q}}
\newcommand{\C}{\mathbb{C}}
\newcommand{\N}{\mathbb{N}}
\newcommand{\K}{\mathbb{K}}
\newcommand{\A}{\mathbb{A}}
\newcommand{\B}{\mathcal{B}}
\newcommand{\Or}{\mathcal{O}}
\newcommand{\F}{\mathbb F}
\newcommand{\m}{\mathfrak m}
\renewcommand{\b}{\mathfrak b}
\renewcommand{\a}{\mathfrak a}
\newcommand{\p}{\mathfrak p}
\newcommand{\I}{\mathfrak I}
\newcommand{\Hom}{\textrm{Hom}}
\newcommand{\disc}{\textrm{disc}}
\newcommand{\Pic}{\textrm{Pic}}
\newcommand{\End}{\textrm{End}}
\newcommand{\Spec}{\textrm{Spec}}
\newcommand{\Frac}{\textrm{Frac}}

\newcommand{\cL}{\mathscr{L}}
\newcommand{\G}{\mathscr{G}}
\newcommand{\D}{\mathscr{D}}
\newcommand{\E}{\mathscr{E}}

\theoremstyle{plain}
\newtheorem{thm}{Théoreme}
\newtheorem{lem}{Lemme}
\newtheorem{prop}{Proposition}
\newtheorem{cor}{Corollaire}
\newtheorem{heur}{Heuristique}
\newtheorem{rem}{Remarque}
\newtheorem{rembis}{Remarque}
\newtheorem{note}{Note}

\theoremstyle{definition}
\newtheorem{conj}{Conjecture}
\newtheorem*{eq}{Équivalences}
\newtheorem{prob}{Problème}
\newtheorem{quest}{Question}
\newtheorem{prot}{Protocole}
\newtheorem{algo}{Algorithme}
\newtheorem{defn}{Définition}
\newtheorem{defnbis}{Définition}
\newtheorem{ex}{Exemple}
\newtheorem{exo}{Exercices}

\theoremstyle{remark}

\definecolor{wgrey}{RGB}{148, 38, 55}
\definecolor{wgreen}{RGB}{100, 200,0} 
\hypersetup{
    colorlinks=true,
    linkcolor=wgreen,
    urlcolor=wgrey,
    filecolor=wgrey
}

\title{Vecteurs de Witt}
\date{}

\begin{document}
\maketitle

\section{Définitions}
Pour $p$ un premier et $A$ un anneau commutatif
unitaire déf 
\[w_n(X_0,\ldots,X_n):=X_0^{p^n}+pX_1^{p^{n-1}}+\ldots+p^nX_n\]
dans $R=\Z[X_0,\ldots,X_n,\ldots]$.

On peut prouver que pour tout
$F(X,Y)\in \Z[X,Y]$ il existe 
\[(f_i(\bar X,\bar Y))_{i\in \N}\in (R\times R)^\N\] tel que
\[w_n(f_0(\bar X,\bar Y),\ldots, f_n(\bar X,\bar Y))=F(w_n(\bar X),w_n(\bar Y))\]
en particulier pour $F(X,Y)=X+Y$ ou $X.Y$. Je note 
$\Phi(F)=(f_i)_{i\in \N}$ les polynômes associés à $F$. 
Les $p,n$-vecteurs de Witt maintenant c'est
\[W_n(A):=A^n\]
muni de l'addition \[x+_{W(A)}y:=(\Phi(X+Y)_i(x,y))_{i\in n}\] et
\[x._{W(A)}y:=(\Phi(X.Y)_i(x,y))_{i\in n}\]. Puis les
$p$-vecteurs de Witt c'est $W(A):=W_{\omega}(A)$.

\section{Calcul}
En pratique pour le passage à la caractéristique $p$ faut calculer
d'abord dans $\Z$ les $h_n$ puis réduire modulo $p$. 

En fait c'est plus compliqué je l'explique dans \ref{calcul}.

\subsection{Preuve de l'identité : congruences}
L'unicité et la construction de $\Phi(F(X,Y))$ est directe par 
récurrence. A priori 
\[\Phi(F(X,Y))\in \Z[p^{-1}][\overline X]^\N\]
et on veut 
\[\Phi(F(X,Y))\in \Z[\overline X]^\N\]
et pour ça faut montrer par récurrence que pour tout $n\geq 0$
\[F(w_n(\bar X),w_n(\bar Y))=w_n(\Phi(F(X,Y)))\mod p^n\]
L'idée c'est déjà que
\begin{equation}
w_{n-1}\circ\varphi(x)=w_n(x)\mod p^n
\end{equation}
puis que pour $f\in (X_0,\ldots, X_n)$ on a 
\begin{equation}
f^{p^m}\equiv f^{p^{m-1}}\circ\varphi\mod p^m
\end{equation}
d'où si 
\begin{equation}
F(w_{n-1}(\bar X),w_{n-1}(\bar Y))= w_{n-1}(\Phi(F(X,Y)))
\end{equation}
on obtient 
\begin{align*}
  F(w_n(\bar X),w_n(\bar Y))&=F(w_{n-1}\circ\varphi(\bar X),w_{n-1}\circ\varphi(\bar Y))\mod p^n\\
                            &=w_{n-1}(h_0\circ\varphi,\ldots, h_{n-1}\circ\varphi)\mod p^n\\
                            &=w_{n-1}(h_0^p,h_1^p,\ldots, h_{n-1}^p)\mod p^n\\
                            &=w_n(h_0,\ldots, h_n)\mod p^n
\end{align*}
où la première égalité est dûe à $(3)$ la deuxième à $(1)$ et 
la troisième à $(2)$. La dernière c'est à nouveau $(1)$

\subsection{Preuve de (2)}
L'équation $(2)$ est obtenue simplement parce que si 
$A=B\mod p$ alors 
\[A^{p^m}-B^{p^m}=(A^{p^{m-1}}-B^{p^{m-1}})\left(\sum_{i=0}^{p-1} A^{p^{m-1}i}B^{p^{m-1}(p-1-i)}\right)\]
sauf que par récurrence le premier terme est divisible par
$p^{m-1}$ et le deuxième vaut $0$ modulo $p$ via la congruence
$A=B\mod p$ d'où le résultat.


\section{$A$ parfait de caractéristique $p>0$}
Dans ces conditions $W(A)$ est muni de la topologie $p$-adique et
$p^n.x=V^n\varphi^n(x)$. Ça se voit au moment du calcul si 
$x\in V^n(W(A))$ au moment du calcul de $x^2$ par exemple on voit
que $\Phi(X.Y)_n(x,x)=p^nx_n^2$ qui se réduit en $0\mod p$. Et
la suite aussi.

\subsection{Pré-calcul}
En fait si $h_0=h\mod p$ alors 
\[p^ih_0^{p^{n-i}}=p^ih^{p^{n-i}}\mod p^{n+1}\]
en particulier, une fois qu'on a calculé $\Phi(F)_i$ on peut
prendre le lift qu'on veut mod $p$ et ça changera rien pour
$\Phi(F)_j\mod p$, $j>i$. C'est pratique mais ça arrange pas non
plus tout.


\subsection{Topologie}
Ducoup on peut munir $W(A)$ de la topologie $p$-adique via les
$p^nW(A)=:I_n(A)=\ker(W(A)\to W_n(A))$. 

\subsection{Lift de Teichmüller}
Le lift de Teichmüller est donné par
\[[\_]\colon a\mapsto (a,0,\ldots)\]
et c'est clairement multiplicatif.
\subsection{Verschiebung et Frobenius}
L'opérateur Verschiebung (shift) de dual le Frobenius vérifie
$V\varphi=\varphi V=p$ et on a 
\[p.x=(0,x_0^p,x_1^p,\ldots)\]
via l'identité 
\[w_n(\Phi(p.X))=pw_n(X)\]
et une récurrence. 

\subsection{Écriture canonique et propriété universelle}
De $p=\varphi.V$ on obtient la caractéristique $0$ et l'écriture
canonique
\[x=\sum_{i\in \N}p^n[x_n^{p^{-n}}]\]
qui montre la $p$-complétude et la caractéristique $0$. Pour
la propriété universelle, l'idée c'est que un anneau $p$-adiquement
complet $R$ avec $R/p=A$ comme anneau résiduel vérifie que tout 
élément $x\in R$ s'écrit comme $ x=\sum p^ia_i$ avec $a_i$ un
représentant. Puis $W(A)\to R$ est défini terme à terme. Ça
marche parce que $\Phi(F(X,Y))$ est dans $\F_p[\overline X,\overline Y]^\N$
ou $\Z[\ldots]^\N$. D'où $w_n(\varphi(a+a'))$



\subsection{Racines de l'unité, $W(\F_p)$}
Vincent pense que l'idée c'est de canoniser le lift de teichmüller.
Ce qui est plutôt cohérent, i.e. le système de représentants du
corps résiduel est donné par les $([i])_{i\in A}$.

Mais en fait il y'a une suite intéressante à cette histoire. 
Voir l'article de M. Hazewinkel

\section{Anneaux de valuations}
Une première remarque : les idéaux principaux ont valuation
minorée $>0$. En particulier $\m_E$ est jamais principal. Ducoup
de même $\alpha.\m_E$ non plus pour $\alpha\in \m_E$. Enfin si
tu contiens $\alpha$ tu contiens $B(0, |\alpha|)$ de sorte que
$\m_E$ est le seul idéal de valuation $0$ puis on a tout les
idéaux !


\section{Calcul en pratique}\label{calcul}
Donc on veut évaluer $\Phi(F(X,Y))_n=:f_n$. On a $f_n\in \Z[\bar X,
\bar Y]$ ou $\in \F_p[\bar X,\bar Y]$. Mais en pratique en fait
pour évaluer y faut écrire $f_n\in A[\bar X,\bar Y]$. Pour 
$a\in W(A)=A^\N$ je note $I_a:=(X_i-a_i,i\in \N)$.

\subsection{L'intuition}
Donc on veut écrire 
$p^nf_n(a,b)=(w_n(F(a,b))-f_0^{p^n}-\ldots p^{n-1}f_{n-1}^p)$.
Et ça on peut pas l'écrire dans $A$ si $A$ est de char $p>0$.

En fait pour calculer $f_n(a,b)$ on peut simplement remarquer que
la congruence 
\[F(w_n(X),w_n(Y))-(w_n(\Phi(F(X,Y)))-p^n\Phi(F(X,Y))_n)
=0\mod p^n\]
implique que 
\[F(w_n(b),w_n(a))-(w_n(\Phi(F(a,b)))-p^n\Phi(F(a,b))_n)
=0\mod p^n\]
en particulier, l'évaluation de $f_n$ en $(a,b)$ dépend seulement
de l'évaluation des $f_i$ en $(a,b)$.




\end{document}
