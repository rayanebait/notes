\documentclass[a4paper,12pt]{article}
\usepackage{amsmath,  amsthm,enumerate}
\usepackage{csquotes}
\usepackage[provide=*,french]{babel}
\usepackage[dvipsnames]{xcolor}
\usepackage{quiver, tikz}

%symbole caligraphique
\usepackage{mathrsfs}

%hyperliens
\usepackage{hyperref}

%pseudo-code
\usepackage{algpseudocode}
\usepackage{algorithm}
\makeatletter
  \renewcommand{\ALG@name}{Algorithme}
  \makeatother
\usepackage{fancyhdr}

%bibliographie
\usepackage[
backend=biber,
style=alphabetic,
sorting=ynt
]{biblatex}

\addbibresource{bib.bib}


\definecolor{wgrey}{RGB}{148, 38, 55}

\setlength\parindent{24pt}

\newcommand{\Z}{\mathbb{Z}}
\newcommand{\R}{\mathbb{R}}
\newcommand{\rel}{\omathcal{R}}
\newcommand{\Q}{\mathbb{Q}}
\newcommand{\C}{\mathbb{C}}
\newcommand{\N}{\mathbb{N}}
\newcommand{\K}{\mathbb{K}}
\newcommand{\A}{\mathbb{A}}
\newcommand{\B}{\mathcal{B}}
\newcommand{\Or}{\mathcal{O}}
\newcommand{\F}{\mathbb F}
\newcommand{\m}{\mathfrak m}
\renewcommand{\b}{\mathfrak b}
\renewcommand{\a}{\mathfrak a}
\newcommand{\p}{\mathfrak p}
\newcommand{\I}{\mathfrak I}
\newcommand{\Hom}{\textrm{Hom}}
\newcommand{\disc}{\textrm{disc}}
\newcommand{\Pic}{\textrm{Pic}}
\newcommand{\End}{\textrm{End}}
\newcommand{\Spec}{\textrm{Spec}}
\newcommand{\Frac}{\textrm{Frac}}

\newcommand{\cL}{\mathscr{L}}
\newcommand{\G}{\mathscr{G}}
\newcommand{\D}{\mathscr{D}}
\newcommand{\E}{\mathscr{E}}
\newcommand{\U}{\mathscr{U}}

\theoremstyle{plain}
\newtheorem{thm}{Théoreme}
\newtheorem{lem}{Lemme}
\newtheorem{prop}{Proposition}
\newtheorem{cor}{Corollaire}
\newtheorem{heur}{Heuristique}
\newtheorem{rem}{Remarque}
\newtheorem{rembis}{Remarque}
\newtheorem{note}{Note}

\theoremstyle{definition}
\newtheorem{conj}{Conjecture}
\newtheorem*{eq}{Équivalences}
\newtheorem{prob}{Problème}
\newtheorem{quest}{Question}
\newtheorem{prot}{Protocole}
\newtheorem{algo}{Algorithme}
\newtheorem{defn}{Définition}
\newtheorem{defnbis}{Définition}
\newtheorem{ex}{Exemple}
\newtheorem{exo}{Exercices}

\theoremstyle{remark}

\definecolor{wgrey}{RGB}{148, 38, 55}
\definecolor{wgreen}{RGB}{100, 200,0} 
\hypersetup{
    colorlinks=true,
    linkcolor=wgreen,
    urlcolor=wgrey,
    filecolor=wgrey
}

\title{Extensions profondément ramifiées et presques étales}
\date{}

\begin{document}
\maketitle

Le setup c'est $K$ un corps local de caractéristique $0$,
$K_\infty/K$ une extension algébrique, $K_\infty=\cup_{n\geq 0} K_n$
avec $[K_n:K]$ finie. Puis $L_\infty/K_\infty$ finie et $L_\infty=K_\infty(\alpha)$,
ensuite $L_n:=K_n(\alpha)$ et on suppose
$\mu_\alpha\in K_{0}[X]$. On a $L=\cup_{n\geq 0} L_n$.

\begin{rem}
  La différente $\D_{L_\infty/K_\infty}$ est juste
  définie par $(\Or_{L_\infty}^\wedge)^{-1}$ ce qui 
  fait sens. On a pas forcément les propriétés 
  habituelles a priori.
\end{rem}

\section{$\D_{K_\infty/K}$ et $\D_{L_\infty/K_\infty}$}
On déf :
\[\D_{K_\infty/K}:=\cap_{n=0}^\infty \D_{K_n/K}\Or_{K_\infty}\]
c'est bien défini et dépend par des $K_n$ choisis. Alors
on a 
\[\D_{L_n/K_n}\Or_{L_m}\subset \D_{L_m/K_m}\]
puis
\[\D_{L_\infty/K_\infty}\Or_{L_m}=\bigcup_{n=0}^\infty(\D_{L_n/K_n}\Or_{L_\infty})\]



\section{Extensions profondément ramifiées (Coates-Greenberg)}
Le conducteur est donné par 
\[c(M)=inf\{\nu|M\subset \bar K^{(\nu)}\}+1\]
Le théorème principal c'est :
\begin{thm}[Coates-Greenberg]


\end{thm}

\end{document}
