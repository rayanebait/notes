\documentclass[a4paper,12pt]{article}
\usepackage{amsmath,  amsthm,enumerate}
\usepackage{csquotes}
\usepackage[provide=*,french]{babel}
\usepackage[dvipsnames]{xcolor}
\usepackage{quiver, tikz}

%symbole caligraphique
\usepackage{mathrsfs}

%hyperliens
\usepackage{hyperref}

%pseudo-code
\usepackage{algpseudocode}
\usepackage{algorithm}
\makeatletter
  \renewcommand{\ALG@name}{Algorithme}
  \makeatother
\usepackage{fancyhdr}

%bibliographie
\usepackage[
backend=biber,
style=alphabetic,
sorting=ynt
]{biblatex}

\addbibresource{bib.bib}


\definecolor{wgrey}{RGB}{148, 38, 55}

\setlength\parindent{24pt}

\newcommand{\Z}{\mathbb{Z}}
\newcommand{\R}{\mathbb{R}}
\newcommand{\rel}{\omathcal{R}}
\newcommand{\Q}{\mathbb{Q}}
\newcommand{\C}{\mathbb{C}}
\newcommand{\N}{\mathbb{N}}
\newcommand{\K}{\mathbb{K}}
\newcommand{\A}{\mathbb{A}}
\newcommand{\B}{\mathcal{B}}
\newcommand{\Or}{\mathcal{O}}
\newcommand{\F}{\mathbb F}
\newcommand{\m}{\mathfrak m}
\renewcommand{\b}{\mathfrak b}
\renewcommand{\a}{\mathfrak a}
\newcommand{\p}{\mathfrak p}
\newcommand{\I}{\mathfrak I}
\newcommand{\Hom}{\textrm{Hom}}
\newcommand{\disc}{\textrm{disc}}
\newcommand{\Pic}{\textrm{Pic}}
\newcommand{\End}{\textrm{End}}
\newcommand{\Spec}{\textrm{Spec}}
\newcommand{\Frac}{\textrm{Frac}}

\newcommand{\cL}{\mathscr{L}}
\newcommand{\G}{\mathscr{G}}
\newcommand{\D}{\mathscr{D}}
\newcommand{\E}{\mathscr{E}}
\newcommand{\U}{\mathscr{U}}

\theoremstyle{plain}
\newtheorem{thm}{Théoreme}
\newtheorem{lem}{Lemme}
\newtheorem{prop}{Proposition}
\newtheorem{cor}{Corollaire}
\newtheorem{heur}{Heuristique}
\newtheorem{rem}{Remarque}
\newtheorem{rembis}{Remarque}
\newtheorem{note}{Note}

\theoremstyle{definition}
\newtheorem{conj}{Conjecture}
\newtheorem*{eq}{Équivalences}
\newtheorem{prob}{Problème}
\newtheorem{quest}{Question}
\newtheorem{prot}{Protocole}
\newtheorem{algo}{Algorithme}
\newtheorem{defn}{Définition}
\newtheorem{defnbis}{Définition}
\newtheorem{ex}{Exemple}
\newtheorem{exo}{Exercices}

\theoremstyle{remark}

\definecolor{wgrey}{RGB}{148, 38, 55}
\definecolor{wgreen}{RGB}{100, 200,0} 
\hypersetup{
    colorlinks=true,
    linkcolor=wgreen,
    urlcolor=wgrey,
    filecolor=wgrey
}

\title{Lemme de Krasner}
\date{}

\begin{document}
\maketitle

\section{Preuve}
Théorie de Galois et invariance de la valeur absolue.

\section{Un exemple : $\Q_p(\zeta_p)=\Q_p((-p)^{\frac{1}{p-1}})$}
C'est dingo : On peut montrer que si 
$\alpha=(-p)^{\frac{1}{p-1}}$ alors il existe $j$ tel que
\[v_p(\alpha-(\zeta_p^j-1))\geq \frac{1}{p-1}+\epsilon\]
pour $0 <\epsilon$ sachant que 
$v_p(\zeta_p^j-\zeta_p^k)=\frac{1}{p-1}$ si 
$j\ne k\mod p$. Pour ça on peut remarquer que
\[\prod_{j=1}^{p-1} (1-\zeta_p^j)=p=-\alpha^{p-1}\]
d'où
\[\prod_{j=1}^{p-1} \frac{1-\zeta_p^j}{\alpha}=-1.\]
Maintenant si on calcule 
$\prod_{j=1}^{p-1}(\alpha-(1-\zeta_p^j))$ on remarque que
en notant $J_i$ les sous-ensembles de $\{1,\ldots, p-1\}$
de taille $i$ :
\begin{align*}
  \prod_{j=1}^{p-1}(\alpha-(1-\zeta_p^j))&=1+S+(-1)\\
                                        &=\sum_{i=1}^{p-2}\sum_{(j_1,\ldots,j_i)\in J_i}\alpha^i(1-\zeta_p^{j_1})\ldots(1-\zeta_p^{j_i})\\
\end{align*}
maintenant la valuation est donnée par les sommes
de produits de taille $1$ (trace$=p$) et on a
\[v_p(\prod_{j=1}^{p-1}(\alpha-(1-\zeta_p^j))\geq v_p(\alpha)+v_p(p)\]
en particulier il doit exister $j$ tel que 
\[v_p(\alpha-(1-\zeta_p^j))\geq \frac{1}{(p-1)^2}+\frac{1}{p-1}\]
d'où ce qu'on veut.

\begin{rem}
  En fait je m'étais trompé sur le calcul mais j'ai
  mieux : On évalue $\phi_p(X+1)$ presque en $\alpha$
  et comme $X\phi_p(X+1)=(X+1)^p-1$ comme d'hab chaque
  coeff
  est divisible par $p$ (!) d'où les autres sommes sont
  de valuations $\geq v_p(\alpha^i)+v_p(p)$ et le
  résultat. Attention nous on a une cancellation alors
  que $v_p(\phi_p(\alpha+1))$ y'a le coeff dominant
  qui est $1$.
\end{rem}




\end{document}
