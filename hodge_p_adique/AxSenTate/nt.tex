\documentclass[a4paper,12pt]{article}
\usepackage{amsmath,  amsthm,enumerate}
\usepackage{csquotes}
\usepackage[provide=*,french]{babel}
\usepackage[dvipsnames]{xcolor}
\usepackage{quiver, tikz}

%symbole caligraphique
\usepackage{mathrsfs}

%hyperliens
\usepackage{hyperref}

%pseudo-code
\usepackage{algpseudocode}
\usepackage{algorithm}
\makeatletter
  \renewcommand{\ALG@name}{Algorithme}
  \makeatother
\usepackage{fancyhdr}

%bibliographie
\usepackage[
backend=biber,
style=alphabetic,
sorting=ynt
]{biblatex}

\addbibresource{bib.bib}


\definecolor{wgrey}{RGB}{148, 38, 55}

\setlength\parindent{24pt}

\newcommand{\Z}{\mathbb{Z}}
\newcommand{\R}{\mathbb{R}}
\newcommand{\rel}{\omathcal{R}}
\newcommand{\Q}{\mathbb{Q}}
\newcommand{\C}{\mathbb{C}}
\newcommand{\N}{\mathbb{N}}
\newcommand{\K}{\mathbb{K}}
\newcommand{\A}{\mathbb{A}}
\newcommand{\B}{\mathcal{B}}
\newcommand{\Or}{\mathcal{O}}
\newcommand{\F}{\mathbb F}
\newcommand{\m}{\mathfrak m}
\renewcommand{\b}{\mathfrak b}
\renewcommand{\a}{\mathfrak a}
\newcommand{\p}{\mathfrak p}
\newcommand{\I}{\mathfrak I}
\newcommand{\Hom}{\textrm{Hom}}
\newcommand{\disc}{\textrm{disc}}
\newcommand{\Pic}{\textrm{Pic}}
\newcommand{\End}{\textrm{End}}
\newcommand{\Spec}{\textrm{Spec}}
\newcommand{\Frac}{\textrm{Frac}}

\newcommand{\cL}{\mathscr{L}}
\newcommand{\G}{\mathscr{G}}
\newcommand{\D}{\mathscr{D}}
\newcommand{\E}{\mathscr{E}}
\newcommand{\U}{\mathscr{U}}

\theoremstyle{plain}
\newtheorem{thm}{Théoreme}
\newtheorem{lem}{Lemme}
\newtheorem{prop}{Proposition}
\newtheorem{cor}{Corollaire}
\newtheorem{heur}{Heuristique}
\newtheorem{rem}{Remarque}
\newtheorem{rembis}{Remarque}
\newtheorem{note}{Note}

\theoremstyle{definition}
\newtheorem{conj}{Conjecture}
\newtheorem*{eq}{Équivalences}
\newtheorem{prob}{Problème}
\newtheorem{quest}{Question}
\newtheorem{prot}{Protocole}
\newtheorem{algo}{Algorithme}
\newtheorem{defn}{Définition}
\newtheorem{defnbis}{Définition}
\newtheorem{ex}{Exemple}
\newtheorem{exo}{Exercices}

\theoremstyle{remark}

\definecolor{wgrey}{RGB}{148, 38, 55}
\definecolor{wgreen}{RGB}{100, 200,0} 
\hypersetup{
    colorlinks=true,
    linkcolor=wgreen,
    urlcolor=wgrey,
    filecolor=wgrey
}

\title{Ax-Sen-Tate}
\date{}

\begin{document}
\maketitle
Je suis la preuve de Colmez avec quelques détails en
plus.
Le but c'est de montrer que pour $K$ un corps 
local de caractéristique $0$ (y'a une version en char $p\geq 1$)
on a 
\[\C_K^{G_K}=K\]
et plus généralement pour $\Q_p\subset L\subset \C_p$
avec $L$ complet :
\[\widehat{\overline{L}}^{G_L}=\widehat{\overline{L}^{G_L}}\]

Intuitivement, si $L\subset M\subset \bar L$ et qu'on
peut montrer qu'on peut borner continûment la distance 
entre $M$ et un élément $\alpha\in \bar L$ par un truc
qui dépend de $G_M.\alpha$ et $[M(\alpha):M]$.

\section{Preuve}
En notant 
\[\Delta_L(\alpha):=\sup_{g\in G_L}|g\alpha-\alpha|\]
pour $\alpha\in \C_L$. Maintenant si $\alpha_n\to \alpha$
dans $\bar L$ on a 
$\Delta_L(\alpha)\leq \Delta_L(\alpha_n)$ pour $n$ grand.

Le truc cool maintenant c'est que pour $x\in \bar L$
on peut trouver $a\in L$ tel que 
\[|a-x|\leq c_p\Delta_L(x)\]
pour une constante $c_p$ qui dépend que de $p$ d'où si
$\alpha_n$ est fixe par $\G_L$ alors $\alpha$ aussi
puis $\alpha$ est dans l'adhérence (la complétion) de
$L[\alpha_n]$ dans $\C_L$!

\section{Lemme principal}
\subsection{Borne 0}
Un truc connu pour $P(X)\in \Or_{\bar \Q_p}[X]$
et $P(X)=X^n+a_{n-1}X^{n-1}+\ldots+a_0$ c'est la
borne :
\[|a_i|\leq |Rac(P)|^{n-i}\]
\subsection{Borne I : Calculs}
\subsubsection{Cas général}
On a 
$P^{(q)}(X)=a_n\frac{n!}{(n-q)!}X^{n-q}+\ldots+q!a_q$
et $a_n=1$ d'où 
$b=\frac{a_q}{\begin{pmatrix} n\\ q\end{pmatrix}}$ est
le produit des racines de $P^{(q)}$ au signe près.
En plus pour le coeff constant un analogue de la borne
$0$ dit que
il existe une racine $\beta\in Rac(P^{(q)})$ telle que
\[|\beta|\leq |b|^{n-q}\]
reste plus qu'à calculer 
$|\begin{pmatrix} n\\ q\end{pmatrix}|^{1/(n-q)}$.
\subsubsection{Ce qu'on veut}
On peut prendre $q=p^{v_p(n)}=p^{k+1}$ si $n$ est pas une
puissance de $p$ et $q=n/p$ sinon. Alors dans ces cas
là on calcule bien
\[|\begin{pmatrix} n\\ q\end{pmatrix}|^{1/(n-q)}=\begin{cases} 1,~\textrm{cas 1}\\\frac{1}{p^{\frac{1}{p^{k+1}}-\frac{1}{p^{k}}}},~\textrm{autre cas}\end{cases}\]
parce que 
\[\begin{pmatrix} n\\ q\end{pmatrix}=\frac{n}{q}\prod_{i=1}^{q-1}\frac{n-i}{i}\]






\end{document}
