\documentclass[a4paper,12pt]{article}
\usepackage{amsmath,  amsthm,enumerate}
\usepackage{csquotes}
\usepackage[provide=*,french]{babel}
\usepackage[dvipsnames]{xcolor}
\usepackage{quiver, tikz}

%symbole caligraphique
\usepackage{mathrsfs}

%hyperliens
\usepackage{hyperref}

%pseudo-code
\usepackage{algpseudocode}
\usepackage{algorithm}
\makeatletter
  \renewcommand{\ALG@name}{Algorithme}
  \makeatother
\usepackage{fancyhdr}

%bibliographie
\usepackage[
backend=biber,
style=alphabetic,
sorting=ynt
]{biblatex}

\addbibresource{bib.bib}


\definecolor{wgrey}{RGB}{148, 38, 55}

\setlength\parindent{24pt}

\newcommand{\Z}{\mathbb{Z}}
\newcommand{\R}{\mathbb{R}}
\newcommand{\rel}{\omathcal{R}}
\newcommand{\Q}{\mathbb{Q}}
\newcommand{\C}{\mathbb{C}}
\newcommand{\N}{\mathbb{N}}
\newcommand{\K}{\mathbb{K}}
\newcommand{\A}{\mathbb{A}}
\newcommand{\B}{\mathcal{B}}
\newcommand{\Or}{\mathcal{O}}
\newcommand{\F}{\mathbb F}
\newcommand{\m}{\mathfrak m}
\renewcommand{\b}{\mathfrak b}
\renewcommand{\a}{\mathfrak a}
\newcommand{\p}{\mathfrak p}
\newcommand{\I}{\mathfrak I}
\newcommand{\Hom}{\textrm{Hom}}
\newcommand{\disc}{\textrm{disc}}
\newcommand{\Pic}{\textrm{Pic}}
\newcommand{\End}{\textrm{End}}
\newcommand{\Spec}{\textrm{Spec}}
\newcommand{\Frac}{\textrm{Frac}}

\newcommand{\cL}{\mathscr{L}}
\newcommand{\G}{\mathscr{G}}
\newcommand{\D}{\mathscr{D}}
\newcommand{\E}{\mathscr{E}}
\newcommand{\U}{\mathscr{U}}

\theoremstyle{plain}
\newtheorem{thm}{Théoreme}
\newtheorem{lem}{Lemme}
\newtheorem{prop}{Proposition}
\newtheorem{cor}{Corollaire}
\newtheorem{heur}{Heuristique}
\newtheorem{rem}{Remarque}
\newtheorem{rembis}{Remarque}
\newtheorem{note}{Note}

\theoremstyle{definition}
\newtheorem{conj}{Conjecture}
\newtheorem*{eq}{Équivalences}
\newtheorem{prob}{Problème}
\newtheorem{quest}{Question}
\newtheorem{prot}{Protocole}
\newtheorem{algo}{Algorithme}
\newtheorem{defn}{Définition}
\newtheorem{defnbis}{Définition}
\newtheorem{ex}{Exemple}
\newtheorem{exo}{Exercices}

\theoremstyle{remark}

\definecolor{wgrey}{RGB}{148, 38, 55}
\definecolor{wgreen}{RGB}{100, 200,0} 
\hypersetup{
    colorlinks=true,
    linkcolor=wgreen,
    urlcolor=wgrey,
    filecolor=wgrey
}

\title{Ax-Sen-Tate}
\date{}

\begin{document}
\maketitle
Je suis la preuve de Colmez avec quelques détails en
plus.
Le but c'est de montrer que pour $K$ un corps 
local de caractéristique $0$ (y'a une version en char $p\geq 1$)
on a 
\[\C_K^{G_K}=K\]
et plus généralement pour $\Q_p\subset L\subset \C_p$
avec $L$ complet :
\[\widehat{\overline{L}}^{G_L}=\widehat{\overline{L}^{G_L}}\]

Intuitivement, si $L\subset M\subset \bar L$ et qu'on
peut montrer qu'on peut borner continûment la distance 
entre $M$ et un élément $\alpha\in \bar L$ par un truc
qui dépend de $G_M.\alpha$ et $[M(\alpha):M]$.

\section{Preuve}
En notant 
\[\Delta_L(\alpha):=\sup_{g\in G_L}|g\alpha-\alpha|\]
pour $\alpha\in \C_L$. Maintenant si $\alpha_n\to \alpha$
dans $\bar L$ on a 
$\Delta_L(\alpha)\leq \Delta_L(\alpha_n)$ pour $n$ grand.

Le truc cool maintenant c'est que pour $x\in \bar L$
on peut trouver $a\in L$ tel que 
\[|a-x|\leq c_p\Delta_L(x)\]
pour une constante $c_p$ qui dépend que de $p$ d'où si
$\alpha_n$ est fixe par $\G_L$ alors $\alpha$ aussi
puis $\alpha$ est dans l'adhérence (la complétion) de
$L[\alpha_n]$ dans $\C_L$!

\section{Lemme principal}
On montre que pour $\alpha\in \bar L$ il existe $a\in M$
tel que $|a-\alpha|\leq c_p\Delta_M(\alpha)$.
\subsection{Borne 0}
Un truc connu pour $P(X)\in \Or_{\bar \Q_p}[X]$
et $P(X)=X^n+a_{n-1}X^{n-1}+\ldots+a_0$ c'est la
borne :
\[|a_i|\leq |Rac(P)|^{n-i}\]
\subsection{Borne I : Calculs}
\subsubsection{Cas général}
On a 
$P^{(q)}(X)=a_n\frac{n!}{(n-q)!}X^{n-q}+\ldots+q!a_q$
et $a_n=1$ d'où 
$b=\frac{a_q}{\begin{pmatrix} n\\ q\end{pmatrix}}$ est
le produit des racines de $P^{(q)}$ au signe près.
En plus pour le coeff constant un analogue de la borne
$0$ dit que
il existe une racine $\beta\in Rac(P^{(q)})$ telle que
\[|\beta|\leq |b|^{n-q}\]
reste plus qu'à calculer 
$|\begin{pmatrix} n\\ q\end{pmatrix}|^{1/(n-q)}$.
\subsubsection{Ce qu'on veut}
On peut prendre $q=p^{v_p(n)}=p^{k+1}$ si $n$ est pas une
puissance de $p$ et $q=n/p$ sinon. Alors dans ces cas
là on calcule bien :
\[|\begin{pmatrix} n\\ q\end{pmatrix}|^{1/(n-q)}=\begin{cases} 1,~\textrm{cas 1}\\\frac{1}{p^{\frac{1}{p^{k+1}}-\frac{1}{p^{k}}}},~\textrm{autre cas}\end{cases}\]
parce que 
\[\begin{pmatrix} n\\ q\end{pmatrix}=\frac{n}{q}\prod_{i=1}^{q-1}\frac{n-i}{i}\]
maintenant via les hypothèses $|n-i|=|i|$ pour
$1\leq i\leq q-1$ via $p^ka-i=p^l.u$ dit que
$i=p^l(p^{k-l}a-u)$.

\begin{rem}
  Autrement, entre $n$ et $n-p^k$ y'a "qu'une seule
  copie de $p^k$" d'où les mêmes "$p$-éléments" qu'entre
  $1$ et $p^k$.
\end{rem}

\subsection{Preuve du Lemme}
On peut étant donné $P(X)$ prendre $Q(X):=P(X+\alpha)$
alors $\beta\in Rac(P)$ est de la forme 
$g.\alpha-\alpha$ d'où $|Rac(Q)|\leq \Delta_M(\alpha)$.
Ensuite on a $\gamma\in Rac(Q^{(q)})$ telle que 
on ait $|\gamma|\leq c_p\Delta_M(\alpha)$. Maintenant
le trick bizarre mais naturel en fait : 
$d(X+\alpha)/dX=1$ d'où il existe 
$\beta\in Rac(P^{(q)})$ telle que 
\[|\gamma|=|\beta-\alpha|\leq c_p\Delta_M(\alpha)\]
sauf que $[M(\beta):M]\leq [M(\alpha):M]-1$ ! D'où
par récurrence sachant que 
$\Delta_M(\beta)\leq \Delta_M(\alpha)$ on peut conclure.

\subsection{Procédé récursif}
Si on note $\alpha=:\alpha_0$ et $\beta=:\alpha_1$, 
$n_i=[M(\alpha_i):M]$, $q_i=p^{v_p(n_i)}$ ou 
$q_i=p^{v_p(n_i)-1}$ et $k_i$ cette puissance puis
$P_0(X)=P(X)$, 
$P_1(X)=Irr(\alpha_1)|P_0^{(q)}(X)/a_{n-q}^{(1)}$
avec $a_i^{(j)}$ le $i$-ème coefficient de 
$P_{j-1}^{(q_{j-1})}$ on obtient à chaque étape 
une racine $\alpha_i$ de $P_i$ telle que 
$\Delta_M(\alpha_i)\leq \Delta_M(\alpha)$ et 
$|\alpha_i-\alpha_{i-1}|\leq c_p \Delta_M(\alpha_{i-1})$
puis éventuellement $\alpha_m\in M$ et 
\[|\alpha_m-\alpha|\leq c_p\Delta_M(\alpha)\]

\subsection{Pourquoi $q=p^{k}$ ?}
Ça a l'air d'être pour étendre la preuve à la 
caractéristique $p\geq 1$ pour pouvoir obtenir
$\Delta_M(\alpha_i)$ vu que quand y'a des extensions
inséparables c'est embêtant.
\subsection{Via la trace}
En caractéristique $0$ si on pose $q=n-1$ on calcule
\[|\beta|\leq|a_{n-1}|/|n|\leq \delta/p^{v_p(n)}\]
de sorte que si $n\wedge p=1$ alors directement 
$|\beta-\alpha|\leq \Delta_M(\alpha)$ et $\beta\in M$.

EN FAIT NON : peut-être qu'il y'a un problème avec 
la ramification sauvage et la surjectivité de la trace?


\section{Avec $K\subset K_\infty\subset \C_K$}
En fait l'idée d'approcher $\alpha$ par sa trace dans
$M$ peut être systématisée. On peut poser $K_\infty/K$
une $\Z_p$-extension de sorte qu'elle est pas de 
conducteur finie. 

Avec Coates-Greenberg on a la condition équivalente
sur la trace qui la rend surjective pour toutes les
extensions finies de $K_\infty$.

Si on peut prouver que






\end{document}
