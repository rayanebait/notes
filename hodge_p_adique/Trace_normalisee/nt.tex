\documentclass[a4paper,12pt]{article}
\usepackage{amsmath,  amsthm,enumerate}
\usepackage{csquotes}
\usepackage[provide=*,french]{babel}
\usepackage[dvipsnames]{xcolor}
\usepackage{quiver, tikz}

%symbole caligraphique
\usepackage{mathrsfs}

%hyperliens
\usepackage{hyperref}

%pseudo-code
\usepackage{algpseudocode}
\usepackage{algorithm}
\makeatletter
  \renewcommand{\ALG@name}{Algorithme}
  \makeatother
\usepackage{fancyhdr}

%bibliographie
\usepackage[
backend=biber,
style=alphabetic,
sorting=ynt
]{biblatex}

\addbibresource{bib.bib}


\definecolor{wgrey}{RGB}{148, 38, 55}

\setlength\parindent{24pt}

\newcommand{\Z}{\mathbb{Z}}
\newcommand{\R}{\mathbb{R}}
\newcommand{\rel}{\omathcal{R}}
\newcommand{\Q}{\mathbb{Q}}
\newcommand{\C}{\mathbb{C}}
\newcommand{\N}{\mathbb{N}}
\newcommand{\K}{\mathbb{K}}
\newcommand{\A}{\mathbb{A}}
\newcommand{\B}{\mathcal{B}}
\newcommand{\Or}{\mathcal{O}}
\newcommand{\F}{\mathbb F}
\newcommand{\m}{\mathfrak m}
\renewcommand{\b}{\mathfrak b}
\renewcommand{\a}{\mathfrak a}
\newcommand{\p}{\mathfrak p}
\newcommand{\I}{\mathfrak I}
\newcommand{\Hom}{\textrm{Hom}}
\newcommand{\disc}{\textrm{disc}}
\newcommand{\Pic}{\textrm{Pic}}
\newcommand{\End}{\textrm{End}}
\newcommand{\Spec}{\textrm{Spec}}
\newcommand{\Frac}{\textrm{Frac}}

\newcommand{\cL}{\mathscr{L}}
\newcommand{\G}{\mathscr{G}}
\newcommand{\D}{\mathscr{D}}
\newcommand{\E}{\mathscr{E}}
\newcommand{\U}{\mathscr{U}}

\theoremstyle{plain}
\newtheorem{thm}{Théoreme}
\newtheorem{lem}{Lemme}
\newtheorem{prop}{Proposition}
\newtheorem{cor}{Corollaire}
\newtheorem{heur}{Heuristique}
\newtheorem{rem}{Remarque}
\newtheorem{rembis}{Remarque}
\newtheorem{note}{Note}

\theoremstyle{definition}
\newtheorem{conj}{Conjecture}
\newtheorem*{eq}{Équivalences}
\newtheorem{prob}{Problème}
\newtheorem{quest}{Question}
\newtheorem{prot}{Protocole}
\newtheorem{algo}{Algorithme}
\newtheorem{defn}{Définition}
\newtheorem{defnbis}{Définition}
\newtheorem{ex}{Exemple}
\newtheorem{exo}{Exercices}

\theoremstyle{remark}

\definecolor{wgrey}{RGB}{148, 38, 55}
\definecolor{wgreen}{RGB}{100, 200,0} 
\hypersetup{
    colorlinks=true,
    linkcolor=wgreen,
    urlcolor=wgrey,
    filecolor=wgrey
}

\title{Trace normalisée}
\date{}

\begin{document}
\maketitle
On prends $K$ un corps local de caractéristique $0$
et $K_\infty/K$ une $\Z_p$-extension qui est pas de
conducteur fini (en gros on peut se ramener à une
$\Z_p$-extension totalement ramifiée). Je note
$\Gamma = G_K/G_{K_\infty}\simeq \Z_p$ et $\gamma$
un générateur topologique, puis $\gamma_n:=\gamma^{p^n}$.

\section{Idée}
L'idée c'est que $K_\infty/K$ est alors profondément
ramifiée d'où on a Ax-Sen-Tate :
\[\C_K^{G_{K_\infty}}=\widehat K_\infty\]
puis $\C_K^{G_K}=\widehat{K_\infty}^\Gamma$.

\subsection{Première partie}
C'est Ax-Sen-Tate pour les extensions profondéments 
ramifiées.
\subsection{Deuxième cas}
Si $Tr_{K_\infty/K}(x):=\frac{1}{p^n}Tr_{K_n/K}(x)$
alors :
\begin{enumerate}
  \item $|Tr_{K_\infty/K}(x)|\leq c|\gamma.x-x|_K$
  \item $Tr_{K_\infty/K}(.)$ est continue et s'étend à $\widehat{K_\infty}$.
  \item On obtient $\widehat K_\infty=K\oplus \widehat K_\infty^\circ$ où
    le deuxième terme c'est $\ker(Tr_{K_\infty/K})$.
  \item $\gamma-1$ est bijectif d'inverse continu sur $\widehat K_\infty^\circ$.
\end{enumerate}
Le dernier point permet de montrer que 
$\widehat{K_\infty}^\Gamma = K^\Gamma=K$
par injectivité de $\gamma-1$.

\subsection{Extension : $\C_K(\eta)^{G_K}=0$}
On peut ensuite étendre le résultat et prouver que
\begin{enumerate}
  \item Si $\tau = \gamma -\lambda$ où $\lambda\in U_K^{(1)}$ est
    pas une racine de l'unité alors $\tau$ est bijectif d'inverse
    continu sur $\widehat{K_\infty}$.
\end{enumerate}
En particulier étant donné un caractère non trivial de 
$G_K$, $\eta$, si $\eta(\gamma)=\lambda$ alors 
$\widehat{K_\infty(\eta)}^\Gamma=0$ par injectivité de
$\gamma-\lambda$. On prouve ensuite que
\[\C_K(\eta)^{G_K}=0\]
parce que 
$\C_K(\eta)^{G_{K_\infty}}=\C_K^{G_{K_\infty}}(\eta)$.
(directement via la déf)

\section{Preuves}
\subsection{(1.2.1.)}
\subsubsection{Idée de la récurrence}
On fait une récurrence sur $n$ pour montrer que si
$x\in K_n$ alors
\[|Tr_{K_\infty/K}(x)-x|\leq c_n|\gamma(x)-x|_K\]
l'idée c'est que si $y=\frac{1}{p}Tr_{K_n/K_{n-1}}(x)$
alors $Tr_{K_\infty/K}(x)=Tr_{K_\infty/K}(y)$.
D'où 
\[|Tr_{K_\infty/K}(x)-x|_K=|Tr_{K_\infty/K}(y)-y+\frac{1}{p}Tr_{K_n/K_{n-1}}(x)-x|_K\]
Maintenant le terme de droite se majore facile via
$3.$ et le terme de gauche c'est plus dur on a 
\[Tr_{K_\infty/K}(y)-y=\frac{1}{p}Tr_{K_n/K_{n-1}}(\gamma(x)-x)\]
et faut utiliser les formules de comparaisons de 
valuations de traces.

\subsubsection{Les preuves}
Dans l'ordre pour $1.2.1.$ on a :
\begin{enumerate}
  \item $v_K(Tr_{K_n/K_{n-1}})\geq v_K(x)+v_K(D_{K_n/K_{n-1}}) - \frac{1}{p^{n-1}}$ par les formules 
    d'estimation de valuations de traces.
  \item En déroulant avec le théorème de Tate
    \[|Tr_{K_n/K_{n-1}}(x)|\leq |p|_K^{1-\frac{b}{p^{n-1}}}|x|_K\]
  \item Via $Tr_{K_n/K_{n-1}}(x)=\sum_{k=0}^{p-1}\gamma_{n-1}^k(x)$, $\gamma_n=\gamma^{p^n}$:
    \[|\frac{1}{p}Tr_{K_n/K_{n-1}}(x)-x|_K\leq|p|_K^{-1}|\gamma.x-x|_K\]
\end{enumerate}
\begin{rem}
  Faut penser $\alpha^k-1=(1+\alpha+\ldots+\alpha^{k-1})(\alpha-1)$).
\end{rem}
Le $3.$ est assez direct 

on a une récurrence via si 
$y=\frac{1}{p}Tr_{K_n/K_{n-1}}(x)$ alors
\[Tr_{K_\infty/K}(x)=Tr_{K_\infty/K}(y)\]
et 
\[|Tr_{K_\infty/K}(x)-x|_K=|(Tr_{K_\infty/K}(y)-y)+\left(\frac{1}{p}Tr_{K_n/K_{n-1}}(x)-x\right)|_K\]
Y faut juste estimer la constante $c$ via les 
$p^{-b/p^{n-1}}$.

\subsection{(1.2.2.)}
On a 
\[|Tr_{K_\infty/K}(x)|_K\leq c.|x|_K\]
d'où la continuité en $0$ puis la continuité partout
par linéarité de la trace.

\subsection{(1.2.3.)}
Direct.
\subsection{(1.2.4.)}
Y s'agit juste de montrer que l'inverse est continu. Son
existence est facile, je l'appelle $\rho$. On a 
\[|x|_K=|x-Tr_{K_\infty/K}(x)+Tr_{K_\infty/K}(x)|_K\leq c.|(\gamma-1)x|_K\]
pour tout $x\in \widehat{K_\infty^\circ}$ par les
sections d'avant. En 
particulier comme 
$\rho(\widehat{K_\infty^\circ})=\widehat{K_\infty^\circ}$
on a
\[|\rho(x)|_K\leq c.|x|_K\]
puis la continuité.


\section{Extension}
Pour $\lambda\in U_K^{(1)}$ l'idée c'est que
\[(\gamma-1+1-\lambda)\rho=1+(1-\lambda).\rho\]
et $|1-\lambda|_K<1$. Le truc de droite est de la
forme $1-a$ et donc ça suggère de regarder l'opérateur
\[\sum_{i=0}^\infty (1-a)^i\]
Maintenant si $|1-\lambda|_K<c^{-1}$ la série
\[\sum_{i=0}^{+\infty} (\lambda-1)^i\rho^i\]
converge absolument ! Dans le cas général
il existe $n$ tel que $|1-\lambda^{p^n}|_K<c^{-1}$
d'où on peut conclure si $\lambda$ est pas une
racine de l'unité.

\section{Cohomologie}
On a montré que 
\[H^0(\Gamma, \widehat{K_\infty(\eta)})=\begin{cases} K,~\eta=1\\0,~\eta(\gamma)\notin \mu_K\end{cases}\]
En fait on voit aussi directement que
\[H^1(\Gamma, \widehat{K_\infty(\eta)})=\begin{cases} K,~\eta=1\\0,~\eta(\gamma)\notin \mu_K\end{cases}\]
en remarquant qu'un cocycle $f$ est 
\begin{enumerate}
  \item entièremeent determiné par $f(\gamma)$,
  \item $\gamma-\eta(\gamma)=\eta(\gamma)(\rho(\gamma)-1)$ dans $End(\widehat{K_\infty(\eta)})$.
\end{enumerate}
En particulier 
\[H^1(\Gamma,\widehat{K_\infty(\eta)})\simeq \frac{\widehat{K_\infty(\eta)}}{im(\gamma-\eta(\gamma))}\]
d'où le résultat par les parties d'avant.



\end{document}
