\documentclass[a4paper,12pt]{book}
\usepackage{amsmath,  amsthm,enumerate}
\usepackage{csquotes}
\usepackage[provide=*,french]{babel}
\usepackage[dvipsnames]{xcolor}
\usepackage{quiver, tikz}

%symbole caligraphique
\usepackage{mathrsfs}

%hyperliens
\usepackage{hyperref}

%pseudo-code
\usepackage{algpseudocode}
\usepackage{algorithm}
\makeatletter
  \renewcommand{\ALG@name}{Algorithme}
  \makeatother
\usepackage{fancyhdr}

\pagestyle{fancy}
\addtolength{\headwidth}{\marginparsep}
\addtolength{\headwidth}{\marginparwidth}
\renewcommand{\chaptermark}[1]{\markboth{#1}{}}
\renewcommand{\sectionmark}[1]{\markright{\thesection\ #1}}
\fancyhf{}
\fancyfoot[C]{\thepage}
\fancyhead[LO]{\textit \leftmark}
\fancyhead[RE]{\textit \rightmark}
\renewcommand{\headrulewidth}{0pt} % and the line
\fancypagestyle{plain}{%
    \fancyhead{} % get rid of headers
}

%bibliographie
\usepackage[
backend=biber,
style=alphabetic,
sorting=ynt
]{biblatex}

\addbibresource{bib.bib}

\usepackage{appendix}
\renewcommand{\appendixpagename}{Annexe}

\definecolor{wgrey}{RGB}{148, 38, 55}

\setlength\parindent{24pt}

\newcommand{\Z}{\mathbb{Z}}
\newcommand{\R}{\mathbb{R}}
\newcommand{\rel}{\omathcal{R}}
\newcommand{\Q}{\mathbb{Q}}
\newcommand{\C}{\mathbb{C}}
\newcommand{\N}{\mathbb{N}}
\newcommand{\K}{\mathbb{K}}
\newcommand{\A}{\mathbb{A}}
\newcommand{\B}{\mathcal{B}}
\newcommand{\Or}{\mathcal{O}}
\newcommand{\F}{\mathbb F}
\newcommand{\m}{\mathfrak m}
\renewcommand{\b}{\mathfrak b}
\renewcommand{\a}{\mathfrak a}
\newcommand{\p}{\mathfrak p}
\newcommand{\I}{\mathfrak I}
\newcommand{\Hom}{\textrm{Hom}}
\newcommand{\disc}{\textrm{disc}}
\newcommand{\Pic}{\textrm{Pic}}
\newcommand{\End}{\textrm{End}}
\newcommand{\Spec}{\textrm{Spec}}
\newcommand{\Frac}{\textrm{Frac}}

\newcommand{\cL}{\mathscr{L}}
\newcommand{\G}{\mathscr{G}}
\newcommand{\D}{\mathscr{D}}
\newcommand{\E}{\mathscr{E}}

\theoremstyle{plain}
\newtheorem{thm}{Théoreme}
\newtheorem{lem}{Lemme}
\newtheorem{prop}{Proposition}
\newtheorem{cor}{Corollaire}
\newtheorem{heur}{Heuristique}
\newtheorem{rem}{Remarque}
\newtheorem{rembis}{Remarque}
\newtheorem{note}{Note}

\theoremstyle{definition}
\newtheorem{conj}{Conjecture}
\newtheorem*{eq}{Équivalences}
\newtheorem{prob}{Problème}
\newtheorem{quest}{Question}
\newtheorem{prot}{Protocole}
\newtheorem{algo}{Algorithme}
\newtheorem{defn}{Définition}
\newtheorem{defnbis}{Définition}
\newtheorem{ex}{Exemple}
\newtheorem{exo}{Exercices}

\theoremstyle{remark}

\definecolor{wgrey}{RGB}{148, 38, 55}
\definecolor{wgreen}{RGB}{100, 200,0} 
\hypersetup{
    colorlinks=true,
    linkcolor=wgreen,
    urlcolor=wgrey,
    filecolor=wgrey
}

\title{Groupes de ramification}
\date{}

\begin{document}
\maketitle
Mes notes dans celles de corps\_locaux/decomposition\_extensions
sont vraiment bien. 

\section{Résumé}
On se place sur $L/K$ galoisienne de corps complets avec 
$Gal(L/K)=G$, en particulier $G=D_{\m_L}$. On déf $G_i$ via 
\[G_i=\ker(Gal(L/K)\to \Or_L/\m_L^{i+1})\]
i.e. $G_{-1}=G$ et $G_0=I_{\m_L}$, et $G_i\supset G_{i+1}$ ça se
traduit en
$v_L(g(x)-x)\geq i+1$ pour tout $x\in \Or_L$. Si
$\Or_L=\Or_K[\alpha]$, dans le cas des corps locaux de car $0$ 
c'est vrai, on peut juste regarder sur $\alpha$. On déf
\[i_G(g):=v_L(g(\alpha)-\alpha)\]
pour $g\in G_0$, on a $\Or_L=\Or_{L^{G_0}}[\pi_L]$ d'où 
\[i_G(g):=v_L(g(\pi_L)-\pi_L)\]
et même $g(\pi_L)/\pi_L\in U_L^{(i_G(g))}$


\chapter{Remarques}
\section{Le cas complet où $k_L-k_K$ est purement inséparable.}
On s'y en retrouve fait souvent : $L^I-L$ pour le cas galoisien
complet, $K^{un}-L$, $K^{tam}-L$. 
\section{Descriptions des $G_i$}
On a pour $g\in G_i$ la déf générale : pour tout $x$ : 
$v_L(gx-x)\geq i+1$, pour $\Or_L=\Or_K[\alpha]$, 
$v_L(g\alpha-\alpha)\geq i+1 $, et sinon pour $i\geq 0$ dans
le cas des corps locaux ma préf :
\[v_L(g(\pi_L)/\pi_L-1)\geq i\]
pour la troisième $g$ fixe $L^I$ et $L-L^I$ est totalement ramifiée
(hypothèse corps local, $k_I-k_L$ est purement inséparable donc
triviale) d'où $\Or_L=\Or_{L^I}[\pi_L]$, puis $x=\sum a_i\pi_L^i$
et (!)
\[gx-x=\sum a_i((g\pi_L)^i -\pi_L)\]
et $i_{L/K}(g)=i_{L/L^I}(g)$.

\section{Comparaison avec les $U_L^{(i)}$}
On regarde $i\geq 0$!!
Donc la troisième description est plus claire là. On regarde 
\[U_L^{(i)}=\ker(\Or_L^\times\to (\Or_L/\m_L^i)^\times)\]
on a 
$U_L^{(i)}=\begin{cases} \Or_L^\times,~i=0\\ 1+\m_L^i~sinon\end{cases}$.
on a $g\in G_i\equiv g(\pi_L)/\pi_L\in U_L^{(i)}$. Et en plus
\[G_i\to U_L^{(i)}/U_L^{(i+1)}\]
via $g\mapsto g(\pi_L)/\pi_L$ est un m.g. L'idée clé c'est que
$\sigma(x)/x\in 1+x^{-1}\m_K^{(i+1)}$ d'où $U_L^{(i+1)}$ si
$x=u\in \Or_K^\times$ et $U_L^{(i)}$ si $x=\pi_K$ (!).


\section{Les $U_L^{(i)}$ et $k_{L_0}$}
\begin{rem}
    Pour l'incompréhension, en fait pour $G_i$, $i\geq 0$
    on se restreint à $K=L_0$ d'où la flèche
    \[U_L^{(i)}/U_L^{(i+1)}\to k_{K}\]
    a la bonne image parce que $k_{L_0}=k_L$.
\end{rem}
On a $U_L^{(i)}/U_L^{(i+1)}\to
\begin{cases}k_K^\times,~i=0\\ k_K,~i>0\end{cases}$. Donnés par
$x\mapsto x$ et $1+\pi_K^ix\mapsto x$. En particulier
$G_0/G_1=I/P\hookrightarrow k_K^\times$ d'où d'ordre $\wedge p=1$
et $G_i/G_{i+1}\hookrightarrow k_K$ d'où un $p$-groupe. En 
particulier 
$|I|=|I/P|.\prod_{i=1}^\infty |G_i/G_{i+1}|=t.p^{v_p(|I|)}$.
Enfin, $L^I-L^P$ a pour groupe de galois $G_0/G_1$ est totalement
ramifiée et cyclique vu que sous-groupe fini multiplicatif d'un
corps. Ou sinon, vu que tame et a les racines de l'unité. 
\begin{rem}
    Penser que pour $g\ne g'$, $g(\pi_L)/\pi_L\ne g'(\pi_L)/\pi_L$
    d'où des racines de l'unité différente quand
    $\pi_L^e=\pi_K$.
\end{rem}



\end{document}
