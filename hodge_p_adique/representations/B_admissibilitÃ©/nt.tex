\documentclass[a4paper,12pt]{article}
\usepackage{amsmath,  amsthm,enumerate}
\usepackage{csquotes}
\usepackage[provide=*,french]{babel}
\usepackage[dvipsnames]{xcolor}
\usepackage{quiver, tikz}

%symbole caligraphique
\usepackage{mathrsfs}

%hyperliens
\usepackage{hyperref}

%pseudo-code
\usepackage{algpseudocode}
\usepackage{algorithm}
\makeatletter
  \renewcommand{\ALG@name}{Algorithme}
  \makeatother
\usepackage{fancyhdr}





\definecolor{wgrey}{RGB}{148, 38, 55}

\setlength\parindent{24pt}

\newcommand{\Z}{\mathbb{Z}}
\newcommand{\R}{\mathbb{R}}
\newcommand{\rel}{\omathcal{R}}
\newcommand{\Q}{\mathbb{Q}}
\newcommand{\C}{\mathbb{C}}
\newcommand{\N}{\mathbb{N}}
\newcommand{\K}{\mathbb{K}}
\newcommand{\A}{\mathbb{A}}
\newcommand{\D}{\mathbb{D}}
\newcommand{\B}{\mathcal{B}}
\newcommand{\Or}{\mathcal{O}}
\newcommand{\F}{\mathbb F}
\newcommand{\m}{\mathfrak m}
\renewcommand{\b}{\mathfrak b}
\renewcommand{\a}{\mathfrak a}
\newcommand{\p}{\mathfrak p}
\newcommand{\I}{\mathfrak I}
\newcommand{\Hom}{\textrm{Hom}}
\newcommand{\disc}{\textrm{disc}}
\newcommand{\Pic}{\textrm{Pic}}
\newcommand{\End}{\textrm{End}}
\newcommand{\Spec}{\textrm{Spec}}
\newcommand{\Frac}{\textrm{Frac}}

\newcommand{\cL}{\mathscr{L}}
\newcommand{\G}{\mathscr{G}}
\newcommand{\E}{\mathscr{E}}

\newcommand{\RepBp}{\underline{\textrm{Rep}}_{B}(G_K)}
\newcommand{\RepKbar}{\underline{\textrm{Rep}}_{\bar K}(G_K)}
\newcommand{\RepCK}{\underline{\textrm{Rep}}_{\C K}(G_K)}
\newcommand{\RepBcrys}{\underline{\textrm{Rep}}_{B_{\textrm{crys}}}(G_K)}
\newcommand{\RepBdR}{\underline{\textrm{Rep}}_{B_{\textrm{dR}}}(G_K)}
\newcommand{\RepBHT}{\underline{\textrm{Rep}}_{B_{\textrm{HT}}}(G_K)}
\newcommand{\Repp}{\underline{\textrm{Rep}}_{\Q_p}(G_K)}
\newcommand{\Repl}{\underline{\textrm{Rep}}_{\Q_l}(G_K)}

\theoremstyle{plain}
\newtheorem{thm}{Théoreme}
\newtheorem{lem}{Lemme}
\newtheorem{prop}{Proposition}
\newtheorem{cor}{Corollaire}
\newtheorem{heur}{Heuristique}
\newtheorem{rem}{Remarque}
\newtheorem{rembis}{Remarque}
\newtheorem{note}{Note}

\theoremstyle{definition}
\newtheorem{conj}{Conjecture}
\newtheorem*{eq}{Équivalences}
\newtheorem{prob}{Problème}
\newtheorem{quest}{Question}
\newtheorem{prot}{Protocole}
\newtheorem{algo}{Algorithme}
\newtheorem{defn}{Définition}
\newtheorem{defnbis}{Définition}
\newtheorem{ex}{Exemple}
\newtheorem{exo}{Exercices}

\theoremstyle{remark}

\definecolor{wgrey}{RGB}{148, 38, 55}
\definecolor{wgreen}{RGB}{100, 200,0} 
\hypersetup{
    colorlinks=true,
    linkcolor=wgreen,
    urlcolor=wgrey,
    filecolor=wgrey
}

\title{$\RepBp$}
\date{}

\begin{document}
\maketitle

\section{$B$ et $C$-admissibilité}
La $\Q_p$-algèbre $B$ est supposée commutative intègre.
En plus elle est $G_K$-régulière, i.e. 
$B^{G_K}=C^{G_K}=E$ et 
$\{b\in B| G_K(b.\Q_p)=b.\Q_p\}\subset B^\times$.

\subsection{$\dim_E \D_B(V)\leq\dim_{\Q_p}V$}
On peut montrer que 
\[\alpha_B\colon \D_B(\_)\otimes_{E}B\to V\otimes_{\Q_p}B\]
est injective, à noter que $(d_i\otimes 1)_i$ est lin
indép sur $C$ !
\begin{note}
  La preuve : $x=\sum d_i\otimes c_i$ de taille minimale
  et $c_m=1$ ensuite $\alpha_C$ est $G_K$-équivariant 
  d'où $gx-x$ est dans le noyau puis on conclut via la
  taille.
\end{note}

\subsection{$\alpha_B$ et admissibilité}
Si $\dim_E \D_B(V)=\dim_{\Q_p}(V)$ alors $\alpha_B$
est un isomorphisme ! L'idée est vraiment jolie. On
prends $v=(v_i)$ une base de $V$ et $d=(d_i)$ de 
$\D_B(V)$. Puis $d=A.v$ avec $A\in M_n(B)$. Maintenant
en posant $\wedge_{i=1}^n d_i=x$ et 
$\wedge_{i=1}^n v_i=y$ on obtient $x=\det(A).y$. Et
l'action induite de $G_K$ est celle d'un caractère
$\eta$! En particulier,
\[x=g.x=g(\det(A))\eta(g).y\]
d'où $g(\det(A))=\det(A)\eta(g)^{-1}\in \Q_p.\det(A)$.
Puis $A$ est inversible dans $B$!

\begin{rem}
  Ça montre la surjectivité de $\alpha_B$. Faut
  juste déterminer pq $g.y=\eta(g).y$ est un caractère
  $G_K\to \Z_p^\times$.
\end{rem}

\subsection{Ce que ça dit}
Cet isomorphisme dit qu'en fait on peut obtenir
une base $G_K$-invariante de $V\otimes_{\Q_p}B$. Vu que
si $\bigoplus_{i=1}^n d_i.K=\D_B(V)$ alors 
$\alpha_B(d_i)=d_i$ et $gd_i=d_i$.

En particulier, l'action est semi-linéaire mais triviale
sur $\D_B(V)$ d'où on peut se concentrer sur l'action
sur $B$ !

Ça ressemble donc à Hilbert 90 en un sens.

\section{$\RepBp$ est tannakienne.}

\subsection{Exactitude de $\D_B(\_)$}
Quand on a $0\to V'\to V\to V''$ on a tjr 
\[\dim_K V\leq \dim_K V' + \dim_K V''\]
et égalité si $0\to V'\to V\to V''\to 0$ est exacte.
Maintennat il s'agit juste de remarquer que si la
première suite de
% https://q.uiver.app/#q=WzAsMTcsWzEsMCwiMCJdLFsyLDAsIlYnIl0sWzMsMCwiViJdLFs0LDAsIlYnJyJdLFs1LDAsIjAiXSxbMSwxLCIwIl0sWzIsMSwiVidcXG90aW1lc197XFxRX3B9QiJdLFszLDEsIlZcXG90aW1lc197XFxRX3B9QiJdLFs0LDEsIlYnJ1xcb3RpbWVzX3tcXFFfcH1CIl0sWzUsMSwiMCJdLFsxLDIsIjAiXSxbMiwyLCJEX0IoVicpIl0sWzMsMiwiRF9CKFYpIl0sWzQsMiwiRF9CKFYnJykiXSxbMCwxXSxbMCwyXSxbMCwwXSxbMCwxXSxbMSwyXSxbMiwzXSxbMyw0XSxbNSw2XSxbNiw3XSxbNyw4XSxbOCw5XSxbMTAsMTFdLFsxMSwxMl0sWzEyLDEzXSxbMTYsMTQsIlxcb3RpbWVzX3tcXFFfcH1CIiwyLHsibGV2ZWwiOjJ9XSxbMTQsMTUsIihcXF8pXntHX0t9IiwyLHsibGV2ZWwiOjJ9XV0=
\[\begin{tikzcd}
	{} & 0 & {V'} & V & {V''} & 0 \\
	{} & 0 & {V'\otimes_{\Q_p}B} & {V\otimes_{\Q_p}B} & {V''\otimes_{\Q_p}B} & 0 \\
	{} & 0 & {\D_B(V')} & {\D_B(V)} & {\D_B(V'')}
	\arrow["{\otimes_{\Q_p}B}"', Rightarrow, from=1-1, to=2-1]
	\arrow[from=1-2, to=1-3]
	\arrow[from=1-3, to=1-4]
	\arrow[from=1-4, to=1-5]
	\arrow[from=1-5, to=1-6]
	\arrow["{(\_)^{G_K}}"', Rightarrow, from=2-1, to=3-1]
	\arrow[from=2-2, to=2-3]
	\arrow[from=2-3, to=2-4]
	\arrow[from=2-4, to=2-5]
	\arrow[from=2-5, to=2-6]
	\arrow[from=3-2, to=3-3]
	\arrow[from=3-3, to=3-4]
	\arrow[from=3-4, to=3-5]
\end{tikzcd}\]
est exacte et $V$ est $B$-admissible par alors les
deux autres sont $B$-admissible ET la suite du bas 
devient exacte. En particulier, \[\D_B(\_)\colon
\RepBp\to \textrm{Vect}_E\]
est exact.
\subsection{Admissibilité de $V'\otimes_{\Q_p}V''$}
Étant donné $V'$ et $V''$ dans $\RepBp$ on a
\begin{align*}
  \D_B(V'\otimes_{\Q_p}V'')&=(V'\otimes_{\Q_p}B\otimes_{\Q_p}V'')^{(G_K)}\\
			  &=(\D_B(V')\otimes_E B\otimes_{\Q_p}V'')^{G_K}\\
			  &=(\D_B(V')\otimes_E\D_B(V'')\otimes_E B)^{G_K}\\
\end{align*}
Et le truc de droite c'est $\D_B(V')\otimes_E\D_B(V'')$.
D'où le résultat.

\subsection{$V$ est $B$-admissible ssi $V^*$ aussi}
Soit $V\in\RepBp$ alors 
$\D_B(V)\otimes_E B\simeq V\otimes_{\Q_p}B$ d'où
\begin{align*}
  (V^*\otimes_{\Q_p}B)^{G_K}&=(\Hom_{\Q_p}(V,\Q_p)\otimes_{\Q_p}B)^{G_K}\\
			    &=(\Hom_{\Q_p}(V, B))^{G_K}\\
			    &=(\Hom_B(V\otimes_{\Q_p}B,B))^{G_K}\\
			    &=(\Hom_B(\D_B(V)\otimes_E B,B))^{G_K}\\
			    &=(\Hom_E(\D_B(V),B))^{G_K}\\
			    &=\Hom_E(\D_B(V),E)\\
			    &=\D_B(V)^*
\end{align*}
où l'avant dernière égalité est dûe au fait que les
$G_K$-invariant du Hom c'est les flèches 
$G_K$-équivariantes et que $\D_B(V)$ est $G_K$-invariant.

\subsection{Admissibilité de $\Hom_{\Q_p}(V',V'')$}
Il suffit de remarquer que \[\Hom_{\Q_p}(V',V'')\simeq
(V'\otimes_{\Q_p}(V'')^*)^*\]


\section{Les premiers exemples}
On peut regarder 
$B=\bar K,K^{un},\C_K,B_{dR},B_{crys},B_{ht},B_{st}$
et encore d'autres. Ici je regarde juste $\bar K$ et 
$\C_K$.
\subsection{$\bar K$-admissibilité}
Les représentations $\bar K$-admissibles sont les
$\rho\colon G_K\to GL_n(\Q_p)$ telles que $\rho$ 
est d'image finie. Si $V\in\RepKbar$ et $d=(d_i)_i$
engendre $\D_{\bar K}(V)$, $v=(v_i)_i$ engendre $V$.
Alors $d_i=\sum a_{ij}\otimes v_j$ et l'idée c'est
que $A$ est dans $GL_n(L)$ pour $L/K$ finie puis
$G_L$ agit triv sur $v$. D'où
\[G_K\to Gal(L/K)\to Aut_{\Q_p}(V)\]

\subsection{$\C_K$-admissibilité}
Un peu le seum de pas l'avoir vu en cours : c'est la
théorie de Sen. En gros on a 
\begin{thm}[Sen]
  $\rho\colon G_K\to Aut_{\Q_p}(V)$ est $\C_K$-admissible ssi $\rho(I_K)$ est fini.
  finie.
\end{thm}
Un très bon non-exemple c'est $\Q_p(\chi_K)=\Q_p(1)$
où $\chi_K$ est le caractère cyclotomique. Comme y'a
une infinité de ramification cette représentation
est pas $\C_K$-admissible.

\section{$B_{HT}$-admissibilité}
On pose $B_{HT}:=\C_K[t,t^{-1}]$ avec l'action :
\[g\sum a_it^i=\sum g(a_i)\chi_K^i(g)t^i\]
i.e. $t$ est la période de $\mathbb G_m$. 
\begin{rem}
  Pour rappel sur $\C$, $\C-0$ est de genre $1$ et à
  homotopie près le $\pi_1$ est engendré par $S^1$.
  D'où $t=2\pi i$ ici !
\end{rem}
\subsection{$B_{HT}$ est régulière}
D'abord $B_{HT}\subset \C_K(t)\subset \C_K((t))$ et
à droite on sait que la $G_K$-invariance est sur les
monômes ! Or $\C_K(\chi_K^i)^{G_K}=0$ si $i\ne 0$ et $K$
sinon.
Faut prouver que $B_{HT}$ est régulière. L'idée est
qu'un $b$ t.q la ligne $\Q_p.b$ est $G_K$-stable est
un monôme $a_it^i$. Faut utiliser 
$\C_K(\chi_K^i)^{G_K}=0$.
\subsection{$\Repp\to \textrm{Grad}_K$}
On a 
\[(V\otimes_{\Q_p}\C_K[t,t^{-1}])^{G_K}=(\bigoplus_{i\in\Z}V\otimes \C_Kt^i)^{G_K}=\bigoplus_{i\in\Z}(V\otimes \C_Kt^i)^{G_K}\]
parce que les $:=V\otimes \C_Kt^i$ sont $G_K$-stables.
On peut déf
\[gr^i(\D_{B_{HT}}(V)):=(V\otimes \C_Kt^i)^{G_K}\]
et obtenir
\[\Repp\to \textrm{Grad}_K\]
exact à gauche. Puis exact et fidèle en restreignant
à $\RepBHT$ par la théorie générale. 
\subsection{Graduation et critère d'admissibilité}
En gros il suffit de prouver l'isomorphisme en degré 
$0$ pour être $B_{HT}$-admissible.
\newline

Étant donné $V\in \Repp$ on forme 
$V(i)\subset V\otimes_{\Q_p}\C_K$ les $v$ tels que 
$g.v=\chi_K(g)^iv$ pour tout $g\in G_K$. Puis on a
via $v\mapsto vt^{-i}\otimes t^i$
\[V(i)\simeq gr^{-1}\D_{HT}(V)\otimes_K Kt^i\]
\begin{note}
  Le calcul est direct!
\end{note}
\noindent ensuite si $V\{i\}:=V(i)\otimes_K \C_K$ alors 
\[V\{i\}\simeq gr^{-i}\D_{HT}(V)\otimes_K\C_Kt^i\]
puis
\[\bigoplus_{i\in \Z}V\{i\}\simeq gr^0(\D_{HT}(V)\otimes_K B_{HT})\]
et le diagramme suivant 
% https://q.uiver.app/#q=WzAsNixbMCwwLCJcXGJpZ29wbHVzX3tpXFxpblxcWn1WXFx7aVxcfSJdLFsxLDAsIlZcXG90aW1lc197XFxRX3B9XFxDX0siXSxbMSwxLCJWXFxvdGltZXNfe1xcUV9wfVxcQ19LIl0sWzEsMiwiVlxcb3RpbWVzX3tcXFFfcH1CX3tIVH0iXSxbMCwxLCJncl4wKFxcRF97SFR9KFYpXFxvdGltZXNfS0Jfe0hUfSkiXSxbMCwyLCJcXERfe0hUfShWKVxcb3RpbWVzX0sgQl97SFR9Il0sWzAsMV0sWzQsMl0sWzAsNCwiIiwxLHsic3R5bGUiOnsiaGVhZCI6eyJuYW1lIjoibm9uZSJ9fX1dLFs0LDVdLFs1LDNdLFsyLDNdLFsxLDIsIiIsMSx7InN0eWxlIjp7ImhlYWQiOnsibmFtZSI6Im5vbmUifX19XV0=
\[\begin{tikzcd}
	{\bigoplus_{i\in\Z}V\{i\}} & {V\otimes_{\Q_p}\C_K} \\
	{gr^0(\D_{HT}(V)\otimes_KB_{HT})} & {V\otimes_{\Q_p}\C_K} \\
	{\D_{HT}(V)\otimes_K B_{HT}} & {V\otimes_{\Q_p}B_{HT}}
	\arrow[hook,from=1-1, to=1-2]
	\arrow[no head, from=1-1, to=2-1]
	\arrow[no head, from=1-2, to=2-2]
	\arrow[from=2-1, to=2-2]
	\arrow[from=2-1, to=3-1]
	\arrow[from=2-2, to=3-2]
	\arrow["\alpha_{HT}"',from=3-1, to=3-2]
\end{tikzcd}\]
où la flèche du haut est la restriction au degré $0$ 
ducoup. En particulier l'iso
en haut induit l'iso en bas qui est l'admissibilité vu
que $\alpha_{HT}$ respecte la graduation.

\begin{rem}
  Pour rappel l'iso a pas besoin d'être équivariant
  c'est juste un calcul de dimension. 
\end{rem}
\subsection{Poids de Hodge-Tate}
Dans la décomposition $V\otimes_{\Q_p}\C_K\simeq \bigoplus_{i\in \Z}V\{i\}$
les $i$ tels que $V\{i\}\ne 0$ sont les poids de 
Hodge-Tate. Maintenant $m_i:=\dim_{\C_K}V\{i\}$ est la 
multiplicité du poids et 
$V\{i\}\simeq \C_K(\chi_K^i)^{m_i}$. D'où
\[V\otimes_{\Q_p}\C_K\simeq \bigoplus_{i\in \Z}\C_K(i)^{m_i}.\]


\section{$B_{dR}$-admissibilité}


\section{$B_{crys}$-admissibilité}

\section{Hiérarchie}

\section{Trucs sur les complétions}
\subsection{Anneaux de valuations discrets complets}
Étant donné $A$ commutatif intègre ayant un idéal
principal maximal $\m$ : la flèche
\[\iota\colon A\to \varprojlim A/\m^n\]
est injective et $\widehat A:=\varprojlim A/\m^n$ est
un DVR complet.

\section{Trucs sur les modules}
\subsection{Relation de taille minimale}
Étant donné $M$ un $R$-module et 
\[\pi\colon R^{(I)}\to M\to 0\]
avec $\pi$ la flèche de gauche on peut regarder
$r\in \pi^{-1}(m)$ pour $m\in M$. Puis $|supp(r)|$
est fini par déf. D'où on peut regarder une relation
de taille minimale pour $m$. Par déf si 
$\sum_{j\in J} a_j m_j=m$ pour $n=|J|$ minimal
et un des $a_j$ est inversible bah $(m_j)_j$ est libre ! 
Sinon le $j=j_0$ t.q $a_{j_0}$ est inversible, $m_{j_0}$,
se réécrit en fonction des autres !!

\begin{rem}
  Donc quand y'a un corps en jeu c'est direct une famille
  libre.
\end{rem}
Par contre on voit aussi que passer au corps de fraction peut
réduire la taille des relations logique.

\subsection{Dans un sous-module}
On peut pareil demander pour $N\leq M$ un élément de $N$
d'écriture de taille minimale non nul. Vu que $N$ contient pas
forcément de générateur $m\in M$ c'est intéressant, pour le 
calcul de noyau par exemple.

\subsection{Applications aux produits tensoriels}
On peut donc prendre des écritures de taille minimales
$\sum d_i\otimes c_i$. Ça permet par exemple de faire
le trick d'Artin !

\subsection{Modules libres sont plats}
On a 
\[R^n\otimes_R B\simeq B^n\]
via
\begin{align*}
  \Hom_R(R^n\otimes_R B, C)&=\Hom_R(R^n,\Hom_R(B,C))\\
                           &=(\Hom_R(R,\Hom_R(B,C)))^n\\
                           &=(\Hom_R(B,C)^n)\\
                           &=\Hom_R(B^n,C)
\end{align*}
D'où l'isomorphisme puis le résultat vu que on peut
calculer des antécédents terme à terme pour la
surjectivité à droite.




\end{document}
