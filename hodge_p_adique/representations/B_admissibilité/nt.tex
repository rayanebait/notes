\documentclass[a4paper,12pt]{article}
\usepackage{amsmath,  amsthm,enumerate}
\usepackage{csquotes}
\usepackage[provide=*,french]{babel}
\usepackage[dvipsnames]{xcolor}
\usepackage{quiver, tikz}

%symbole caligraphique
\usepackage{mathrsfs}

%hyperliens
\usepackage{hyperref}

%pseudo-code
\usepackage{algpseudocode}
\usepackage{algorithm}
\makeatletter
  \renewcommand{\ALG@name}{Algorithme}
  \makeatother
\usepackage{fancyhdr}





\definecolor{wgrey}{RGB}{148, 38, 55}

\setlength\parindent{24pt}

\newcommand{\Z}{\mathbb{Z}}
\newcommand{\R}{\mathbb{R}}
\newcommand{\rel}{\omathcal{R}}
\newcommand{\Q}{\mathbb{Q}}
\newcommand{\C}{\mathbb{C}}
\newcommand{\N}{\mathbb{N}}
\newcommand{\K}{\mathbb{K}}
\newcommand{\A}{\mathbb{A}}
\newcommand{\B}{\mathcal{B}}
\newcommand{\Or}{\mathcal{O}}
\newcommand{\F}{\mathbb F}
\newcommand{\m}{\mathfrak m}
\renewcommand{\b}{\mathfrak b}
\renewcommand{\a}{\mathfrak a}
\newcommand{\p}{\mathfrak p}
\newcommand{\I}{\mathfrak I}
\newcommand{\Hom}{\textrm{Hom}}
\newcommand{\disc}{\textrm{disc}}
\newcommand{\Pic}{\textrm{Pic}}
\newcommand{\End}{\textrm{End}}
\newcommand{\Spec}{\textrm{Spec}}
\newcommand{\Frac}{\textrm{Frac}}

\newcommand{\cL}{\mathscr{L}}
\newcommand{\G}{\mathscr{G}}
\newcommand{\D}{\mathscr{D}}
\newcommand{\E}{\mathscr{E}}

\newcommand{\RepBp}{\underline{\textrm{Rep}}_{\Q_p,B}(G_K)}
\newcommand{\Repp}{\underline{\textrm{Rep}}_{\Q_p}(G_K)}
\newcommand{\Repl}{\underline{\textrm{Rep}}_{\Q_l}(G_K)}

\theoremstyle{plain}
\newtheorem{thm}{Théoreme}
\newtheorem{lem}{Lemme}
\newtheorem{prop}{Proposition}
\newtheorem{cor}{Corollaire}
\newtheorem{heur}{Heuristique}
\newtheorem{rem}{Remarque}
\newtheorem{rembis}{Remarque}
\newtheorem{note}{Note}

\theoremstyle{definition}
\newtheorem{conj}{Conjecture}
\newtheorem*{eq}{Équivalences}
\newtheorem{prob}{Problème}
\newtheorem{quest}{Question}
\newtheorem{prot}{Protocole}
\newtheorem{algo}{Algorithme}
\newtheorem{defn}{Définition}
\newtheorem{defnbis}{Définition}
\newtheorem{ex}{Exemple}
\newtheorem{exo}{Exercices}

\theoremstyle{remark}

\definecolor{wgrey}{RGB}{148, 38, 55}
\definecolor{wgreen}{RGB}{100, 200,0} 
\hypersetup{
    colorlinks=true,
    linkcolor=wgreen,
    urlcolor=wgrey,
    filecolor=wgrey
}

\title{$\RepBp$}
\date{}

\begin{document}
\maketitle

\section{Trucs sur les modules}
\subsection{Relation de taille minimale}
Étant donné $M$ un $R$-module et 
\[\pi\colon R^{(I)}\to M\to 0\]
avec $\pi$ la flèche de gauche on peut regarder
$r\in \pi^{-1}(m)$ pour $m\in M$. Puis $|supp(r)|$
est fini par déf. D'où on peut regarder une relation
de taille minimale pour $m$. Par déf si 
$\sum_{j\in J} a_j m_j=m$ pour $n=|J|$ minimal
et un des $a_j$ est inversible bah $(m_j)_j$ est libre ! 
Sinon le $j=j_0$ t.q $a_{j_0}$ est inversible, $m_{j_0}$,
se réécrit en fonction des autres !!

\begin{rem}
  Donc quand y'a un corps en jeu c'est direct une famille
  libre.
\end{rem}
Par contre on voit aussi que passer au corps de fraction peut
réduire la taille des relations logique.

\subsection{Dans un sous-module}
On peut pareil demander pour $N\leq M$ un élément de $N$
d'écriture de taille minimale non nul. Vu que $N$ contient pas
forcément de générateur $m\in M$ c'est intéressant, pour le 
calcul de noyau par exemple.

\subsection{Applications aux produits tensoriels}
On peut donc prendre des écritures de taille minimales
$\sum d_i\otimes c_i$. Ça permet par exemple de faire
le trick d'Artin !

\subsection{Modules libres sont plats}
On a 
\[R^n\otimes_R B\simeq B^n\]
via
\begin{align*}
  \Hom_R(R^n\otimes_R B, C)&=\Hom_R(R^n,\Hom_R(B,C))\\
                           &=(\Hom_R(R,\Hom_R(B,C)))^n\\
                           &=(\Hom_R(B,C)^n)\\
                           &=\Hom_R(B^n,C)
\end{align*}
D'où l'isomorphisme puis le résultat vu que on peut
calculer des antécédents terme à terme pour la
surjectivité à droite.



\section{$B$ et $C$-admissibilité}
La $\Q_p$-algèbre $B$ est supposée commutative intègre.
En plus elle est $G_K$-régulière, i.e. 
$B^{G_K}=C^{G_K}=E$ et 
$\{b\in B| G_K(b.\Q_p)=b.\Q_p\}\subset B^\times$.

\subsection{$\dim_E D_B(V)\leq\dim_{\Q_p}V$}
On peut montrer que 
\[\alpha_B\colon D_B(\_)\otimes_{E}B\to V\otimes_{\Q_p}B\]
est injective, à noter que $(d_i\otimes 1)_i$ est lin indép
sur $C$ !

\subsection{$\alpha_B$ et admissibilité}
Si $\dim_E D_B(V)=\dim_{\Q_p}(V)$ alors $\alpha_B$
est un isomorphisme ! L'idée est vraiment jolie. On
prends $v=(v_i)$ une base de $V$ et $d=(d_i)$ de 
$D_B(V)$. Puis $d=A.v$ avec $A\in M_n(B)$. Maintenant
en posant $\wedge_{i=1}^n d_i=x$ et 
$\wedge_{i=1}^n v_i=y$ on obtient $x=\det(A).y$. Et
l'action induite de $G_K$ est celle d'un caractère
$\eta$! En particulier,
\[x=g.x=g(\det(A))\eta(g).y\]
d'où $g(\det(A))=\det(A)\eta(g)^{-1}\in \Q_p.\det(A)$.
Puis $A$ est inversible dans $B$!

\begin{rem}
  Ça montre la surjectivité de $\alpha_B$. Faut
  juste déterminer pq $g.y=\eta(g).y$ est un caractère
  $G_K\to \Z_p^\times$.
\end{rem}

\section{$\RepBp$ est tannakienne.}

\subsection{Exactitude de $D_B(\_)$}
Quand on a $0\to V'\to V\to V''$ on a tjr 
\[\dim_K V\leq \dim_K V' + \dim_K V''\]
et égalité si $0\to V'\to V\to V''\to 0$ est exacte.
Maintennat il s'agit juste de remarquer que si la
première suite de
% https://q.uiver.app/#q=WzAsMTcsWzEsMCwiMCJdLFsyLDAsIlYnIl0sWzMsMCwiViJdLFs0LDAsIlYnJyJdLFs1LDAsIjAiXSxbMSwxLCIwIl0sWzIsMSwiVidcXG90aW1lc197XFxRX3B9QiJdLFszLDEsIlZcXG90aW1lc197XFxRX3B9QiJdLFs0LDEsIlYnJ1xcb3RpbWVzX3tcXFFfcH1CIl0sWzUsMSwiMCJdLFsxLDIsIjAiXSxbMiwyLCJEX0IoVicpIl0sWzMsMiwiRF9CKFYpIl0sWzQsMiwiRF9CKFYnJykiXSxbMCwxXSxbMCwyXSxbMCwwXSxbMCwxXSxbMSwyXSxbMiwzXSxbMyw0XSxbNSw2XSxbNiw3XSxbNyw4XSxbOCw5XSxbMTAsMTFdLFsxMSwxMl0sWzEyLDEzXSxbMTYsMTQsIlxcb3RpbWVzX3tcXFFfcH1CIiwyLHsibGV2ZWwiOjJ9XSxbMTQsMTUsIihcXF8pXntHX0t9IiwyLHsibGV2ZWwiOjJ9XV0=
\[\begin{tikzcd}
	{} & 0 & {V'} & V & {V''} & 0 \\
	{} & 0 & {V'\otimes_{\Q_p}B} & {V\otimes_{\Q_p}B} & {V''\otimes_{\Q_p}B} & 0 \\
	{} & 0 & {D_B(V')} & {D_B(V)} & {D_B(V'')}
	\arrow["{\otimes_{\Q_p}B}"', Rightarrow, from=1-1, to=2-1]
	\arrow[from=1-2, to=1-3]
	\arrow[from=1-3, to=1-4]
	\arrow[from=1-4, to=1-5]
	\arrow[from=1-5, to=1-6]
	\arrow["{(\_)^{G_K}}"', Rightarrow, from=2-1, to=3-1]
	\arrow[from=2-2, to=2-3]
	\arrow[from=2-3, to=2-4]
	\arrow[from=2-4, to=2-5]
	\arrow[from=2-5, to=2-6]
	\arrow[from=3-2, to=3-3]
	\arrow[from=3-3, to=3-4]
	\arrow[from=3-4, to=3-5]
\end{tikzcd}\]
est exacte et $V$ est $B$-admissible par alors les
deux autres sont $B$-admissible ET la suite du bas 
devient exacte. En particulier, \[D_B(\_)\colon
\RepBp\to \textrm{Vect}_E\]
est exact.
\subsection{Admissibilité de $V'\otimes_{\Q_p}V''$}
Étant donné $V'$ et $V''$ dans $\RepBp$ on a
\begin{align*}
  D_B(V'\otimes_{\Q_p}V'')&=(V'\otimes_{\Q_p}B\otimes_{\Q_p}V'')^{(G_K)}\\
			  &=(D_B(V')\otimes_E B\otimes_{\Q_p}V'')^{G_K}\\
			  &=(D_B(V')\otimes_ED_B(V'')\otimes_E B)^{G_K}\\
\end{align*}
Et le truc de droite c'est $D_B(V')\otimes_ED_B(V'')$.
D'où le résultat.

\subsection{$V$ est $B$-admissible ssi $V^*$ aussi}
Soit $V\in\RepBp$ alors 
$D_B(V)\otimes_E B\simeq V\otimes_{\Q_p}B$ d'où
\begin{align*}
  (V^*\otimes_{\Q_p}B)^{G_K}&=(\Hom_{\Q_p}(V,\Q_p)\otimes_{\Q_p}B)^{G_K}\\
			    &=(\Hom_{\Q_p}(V, B))^{G_K}\\
			    &=(\Hom_B(V\otimes_{\Q_p}B,B))^{G_K}\\
			    &=(\Hom_B(D_B(V)\otimes_E B,B))^{G_K}\\
			    &=(\Hom_E(D_B(V),B))^{G_K}\\
			    &=\Hom_E(D_B(V),E)\\
			    &=D_B(V)^*
\end{align*}
où l'avant dernière égalité est dûe au fait que les
$G_K$-invariant du Hom c'est les flèches 
$G_K$-équivariantes et que $D_B(V)$ est $G_K$-invariant.

\subsection{Admissibilité de $\Hom_{\Q_p}(V',V'')$}
Il suffit de remarquer que \[\Hom_{\Q_p}(V',V'')\simeq
(V'\otimes_{\Q_p}(V'')^*)^*\]


\section{Les bons exemples}
On peut regarder 
$B=\bar K,K^{un},\C_K,B_{dR},B_{crys},B_{ht},B_{st}$
et encore d'autres. 





\section{À faire}
Comprendre les représentations $\Z_p(1), \Z_p(i):=\Z_p(1)^{\otimes i}$
et leur lien avec les twists de faisceaux. Ducoup comprendre les 
représentation de dimension $1$/abéliennes.

Voir quelques représentations de dimension $2$, genre courbes 
elliptiques. Juste se donner une idée des méthodes.

Capter Ax-Sen-Tate qui est un des plus importants, appliquer à 
la correspondance de galois entre $E$ et $E^\flat$.

Capter Hilbert $90$ et ses applications, accessoirement 





\section{Représentations $p$-adiques}


\end{document}
