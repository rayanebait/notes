\documentclass[a4paper,12pt]{article}
\usepackage{amsmath,  amsthm,enumerate}
\usepackage{csquotes}
\usepackage[provide=*,french]{babel}
\usepackage[dvipsnames]{xcolor}
\usepackage{quiver, tikz}

%symbole caligraphique
\usepackage{mathrsfs}

%hyperliens
\usepackage{hyperref}


\definecolor{wgrey}{RGB}{148, 38, 55}

\setlength\parindent{24pt}

\newcommand{\Z}{\mathbb{Z}}
\newcommand{\R}{\mathbb{R}}
\newcommand{\rel}{\omathcal{R}}
\newcommand{\Q}{\mathbb{Q}}
\newcommand{\C}{\mathbb{C}}
\newcommand{\N}{\mathbb{N}}
\newcommand{\K}{\mathbb{K}}
\newcommand{\A}{\mathbb{A}}
\newcommand{\B}{\mathcal{B}}
\newcommand{\Or}{\mathcal{O}}
\newcommand{\F}{\mathbb F}
\newcommand{\m}{\mathfrak m}
\renewcommand{\b}{\mathfrak b}
\renewcommand{\a}{\mathfrak a}
\newcommand{\p}{\mathfrak p}
\newcommand{\I}{\mathfrak I}
\newcommand{\Hom}{\textrm{Hom}}
\newcommand{\disc}{\textrm{disc}}
\newcommand{\Pic}{\textrm{Pic}}
\newcommand{\End}{\textrm{End}}
\newcommand{\Spec}{\textrm{Spec}}
\newcommand{\Frac}{\textrm{Frac}}

\newcommand{\cL}{\mathscr{L}}
\newcommand{\G}{\mathscr{G}}
\newcommand{\D}{\mathscr{D}}
\newcommand{\E}{\mathscr{E}}

\theoremstyle{plain}
\newtheorem{thm}{Théoreme}
\newtheorem{lem}{Lemme}
\newtheorem{prop}{Proposition}
\newtheorem{cor}{Corollaire}
\newtheorem{heur}{Heuristique}
\newtheorem{rem}{Remarque}
\newtheorem{rembis}{Remarque}
\newtheorem{note}{Note}

\theoremstyle{definition}
\newtheorem{conj}{Conjecture}
\newtheorem*{eq}{Équivalences}
\newtheorem{prob}{Problème}
\newtheorem{quest}{Question}
\newtheorem{prot}{Protocole}
\newtheorem{algo}{Algorithme}
\newtheorem{defn}{Définition}
\newtheorem{defnbis}{Définition}
\newtheorem{ex}{Exemple}
\newtheorem{exo}{Exercices}

\theoremstyle{remark}

\definecolor{wgrey}{RGB}{148, 38, 55}
\definecolor{wgreen}{RGB}{100, 200,0} 
\hypersetup{
    colorlinks=true,
    linkcolor=wgreen,
    urlcolor=wgrey,
    filecolor=wgrey
}

\title{Corps perfectoides}
\date{}

\begin{document}
\maketitle
\section{Perfections}
Je note $Ring_{p,sperf}$ (resp. $Ring_{p,perf}$) la catégorie des
anneaux commutatifs semi-parfaits (resp. parfaits) de
caractéristique $p>0$. 
Étant donné un anneau commutatif unitaire semi-parfait $R$,
y'a deux adjonctions
\[(\_)_{perf}\colon Ring_{p, sperf}\leftrightharpoons Ring_{p,perf}\colon Forget\]
et
\[Forget\colon Ring_{p, perf}\leftrightharpoons Ring_{p,sperf}\colon (\_)^{perf}\]
où la première est donnée par l'universalité de 
$R\to (R)_{perf}:=\varinjlim_{\varphi}R$
($R$ est le dernier terme du diagramme $\N\to Ring$ où $i+1\to i$
est donné par $\varphi$ et $i\mapsto R$).

Et la deuxième par l'universalité de $R^{perf}\to R$.

\subsection{Description usuelle}
On a toujours $R^{perf}=\{(x_n)_{n\geq 0}| x_{n+1}^p=x_n\}$. 
Les lois sont termes à termes.

\section{Corps perfectoide}
Un corps non-archimédien (NA) $(E,|.|_E)$ est perfectoide si
\begin{enumerate}
    \item $|.|_E$ est pas discrète.
    \item $E$ est complet.
    \item $\Or_E/p$ est semi-parfait.
\end{enumerate}
Y'a une prop qui dit (en char$=0$) que $E=\widehat F$ est la
complétion d'un corps profondément ramifié $F$.

\section{Tilt}
Le tilt d'un anneau commutatif semi-parfait est donné par 
$(R/p)^{perf}$. Dans le cas de $E$ perfectoide de caractéristique 
$0$, on déf $E^\flat$ comme $Frac((\Or_E/p)^{perf})$. 
\subsection{Étude de $\Or_{E^\flat}$}
Pour $(x_n)_n\in \Or_{E^\flat}$ on lift $x_n$ en $\widehat x_n$.
Alors $\lim_{m\to \infty} \widehat x_{n+m}^{p^m}$ converge en
$x^{(n)}$ tels que 
\[(x^{(n+1)})^p=x^{(n)}\]
Maintenant la limite ensembliste
\[\varprojlim_{x\mapsto x^p}\Or_E\]
a une structure d'anneau via \[(x^{(n)})_n+(y^{(n)})_n=(\lim_{m\to\infty} (x^{(n+m)}+y^{(n+m)})^{p^m})_n\]
et $(x^{(n)})_n(y^{(n)})_n=(x^{(n)}y^{(n)})_n$. Ça donne
un isomorphisme
\[\Or_{E^\flat}\to \varprojlim_{x\to x^p}\Or_E\]
et on définit (par extension multiplicative)
\[|x/y|_{E^\flat}:=|x^{(0)}|_E/|y^{(0)}|_E\]
on a alors
\[|x|_{E^\flat}\leq |p|^{p^n}\Leftrightarrow x_n=0\]
simplement parce que $x^{(n)}=0\mod p$ d'où
\[(x^{(n)})^{p^n}=0\mod p^{p^n}\]

\subsection{Valeur absolue}
On a 
\begin{enumerate}
    \item $|E^\flat|=|E|$ de manière évidente.
    \item $|E|$ est $p$-divisible.
    \item $\m_E$ est plat.
    \item $\m_E^2=\m_E$.
\end{enumerate}
\subsubsection{Anneaux de valuations et ordre sur les idéaux}
En fait si $|x|\leq |y|$, $xy^{-1}\in\Or_{E}$ de sorte que 
\[x=(xy^{-1})y\in y\Or_E\]
d'où $(x)\subset (y)$. Pour les idéaux de type fini c'est facile
aussi. En général ca se fait bien aussi je pense. 

\subsubsection{$p$-divisibilité}
L'idée est bête on a $x=y^p+pz$ et si $|x|>|p|$ la 
$p$-divisibilité. Pour les autres cas voir le Bhatt.



\end{document}
