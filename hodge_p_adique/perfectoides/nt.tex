\documentclass[a4paper,12pt]{book}
\usepackage{amsmath,  amsthm,enumerate}
\usepackage{csquotes}
\usepackage[provide=*,french]{babel}
\usepackage[dvipsnames]{xcolor}
\usepackage{quiver, tikz}

%symbole caligraphique
\usepackage{mathrsfs}

%hyperliens
\usepackage{hyperref}

%pseudo-code
\usepackage{algpseudocode}
\usepackage{algorithm}
\makeatletter
  \renewcommand{\ALG@name}{Algorithme}
  \makeatother
\usepackage{fancyhdr}

\pagestyle{fancy}
\addtolength{\headwidth}{\marginparsep}
\addtolength{\headwidth}{\marginparwidth}
\renewcommand{\chaptermark}[1]{\markboth{#1}{}}
\renewcommand{\sectionmark}[1]{\markright{\thesection\ #1}}
\fancyhf{}
\fancyfoot[C]{\thepage}
\fancyhead[LO]{\textit \leftmark}
\fancyhead[RE]{\textit \rightmark}
\renewcommand{\headrulewidth}{0pt} % and the line
\fancypagestyle{plain}{%
    \fancyhead{} % get rid of headers
}

%bibliographie
\usepackage[
backend=biber,
style=alphabetic,
sorting=ynt
]{biblatex}

\addbibresource{bib.bib}

\usepackage{appendix}
\renewcommand{\appendixpagename}{Annexe}

\definecolor{wgrey}{RGB}{148, 38, 55}

\setlength\parindent{24pt}

\newcommand{\Z}{\mathbb{Z}}
\newcommand{\R}{\mathbb{R}}
\newcommand{\rel}{\omathcal{R}}
\newcommand{\Q}{\mathbb{Q}}
\newcommand{\C}{\mathbb{C}}
\newcommand{\N}{\mathbb{N}}
\newcommand{\K}{\mathbb{K}}
\newcommand{\A}{\mathbb{A}}
\newcommand{\B}{\mathcal{B}}
\newcommand{\Or}{\mathcal{O}}
\newcommand{\F}{\mathbb F}
\newcommand{\m}{\mathfrak m}
\renewcommand{\b}{\mathfrak b}
\renewcommand{\a}{\mathfrak a}
\newcommand{\p}{\mathfrak p}
\newcommand{\I}{\mathfrak I}
\newcommand{\Hom}{\textrm{Hom}}
\newcommand{\disc}{\textrm{disc}}
\newcommand{\Pic}{\textrm{Pic}}
\newcommand{\End}{\textrm{End}}
\newcommand{\Spec}{\textrm{Spec}}
\newcommand{\Frac}{\textrm{Frac}}

\newcommand{\cL}{\mathscr{L}}
\newcommand{\G}{\mathscr{G}}
\newcommand{\D}{\mathscr{D}}
\newcommand{\E}{\mathscr{E}}

\theoremstyle{plain}
\newtheorem{thm}{Théoreme}
\newtheorem{lem}{Lemme}
\newtheorem{prop}{Proposition}
\newtheorem{cor}{Corollaire}
\newtheorem{heur}{Heuristique}
\newtheorem{rem}{Remarque}
\newtheorem{rembis}{Remarque}
\newtheorem{note}{Note}

\theoremstyle{definition}
\newtheorem{conj}{Conjecture}
\newtheorem*{eq}{Équivalences}
\newtheorem{prob}{Problème}
\newtheorem{quest}{Question}
\newtheorem{prot}{Protocole}
\newtheorem{algo}{Algorithme}
\newtheorem{defn}{Définition}
\newtheorem{defnbis}{Définition}
\newtheorem{ex}{Exemple}
\newtheorem{exo}{Exercices}

\theoremstyle{remark}

\definecolor{wgrey}{RGB}{148, 38, 55}
\definecolor{wgreen}{RGB}{100, 200,0} 
\hypersetup{
    colorlinks=true,
    linkcolor=wgreen,
    urlcolor=wgrey,
    filecolor=wgrey
}

\title{Théorie de Hodge $p$-adique}
\date{}

\begin{document}
\maketitle
Programme: 1. ramification 2. corps perfectoides, extensions
profondément ramifiées, extensions arithmétiquement profinies.
3. corps des normes 4. représentation $p$-adiques des corps locaux.

Fontaine 1970-1980. Généralisation par Scholze.

\chapter{Préliminaires}
\section{Corps non-archimédien}

\section{Lemme de Krasner}
Soit $\alpha\in \bar K$ et $f=\mu_\alpha$. Soit aussi
$\alpha_1,...,\alpha_n$ ses racines avec $\alpha=\alpha_1$.
On note $d_\alpha:=min_{i>1}\{|\alpha-\alpha_i|\}$.
Supposons que pour $\beta\in \bar K$ on ait
\[|\beta-\alpha|_K<d_\alpha|\]
alors $K(\alpha)\subset K(\beta)$.
\section{Plus petit corps algébriquement clos complet}
Étant donné $K$ non archimédien complet. On a
\[\widehat{\hat K}=:\C_K\]
est complet et algébriquement clos. Grâce à Krasner.

\section{Action de $G_K=Gal(\bar K/K)$}
Avec la topologie discrète sur $G_K$. On a
pour tout $g,x$, $|gx|=|x|$!
On peut l'étendre à $\C_K$. Question :
\[\C_K^{G_K}=K?\]
Pas trivial mais oui. On le prouve plus tard.
C'est un théorème d'Ax-Tate.

\section{Corps locaux}
Un corps local est un corps complet de valuation
discrète et de corps résiduel fini.
\begin{rem}
    On sera souvent dans ce cas. Mais les théorèmes
    seront souvent vrai si le corps résiduel est seulement
    parfait.
\end{rem}
Maintenant $K$ est local et $|k_K|=q$.
\begin{prop}
    Pour tout $x\in k_K$ il existe un unique $[x]$
    t.q $[x]^q=[x]$. En plus la flèche, $k_K^*\to K^*$
    $x\mapsto [x]$ est un iso de $k_K^*\simeq \mu_{q-1}$.
\end{prop}

\subsection{Classification, vecteurs de Witt?}
Soit $K$ un corps local. Si $char(K)=0$, alors
$K$ est une extension finie (à iso près) de $\Q_p$.
Si $char(K)=p$ alors $K\simeq k_K((X))$ avec la
valuation $X$-adique. Sachant que $k_K=\F_q$ ducoup.

\section{Filtration}
Étant donné $K$ local, on note
\[U_K^{(i)}=\begin{cases}U_K,~i=0\\ 1+\pi_K^i\Or_K,~\geq 1\end{cases}\]
C'est juste $U_K^{(i)}=\bar B(1,i)$. On a
\[U_K^{(i)}/U_K^{(i+1)}=\begin{cases}k_K^*,~i=0\\ k_K^+~i\geq 0\end{cases}\]

\section{Vocabulaire dans le cas des corps locaux.}
Si $L/K$ est finie de corps locaux. On dit qu'elle est
\begin{itemize}
    \item Non ramifiée si $e=1$. (les corps résiduels sont finis!)
    \item Totalement ramifiée si $e=[L:K]$.
    \item Totalement modérément ramifiée si $e=[L:K]$ et
        $p\nmid e$.
    \item Totalement sauvagement ramifiée si $e=[L:K]$ et
        $e=p^k=[L:K]$.
\end{itemize}
Dans $K-\bar K$ on peut prendre $K^{un}$ l'union/le compositum
(?) des sous extensions non ramifiées. Maintenant, 
\[Gal(K^{un}/K)\simeq Gal(\bar k_K/k_K)\simeq \hat \Z\]
est engendré par le Frobenius (ça c'est bizarre y me semble
que c'est faux, neukirch class field theory première page mdr,
bah c'était faux mdr).
En particulier, on a $Frob_K\in Gal(K^{un}/K)$ t.q
\[Frob_K(x)=x^q\mod m_K\]
pour tout $x\in \Or_L$ t.q $L/K$ non ramifiée. 
On se restreint donc à $\bar K-K^{un}$ et on note
$I_K=Gal(\bar K/K^{un})$.

\begin{rem}
    Exercice, si $char K=0$, alors y'a qu'un nb fini
    d'extensions de degré $\leq n$. En caractérstique 
    positive c'est faux, faut ajouter séparable de degré
    $\leq n$ (apparemment c dur).
\end{rem}

\section{La différente}
Étant donné $L/K$ finie séparable de corps locaux. On regarde
la forme trace $L\times L\to K$. Donnée par $tr(x,y)=Tr_{L/K}(xy)$.
On note
\[\Or_L'=\{x\in L| tr(x,y)\in\Or_K,\forall y\in \Or_L\}\]
c'est un idéal fractionnaire de $L$ qui contient $\Or_L$.
On note $\D_{L/K}=(\Or_L')^{-1}$, c'est un idéal de $\Or_L$.
Ca se définit bien dans des anneaux de Dedekind généraux.
On a 
\[\D_{L/K}=\D_{L/F}\D_{F/K}\]
et on déf/a
\[v_L(\D_{L/K})=\min\{\ldots\}\]
et c'est simple vu que $\D_{L/K}=(\pi_L^m)$.
On a 
\begin{prop}
    On a 
    \begin{enumerate}
        \item Si $\Or_L=\Or_K[\alpha]$ et $f=\mu_\alpha$.
    On a $\D_{L/K}=(f'(\alpha))$!
        \item $\D_{L/K}=\Or_L$ si et seulement si $L/K$ est
    non ramifiée.
        \item $v_L(\D_{L/K})\geq e-1$, $e=e(L/K)$.
        \item $v_L(\D_{L/K})=e-1$ ssi $p\nmid e$.
    \end{enumerate}
\end{prop}
\begin{proof}[1.]
    Étant donné $\alpha_i$ les racs de $f$. Claim :
    $\sum \frac{f(X)}{(X-\alpha_i)}\frac{ \alpha_i^r}{f'(\alpha_i)}=X^r$
    pour $0\leq r\leq n-1$ (c'est presque la dérivée). Pour le prouver,
    on peut évaluer en $n$ points. On évalue en les $\alpha_i$ et c'est clair.
    Pour une racine fixée $\alpha=\alpha_1$, on a 
    \[f(X)/(X-\alpha)=\sum_{i=0}^{n-1} b_i X^i\]
    avec $b_i\in \Or_L=\Or_K[\alpha]$. Maintenant on remarque
    que les termes de
    $\sum \frac{f(X)}{(X-\alpha_i)}\frac{ \alpha_i^r}{f'(\alpha_i)}=X^r$
    sont conjugués sous l'action de $G_{L/K}$ d'où on peut calculer
    \[\sum_{k=0}^{n-1} X^kTr_{L/K}(b_k\alpha^r/f'(\alpha))=X^r\]
    puis
    \[Tr_{L/K}(b_k\alpha^r/f'(\alpha))=\begin{cases}1,~k=r\\0,~sinon\end{cases}\]
    en particulier, on a une base duale de $(\alpha^i)_{i=1,\ldots,n-1}$ pour
    la trace : $(b_k/f'(\alpha))_k$. Deuxième claim : $(b_k)_k$ engendre $\Or_L$
    sur $\Or_K$. On peut le faire en comparant les coefficients de
    \[f(X)=(X-\alpha)(\sum b_i X^i)\]
    et par induction(?). Du claim on déduit que $\Or_L'=1/f'(\alpha)\Or_L$ puis
    le résultat.
\end{proof}
\begin{proof}[2.]
    On montre non ramifiée implique $\D_{L/K}=\Or_L$. On peut prendre
    $\alpha$ tq $k_K(\bar \alpha)=k_L$. Avec $f=\mu_\alpha$. On a 
    $\bar f'(\bar \alpha)\ne 0$, d'où $f'(\alpha)\in \Or_L^\times$
    et le résultat. L'inverse est clair, on prouve que $[L:K]=f$.
\end{proof}
\begin{proof}[3.]
    On a $\Or_L=\Or_K[\pi_L]$. Et $\mu_{\pi_L}=f$ Eisenstein.
    D'où $f'(\pi_L)=\sum (e-i)\pi_L^{e-i-1}$ et $v_L(e\pi_L^{e-1})=
    v_L(e)+(e-1)$ plus $v_L(a_i)\geq e$ d'où $v_L(f'(\pi_L))\geq e-1$.
    
\end{proof}
\begin{proof}[4.]
    Si $p\nmid e$ on a $v_L(e\pi_L^{e-1})=e-1<v_L(a_i)$. Si $p\mid e$
    on a $v_L(f'(\pi_L))\geq e$.
\end{proof}
\begin{proof}[5. (?)]
    Si $K-L_0-L$ est tq $L_0=K^{un}$ alors $\D_{L/K}=\D_{L/L_0}\D_{L_0/K}=\D_{L/L_0}$.
    Alors $\D_{L/K}\Or_L\equiv L=L_0\equiv L/K$ est non ramifiée.
\end{proof}
\section{Filtration de ramification}
On regarde $L/K$ galoisienne ($G$) de corps locaux. Alors
pour $i\geq -1$, $G_i=\{g\in G|\forall x\in \Or_L,~v_L(g(x)-x)\geq i+1\}$.
On voit facilement que c'est des sous groupes distingués de $G$.
On a en plus $G_{-1}=G$ et $G_0=I_K$. Puis pour $i\geq 0$, on a 
\[G_i=\{g\in G|v_L((g(\pi_L)/\pi_L)-1)\geq i\}\]
(ca marche psq $i\geq 0$)
on écrit, $\sum a_i\pi_L^i=x\in \Or_L$ avec
$a_i\in \Or_{L_0}$ ($L-L_0$ tot ram!) d'où $g(a_j)=a_j$ pour $g\in G_i$
vu que $G_i\subset I_K$. Ensuite c'est comme d'hab.
Maintenant on a 
\[G_i\to U_L^{(i)}/U_L^{(i+1)}\]
via $g\mapsto g(\pi_L)/\pi_L$ qui induit un morphisme
injectif $G_i/G_{i+1}$. En particulier 
\[G_i/G_{i+1}\textrm{ est abélien.}\]
on en déduit que les extensions galoisiennes de corps
locaux sont résolubles.
\begin{rem}
    Dans la somme, le $-1$ apparaît car quand $|G_{i_0}|=1$,
    on fait $-i_0-1$ et y'a $i_0$ termes jusqu'ici.
\end{rem}
\begin{prop}
    $v_L(\D_{L/K})=\sum_{i=0}^\infty(|G_i|-1)$.
\end{prop}
\begin{proof}
    On prend $\alpha\in \Or_L$ tq $\Or_L=\Or_K[\alpha]$ et $f=\mu_\alpha$.
    On déf $i_G(g)=v_L(g\alpha-\alpha)$. On a 
    \[i_G(g)=i+1\equiv g\in G_i\]
    puis
    \begin{align*}
    v_L(\D_{L/K})=v_L(f'(\alpha))&=\sum_{i=2}^nv_L(\alpha-\alpha_i)\\
                                 &=\sum_{g\in G-id}v_L(\alpha-g\alpha)\\
                                 &=\sum_{g\in G-id}i_G(g)\\
                                 &=\sum_{i=0}^\infty (|G_i|-|G_{i+1}|)(i+1)\\
                                 &=\sum_{i=0}^\infty(|G_i|-1)
    \end{align*}
\end{proof}

\begin{lem}
    Sur $L/K$ galoisienne de corps locaux, avec 
    $\Or_L=\Or_K[\alpha]$. On a 
    \begin{enumerate}
        \item $i_{L/K}(g_1g_2)\geq min(i_{L/K}(g_1),i_{L/K}(g_2))$
        \item $i_{L/K}(g_2g_1g_2^{-1})=i_{L/K}(g_1)$
        \item $g\in G_i\Leftrightarrow i_{L/K}(g)\geq 1+i$
        \item $v_L(\D_{L/K})=\sum_{i=0}^\infty (|G_i|-1)=\sum_{g\ne id} i_{L/K}(g)$
    \end{enumerate}
\end{lem}
\begin{proof}
    Pour $1.$ on écrit juste
    $g_1g_2(\alpha)-\alpha=
    g_1(g_2\alpha-\alpha)+g_1(\alpha)-\alpha$. Le $2.$ est
    clair.
\end{proof}

\chapter{Ramification en tours}
\section{Ramification dans les tours d'extensions}
On regarde $L-F-K$ avec $H=Gal(L/F)\leq G$ normal. On a 
$H_i=G_i\cap H$ directement vu que $i_{L/E}$ dépend de $v_L$.
On a en plus $\D_{L/K}=\D_{L/F}\D_{F/K}$ d'où $v_L(\D_{L/K})=v_L(
\D_{L/F})+v_L(\D_{F/E})=v_L(\D_{L/F})+e_{L/F}v_F(\D_{F/K})$
où à droite on prends $v_L,v_F$ normalisées. Maintenant
\begin{align*}
    v_L(\D_{L/F})&=\sum_{h\ne 1,h\in H}i_{L/F}(h)\\
                 &=\sum_{h\ne 1,h\in H} i_{L/K}(h)\\
\end{align*}
en corollaire et via $v_L(\D_{L/K})=\sum_{g\ne id} i_{L/K}(g)$ :
\begin{align*}
    e_{L/F}v_F(\D_{F/K})&=\sum_{g\in G-H}i_{L/K}(g)\\
\end{align*}
\begin{prop}
    Pour tout $s\in G/H$, $e_{L/F}i_{F/K}(s)=\sum_{g\in sH} i_{L/K}(g)$.
\end{prop}
\begin{proof}
    On note $\Or_L=\Or_F[\alpha]$ et $\mu_\alpha=f\in\Or_F[X]$.
    On regarde $s(f)(X)-f(X)$, on a 
    $s(f)-f=0\mod \pi_F^{i_{F/K}(s)}$. D'où 
    \[v_L(s(f)(\alpha))\geq e_{L/F}i_{F/K}(s)\]
    maintenant si $f(X)=\prod_{h\in H}(X-h(\alpha))$ alors
    $s(f)(X)=\prod_{g\in sH}(X-g(\alpha))$ puis 
    $s(f)(\alpha)=\prod_{g\in sH}(\alpha-g(\alpha))$. On en dduit
    que 
    \[\sum_{g\in sH}v_L(g(\alpha)-\alpha)\geq e_{L/K}i_{F/K}(s)\]
    et on utilise le dernier corollaire pour obtenir :
    \begin{align*}
        \sum_{s\ne 1}(\sum_{g\in\pi^{-1}s=sH}i_{L/K}(g))&=\sum_{g\in G-H} i_{L/K}(g)\\
                                                        &=e_{L/F}v_F(\D_{F/K})\\
                                                        &=e_{L/F}\sum_{s\ne 1}i_{F/K}(s)
    \end{align*}
    et on en déduit l'autre inégalité grâce à la première.
\end{proof}
Maintenant, on considère $j(s)=\max(i_{L/K}(g);\bar g=s)$.
Il existe $\tilde g\in G$ tel que $j(s)=i_{L/K}(\tilde g)$.
Maintenant $s=\tilde gH$ d'où $g=\tilde g.h$ pour $g\in sH$. 
On peut en déduire que
\[i_{L/K}(g)\geq \min\{i_{L/K}(\tilde g),i_{L/K}(h)\}\]
et 
\[i_{L/K}(h)\geq \min\{i_{L/K}(\tilde g^{-1})=i_{L/K}(\tilde g),i_{L/K}(g)\}\]
puis comme $i_{L/K}(\tilde g)\geq i_{L/K}(g)$ on a
\[\min\{i_{L/K}(\tilde g),i_{L/K}(h)\}\geq i_{L/K}(g)\]
d'où $i_{L/K}(g)=\min\{i_{L/K}(\tilde g),i_{L/K}(h)\}$.
\begin{cor}
    $e_{L/K}i_{F/K}(s)=\sum_{h\in H}\min\{j(s),i_{L/F}(h)\}$.
\end{cor}
On note $G_x=G_{[x]+1}$ pour $x\in [-1,+\infty[$. Puis
\[\varphi_{L/K}(x)=\begin{cases} x,~x\in [-1,0]\\
    \int_0^x \frac{dt}{(G_0:G_t)},~x\geq 0
\end{cases}\]
qui est la première fonction de Herbrand. En notant $g_m=|G_m|$
on a 
\[\varphi_{L/K}(x)=\frac{1}{g_0}(g_1+\ldots+g_m+g_{m+1}(x-m))\]
pour $m<x\leq m+1$.

\begin{lem}
    $\varphi_{L/K}(x)=\frac{1}{g_0}\sum_{g\in G,g\ne 1}\min\{i_{L/K}(g),x+1\}-1$
\end{lem}
\begin{proof}
    La fonction à droite est continue et vaut 
    \[\frac{1}{g_0}\sum_{g\ne 1}\min\{i_{L/K}(g),0\}-1=-1\]
    en $-1$ donc coincide avec $\varphi_{L/K}(-1)$. En plus en
    comparant leurs dérivées on voit qu'elles sont égales. À
    gauche on a $\varphi_{L/K}'(x)=\frac{1}{(G_0:G_x)}$ et
    si $m<x<m+1$ alors $\varphi_{L/K}(x)=g_{m+1}/g_0$. Et
    à droite la dérivée c'est $1$ ou $0$ en fonction de $g$,
    et surtout $i_{L/K}(g)\geq m+2$ implique $g\in G_{m+1}$ donc
    via la somme on est bon. La dérivée compte les $g$ dans 
    $G_{m+1}$ entre $m+1<x+1<m+2$.
\end{proof}

\begin{lem}
    Pour tout $s\in G/H - id$, 
    $\varphi_{L/F}(j(s)-1)=i_{F/K}(s)-1$.
\end{lem}
\begin{proof}
    On a $e_{L/F}i_{F/K}(s)=\sum_{h\in H}\min\{j(s),i_{L/K}(h)\}$
    et 
\[\varphi_{L/F}(j(s)-1)=\frac{1}{h_0=e_{L/F}}\sum_{h\in H-\{id\}} \min\{i_{L/K}(h),j(s)\}-1\]
    d'où le résultat en comparant.
\end{proof}

\begin{thm}[Théorème de Herbrand]
    Pour tout $x\geq 1$ on a $(G/H)_{\varphi_{L/K}(x)}=G_xH/H(=
    G_x/H\cap G_x=G_x/H_x)$.
\end{thm}
\begin{rem}
    Au sens des groupes de ramifications pas des coinvariants mdr.
\end{rem}
\begin{proof}
    Pour $s\in (G/H)_{\varphi_{L/K}(x)}$ équivaut à 
    $i_{F/K}(s)\geq \varphi_{L/F}(x)+1$ d'où par le dernier
    lemme ça équivaut à
    \[\varphi_{L/F}(j(s)-1)\geq \varphi_{L/F}(x)\]
    mais la fonction de herbrand est strictement croissante d'où
    $j(s)\geq x+1$, i.e. $\max\{i_{L/K}(g)|\bar g=s\}$, et
    il existe $\tilde g$ qui atteint le max. Ça se traduit en
    $i_{L/K}(\tilde g)\in G_x$ et $\overline{\tilde g}=s$. D'où
    le résultat.
\end{proof}
On note maintenant $\psi_{L/K}\colon [-1,\infty[\to \R$ ($[-1,\infty[\to [-1,\infty[$ même) l'inverse de $\varphi_{L/K}$.
\begin{thm}
    $\varphi_{L/K}=\varphi_{F/K}\circ\varphi_{L/F}$ et
    $\psi_{L/K}=\psi_{L/F}\circ\psi_{F/K}$.
\end{thm}
\begin{proof}
    On montre que la première. Elles ont la même valeur à gauche
    et à droite. On calcule la dérivée. On a
    \begin{align*}
        \varphi_{F/K}\circ\varphi_{L/F}'(x)&=\varphi_{F/K}'(x).\varphi'_{L/F}(x)\\
                                           &=\frac{1}{((G/H)_0\colon (G/H)_{\varphi_{L/F}(x)})}.\frac{1}{(H_0:H_x)}\\
                                           &=\frac{1}{((G/H)_0:(G_x/H_x))}.\frac{1}{(H_0:H_x)}\\
                                           &=\frac{1}{(G_0:G_x)}\\
                                           &=\varphi_{L/K}'(x)
    \end{align*}

\end{proof}
\begin{defn}
    On définit les groupes d'énumération supérieurs comme
    \[G^{(x)}=G_{\psi_{L/K}=(x)}\]
    de manière équivalente 
    \[G^{(\varphi_{L/K}(x))}=G_x\]
\end{defn}
\begin{thm}[Hasse-Herbrand]
    On a $(G/H)^{(x)}=G^{(x)}H/H$.
\end{thm}
\begin{proof}
    La preuve est directe : $(G/H)^{(x)}=(G/H)_{\psi_{F/K}}(x)$.
    Comme $(G/H)_{\varphi_{L/F}(x)}=G_xH/H$ on remplace $x$
    par
    $\varphi^{-1}_{L/F}\circ \psi_{F/K}(x)=\psi_{L/F}\circ\psi_{F/K}(x)=\psi_{L/K}(x)$. D'où le résultat. On a utilisé
    le théorème 2!
\end{proof}

On a $G_K=Gal(K^{sep}/K)=\varprojlim_{L/K~finie}Gal(L/K)$. 
Maintenant le théorème de Hasse-Herbrand la projection
\[Gal(L/K)\to Gal(F/K)\]
induit une surjection $Gal(L/K)^{(x)}\to Gal(F/K)^{(x)}$. Et
même \[G_K^{(x)}=\varprojlim_{L/K}Gal(L/K)^{(x)}\] une filtration
de $G_K$!! On définit maintenant les sauts de ramification.
\begin{defn}
    La filtration de ramification a un saut en $v\geq -1$ si
    $G_K^{v+\epsilon}\subsetneq G_K^{(v)}$ pour tout 
    $\epsilon \geq 0$.
\end{defn}
\begin{rem}
    Dans cette notation $v$ est rationnel aux sauts. Et apparemment
    en fait y'a des sauts à TOUT les rationnels ptdr. Donc la
    filtration est très compliquée.
\end{rem}
\begin{ex}
    On regarde
% https://q.uiver.app/#q=WzAsMixbMCwxLCJcXFFfcCJdLFswLDAsIks6PVxcUV9wKFxcemV0YV97cF5ufSkiXSxbMCwxXSxbMSwwLCJHPUdhbChLL1xcUV9wKSIsMCx7Im9mZnNldCI6LTMsImN1cnZlIjotNX1dXQ==
\[\begin{tikzcd}
	{K:=\Q_p(\zeta_{p^n})} \\
	{\Q_p}
	\arrow["{G=Gal(K/\Q_p)}", shift left=3, curve={height=-30pt}, from=1-1, to=2-1]
	\arrow[from=2-1, to=1-1]
\end{tikzcd}\]
    et on a $\chi\colon G\simeq(\Z/p^n\Z)^\times$ via le
    caractère cyclotomique, donné par 
    $g(\zeta_{p^n})=\zeta_{p^n}^{\chi(g)}$. On a une filtration
    naturelle 
    $\Gamma[m]=\ker((\Z/p^n\Z)^\times\to(\Z/p^m\Z)^\times)$.
    On sait que $K/\Q_p$ est totalement ramifiée et 
    $\pi=\zeta_{p^n}-1$ est une uniformisante de $K$ car racine
    d'un polynôme d'eisenstein (!!!). On a 
\[g\pi-\pi=g\zeta_{p^n}-\zeta_{p^n}=\zeta_{p^n}(\zeta_{p^n}^{\chi(g)-1}-1)\]
d'où $g\in G_m$ ssi $v_K(\zeta_{p^n}^{\chi(g)-1}-1)\geq m+1$.
En traduisant : via 
\[\chi(g)\equiv 1\mod p^m\Leftrightarrow \chi(g)\in \Gamma[m]\]
    on a \[\zeta_{p^n}^{\chi(g)-1}-1=\zeta_{p^n}^{p^m.u}-1\equiv 
    v((\zeta_{p^n}-1)^{p^m})=0\mod (\zeta_{p^{n-m}}-1)\]
    pour des unités $u,v$. En particulier
    $\chi(G_m)=\Gamma(k]$
    tel que $p^{k-1}\leq m< p^k$. On peut maintenant calculer
    la fonction de Hasse-Herbrand : on a des sauts en $p^i-1$
    et elle vaut $\varphi_{K/\Q_p}(p^i-1)=i$. En énumération
    supèrieure on a $G^{(x)}=\Gamma[x]$!
\end{ex}

\begin{prop}
    $v_K(\D_{L/K})=\int_{-1}^{+\infty}(1-\frac{1}{|G^{(v)}|})dv$
    avec $v_K$ normalisée.
\end{prop}
\begin{proof}
    Preuve formelle.
\end{proof}

Étant donné $G_K$ le groupe de galois absolu de $K$, on définit
pour $L/K$ tq $Gal(\bar K/L)=H$: $G^{(v)}:=G_K^{(v)}H/H$ puis
$\bar K^{(v)}=(\bar K)^{G^{(v)}}$. Pour $L/K$, on définit
$L^{(v)}=L\cap \bar K^{(v)}=L^{G^{(v)}}$. Alors 
$|G^{(v)}|=[L:L^{(v)}]$ et on a
\[v_K(\D_{L/K})=\int_{-1}^{\infty}(1-\frac{1}{[L:L^{(v)}]})dv\]
qui fait sens même si $L/K$ est pas galoisienne, on utilise 
seulement que $\bar K/L$ est galoisienne.
\begin{rem}
    C'est vrai pour toute extension séparable $L/K$.
\end{rem}

\chapter{Groupes de galois de corps locaux}
On étudie toujours $Gal(\bar K/K)$ pour $K$ local. On a une
correspondance groupes corps pour certains groupes/corps. 
En premier $K^{un}/K$ tq 
$Gal(K^{un}/K)\simeq Gal(\bar k_K/k_K)\simeq \hat Z$. Aussi,
$K^{tam}=\cup_{e\wedge p=1} K^{un}[\pi_K^{1/e}]$. Et on a 
via la théorie de Kummer que 
\[Gal(K^{tam}/K^{un})\simeq \varprojlim_{e\wedge p=1}(\Z/e\Z)\simeq
\prod_{l\ne p} \Z_l\]
maintenant $Gal(\bar K/K^{tam})=P_K$ est une limite projective
de $p$-groupes finis. On sait apparemment donner générateurs et
relations. Mais ça définit pas le corps uniqument.
\begin{thm}[Mochizuki,Abrashkin]
    $K$ équivaut à $G_K$ muni de la filtration de ramification.
\end{thm}
\begin{exo}
    Montrer que
    \[P_K=Gal(\bar K/K^{tam})=\overline{\cup_{\epsilon>0}G_K^{(\epsilon)}}\]
\end{exo}
\section{Corps de classe local (Hasse, Artin-Tate)}
Une extension $L/K$ est abélienne si galoisienne de groupe de 
galois abélien. On note $K^{ab}$ le compositum des extensions
abéliennes finies. On a 
\[Gal(K^{ab}/K)=G_K/[G_K:G_K]\]
l'abéllianisé topologique (faut prendre la clôture des 
commutateurs).
\begin{thm}
    Il existe un morphisme injectif
    \[\theta_K\colon K^\times\to Gal(K^{ab}/K)\]
    de groupes topologiques tel que
    \begin{enumerate}
        \item $\theta_K$ est continu d'image dense. Pour remarque,
            ça peut pas être surj car compact à droite et pas
            compact à gauche.
        \item Soit $L/K$ une extension finie abélienne et
            soit $N_{L/K}\colon L^\times\to K^\times$ la norme.
            On a un carré (!)
% https://q.uiver.app/#q=WzAsNCxbMCwwLCJLXlxcdGltZXMiXSxbMSwwLCJHYWwoS157YWJ9L0spIl0sWzEsMSwiR2FsKEwvSykiXSxbMCwxLCJLXlxcdGltZXMvTl97TC9LfShMXlxcdGltZXMpIl0sWzAsMSwiXFx0aGV0YV9LIiwwLHsic3R5bGUiOnsidGFpbCI6eyJuYW1lIjoiaG9vayIsInNpZGUiOiJ0b3AifX19XSxbMSwyLCIiLDAseyJzdHlsZSI6eyJoZWFkIjp7Im5hbWUiOiJlcGkifX19XSxbMCwzLCIiLDIseyJzdHlsZSI6eyJoZWFkIjp7Im5hbWUiOiJlcGkifX19XSxbMywyLCJcXHNpbWVxIiwxLHsic3R5bGUiOnsiYm9keSI6eyJuYW1lIjoibm9uZSJ9LCJoZWFkIjp7Im5hbWUiOiJub25lIn19fV1d
\[\begin{tikzcd}
	{K^\times} & {Gal(K^{ab}/K)} \\
	{K^\times/N_{L/K}(L^\times)} & {Gal(L/K)}
	\arrow["{\theta_K}", hook, from=1-1, to=1-2]
	\arrow[two heads, from=1-1, to=2-1]
	\arrow[two heads, from=1-2, to=2-2]
	\arrow["\simeq"{description}, draw=none, from=2-1, to=2-2]
\end{tikzcd}\]
        \item Le carré
% https://q.uiver.app/#q=WzAsNixbMCwwLCJLXlxcdGltZXMiXSxbMSwwLCJHYWwoS157YWJ9L0spIl0sWzEsMSwiR2FsKEtee3VufS9LKSJdLFswLDEsIlxcWiJdLFswLDIsIjEiXSxbMSwyLCJGcm9iX0siXSxbMCwxLCJcXHRoZXRhX0siLDAseyJzdHlsZSI6eyJ0YWlsIjp7Im5hbWUiOiJob29rIiwic2lkZSI6InRvcCJ9fX1dLFsxLDIsIiIsMCx7InN0eWxlIjp7ImhlYWQiOnsibmFtZSI6ImVwaSJ9fX1dLFswLDMsInZfSyIsMix7InN0eWxlIjp7ImhlYWQiOnsibmFtZSI6ImVwaSJ9fX1dLFszLDIsIlxcc2ltZXEiLDEseyJzdHlsZSI6eyJib2R5Ijp7Im5hbWUiOiJub25lIn0sImhlYWQiOnsibmFtZSI6Im5vbmUifX19XSxbNCw1LCIiLDIseyJzdHlsZSI6eyJ0YWlsIjp7Im5hbWUiOiJtYXBzIHRvIn19fV1d
\[\begin{tikzcd}
	{K^\times} & {Gal(K^{ab}/K)} \\
	\Z & {Gal(K^{un}/K)} \\
	1 & {Frob_K}
	\arrow["{\theta_K}", hook, from=1-1, to=1-2]
	\arrow["{v_K}"', two heads, from=1-1, to=2-1]
	\arrow[two heads, from=1-2, to=2-2]
	\arrow["\simeq"{description}, draw=none, from=2-1, to=2-2]
	\arrow[maps to, from=3-1, to=3-2]
\end{tikzcd}\]
    \end{enumerate}
    commute.
\end{thm}
On avait définit 
$U_K^{(i)}=\ker(\Or_K^\times\to (\Or_K/\m_K^i)^\times)$. On définit
\[U_K^{(x)}:=U_K^{(x)}=U_K^{(i)},i-1<x\leq i\]
alors (!)
\[\theta_K(U_K^{(x)})=Gal(K^{ab}/K)^{(x)}\]
\begin{rem}
il y'a 3 preuves apparemment :
\begin{enumerate}
    \item via les corps globaux (?!) et le corps de classe global.
    \item preuves locales difficiles. (dwork)
    \item Lubin-Tate theory (!). (donne explicitement une 
        construction $K^{ab}$)
\end{enumerate}
\end{rem}

Si $L/K$ est abélienne (potentiellement infinie) alors $G=Gal(L/K)
\simeq G_K^{ab}/H$ et $G^{(x)}\simeq (G_K^{ab})^{(x)}H/H$. 
Maintenant $G$ a un saut de ramification en $v$ si pour tout 
$\epsilon>0$, $G^{(v+\epsilon)}\ne G^{(v)}$. La filtration
$U_K^{(i)}$ sur $U_K$ a des sauts a tout entiers $i\geq 0$ d'où
$G_K^{ab}$ aussi par le théorème. En particulier, $G^{(x)}$
saute aussi qu'aux entiers (Hasse-Arf). Maintenant, 
\[U_K^{(i)}/U_K^{(i+1)}\]
est $p$-élémentaire pour $i\geq 1$, d'où pareil pour
$(G_K^{ab})^{(i)}/(G_K^{ab})^{(i+1)}$ (!) pour $i\geq 1$.
On note $v_0,v_1,\ldots$ les sauts de ramifications de $G=Gal(L/K)$
alors $G=G^{(v_0)}\supset G^{(v_1)}\supset\ldots$
et $G^{(v_i)}/G^{(v_{i+1})}$ sont $p$-élementaires.

\section{Ramification dans les $\Z_p$-extensions}
\begin{defn}
    Une $\Z_p$-extension est galoisienne de groupe de galois
    $\Z_p$.
\end{defn}
\begin{note}
    On note $K_\infty/K$ une $\Z_p$-extension totalement ramifiée
    de corps locaux de caractéristique $0$. Et 
    $\Gamma=Gal(K_\infty/K)\simeq\Z_p$. On note $\Gamma(n)\simeq
    p^n\Z_p$. On note $K_n=K_\infty^{\Gamma(n)}$.
\end{note}
\begin{thm}[Tate]
    Soit $e_K=e(K/\Q_p)$
    \begin{enumerate}
        \item Il existe $i_0\in \Z$ tq $v_{i+1}=v_i+e_K$ pour
            $i\geq i_0$.
        \item Il existe $c$ tq pour tout $n\geq 1$
            \[v_K(\D_{K_n/K})=c+e_Kn+\frac{a_n}{p^n}\]
            où $(a_n)_n$ est bornée.
    \end{enumerate}
\end{thm}
\begin{lem}
    Dans $K/\Q_p$, $e_K=e(K/\Q_p)$. On note
    \begin{enumerate}
        \item $\log(1+x):=\sum_{m\geq 1} (-1)^{m+1}\frac{x^m}{m}$
            converge si $x\in \m_K$ et a valeur dans $K$ (!).
        \item $\exp(x):=\sum_{n\geq 1} \frac{x^n}{n!}$ converge
            sur $\m_K^r$ avec $r>e_K/(p-1)$ entier. 
        \item Pour tout $r$ comme avant, $\log(1+x)$ et $\exp(x)$
            sont des homomorphismes 
            \[\log\colon U_K^{(r)}\to \m_K^r\]
            et \[\exp\colon \m_K^r\to U_K^{(r)}\]
            inverses l'un de l'autre.
    \end{enumerate}
\end{lem}
\begin{proof}
    Exercice (!).
\end{proof}
\begin{cor}
    Pour tout $r>\frac{e_K}{p-1}$, 
    \[(U_K^{(r)})^p=U_K^{(r+e_K)}\]
\end{cor}
\begin{proof}
    $\log((U_K^{(r)})^p)=p\m_K^r=\m_K^{r+e_K}=\log(U_K^{(r+e_K)})$.
    (!)
\end{proof}
\begin{proof}[Théorème 6, (i)]
    Cours. Dans
    % https://q.uiver.app/#q=WzAsNyxbMCwzLCJLIl0sWzAsMiwiS157dW59Il0sWzAsMSwiS157YWIsdGFtfSJdLFswLDAsIktee2FifSJdLFszLDEsIktfXFxpbmZ0eSJdLFszLDIsIksiXSxbMywwLCJLXnthYn0iXSxbMiwzXSxbMSwyXSxbMCwxXSxbMywyLCJVX0teeygxKX0iLDAseyJvZmZzZXQiOi0zLCJjdXJ2ZSI6LTMsInN0eWxlIjp7ImhlYWQiOnsibmFtZSI6Im5vbmUifX19XSxbMiwxLCJVX0svVV9LXnsoMSl9IiwwLHsib2Zmc2V0IjotMywiY3VydmUiOi0zfV0sWzEsMCwiXFxoYXQgXFxaIiwwLHsib2Zmc2V0IjotMywiY3VydmUiOi0yfV0sWzUsNF0sWzQsNl0sWzQsNSwiXFxaX3AiLDAseyJvZmZzZXQiOi0zLCJjdXJ2ZSI6LTIsInN0eWxlIjp7ImhlYWQiOnsibmFtZSI6Im5vbmUifX19XSxbNiw0LCJHIiwwLHsib2Zmc2V0IjotMywiY3VydmUiOi0yfV1d
\[\begin{tikzcd}
	{K^{ab}} &&& {K^{ab}} \\
	{K^{ab,tam}} &&& {K_\infty} \\
	{K^{un}} &&& K \\
	K
	\arrow["{U_K^{(1)}}", shift left=3, curve={height=-18pt}, no head, from=1-1, to=2-1]
	\arrow["G", shift left=3, curve={height=-12pt}, from=1-4, to=2-4]
	\arrow[from=2-1, to=1-1]
	\arrow["{U_K/U_K^{(1)}}", shift left=3, curve={height=-18pt}, from=2-1, to=3-1]
	\arrow[from=2-4, to=1-4]
	\arrow["{\Z_p}", shift left=3, curve={height=-12pt}, no head, from=2-4, to=3-4]
	\arrow[from=3-1, to=2-1]
	\arrow["{\hat \Z}", shift left=3, curve={height=-12pt}, from=3-1, to=4-1]
	\arrow[from=3-4, to=2-4]
	\arrow[from=4-1, to=3-1]
\end{tikzcd}\]
    et $K_\infty\cap K^{ab,tam}=K$ (wild a gauche). L'idée pour
    les groupes de galois : via $\theta_K$, $U_K^{(0)}$ correspond
    à l'inertie (car $K^un$ correspond à l'inertie), et $U_K^{(1)}$
    est maximal pro-$p$ via les 
    quotients :
% https://q.uiver.app/#q=WzAsMTAsWzAsMSwiMCJdLFsxLDEsIkdhbChLXnthYn0vS157YWIsdGFtfSkiXSxbMSwwLCJVX0teeygxKX0iXSxbMiwxLCJHX0tee2FifSJdLFszLDEsIkdhbChLXnthYix0YW19KS9LIl0sWzIsMiwiXFxHYW1tYSJdLFszLDIsIjEiXSxbNCwxLCIwIl0sWzAsMiwiXFxleGlzdHMgIE5cXHN1YnNldCBVX0teeygxKX06Il0sWzEsMiwiVV9LXnsoMSl9L04iXSxbMCwxXSxbMSwyLCJcXHNpbWVxIiwxLHsic3R5bGUiOnsiYm9keSI6eyJuYW1lIjoibm9uZSJ9LCJoZWFkIjp7Im5hbWUiOiJub25lIn19fV0sWzEsM10sWzMsNF0sWzUsNl0sWzQsN10sWzMsNV0sWzksNV0sWzEsOV0sWzYsNCwiXFxiaWdjdXAiLDEseyJzdHlsZSI6eyJib2R5Ijp7Im5hbWUiOiJub25lIn0sImhlYWQiOnsibmFtZSI6Im5vbmUifX19XV0=
\[\begin{tikzcd}
	& {U_K^{(1)}} \\
	0 & {Gal(K^{ab}/K^{ab,tam})} & {G_K^{ab}} & {Gal(K^{ab,tam})/K} & 0 \\
	{\exists  N\subset U_K^{(1)}:} & {U_K^{(1)}/N} & \Gamma & 1
	\arrow[from=2-1, to=2-2]
	\arrow["\simeq"{description}, draw=none, from=2-2, to=1-2]
	\arrow[from=2-2, to=2-3]
	\arrow[from=2-2, to=3-2]
	\arrow[from=2-3, to=2-4]
	\arrow[from=2-3, to=3-3]
	\arrow[from=2-4, to=2-5]
	\arrow["\simeq",from=3-2, to=3-3]
	\arrow[from=3-3, to=3-4]
	\arrow["\bigcup"{description}, draw=none, from=3-4, to=2-4]
\end{tikzcd}\]
ensuite $\Gamma^{(v)}$ correspond à 
$U_K^{(v)}N/N\simeq U_K^{(v)}/(U_K^{(v)}\cap H)$. La suite on 
prends $\overline{<\gamma>}=\Gamma$ et $i_0$ tq 
$\gamma^{p^{i_0}}\in \Gamma(i_0)$ s'envoie dans 
$U_K^{(m)}/(U_K^{(m)}\cap H)$ pour $m>e_K/(p-1)$(!). On prends
le plus petit $m=m_0$. Alors $\gamma_0:=\gamma^{p^{i_0}}$ engendre 
$\Gamma^{(m_0)}$, d'où $\Gamma(i_0)=\Gamma^{(m_0)}$. Maintenant
$\overline{<\gamma_0^{p^n}>}=\Gamma(i_0+n)$, aussi $\gamma_0^{p^n}$
s'envoie dans $(U_K^{(m_0)})^{p^n}/(U_K^{(m_0)})^{p^n}\cap H$
qui est $U_K^{(m_0+e_Kn)}/U_K^{(m_0+e_Kn)}\cap H$ par le 
corollaire. Le résultat tombe ensuite via $\Gamma^{(v_{i_0}+n)}=
\Gamma^{(m_0+e_Kn)}$ d'où $v_i-v_{i_0}=e_K(i-i_0)$ pour 
$i\geq i_0$.
\end{proof}


\chapter{Extensions presque étales}
Corps locaux de char $0$.
\section{Normes et traces}
\begin{lem}
    $Tr_{L/K}(\m_L^n)=\m_K^r$ avec $r=[(v_L(\D_{L/K})+n)/e_{L/K}]$.
\end{lem}
On prends $L/K$ galois avec $[L:K]=p$ et $G=\Z/p\Z$ et $t$ le
seul saut de ramification : $G_t=G$ et $G_{t+1}=\{e\}$. On a
\[v_L(\D_{L/K})=\sum (|G_i|-1)=(t+1)(p-1)\]
\begin{lem}
    Pour $x\in \m_L^n$ on a 
    \[N_{L/K}(1+x)=1+N_{L/K}(x)+Tr_{L/K}(x)\mod \m_K^s\]
    avec $s=[\frac{(p-1)(t+1)+2n}{p}]$.
\end{lem}
\begin{proof}
    On a 
    \begin{align*}
        N_{L/K}(1+x)&=\prod_{g\in G}(1+g.x)\\
                    &=1+N_{L/K}(x)+Tr_{L/K}(x)+\sum_{g_1\ne g_2}g_1(x)g_2(x)+\ldots+\sum_{g_1\ne\ldots\ne g_{p-1}}\prod g_i(x)\\
    \end{align*}
    Dans chaque somme, on prend des éléments dans 
    $C_j=\{K\subset G;|K|=j\}$ (en prenant $i<j$ c'est pas trivial
    l'action de $G$). Maintenant $G$
    agit terme à terme sur $C_j$ qu'on décompose en orbites
    de tailles $p$. Omg, on peut écrire les sommes comme sommes
    de traces (chaque orbite (!!!)) d'éléments de $\m_L^{jn}$ par
    hypothèse, on peut prendre $j=2$. Pour tout $y\in \Or_L$,
    le premier lemme donne $Tr_{L/K}(y)\in \m_K^r$ et
    le résultat suit.
\end{proof}
\begin{cor}
    Pour tout $x\in \Or_L$, 
    $v_K(N_{L/K}(1+x)-1-N_{L/K}(x))\geq t(p-1)/p$
\end{cor}
\begin{proof}
    Si $x\in \Or_L$, par le deuxième lemme $N_{L/K}(1+x)=1+N_{L/K}(x)+Tr_{L/K}(x)\mod \m_K^{\frac{(p-1)(t+1)}{p}}$.
    Maintenant $(p-1)(t+1)/p=\frac{t(p-1)}{p}+1-1/p$ et
    $[(p-1)(t+1)/p]\geq t(p-1)/p$ d'où par le premier lemme 
    $Tr_{L/K}(x)\in \m_K^{t(p-1)/p}$.
\end{proof}

\section{Extensions profondément (deeply) ramifiées}
On prends $K_\infty/K$ une extensions infinie de corps locaux
de caractéristique $0$ et $K_\infty=\cup_{n=0}^\infty K_n$,
$K_n\subset K_{n+1}$, $[K_n:K]<\infty$. À
l'inverse on peut écrire toute extension infinie de cette manière
(exo). On veut définir $\D_{K_\infty/K}$ moralement on veut
définir comme $\cap \D_{K_n/K}$. L'anneau $\Or_{K_\infty}$ est
bien défini et c'est $\cup \Or_{K_n}$. On pose
\[\D_{K_\infty/K}:=\cap (\D_{K_n/K}\Or_{K_\infty})\]
et on a $v_K(\D_{K_\infty/K})=\sup_n\{v_K(\D_{K_n/K})\}$ (étendre
à $\bar K$ ?).

Soit $L_\infty/K_\infty$ une extension finie avec 
$L_\infty=K_\infty(\alpha)$. En particulier, 
$f=\mu_\alpha\in K_{n_0}[X]$! On pose $n_0=0$ pour simplifier.
On pose $L_n=K_n[\alpha]$, on obtient 
% https://q.uiver.app/#q=WzAsNixbMCwyLCJMXzA9S18wW1xcYWxwaGFdIl0sWzEsMSwiTF9uPUtfbltcXGFscGhhXSJdLFsxLDMsIktfMCJdLFsyLDIsIlxcYnVsbGV0Il0sWzIsMCwiTF9cXGluZnR5Il0sWzMsMSwiS19cXGluZnR5Il0sWzAsMV0sWzIsMF0sWzIsM10sWzMsMV0sWzEsNF0sWzMsNV0sWzUsNF1d
\[\begin{tikzcd}
	&& {L_\infty} \\
	& {L_n} && {K_\infty} \\
	{L_0} && K_n \\
	& {K_0}
	\arrow[from=2-2, to=1-3]
	\arrow[from=2-4, to=1-3]
	\arrow[from=3-1, to=2-2]
	\arrow[from=3-3, to=2-2]
	\arrow[from=3-3, to=2-4]
	\arrow[from=4-2, to=3-1]
	\arrow[from=4-2, to=3-3]
\end{tikzcd}\]
\begin{prop}
    On a
    \begin{enumerate}
        \item Si $m\geq n$, alors
            $\D_{L_n/K_n}\subset \D_{L_m/K_m}$
        \item $\D_{L_\infty/K_\infty}= \cup_{n=0}^\infty(\D_{L_n/K_n}\Or_{L_\infty})$
    \end{enumerate}
\end{prop}
\begin{proof}[1]
En terme de valuation on a 
$v_K(\D_{L_n/K_n})\geq v_K(\D_{L_m/K_m})$ et 
$v_K(\D_{L_\infty/K_\infty})=\inf\{v_L(\D_{L_n/K_n})\}$. On a
$|\Hom_{K_n}(L_n,\bar K)|=|\Hom_{K_m}(L_m,\bar K)|$ d'où 
$Tr_{L_m/K_m}|_{L_n}=Tr_{L_n/K_n}$. On prends 
$(e_i)_{i=1,\ldots N}$ une base de $\Or_{L_n}$ en tant que 
$\Or_{K_n}$-module et $(e_i^*)$ la base duale pour la trace, ie
$Tr_{L_n/K_n}(e_ie_j^*)=\delta_{ij}$. On a 
\[\D^{-1}_{L_n/K_n}=\Or_{K_n}e_1^*+\ldots +\Or_{K_n}e_N^*\]
et soit $x\in \D_{L_m/K_m}^{-1}$, pour $m>n$, $(e_i^*)$ est
toujours une base de $\Or_{L_m}$ sur $\Or_{K_m}$. On a 
$Tr_{L_m/K_m}(x.e_j)=\sum_{i=1}^N a_i Tr_{L_m/K_m}(e_i^*e_j)\in \Or_{K_m}$ et par l'observation on peut prendre la trace sur 
$L_n/K_n$, où $(e_i^*)$ est la base duale, d'où
\[Tr_{L_m/K_m}(x.e_j)=a_j\in \Or_{K_m}\]
maintenant $x\in \D^{-1}_{L_m/K_m}$ implique $x=\sum a_ie_i^*$
avec $a_i\in \Or_{K_m}$ d'où $x\in \D^{-1}_{L_n/K_n}\Or_{L_m}$
et $\D^{-1}_{L_m/K_m}\subset \D^{-1}_{L_n/K_n}\Or_{L_m}$ implique
que $\D_{L_n/K_n}\subset \D_{L_m/K_m}$ qui est le résultat voulu.
\end{proof}
\begin{proof}[2]
    Déjà par $1.$ c'est bien un idéal, on regarde $\subset$. On
    prouve que \[(\cup \D_{L_n/K_n}\Or_{L_\infty})^{-1}=\cap(\D^{-1}_{L_n/K_n}\Or_{L_\infty})\subset \D^{-1}_{L_\infty/K_\infty}\]
    la première égalité c'est un exo. Soit $x$ dans l'intersection.
    Et $y\in\Or_{L_\infty}$, on veut prouver que 
    $Tr_{L_\infty/K_\infty}(xy)\in\Or_{K_\infty}$. C'est immédiat
    car y'a un $n$ tel que $x\in \D^{-1}_{L_n/K_n}$ et 
    $y\in\Or_{L_n}$ d'où la trace est dans 
    $\Or_{K_n}\subset\Or_{K_\infty}$. De l'autre côté c'est pareil.
    Si $x\in \D^{-1}_{L_\infty/K_\infty}$ alors 
    $x\in \D^{-1}_{L_n/K_n}$, etc...
\end{proof}

\end{document}
