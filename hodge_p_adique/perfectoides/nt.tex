\documentclass[a4paper,12pt]{article}
\usepackage{amsmath,  amsthm,enumerate}
\usepackage{csquotes}
\usepackage[provide=*,french]{babel}
\usepackage[dvipsnames]{xcolor}
\usepackage{quiver, tikz}

%symbole caligraphique
\usepackage{mathrsfs}

%hyperliens
\usepackage{hyperref}

%Symbole action de groupe
\usepackage{amssymb} 
\def\acts{\curvearrowright}

\definecolor{wgrey}{RGB}{148, 38, 55}

\setlength\parindent{24pt}

\newcommand{\Z}{\mathbb{Z}}
\newcommand{\R}{\mathbb{R}}
\newcommand{\rel}{\omathcal{R}}
\newcommand{\Q}{\mathbb{Q}}
\newcommand{\C}{\mathbb{C}}
\newcommand{\N}{\mathbb{N}}
\newcommand{\K}{\mathbb{K}}
\newcommand{\A}{\mathbb{A}}
\newcommand{\B}{\mathcal{B}}
\newcommand{\Or}{\mathcal{O}}
\newcommand{\F}{\mathbb F}
\newcommand{\m}{\mathfrak m}
\renewcommand{\b}{\mathfrak b}
\renewcommand{\a}{\mathfrak a}
\newcommand{\p}{\mathfrak p}
\newcommand{\I}{\mathfrak I}
\newcommand{\Hom}{\textrm{Hom}}
\newcommand{\disc}{\textrm{disc}}
\newcommand{\Pic}{\textrm{Pic}}
\newcommand{\End}{\textrm{End}}
\newcommand{\Spec}{\textrm{Spec}}
\newcommand{\Frac}{\textrm{Frac}}

\newcommand{\cL}{\mathscr{L}}
\newcommand{\G}{\mathscr{G}}
\newcommand{\D}{\mathscr{D}}
\newcommand{\E}{\mathscr{E}}

\theoremstyle{plain}
\newtheorem{thm}{Théoreme}
\newtheorem{lem}{Lemme}
\newtheorem{prop}{Proposition}
\newtheorem{cor}{Corollaire}
\newtheorem{heur}{Heuristique}
\newtheorem{rem}{Remarque}
\newtheorem{rembis}{Remarque}
\newtheorem{note}{Note}

\theoremstyle{definition}
\newtheorem{conj}{Conjecture}
\newtheorem*{eq}{Équivalences}
\newtheorem{prob}{Problème}
\newtheorem{quest}{Question}
\newtheorem{prot}{Protocole}
\newtheorem{algo}{Algorithme}
\newtheorem{defn}{Définition}
\newtheorem{defnbis}{Définition}
\newtheorem{ex}{Exemple}
\newtheorem{exo}{Exercices}

\theoremstyle{remark}

\definecolor{wgrey}{RGB}{148, 38, 55}
\definecolor{wgreen}{RGB}{100, 200,0} 
\hypersetup{
    colorlinks=true,
    linkcolor=wgreen,
    urlcolor=wgrey,
    filecolor=wgrey
}

\title{Corps perfectoides}
\date{}

\begin{document}
\maketitle


\section{Préliminaires}
\subsection{Espaces séquentiels}
Pour rappel un espace est séquentiel si les ouverts séquentiels
sont ouverts et/ou les fermés séquentiels sont fermés. En 
particulier quand on a des bases dénombrables de voisinage on est
séquentiel!

En particulier je peux tester toute les topologies grâces aux 
suites convergentes. 

En plus comme c'est des anneaux topologiques suffit de tester en
$0$.
\subsection{Perfections}
Je note $Ring_{p,sperf}$ (resp. $Ring_{p,perf}$) la catégorie des
anneaux commutatifs semi-parfaits (resp. parfaits) de
caractéristique $p>0$. 
Étant donné un anneau commutatif unitaire semi-parfait $R$,
y'a deux adjonctions
\[(\_)_{perf}\colon Ring_{p, sperf}\leftrightharpoons Ring_{p,perf}\colon Forget\]
et
\[Forget\colon Ring_{p, perf}\leftrightharpoons Ring_{p,sperf}\colon (\_)^{perf}\]
où la première est donnée par l'universalité de 
$R\to (R)_{perf}:=\varinjlim_{\varphi}R$
($R$ est le dernier terme du diagramme $\N\to Ring$ où $i+1\to i$
est donné par $\varphi$ et $i\mapsto R$).

Et la deuxième par l'universalité de $R^{perf}\to R$.

\subsubsection{Description usuelle}
On a toujours $R^{perf}=\{(x_n)_{n\geq 0}| x_{n+1}^p=x_n\}$. 
Les lois sont termes à termes.


\subsection{Anneaux de valuations}
Une première remarque : les idéaux principaux ont valuation
minorée $>0$. En particulier $\m_E$ est jamais principal. Ducoup
de même $\alpha.\m_E$ non plus pour $\alpha\in \m_E$. Enfin si
tu contiens $\alpha$ tu contiens $B(0, |\alpha|)$ de sorte que
$\m_E$ est le seul idéal de valuation $0$ puis on a tout les
idéaux !

\section{Corps perfectoide}
Un corps non-archimédien (NA) $(E,|.|_E)$ est perfectoide si
\begin{enumerate}
    \item $|.|_E$ est pas discrète.
    \item $E$ est complet.
    \item $\Or_E/p$ est semi-parfait.
\end{enumerate}
Y'a une prop qui dit (en char$=0$) que $E=\widehat F$ est la
complétion d'un corps profondément ramifié $F$.

\section{Tilt}
Le tilt d'un anneau commutatif semi-parfait est donné par 
$(R/p)^{perf}$. Dans le cas de $E$ perfectoide de caractéristique 
$0$, on déf $E^\flat$ comme $Frac((\Or_E/p)^{perf})$. 
\subsection{$E^\flat$ est perfectoide.}
Les suites de Cauchy $(x_i)_{i\geq 0}\subset \Or_E^\flat$ vérifient
$(x_{i,k})_{i\geq 0}$ est stationnaire d'où la complétude. La
perfection est dûe à celle de $\Or_E/p$. Et la valuation non 
discrète on le voit après.
\subsection{Étude de $\Or_{E^\flat}$}
Pour $(x_n)_n\in \Or_{E^\flat}$ on lift $x_n$ en $\widehat x_n$.
Alors $\lim_{m\to \infty} \widehat x_{n+m}^{p^m}$ converge en
$x^{(n)}$ tels que 
\[(x^{(n+1)})^p=x^{(n)}\]
Maintenant la limite ensembliste
\[\varprojlim_{x\mapsto x^p}\Or_E\]
a une structure d'anneau via \[(x^{(n)})_n+(y^{(n)})_n=(\lim_{m\to\infty} (x^{(n+m)}+y^{(n+m)})^{p^m})_n\]
et $(x^{(n)})_n(y^{(n)})_n=(x^{(n)}y^{(n)})_n$. Ça donne
un isomorphisme
\[\Or_{E^\flat}\to \varprojlim_{x\to x^p}\Or_E\]
et on définit (par extension multiplicative)
\[|x/y|_{E^\flat}:=|x^{(0)}|_E/|y^{(0)}|_E\]
on a alors
\[|x|_{E^\flat}\leq |p|^{p^n}\Leftrightarrow x_n=0\]
c'est simplement l'équivalence $x^{(n)}=0\mod p$ équivaut à
\[x^{(0)}=(x^{(n)})^{p^n}=0\mod p^{p^n}\]

\subsection{Valeur absolue}
On a 
\begin{enumerate}
    \item $|E^\flat|=|E|$ de manière évidente.
    \item $|E|$ est $p$-divisible.
    \item $\m_E$ est plat (simple).
    \item $\m_E^2=\m_E$.
\end{enumerate}

La première est assez claire. La troisième est parce que c'est
sans torsion et la quatrième c'est juste que 
$(x)=\overline{B(0,|x|)}$ pour $x\in \m_E$ et $\m_E=B(0,1)$.

\subsubsection{$p$-divisibilité $\to$ ramification sauvage}
L'idée est simple on a $x=y^p+pz$ et si $|x|>|p|$ la 
$p$-divisibilité. Maintenant si $|x|\leq |p|$ y'a un plus petit
$n$ tel que
\[|p|^n<|x|\leq |p|^{n-1}\]
puis $|p|<|x/p^{n-1}\leq 1$.
Enfin si $|x|>|p|$ on a $p=x(p/x)$ d'où y'a un élément de
valuation $|p|^{1/p}$ et de la ramification sauvage (même une
infinité). 

\subsubsection{Plus sur la ramification sauvage}
En fait de $p=x(p/x)$ si $y_1,y_2$ sont tels que
$y_1^p=x\mod p$ et $y_2^p=p/x\mod p$ puis 
$\pi=y_1y_2\mod p$ alors l'élément 
\[x=(\varphi^{-n}(\pi))_{n\geq 0}\in \Or_E^\flat\]
engendre presque une $\Z_p$-extension. Faut remplacer
$\pi$ par un élément de $\bar E$ ayant la même valuation
et regarder
\[K_{n,cyc}:=K(\zeta_p,x^{(n)})\]

\subsubsection{Topologie sur le Tilt (complétude)}
Via un $g\in \Or_E^\flat$ tel que $|g|=|p|^{1/p}$ on pose $t=g^p$.
On a naturellement la flèche surjective
\[\Or_E^\flat\to \Or_E/p\]
et on prouve aussi naturellement que le noyau est $(t)$ simplement
en en remarquant que $g^{(0)} \in (p)$ implique
\[g^{(0)}=a^{(0)}.p\]
puis via $t^{(0)}=u^{(0)}p$ pour une unité on a
\[g^{(0)}=\frac{a^{(0)}}{u^{(0)}}t^{(0)}\]
et de la même manière $g^{(n)}=\frac{a^{(n)}}{u^{(n)}}t^{(n)}$.
On a montré que 
\[\Or_{E^\flat}/t\simeq \Or_E/p\]

\section{Théorème de presque-pureté}
\subsection{Untilt}
\subsubsection{$\theta_E\colon A_{\inf}(\Or_E)\to \Or_E$}
Une manière de relever canoniquement $\Or_E^\flat$ c'est via 
\[W(\Or_E^\flat)\]
et on peut construire via le morphisme (par déf) naturel 
\[w_n\colon W_n(\Or_E)\to \Or_E\]
le morphisme $\theta_E\colon W(\Or_E^\flat)\to \Or_E$ par 
commutatitivité de
% https://q.uiver.app/#q=WzAsNixbMiwwLCJXX24oXFxPcl9FKSJdLFszLDAsIlxcT3JfRSJdLFszLDEsIlxcT3JfRS9wXntuKzF9Il0sWzIsMSwiV19uKFxcT3JfRS9wKSJdLFsxLDEsIldfbihcXE9yX0VeXFxmbGF0KSJdLFswLDEsIldfbihcXE9yX0VeXFxmbGF0KSJdLFswLDEsIndfbiJdLFsxLDJdLFswLDNdLFszLDIsIlxcZXRhX24iLDJdLFs1LDQsIlxcdmFycGhpXnstbn0iLDJdLFs0LDMsInByXzAiLDJdXQ==
\[\begin{tikzcd}
	&& {W_n(\Or_E)} & {\Or_E} \\
	{W_n(\Or_E^\flat)} & {W_n(\Or_E^\flat)} & {W_n(\Or_E/p)} & {\Or_E/p^{n+1}}
	\arrow["{w_n}", from=1-3, to=1-4]
	\arrow[from=1-3, to=2-3]
	\arrow[from=1-4, to=2-4]
	\arrow["{\varphi^{-n}}"', from=2-1, to=2-2]
	\arrow["{pr_0}"', from=2-2, to=2-3]
	\arrow["{\eta_n}"', from=2-3, to=2-4]
\end{tikzcd}\]
\begin{rem}
    Attention au $\varphi^{-n}$ au début.
\end{rem}
À la fin on a 
\[\theta_E(\sum_{n\geq 0} p^n[a_n])=\sum_{n\geq 0} p^na_n^{(0)}\]
\begin{rem}
    Vérifier que $\theta_E \mod n$ coincide bien avec la
    composition ! (déjà réduire à droite modulo $p^{n+1}$ se
    factorise bien par $p^{n+1}W(\Or_E^\flat)$.
\end{rem}
en particulier si $x\in \ker(\theta_E)$ alors 
\[|x_0^{(0)}|_E\leq |p|_E\]
et en fait si $|x_0^{(0)}|_E=|p|$ alors il engendre (!!) le
noyau. La preuve est juste par approximation successive (voir 
Fontaine si jamais).
\begin{rem}
    Via une pseudo-uniformisante du tilt, $t$, si 
    $t^{(0)}=u^{(0)}.p$ alors $[t]-p[u]$ engendre le noyau. En
    particulier dans $\C_p$ on peut prendre $t^{(0)}=p$ et
    $t=(0,p^{1/p},p^{1/p^2},\ldots)$ d'où 
    $[t]-p=\ker(\theta_{\C_p})$.
\end{rem}
\subsubsection{Application}
De
% https://q.uiver.app/#q=WzAsMTUsWzIsMCwiQV97XFxpbmZ9KFxcT3JfRSkiXSxbMiwyLCJBX3tcXGluZn0oXFxPcl97XFxDX0V9KSJdLFsyLDEsIlcoXFxPcl97XFxtYXRoc2NyIEZ9KSJdLFsxLDEsIlxceGkgVyhcXE9yX3tcXG1hdGhzY3IgRn0pIl0sWzEsMiwiXFx4aSBBX3tcXGluZn0oXFxPcl97XFxDX0V9KSJdLFsxLDAsIlxceGkgQV97XFxpbmZ9KFxcT3JfRSkiXSxbMCwwLCIwIl0sWzAsMSwiMCJdLFswLDIsIjAiXSxbMywyLCJcXE9yX3tcXENfRX0iXSxbMywxLCJcXE9yX3tcXG1hdGhzY3IgRl5cXHNoYXJwfSJdLFszLDAsIlxcT3JfRSJdLFs0LDAsIjAiXSxbNCwxLCIwIl0sWzQsMiwiMCJdLFswLDIsIiIsMCx7InN0eWxlIjp7InRhaWwiOnsibmFtZSI6Imhvb2siLCJzaWRlIjoidG9wIn19fV0sWzIsMSwiIiwwLHsic3R5bGUiOnsidGFpbCI6eyJuYW1lIjoiaG9vayIsInNpZGUiOiJ0b3AifX19XSxbNSwwXSxbNSwzLCIiLDIseyJzdHlsZSI6eyJ0YWlsIjp7Im5hbWUiOiJob29rIiwic2lkZSI6InRvcCJ9fX1dLFszLDJdLFszLDQsIiIsMix7InN0eWxlIjp7InRhaWwiOnsibmFtZSI6Imhvb2siLCJzaWRlIjoidG9wIn19fV0sWzQsMV0sWzYsNV0sWzcsM10sWzgsNF0sWzEsOV0sWzIsMTBdLFswLDExXSxbMTEsMTJdLFsxMCwxM10sWzksMTRdLFsxMSwxMCwiIiwxLHsic3R5bGUiOnsidGFpbCI6eyJuYW1lIjoiaG9vayIsInNpZGUiOiJ0b3AifX19XSxbMTAsOSwiIiwxLHsic3R5bGUiOnsidGFpbCI6eyJuYW1lIjoiaG9vayIsInNpZGUiOiJ0b3AifX19XV0=
\[\begin{tikzcd}
	0 & {\xi A_{\inf}(\Or_E)} & {A_{\inf}(\Or_E)} & {\Or_E} & 0 \\
	0 & {\xi W(\Or_{\mathscr F})} & {W(\Or_{\mathscr F})} & {\Or_{\mathscr F^\sharp}} & 0 \\
	0 & {\xi A_{\inf}(\Or_{\C_E})} & {A_{\inf}(\Or_{\C_E})} & {\Or_{\C_E}} & 0
	\arrow[from=1-1, to=1-2]
	\arrow[from=1-2, to=1-3]
	\arrow[hook, from=1-2, to=2-2]
	\arrow[from=1-3, to=1-4]
	\arrow[hook, from=1-3, to=2-3]
	\arrow[from=1-4, to=1-5]
	\arrow[hook, from=1-4, to=2-4]
	\arrow[from=2-1, to=2-2]
	\arrow[from=2-2, to=2-3]
	\arrow[hook, from=2-2, to=3-2]
	\arrow[from=2-3, to=2-4]
	\arrow[hook, from=2-3, to=3-3]
	\arrow[from=2-4, to=2-5]
	\arrow[hook, from=2-4, to=3-4]
	\arrow[from=3-1, to=3-2]
	\arrow[from=3-2, to=3-3]
	\arrow[from=3-3, to=3-4]
	\arrow[from=3-4, to=3-5]
\end{tikzcd}\]
on construit $\mathscr F^\sharp$.

\subsubsection{$\mathscr F^\sharp$ est perfectoide}
Pour la perfection, ça découle de $(\xi,p)=([\xi_0],p)$ et
$|\xi_0|=|p|<1$.

À voir : complétude et anneau de valuation.


\subsubsection{$(\_)^{\sharp^{\flat}}=Id$ et $(\_)^{\flat^{\sharp}}=Id$}
Vu que 
$|\xi_0|_{E^\flat}=|p|$ on a 
\begin{align*}
    \Or_{\mathscr F^\sharp}/p\Or_{\mathscr F^\sharp}&=\Or_\mathscr{F}/\xi_0\Or_\mathscr{F}\\
\end{align*}
puis \[\varprojlim_{\varphi} \Or_\mathscr{F}/\xi_0\Or_\mathscr{F}=\Or_\mathscr{F}\]
parce qu'on applique $(\_)^{perf}$ en remplaçant $p$ par $\xi_0$.
On obtient l'iso \[(\_)^{perf}\simeq (\_/\xi_0)^{perf}\] sauf qu'en
caractéristique $p$ l'iso change pas la structure d'anneau à gauche
et la projection fournit l'iso voulu par unicité de la racine
$p$-ème. En particulier on a montré que 
$\mathscr F^{\sharp^{\flat}}=\mathscr F$.

\subsection{$\C_E^\flat=\C_{E^\flat}$}
On a $\C_{E^\flat}\subset \C_E^\flat$ d'où si 
$\mathscr F=\C_{E^\flat}$ on a $\mathscr F^\sharp$ qui est
perfectoide et algébriquement clos (par d'autres sections) d'où
contient $\C_E$ puis 
\[\C_E^\flat\subset (\mathscr F^\sharp)^\flat=\mathscr F\]
d'où le résultat.

\subsection{$G_{E^\flat}\simeq G_E$}
On construit $G_E\to G_{E^\flat}$ via \[G_E\acts \Or_{\C_E}/p\]
d'où terme à terme \[G_E\acts \Or_{\C_E^\flat}\]
puis la flèche 
$i^\flat\colon G_E\to Aut(\C_E^\flat/E^\flat)=G_{E^\flat}$
en remarquant que par déf l'action fixe $E^\flat$.
\begin{rem}
    À chaque fois on utilise $\C_E^\flat=\C_{E^\flat}$.
\end{rem}
À l'inverse, 
\[\xi\in W(\Or_E^\flat)\]
d'où est fixé par l'action induite 
$G_{E^\flat}\acts W(\Or_{\C_{E^\flat}})$. Alors l'action passe
au quotient
\[G_{E^\flat}\acts W(\Or_{\C_{E^\flat}})/\xi
W(\Or_{\C_{E^\flat}})=\Or_{\C_E}(!)\]
D'où on a $i^\sharp\colon G_{E^\flat}\to Aut(\C_E/E)$. 

\begin{rem}
    De la même manière que via $(E^\flat)^\sharp=E$ et
    $\mathscr F^{\sharp^{\flat}}=\mathscr F$ on peut voir que
    les flèches sont inverses l'une de l'autre !!
\end{rem}
\begin{rem}
    IMPORTANT : C'est important de remarquer que les actions
    sont continues, pour l'équivalence de catégorie.
\end{rem}
\subsection{Presque-pureté}
Les preuves complètes consistent à utiliser Ax-Sen-Tate en 
caractéristique $p$ et $0$ et de la cohomologie (de base). Mais
en gros $\mathscr F^\sharp = \C_E^{G_\mathscr{F}}$ et $F^\flat =
\C_{E^\flat}^{G_F}$ d'où l'équivalence de catégorie.

Le papier de Colmez a une très jolie preuve de Ax-Sen-Tate en
caractéristique 0.



\end{document}
