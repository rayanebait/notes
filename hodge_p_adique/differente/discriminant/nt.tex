\documentclass[a4paper,12pt]{book}
\usepackage{amsmath,  amsthm,enumerate}
\usepackage{csquotes}
\usepackage[provide=*,french]{babel}
\usepackage[dvipsnames]{xcolor}
\usepackage{quiver, tikz}

%symbole caligraphique
\usepackage{mathrsfs}

%hyperliens
\usepackage{hyperref}

%pseudo-code
\usepackage{algpseudocode}
\usepackage{algorithm}
\makeatletter
  \renewcommand{\ALG@name}{Algorithme}
  \makeatother
\usepackage{fancyhdr}

\pagestyle{fancy}
\addtolength{\headwidth}{\marginparsep}
\addtolength{\headwidth}{\marginparwidth}
\renewcommand{\chaptermark}[1]{\markboth{#1}{}}
\renewcommand{\sectionmark}[1]{\markright{\thesection\ #1}}
\fancyhf{}
\fancyfoot[C]{\thepage}
\fancyhead[LO]{\textit \leftmark}
\fancyhead[RE]{\textit \rightmark}
\renewcommand{\headrulewidth}{0pt} % and the line
\fancypagestyle{plain}{%
    \fancyhead{} % get rid of headers
}

%bibliographie
\usepackage[
backend=biber,
style=alphabetic,
sorting=ynt
]{biblatex}

\addbibresource{bib.bib}

\usepackage{appendix}
\renewcommand{\appendixpagename}{Annexe}

\definecolor{wgrey}{RGB}{148, 38, 55}

\setlength\parindent{24pt}

\newcommand{\Z}{\mathbb{Z}}
\newcommand{\R}{\mathbb{R}}
\newcommand{\rel}{\omathcal{R}}
\newcommand{\Q}{\mathbb{Q}}
\newcommand{\C}{\mathbb{C}}
\newcommand{\N}{\mathbb{N}}
\newcommand{\K}{\mathbb{K}}
\newcommand{\A}{\mathbb{A}}
\newcommand{\B}{\mathcal{B}}
\newcommand{\Or}{\mathcal{O}}
\newcommand{\F}{\mathbb F}
\newcommand{\m}{\mathfrak m}
\renewcommand{\b}{\mathfrak b}
\renewcommand{\a}{\mathfrak a}
\newcommand{\p}{\mathfrak p}
\newcommand{\I}{\mathfrak I}
\newcommand{\Hom}{\textrm{Hom}}
\newcommand{\disc}{\textrm{disc}}
\newcommand{\Pic}{\textrm{Pic}}
\newcommand{\End}{\textrm{End}}
\newcommand{\Spec}{\textrm{Spec}}
\newcommand{\Frac}{\textrm{Frac}}

\newcommand{\cL}{\mathscr{L}}
\newcommand{\G}{\mathscr{G}}
\newcommand{\D}{\mathscr{D}}
\newcommand{\E}{\mathscr{E}}

\theoremstyle{plain}
\newtheorem{thm}{Théoreme}
\newtheorem{lem}{Lemme}
\newtheorem{prop}{Proposition}
\newtheorem{cor}{Corollaire}
\newtheorem{heur}{Heuristique}
\newtheorem{rem}{Remarque}
\newtheorem{rembis}{Remarque}
\newtheorem{note}{Note}

\theoremstyle{definition}
\newtheorem{conj}{Conjecture}
\newtheorem*{eq}{Équivalences}
\newtheorem{prob}{Problème}
\newtheorem{quest}{Question}
\newtheorem{prot}{Protocole}
\newtheorem{algo}{Algorithme}
\newtheorem{defn}{Définition}
\newtheorem{defnbis}{Définition}
\newtheorem{ex}{Exemple}
\newtheorem{exo}{Exercices}

\theoremstyle{remark}

\definecolor{wgrey}{RGB}{148, 38, 55}
\definecolor{wgreen}{RGB}{100, 200,0} 
\hypersetup{
    colorlinks=true,
    linkcolor=wgreen,
    urlcolor=wgrey,
    filecolor=wgrey
}

\title{Le Discriminant}
\date{}

\begin{document}
\maketitle
\section{Matrice de Vandermonde}
Deux trucs que j'ai fais à l'instant : 
\[V((\alpha_i)_{i=1,\ldots,m},n):=\begin{pmatrix}1&\alpha_1&\ldots&\alpha_1^{n-1}\\
    1&\alpha_2&\ldots&\alpha_2^{n-1}\\
    \vdots&\vdots&&\vdots\\
    1&\alpha_m&\ldots&\alpha_m^{n-1}\\
\end{pmatrix}\]
alors $\det(V)=\prod_{1\leq i<j\leq n} (\alpha_i-\alpha_j)$.
\section{Inversibilité}
Quand $n=m$, on regarde $X=(x_i)^t$ et $P(Y)=\sum x_iY^i$.
Alors $V.X=0$ veut dire que $P(\alpha_i)=0$ pour chaque $i$.
Sauf que le polynôme est de degré $n-1$ et y'a $n$ racines.
D'où $x_i=0$ pour chaque $i$ dès que les $\alpha_i$ sont 
distincts et l'inversibilité.

\subsection{Preuve magique}
Une preuve rapide, en mettant comme indeterminée les $\alpha_i$,
le déterminant est de degré $n(n-1)/2$ et le produit en
question aussi. Y'a une divisibilité via la section d'avant d'où
le résultat!

\subsection{Autre preuve}
On soustrait $L_1$ la première aux autres. Et là on a
$V((\alpha_i)_{i=1,\ldots,n},n)\leftarrow(\sum \alpha_i^j-\alpha_1^j)$,
et la première colonne c'est juste un $1$ au début. Donc on peut
regarder la sous matrice. On peut factoriser $\prod_{1<i} (\alpha_i-\alpha_1)$
en factorisant le terme à chaque ligne. Maintenant la nouvelle
matrice (faudra l'écrire), c'est $V'=(\sum_{k=1}\alpha_i^{j-k}\alpha_1^k)$. 
Et là on remarque en notant 
    \[T=\begin{pmatrix}1&\alpha_1&\alpha_1^2&\ldots&\alpha_1^{n-1}\\
        0&1&\alpha_1&\ldots&\alpha_1^{n-2}\\
        \vdots&\ldots&\ldots&\ldots&\vdots\\
        0&0&\ldots&\ldots&1\\
    \end{pmatrix}
    \]
que la matrice restante c'est $T.V((\alpha_i)_{i=2,\ldots,n-1},n-1)$
et $\det(T)$ est de déterminant $1$ d'où on peut faire une récurrence.

\begin{rem}
    Non le $n-1$ est pas embêtant dans $V$, les $\alpha_m-\alpha_i$ 
    apparaissent bien c'est pas le même indexage.
\end{rem}

\section{Discriminant}
C'est défini comme $\det(Tr_{L/K}(x_ix_j))$. On a
$\det(Tr_{L/K}(x_ix_j))=\det(\sum_{k} g_kx_ig_kx_j)=\det(g_kx_i)\det(g_ix_k)$.
\begin{rem}
    L'égalité de droite est bien vraie, la somme vient du produit de celle
    de droite et sa transposée!
\end{rem}



\end{document}
