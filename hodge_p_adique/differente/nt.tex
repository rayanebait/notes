\documentclass[a4paper,12pt]{book}
\usepackage{amsmath,  amsthm,enumerate}
\usepackage{csquotes}
\usepackage[provide=*,french]{babel}
\usepackage[dvipsnames]{xcolor}
\usepackage{quiver, tikz}

%symbole caligraphique
\usepackage{mathrsfs}

%hyperliens
\usepackage{hyperref}

%pseudo-code
\usepackage{algpseudocode}
\usepackage{algorithm}
\makeatletter
  \renewcommand{\ALG@name}{Algorithme}
  \makeatother
\usepackage{fancyhdr}

\pagestyle{fancy}
\addtolength{\headwidth}{\marginparsep}
\addtolength{\headwidth}{\marginparwidth}
\renewcommand{\chaptermark}[1]{\markboth{#1}{}}
\renewcommand{\sectionmark}[1]{\markright{\thesection\ #1}}
\fancyhf{}
\fancyfoot[C]{\thepage}
\fancyhead[LO]{\textit \leftmark}
\fancyhead[RE]{\textit \rightmark}
\renewcommand{\headrulewidth}{0pt} % and the line
\fancypagestyle{plain}{%
    \fancyhead{} % get rid of headers
}

%bibliographie
\usepackage[
backend=biber,
style=alphabetic,
sorting=ynt
]{biblatex}

\addbibresource{bib.bib}

\usepackage{appendix}
\renewcommand{\appendixpagename}{Annexe}

\definecolor{wgrey}{RGB}{148, 38, 55}

\setlength\parindent{24pt}

\newcommand{\Z}{\mathbb{Z}}
\newcommand{\R}{\mathbb{R}}
\newcommand{\rel}{\omathcal{R}}
\newcommand{\Q}{\mathbb{Q}}
\newcommand{\C}{\mathbb{C}}
\newcommand{\N}{\mathbb{N}}
\newcommand{\K}{\mathbb{K}}
\newcommand{\A}{\mathbb{A}}
\newcommand{\B}{\mathcal{B}}
\newcommand{\Or}{\mathcal{O}}
\newcommand{\F}{\mathbb F}
\newcommand{\m}{\mathfrak m}
\renewcommand{\b}{\mathfrak b}
\renewcommand{\a}{\mathfrak a}
\newcommand{\p}{\mathfrak p}
\newcommand{\I}{\mathfrak I}
\newcommand{\Hom}{\textrm{Hom}}
\newcommand{\disc}{\textrm{disc}}
\newcommand{\Pic}{\textrm{Pic}}
\newcommand{\End}{\textrm{End}}
\newcommand{\Spec}{\textrm{Spec}}
\newcommand{\Frac}{\textrm{Frac}}

\newcommand{\cL}{\mathscr{L}}
\newcommand{\G}{\mathscr{G}}
\newcommand{\D}{\mathscr{D}}
\newcommand{\E}{\mathscr{E}}

\theoremstyle{plain}
\newtheorem{thm}{Théoreme}
\newtheorem{lem}{Lemme}
\newtheorem{prop}{Proposition}
\newtheorem{cor}{Corollaire}
\newtheorem{heur}{Heuristique}
\newtheorem{rem}{Remarque}
\newtheorem{rembis}{Remarque}
\newtheorem{note}{Note}

\theoremstyle{definition}
\newtheorem{conj}{Conjecture}
\newtheorem*{eq}{Équivalences}
\newtheorem{prob}{Problème}
\newtheorem{quest}{Question}
\newtheorem{prot}{Protocole}
\newtheorem{algo}{Algorithme}
\newtheorem{defn}{Définition}
\newtheorem{defnbis}{Définition}
\newtheorem{ex}{Exemple}
\newtheorem{exo}{Exercices}

\theoremstyle{remark}

\definecolor{wgrey}{RGB}{148, 38, 55}
\definecolor{wgreen}{RGB}{100, 200,0} 
\hypersetup{
    colorlinks=true,
    linkcolor=wgreen,
    urlcolor=wgrey,
    filecolor=wgrey
}

\title{La différente}
\date{}

\begin{document}
\maketitle
Cet article de Keith est cool : \href{https://kconrad.math.uconn.edu/blurbs/gradnumthy/different.pdf}.

\section{Remarques}
Ducoup y'a des nouvelles définitions : les duaux relatifs à
une fonctionnelle (linéaire, i.e. $f\colon M\to M^\wedge$), comme 
$\{x\in L|Tr(xL)\subset \Or_K\}$. Et les opérateurs semi-simple.
I.e. les fonctionnelles qui se comportent comme on le voudrait.

\section{Duaux}
Cet autre article de Keith est cool :\href{https://kconrad.math.uconn.edu/blurbs/linmultialg/dualmod.pdf}.

\subsection{Cas de $M\subset K$, et $(\_\mapsto (m\mapsto \_.m)\colon M\to M^\wedge$}
Étant donné un anneau intègre $R$, $K$ le corps de fractions de
$R$ et $M\subset K$ un $R$-module. On définit
\[\Hom_R(M,R)=M^\wedge\]
alors $M^\wedge\simeq _K(R,M)$ via $\varphi\mapsto \varphi(1)$
et $c\mapsto (m\mapsto c.m)$. C'est simplement que $\varphi(b/b)=
b\varphi(1/b)=\varphi(1)$ d'où $\varphi(1/b)=\varphi(1)/b$. On dit
que $M$ a un dénominateur commun dans $R$ si y'a un $c\in R$ tel
que $M\subset c^{-1}R$.
En gros si c'est un idéal fractionnaire.

Si $R$ est noethérien alors $M$ est de type fini ssi il a un
dénominateur commun dans $R$ (i.e. idéal fractionnaire).
Maintenant $M^\wedge\ne 0$ ssi $_K(R,M)\ne 0$ i.e. $M$ a un
dénominateur commun dans $R$, c'est pas immédiat $\varphi(1)\in K$.
Mais $a/bM\subset R$ équivaut à $aM\subset bR\subset R$.
\subsection{Cas général}
Si $M$ est un $R$-module et $f\colon M\to M^\wedge$ une fonctionnelle
on peut définir $M^\wedge:=\{x\in M\otimes_R K| f(x,M)\subset R\}$
quand ça fait sens. Dans le cas où $M=\Or_L$, $R=\Or_K$ et $f=Tr_{L/K}(\_.\_)$
on obtient
\[\Or_L^\wedge=\{x\in L| Tr_{L/K}(x\Or_L)\subset \Or_K\}\]

\section{Calcul}
Si $M$ est libre de rang $n$ alors $M^\wedge$ aussi via 
l'existence de la base duale (le point clé c'est qu'on peut
définir le morphisme entier sur la base de $M$ et qu'on a toujours
les projections). Quand on est dans le cas $L/K$, $\Or_L/\Or_K$ 
et que $\Or_K$ est principal on se retrouve dans le cas libre et
on peut expliciter les calculs. Dans pour un peu plus de généralités
je prends $R$ principal et $M$ libre de rang $n$. Puis $M\otimes K=K^n$.
Si $f\colon (K^n)^2\times K^n$ est bilinéaire on cherche une base de
$M^\wedge=\{x\in M\otimes K=K^n|f(x,M)\subset R\}$, en écrivant ça plus
explicitement avec $M=\oplus Re_i$. On veut résoudre, avec $x=\sum x_ie_i$,
pour tout $j$ : $f(x,e_j)\in R$. Ça suffit parce qu'alors $f(x,M)\subset R$.
Et donc il suffit de résoudre 
\[\begin{cases}x_1f(e_1,e_1)+x_2f_(e_1,e_2)+&\ldots+x_nf(e_1,e_n)\in R\\
x_1f(e_2,e_1)+x_2f_(e_2,e_2)+&\ldots+x_nf(e_2,e_n)\in R\\
                             &\vdots\\
x_1f(e_n,e_1)+x_2f_(e_n,e_2)+&\ldots+x_nf(e_n,e_n)\in R\\
\end{cases}\]
en écrivant $A=(f(e_i,e_j))_{i,j}$ on résoud $A.X\in R^n$ et ça
c'est juste $X\in A^{-1}R^n$ terme à terme on obtient un système
comme d'hab.

\subsection{Calcul d'inverse}
Si $I$ est un idéal fractionnaire de $K$ et $I=\sum Rx_i$, on veut calculer
$_K(R:I)$, on résoud $yI\subset R$. Autrement dit $yx_i\subset R$.

\section{La différente}
Elle est donnée par 
\[\D_{L/K}:=(\Or_L^\wedge)^{-1}=(\{x\in L| Tr_{L/K}(x\Or_L)\subset \Or_K\})^{-1}\]
avec $I^{-1}=_L(\Or_L:I)$.

\end{document}
