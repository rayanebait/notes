\documentclass[a4paper,12pt]{book}
\usepackage{amsmath,  amsthm,enumerate}
\usepackage{csquotes}
\usepackage[provide=*,french]{babel}
\usepackage[dvipsnames]{xcolor}
\usepackage{quiver, tikz}

%symbole caligraphique
\usepackage{mathrsfs}

%hyperliens
\usepackage{hyperref}

%pseudo-code
\usepackage{algpseudocode}
\usepackage{algorithm}
\makeatletter
  \renewcommand{\ALG@name}{Algorithme}
  \makeatother
\usepackage{fancyhdr}

\pagestyle{fancy}
\addtolength{\headwidth}{\marginparsep}
\addtolength{\headwidth}{\marginparwidth}
\renewcommand{\chaptermark}[1]{\markboth{#1}{}}
\renewcommand{\sectionmark}[1]{\markright{\thesection\ #1}}
\fancyhf{}
\fancyfoot[C]{\thepage}
\fancyhead[LO]{\textit \leftmark}
\fancyhead[RE]{\textit \rightmark}
\renewcommand{\headrulewidth}{0pt} % and the line
\fancypagestyle{plain}{%
    \fancyhead{} % get rid of headers
}

%bibliographie
\usepackage[
backend=biber,
style=alphabetic,
sorting=ynt
]{biblatex}

\addbibresource{bib.bib}

\usepackage{appendix}
\renewcommand{\appendixpagename}{Annexe}

\definecolor{wgrey}{RGB}{148, 38, 55}

\setlength\parindent{24pt}

\newcommand{\Z}{\mathbb{Z}}
\newcommand{\R}{\mathbb{R}}
\newcommand{\rel}{\omathcal{R}}
\newcommand{\Q}{\mathbb{Q}}
\newcommand{\C}{\mathbb{C}}
\newcommand{\N}{\mathbb{N}}
\newcommand{\K}{\mathbb{K}}
\newcommand{\A}{\mathbb{A}}
\newcommand{\B}{\mathcal{B}}
\newcommand{\Or}{\mathcal{O}}
\newcommand{\F}{\mathbb F}
\newcommand{\m}{\mathfrak m}
\renewcommand{\b}{\mathfrak b}
\renewcommand{\a}{\mathfrak a}
\newcommand{\p}{\mathfrak p}
\newcommand{\I}{\mathfrak I}
\newcommand{\Hom}{\textrm{Hom}}
\newcommand{\disc}{\textrm{disc}}
\newcommand{\Pic}{\textrm{Pic}}
\newcommand{\End}{\textrm{End}}
\newcommand{\Spec}{\textrm{Spec}}
\newcommand{\Frac}{\textrm{Frac}}

\newcommand{\cL}{\mathscr{L}}
\newcommand{\G}{\mathscr{G}}
\newcommand{\D}{\mathscr{D}}
\newcommand{\E}{\mathscr{E}}

\theoremstyle{plain}
\newtheorem{thm}{Théoreme}
\newtheorem{lem}{Lemme}
\newtheorem{prop}{Proposition}
\newtheorem{cor}{Corollaire}
\newtheorem{heur}{Heuristique}
\newtheorem{rem}{Remarque}
\newtheorem{rembis}{Remarque}
\newtheorem{note}{Note}

\theoremstyle{definition}
\newtheorem{conj}{Conjecture}
\newtheorem*{eq}{Équivalences}
\newtheorem{prob}{Problème}
\newtheorem{quest}{Question}
\newtheorem{prot}{Protocole}
\newtheorem{algo}{Algorithme}
\newtheorem{defn}{Définition}
\newtheorem{defnbis}{Définition}
\newtheorem{ex}{Exemple}
\newtheorem{exo}{Exercices}

\theoremstyle{remark}

\definecolor{wgrey}{RGB}{148, 38, 55}
\definecolor{wgreen}{RGB}{100, 200,0} 
\hypersetup{
    colorlinks=true,
    linkcolor=wgreen,
    urlcolor=wgrey,
    filecolor=wgrey
}

\title{La différente, À REORGANISER}
\date{}

\begin{document}
\maketitle
\section{Cas Dedekind}
On prends $A$ de Dedekind et $K=Frac(A)$ puis $L/K$
finie séparable et $B$ les entiers de $L$ sur $A$.
\subsection{$B^\wedge$ est fractionnaire}

L'idée c'est de faire le cas d'un module libre et 
de l'utiliser pour montrer le cas de type fini.

Dans l'ordre on montre que pour $M\subset L$ un 
$A$-réseau alors
\begin{enumerate}
    \item Si $M=\oplus w_i.A$ alors $M^\wedge=w_i^\wedge.A$ où $w_i^\wedge$ est la $L$-base duale.
    \item Si $M\subset N$ alors $N^\wedge\subset M^\wedge$ pour deux $A$-modules quelconques.
\end{enumerate}
Ensuite si $L=\bigoplus_i e_iK$ alors y'a un $a\in \Or_K$
tel que $\bigoplus ae_iA\subset B$ d'où
\[\bigoplus_i ae_iA\subset B\subset B^\wedge\subset (\bigoplus_i ae_iA)^\wedge=\bigoplus_i (ae_i)^\wedge A\]
et on peut montrer que $B^\wedge$ est fractionnaire
via un dénominateur commun des $(ae_i^\wedge)_i$.

\subsection{$D_{L/K}$ est un idéal de $B$}
On a $B\subset B^\wedge$ d'où $D_{L/K}\subset B$.

\subsection{Caractérisation}
L'idéal $D_{L/K}^{-1}$ est maximal tel que 
\[Tr_{L/K}(D_{L/K}^{-1})=\Or_K\]
Pour le voir suffit de remarquer que pour $I\subset L$
fractionnaire si on a 
\[Tr_{L/K}(I.\Or_L)=Tr_{L/K}(I)=\Or_K\]
alors $I\subset D_{L/K}^{-1}$ par déf.

\subsection{Transitivité}
On veut montrer que $D_{L/K}=D_{L/M}D_{M/K}$, y suffit
de montrer sur les inverses (Dedekind). En fait on a
\[Tr_{L/M}(D_{L/K}^{-1})\subset D_{L/M}^{-1}\]
via la maximalité puis
\[Tr_{L/M}(D_{L/K}^{-1}D_{M/K})=D_{M/K}Tr_{L/K}(D_{L/K}^{-1})\subset D_{M/K}D_{M/K}^{-1}\subset\Or_K\]
d'où $D_{L/K}^{-1}D_{M/K}\subset D_{L/M}^{-1}$ puis
\[D_{L/K}^{-1}\subset D_{L/M}^{-1}D_{M/K}^{-1}.\]
À l'inverse 
\[Tr_{L/K}(D_{M/K}^{-1}D_{L/M}^{-1})\subset \Or_K\]
d'où le résultat.

\subsection{Corps locaux de caractéristique $0$}
On a toujours $\Or_L=\Or_K[\alpha]$ et on peut prouver
qu'alors si $f=\mu_\alpha$ :
\[D_{L/K}=(f'(\alpha))\]
d'où 
\begin{enumerate}
    \item $v_L(D_{L/K})=e-1$ ssi l'extension est modérée.
    \item $v_L(D_{L/K})\geq e$ en général.
\end{enumerate}

\section{La différente}
Elle est donnée par 
\[\D_{L/K}:=(\Or_L^\wedge)^{-1}=(\{x\in L| Tr_{L/K}(x\Or_L)\subset \Or_K\})^{-1}\]
avec $I^{-1}=_L(\Or_L:I)$.

\subsection{Base duale de $(\alpha^i)_i$ pour la trace}
Pour trouver la base duale de $\D_{L/K}^{-1}$
avec $L=K[\alpha]$ et $\alpha\in \Or_L$ on a en notant 
$P(T)=(T-\alpha)(\sum_{i=1}^{n-1} c_i(\alpha)T^i)$ que 
\[\sum_i \alpha_i^k\frac{P(T)}{P'(\alpha_i)(T-\alpha_i)}=T^k\]
d'où en développant 
$\sum_i \alpha_i^k\frac{c_j(\alpha_i)}{P'(\alpha_i)}=\delta_{ij}$.
Et ça c'est $Tr_{L/K}(\alpha\frac{c_j(\alpha)}{P'(\alpha)})$ d'où
$(\frac{c_j(\alpha)}{P'(\alpha)})_j$ est duale pour $(\alpha^k)_k$.
En plus vu que on peut réécrire 
\[([P(T)-P(\alpha))/(T-\alpha)=\sum_{i=1}^n a_i\frac{(T^i-\alpha^i)}{T-\alpha}\]
et en développant on trouve $\sum_{i=j+1}^n a_i\alpha_{i-1-j}=c_j(\alpha)$.
De $\alpha\in\Or_L$ on a $a_n=1$ d'où la matrice de transition de 
$1,\alpha,\ldots,\alpha^{n-1}$ vers $(c_j(\alpha))_j$ est triangulaire
inférieure avec une diagonale faite de $1$ d'où inversible dans 
$\Or_K$.

En conclusion la base duale
pour la trace de $(\alpha^i)_i$ est $(c_j(\alpha)/f'(\alpha))_j$.
Mais quand $\alpha\in\Or_L$, 
$\Or_K[\alpha]^\wedge=\frac{1}{P'(\alpha)}\Or_K[\alpha]$.
\subsection{La différente quand $\Or_L=\Or_K[\alpha]$}
Par exemple dans des corps locaux de caractéristique $0$ un tel
$\alpha$ existe toujours via $\Or_L=\Or_K[\theta,\pi_L]$ et
le fait que l'extension résiduelle est séparable.

Maintenant $\D_{L/K}=((\Or_L)^\wedge)^{-1}=\frac{1}{P'(\alpha)}\Or_L$.

\begin{rem}
    Directement si $L=K[\alpha]/K$ est non ramifiée, $P'(\alpha)\in \Or_L^\times$
    car $\bar P'(\bar \alpha)\ne 0\mod \pi_L$.
    d'où $\D_{L/K}=\Or_L$.
\end{rem}

\section{Transitivité}
Étant donné $L/F/K$, on a $\D_{L/K}=\D_{L/F}\D_{F/K}$. Suffit de
montrer
que, $\Or_{L/K}^\wedge=\Or_{L/F}^\wedge(\Or_{F/K}^\wedge)$ vu que 
$(IJ)^{-1}=I^{-1}J^{-1}$ c'est un calcul terme à terme. Et pour ça 
le seul cas pas évident c'est $x\in \Or_{L/K}^\wedge$ implique est
un produit.
\subsection{Base produit}
Dans le cas complet ça va, c'est que des dvrs d'où
on a une base produit. 

\section{Caractérisation}
On a $Tr(D_{L/K}^{-1})=\Or_K$ et c'est le plus grand idéal 
fractionnaire tel que c'est vrai. Faut penser à la base
duale pour le voir! 

\chapter{Résumé}
On a $Tr_{L/K}(D_{L/K}^{-1})=\Or_K$, la surjectivité vient de
la base duale. L'autre inclusion est par déf. 

La transitivité on peut prouver que 
$Tr_{L/F}(D_{L/K}^{-1})=D_{F/K}^{-1}$ sachant que si $x\in F$
et $v_i^*$ est dans la base duale de $L$ sur $F$ alors $xv_i^*$ 
est dans $D_{L/K}^{-1}$, la raison c'est que $Tr_{F/K}(x)\in\Or_K$.Par déf.

L'autre côté est simple.

L'existence de la base duale c'est que $Hom(L,K)$ est de
dimension $\leq [L:K]$ et $x\mapsto Tr(\_x)$ est injective par non
dégénérescence.









\end{document}
