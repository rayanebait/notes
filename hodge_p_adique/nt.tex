\documentclass[a4paper,12pt]{book}
\usepackage{amsmath,  amsthm,enumerate}
\usepackage{csquotes}
\usepackage[provide=*,french]{babel}
\usepackage[dvipsnames]{xcolor}
\usepackage{quiver, tikz}

%symbole caligraphique
\usepackage{mathrsfs}

%hyperliens
\usepackage{hyperref}

%pseudo-code
\usepackage{algpseudocode}
\usepackage{algorithm}
\makeatletter
  \renewcommand{\ALG@name}{Algorithme}
  \makeatother
\usepackage{fancyhdr}

\pagestyle{fancy}
\addtolength{\headwidth}{\marginparsep}
\addtolength{\headwidth}{\marginparwidth}
\renewcommand{\chaptermark}[1]{\markboth{#1}{}}
\renewcommand{\sectionmark}[1]{\markright{\thesection\ #1}}
\fancyhf{}
\fancyfoot[C]{\thepage}
\fancyhead[LO]{\textit \leftmark}
\fancyhead[RE]{\textit \rightmark}
\renewcommand{\headrulewidth}{0pt} % and the line
\fancypagestyle{plain}{%
    \fancyhead{} % get rid of headers
}

%bibliographie
\usepackage[
backend=biber,
style=alphabetic,
sorting=ynt
]{biblatex}

\addbibresource{bib.bib}

\usepackage{appendix}
\renewcommand{\appendixpagename}{Annexe}

\definecolor{wgrey}{RGB}{148, 38, 55}

\setlength\parindent{24pt}

\newcommand{\Z}{\mathbb{Z}}
\newcommand{\R}{\mathbb{R}}
\newcommand{\rel}{\omathcal{R}}
\newcommand{\Q}{\mathbb{Q}}
\newcommand{\C}{\mathbb{C}}
\newcommand{\N}{\mathbb{N}}
\newcommand{\K}{\mathbb{K}}
\newcommand{\A}{\mathbb{A}}
\newcommand{\B}{\mathcal{B}}
\newcommand{\Or}{\mathcal{O}}
\newcommand{\F}{\mathbb F}
\newcommand{\m}{\mathfrak m}
\renewcommand{\b}{\mathfrak b}
\renewcommand{\a}{\mathfrak a}
\newcommand{\p}{\mathfrak p}
\newcommand{\I}{\mathfrak I}
\newcommand{\Hom}{\textrm{Hom}}
\newcommand{\disc}{\textrm{disc}}
\newcommand{\Pic}{\textrm{Pic}}
\newcommand{\End}{\textrm{End}}
\newcommand{\Spec}{\textrm{Spec}}
\newcommand{\Frac}{\textrm{Frac}}

\newcommand{\cL}{\mathscr{L}}
\newcommand{\G}{\mathscr{G}}
\newcommand{\D}{\mathscr{D}}
\newcommand{\E}{\mathscr{E}}

\theoremstyle{plain}
\newtheorem{thm}{Théoreme}
\newtheorem{lem}{Lemme}
\newtheorem{prop}{Proposition}
\newtheorem{cor}{Corollaire}
\newtheorem{heur}{Heuristique}
\newtheorem{rem}{Remarque}
\newtheorem{rembis}{Remarque}
\newtheorem{note}{Note}

\theoremstyle{definition}
\newtheorem{conj}{Conjecture}
\newtheorem*{eq}{Équivalences}
\newtheorem{prob}{Problème}
\newtheorem{quest}{Question}
\newtheorem{prot}{Protocole}
\newtheorem{algo}{Algorithme}
\newtheorem{defn}{Définition}
\newtheorem{defnbis}{Définition}
\newtheorem{ex}{Exemple}
\newtheorem{exo}{Exercices}

\theoremstyle{remark}

\definecolor{wgrey}{RGB}{148, 38, 55}
\definecolor{wgreen}{RGB}{100, 200,0} 
\hypersetup{
    colorlinks=true,
    linkcolor=wgreen,
    urlcolor=wgrey,
    filecolor=wgrey
}

\title{Théorie de Hodge $p$-adique}
\date{}

\begin{document}
\maketitle
Programme: 1. ramification 2. corps perfectoides, extensions
profondément ramifiées, extensions arithmétiquement profinies.
3. corps des normes 4. représentation $p$-adiques des corps locaux.

Fontaine 1970-1980. Généralisation par Scholze.

\chapter{Préliminaires}
\section{Corps non-archimédien}

\section{Lemme de Krasner}
Soit $\alpha\in \bar K$ et $f=\mu_\alpha$. Soit aussi
$\alpha_1,...,\alpha_n$ ses racines avec $\alpha=\alpha_1$.
On note $d_\alpha:=min_{i>1}\{|\alpha-\alpha_i|\}$.
Supposons que pour $\beta\in \bar K$ on ait
\[|\beta-\alpha|_K<d_\alpha|\]
alors $K(\alpha)\subset K(\beta)$.
\section{Plus petit corps algébriquement clos complet}
Étant donné $K$ non archimédien complet. On a
\[\widehat{\hat K}=:\C_K\]
est complet et algébriquement clos. Grâce à Krasner.

\section{Action de $G_K=Gal(\bar K/K)$}
Avec la topologie discrète sur $G_K$. On a
pour tout $g,x$, $|gx|=|x|$!
On peut l'étendre à $\C_K$. Question :
\[\C_K^{G_K}=K?\]
Pas trivial mais oui. On le prouve plus tard.
C'est un théorème d'Ax-Tate.

\section{Corps locaux}
Un corps local est un corps complet de valuation
discrète et de corps résiduel fini.
\begin{rem}
    On sera souvent dans ce cas. Mais les théorèmes
    seront souvent vrai si le corps résiduel est seulement
    parfait.
\end{rem}
Maintenant $K$ est local et $|k_K|=q$.
\begin{prop}
    Pour tout $x\in k_K$ il existe un unique $[x]$
    t.q $[x]^q=[x]$. En plus la flèche, $k_K^*\to K^*$
    $x\mapsto [x]$ est un iso de $k_K^*\simeq \mu_{q-1}$.
\end{prop}

\subsection{Classification, vecteurs de Witt?}
Soit $K$ un corps local. Si $\char(K)=0$, alors
$K$ est une extension finie (à iso près) de $\Q_p$.
Si $\char(K)=p$ alors $K\simeq k_K((X))$ avec la
valuation $X$-adique. Sachant que $k_K=\F_q$ ducoup.

\section{Filtration}
Étant donné $K$ local, on note
\[U_K^{(i)}=\begin{cases}U_K,~i=0\\ 1+\pi_K^i\Or_K,~\geq 1\end{cases}\]
C'est juste $U_K^{(i)}=\bar B(1,i)$. On a
\[U_K^{(i)}/U_K^{(i+1)}=\begin{cases}k_K^*,~i=0\\ k_K^+~i\geq 0\end{cases}\]

\section{Vocabulaire dans le cas des corps locaux.}
Si $L/K$ est finie de corps locaux. On dit qu'elle est
\begin{itemize}
    \item Non ramifiée si $e=1$. (les corps résiduels sont finis!)
    \item Totalement ramifiée si $e=[L:K]$.
    \item Totalement modérément ramifiée si $e=[L:K]$ et
        $p\nmid e$.
    \item Totalement sauvagement ramifiée si $e=[L:K]$ et
        $e=p^k=[L:K]$.
\end{itemize}
Dans $K-\bar K$ on peut prendre $K^{un}$ l'union/le compositum
(?) des sous extensions non ramifiées. Maintenant, 
\[Gal(K^{un}/K)\simeq Gal(\bar k_K/k_K)\simeq \hat \Z\]
est engendré par le Frobenius (ça c'est bizarre y me semble
que c'est faux, neukirch class field theory première page mdr,
bah c'était faux mdr).
En particulier, on a $Frob_K\in Gal(K^{un}/K)$ t.q
\[Frob_K(x)=x^q\mod m_K\]
pour tout $x\in \Or_L$ t.q $L/K$ non ramifiée. 
On se restreint donc à $\bar K-K^{un}$ et on note
$I_K=Gal(\bar K/K^{un})$.

\begin{rem}
    Exercice, si $\char K=0$, alors y'a qu'un nb fini
    d'extensions de degré $\leq n$. En caractérstique 
    positive c'est faux, faut ajouter séparable de degré
    $\leq n$ (apparemment c dur).
\end{rem}

\section{La différente}
Étant donné $L/K$ finie séparable de corps locaux. On regarde
la forme trace $L\times L\to K$. Donnée par $tr(x,y)=Tr_{L/K}(xy)$.
On note
\[\Or_L'=\{x\in L| tr(x,y)\in\Or_K,\forall y\in \Or_L\}\]
c'est un idéal fractionnaire de $L$ qui contient $\Or_L$.
On note $\D_{L/K}=(\Or_L')^{-1}$, c'est un idéal de $\Or_L$.
Ca se définit bien dans des anneaux de Dedekind généraux.
On a 
\[\D_{L/K}=\D_{L/F}\D_{F/K}\]
et on déf/a
\[v_L(\D_{L/K})=\min\{\ldots\}\]
et c'est simple vu que $\D_{L/K}=(\pi_L^m)$.
On a 
\begin{prop}
    On a 
    \begin{enumerate}
        \item Si $\Or_L=\Or_K[\alpha]$ et $f=\mu_\alpha$.
    On a $\D_{L/K}=(f'(\alpha))$!
        \item $\D_{L/K}=\Or_L$ si et seulement si $L/K$ est
    non ramifiée.
        \item $v_L(\D_{L/K})\geq e-1$, $e=e(L/K)$.
        \item $v_L(\D_{L/K})=e-1$ ssi $p\nmid e$.
    \end{enumerate}
\end{prop}
\begin{proof}[1.]
    Étant donné $\alpha_i$ les racs de $f$. Claim :
    $\sum \frac{f(X)}{(X-\alpha_i)}\frac{ \alpha_i^r}{f'(\alpha_i)}=X^r$
    pour $0\leq r\leq n-1$ (c'est presque la dérivée). Pour le prouver,
    on peut évaluer en $n$ points. On évalue en les $\alpha_i$ et c'est clair.
    Pour une racine fixée $\alpha=\alpha_1$, on a 
    \[f(X)/(X-\alpha)=\sum_{i=0}^{n-1} b_i X^i\]
    avec $b_i\in \Or_L=\Or_K[\alpha]$. Maintenant on remarque
    que les termes de
    $\sum \frac{f(X)}{(X-\alpha_i)}\frac{ \alpha_i^r}{f'(\alpha_i)}=X^r$
    sont conjugués sous l'action de $G_{L/K}$ d'où on peut calculer
    \[\sum_{k=0}^{n-1} X^kTr_{L/K}(b_k\alpha^r/f'(\alpha))=X^r\]
    puis
    \[Tr_{L/K}(b_k\alpha^r/f'(\alpha))=\begin{cases}1,~k=r\\0,~sinon\end{cases}\]
    en particulier, on a une base duale de $(\alpha^i)_{i=1,\ldots,n-1}$ pour
    la trace : $(b_k/f'(\alpha))_k$. Deuxième claim : $(b_k)_k$ engendre $\Or_L$
    sur $\Or_K$. On peut le faire en comparant les coefficients de
    \[f(X)=(X-\alpha)(\sum b_i X^i)\]
    et par induction(?). Du claim on déduit que $\Or_L'=1/f'(\alpha)\Or_L$ puis
    le résultat.
\end{proof}
\begin{proof}[2.]
    On montre non ramifiée implique $\D_{L/K}=\Or_L$. On peut prendre
    $\alpha$ tq $k_K(\bar \alpha)=k_L$. Avec $f=\mu_\alpha$. On a 
    $\bar f'(\bar \alpha)\ne 0$, d'où $f'(\alpha)\in \Or_L^\times$
    et le résultat. L'inverse est clair, on prouve que $[L:K]=f$.
\end{proof}
\begin{proof}[3.]
    On a $\Or_L=\Or_K[\pi_L]$. Et $\mu_{\pi_L}=f$ Eisenstein.
    D'où $f'(\pi_L)=\sum (e-i)\pi_L^{e-i-1}$ et $v_L(e\pi_L^{e-1})=
    v_L(e)+(e-1)$ plus $v_L(a_i)\geq e$ d'où $v_L(f'(\pi_L))\geq e-1$.
    
\end{proof}
\begin{proof}[4.]
    Si $p\nmid e$ on a $v_L(e\pi_L^{e-1})=e-1<v_L(a_i)$. Si $p\mid e$
    on a $v_L(f'(\pi_L))\geq e$.
\end{proof}
\begin{proof}[5. (?)]
    Si $K-L_0-L$ est tq $L_0=K^{un}$ alors $\D_{L/K}=\D_{L/L_0}\D_{L_0/K}=\D_{L/L_0}$.
    Alors $\D_{L/K}\Or_L\equiv L=L_0\equiv L/K$ est non ramifiée.
\end{proof}
\section{Filtration de ramification}
On regarde $L/K$ galoisienne ($G$) de corps locaux. Alors
pour $i\geq -1$, $G_i=\{g\in G|\forall x\in \Or_L,~v_L(g(x)-x)\geq i+1\}$.
On voit facilement que c'est des sous groupes distingués de $G$.
On a en plus $G_{-1}=G$ et $G_0=I_K$. Puis pour $i\geq 0$, on a 
\[G_i=\{g\in G|v_L((g(\pi_L)/\pi_L)-1)\geq i\}\]
(ca marche psq $i\geq 0$)
on écrit, $\sum a_i\pi_L^i=x\in \Or_L$ avec
$a_i\in \Or_{L_0}$ ($L-L_0$ tot ram!) d'où $g(a_j)=a_j$ pour $g\in G_i$
vu que $G_i\subset I_K$. Ensuite c'est comme d'hab.
Maintenant on a 
\[G_i\to U_L^{(i)}/U_L^{(i+1)}\]
via $g\mapsto g(\pi_L)/\pi_L$ qui induit un morphisme
injectif $G_i/G_{i+1}$. En particulier 
\[G_i/G_{i+1}\textrm{ est abélien.}\]
on en déduit que les extensions galoisiennes de corps
locaux sont résolubles.
\begin{prop}
    $v_L(\D_{L/K})=\sum_{i=0}^\infty(|G_i|-1)$.
\end{prop}
\begin{proof}
    On prend $\alpha\in \Or_L$ tq $\Or_L=\Or_K[\alpha]$ et $f=\mu_\alpha$.
    On déf $i_G(g)=v_L(g\alpha-\alpha)$. On a 
    \[i_G(g)=i+1\equiv g\in G_i\]
    puis
    \begin{align*}
    v_L(\D_{L/K})=v_L(f'(\alpha))&=\sum_{i=2}^nv_L(\alpha-\alpha_i)\\
                                 &=\sum_{g\in G-id}v_L(\alpha-g\alpha)\\
                                 &=\sum_{g\in G-id}i_G(g)\\
                                 &=\sum_{i=0}^\infty (|G_i|-|G_{i+1}|)(i+1)\\
                                 &=\sum_{i=0}^\infty(|G_i|-1)
    \end{align*}
\end{proof}





\end{document}

