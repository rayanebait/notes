\documentclass[a4paper,12pt]{book}
\usepackage{amsmath,  amsthm,enumerate}
\usepackage{csquotes}
\usepackage[provide=*,french]{babel}
\usepackage[dvipsnames]{xcolor}
\usepackage{quiver, tikz}

%symbole caligraphique
\usepackage{mathrsfs}

%hyperliens
\usepackage{hyperref}

%pseudo-code
\usepackage{algpseudocode}
\usepackage{algorithm}
\makeatletter
  \renewcommand{\ALG@name}{Algorithme}
  \makeatother
\usepackage{fancyhdr}

\pagestyle{fancy}
\addtolength{\headwidth}{\marginparsep}
\addtolength{\headwidth}{\marginparwidth}
\renewcommand{\chaptermark}[1]{\markboth{#1}{}}
\renewcommand{\sectionmark}[1]{\markright{\thesection\ #1}}
\fancyhf{}
\fancyfoot[C]{\thepage}
\fancyhead[LO]{\textit \leftmark}
\fancyhead[RE]{\textit \rightmark}
\renewcommand{\headrulewidth}{0pt} % and the line
\fancypagestyle{plain}{%
    \fancyhead{} % get rid of headers
}

%bibliographie
\usepackage[
backend=biber,
style=alphabetic,
sorting=ynt
]{biblatex}

\addbibresource{bib.bib}

\usepackage{appendix}
\renewcommand{\appendixpagename}{Annexe}

\definecolor{wgrey}{RGB}{148, 38, 55}

\setlength\parindent{24pt}

\newcommand{\Z}{\mathbb{Z}}
\newcommand{\R}{\mathbb{R}}
\newcommand{\rel}{\omathcal{R}}
\newcommand{\Q}{\mathbb{Q}}
\newcommand{\C}{\mathbb{C}}
\newcommand{\N}{\mathbb{N}}
\newcommand{\K}{\mathbb{K}}
\newcommand{\A}{\mathbb{A}}
\newcommand{\B}{\mathcal{B}}
\newcommand{\Or}{\mathcal{O}}
\newcommand{\F}{\mathbb F}
\newcommand{\m}{\mathfrak m}
\renewcommand{\b}{\mathfrak b}
\renewcommand{\a}{\mathfrak a}
\newcommand{\p}{\mathfrak p}
\newcommand{\I}{\mathfrak I}
\newcommand{\Hom}{\textrm{Hom}}
\newcommand{\disc}{\textrm{disc}}
\newcommand{\Pic}{\textrm{Pic}}
\newcommand{\End}{\textrm{End}}
\newcommand{\Spec}{\textrm{Spec}}
\newcommand{\Frac}{\textrm{Frac}}

\newcommand{\cL}{\mathscr{L}}
\newcommand{\G}{\mathscr{G}}
\newcommand{\D}{\mathscr{D}}
\newcommand{\E}{\mathscr{E}}

\theoremstyle{plain}
\newtheorem{thm}{Théoreme}
\newtheorem{lem}{Lemme}
\newtheorem{prop}{Proposition}
\newtheorem{cor}{Corollaire}
\newtheorem{heur}{Heuristique}
\newtheorem{rem}{Remarque}
\newtheorem{rembis}{Remarque}
\newtheorem{note}{Note}

\theoremstyle{definition}
\newtheorem{conj}{Conjecture}
\newtheorem*{eq}{Équivalences}
\newtheorem{prob}{Problème}
\newtheorem{quest}{Question}
\newtheorem{prot}{Protocole}
\newtheorem{algo}{Algorithme}
\newtheorem{defn}{Définition}
\newtheorem{defnbis}{Définition}
\newtheorem{ex}{Exemple}
\newtheorem{exo}{Exercices}

\theoremstyle{remark}

\definecolor{wgrey}{RGB}{148, 38, 55}
\definecolor{wgreen}{RGB}{100, 200,0} 
\hypersetup{
    colorlinks=true,
    linkcolor=wgreen,
    urlcolor=wgrey,
    filecolor=wgrey
}

\title{Extensions cyclotomiques de $\Q/\Q_p$}
\date{}

\begin{document}
\maketitle
\chapter{Réduction au cas $\zeta_{p^r}$}
Avec $m=\prod p_i^{r_i}$ dans l'ordre, calcul successif
de $\zeta_m$ de l'anneau d'entier, du groupe de Galois
et peut-être de la ramification.
\section{$\zeta_m=\prod\zeta_{p_i^{r_i}}$}
Étant donné $m=\prod p_i^{r_i}$, on a
$|z=\prod\zeta_{p_i^{r_i}}|=m$ via $z^a=1$
implique $z^a/\zeta_{p_i^{r_i}}^a=\zeta_{p_i^{r_i}}^a$
et en mettant à la puissance $p_i^{r_i}$ des deux 
côtés on obtient $p_i^{r_i}\mid a\prod_{j\ne i}p_j^{r_j}$
d'où $p_i^{r_i}\mid a$ pour tout $i$ et le résultat.
\section{$\Z[\zeta_m]=\Or_{\Q(\zeta_m)}$}
On remarque que $\zeta_{p_i^k}\in\Z[\zeta_m]$ pour
tout $k$. En plus $\prod_i\Z[\zeta_{p_i^{r_i}}]=\Z[\zeta_m]$
ducoup si $\Or_{\Q(\zeta_{p_i^{r_i}})}=\Z[\zeta_{p_i^{r_i}}]$
il suffit de vérifier que 
$\prod_i \Or_{\Q(\zeta_{p_i^{r_i}})}=\Or_{\Q(\zeta_m)}$. Pareil,
il suffit de le vérifier à tout les premiers localement. Sauf que


\chapter{Cas $\zeta_{p^r}$}
Le but c'est de trouver la ramification, montrer
que $\Or_K[\zeta_p]=\Or_L$ pour $K=\Q$ et $\Q_p$.
Calculer le groupe de Galois, Trouver les groupes
de décompositions en lower et upper, écrire la
fonction de Herbrand.

\section{Polynôme minimal et groupe de galois.}
Étant donné $l\ne 1\mod p$, $1+X+\ldots+X^{p-1}\mod l$ est
irréductible. 

\section{Anneau d'entier}





\end{document}
