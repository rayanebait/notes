\documentclass[a4paper,12pt]{book}
\usepackage{amsmath,  amsthm,enumerate}
\usepackage{csquotes}
\usepackage[provide=*,french]{babel}
\usepackage[dvipsnames]{xcolor}
\usepackage{quiver, tikz}

%symbole caligraphique
\usepackage{mathrsfs}

%hyperliens
\usepackage{hyperref}

%pseudo-code
\usepackage{algpseudocode}
\usepackage{algorithm}
\makeatletter
  \renewcommand{\ALG@name}{Algorithme}
  \makeatother
\usepackage{fancyhdr}

\pagestyle{fancy}
\addtolength{\headwidth}{\marginparsep}
\addtolength{\headwidth}{\marginparwidth}
\renewcommand{\chaptermark}[1]{\markboth{#1}{}}
\renewcommand{\sectionmark}[1]{\markright{\thesection\ #1}}
\fancyhf{}
\fancyfoot[C]{\thepage}
\fancyhead[LO]{\textit \leftmark}
\fancyhead[RE]{\textit \rightmark}
\renewcommand{\headrulewidth}{0pt} % and the line
\fancypagestyle{plain}{%
    \fancyhead{} % get rid of headers
}

%bibliographie
\usepackage[
backend=biber,
style=alphabetic,
sorting=ynt
]{biblatex}

\addbibresource{bib.bib}

\usepackage{appendix}
\renewcommand{\appendixpagename}{Annexe}

\definecolor{wgrey}{RGB}{148, 38, 55}

\setlength\parindent{24pt}

\newcommand{\Z}{\mathbb{Z}}
\newcommand{\R}{\mathbb{R}}
\newcommand{\rel}{\omathcal{R}}
\newcommand{\Q}{\mathbb{Q}}
\newcommand{\C}{\mathbb{C}}
\newcommand{\N}{\mathbb{N}}
\newcommand{\K}{\mathbb{K}}
\newcommand{\A}{\mathbb{A}}
\newcommand{\B}{\mathcal{B}}
\newcommand{\Or}{\mathcal{O}}
\newcommand{\F}{\mathbb F}
\newcommand{\m}{\mathfrak m}
\renewcommand{\b}{\mathfrak b}
\renewcommand{\a}{\mathfrak a}
\newcommand{\p}{\mathfrak p}
\newcommand{\I}{\mathfrak I}
\newcommand{\Hom}{\textrm{Hom}}
\newcommand{\disc}{\textrm{disc}}
\newcommand{\Pic}{\textrm{Pic}}
\newcommand{\End}{\textrm{End}}
\newcommand{\Spec}{\textrm{Spec}}

\newcommand{\cL}{\mathscr{L}}
\newcommand{\G}{\mathscr{G}}
\newcommand{\D}{\mathscr{D}}
\newcommand{\E}{\mathscr{E}}

\theoremstyle{plain}
\newtheorem{thm}{Théoreme}
\newtheorem{lem}{Lemme}
\newtheorem{prop}{Proposition}
\newtheorem{cor}{Corollaire}
\newtheorem{heur}{Heuristique}
\newtheorem{rem}{Remarque}
\newtheorem{rembis}{Remarque}
\newtheorem{note}{Note}

\theoremstyle{definition}
\newtheorem{conj}{Conjecture}
\newtheorem*{eq}{Équivalences}
\newtheorem{prob}{Problème}
\newtheorem{quest}{Question}
\newtheorem{prot}{Protocole}
\newtheorem{algo}{Algorithme}
\newtheorem{defn}{Définition}
\newtheorem{defnbis}{Définition}
\newtheorem{ex}{Exemple}
\newtheorem{exo}{Exercices}

\theoremstyle{remark}

\definecolor{wgrey}{RGB}{148, 38, 55}
\definecolor{wgreen}{RGB}{100, 200,0} 
\hypersetup{
    colorlinks=true,
    linkcolor=wgreen,
    urlcolor=wgrey,
    filecolor=wgrey
}

\title{Décompositions d'extensions}
\date{}

\begin{document}
\maketitle
\section{Non ramifiée, modérément, sauvagement, totalement ramifié}
Pour le vocabulaire : Avec $L/K$ extension de corps de valuations
discrètes, i.e. $v_K$ discrète fixée.
\begin{enumerate}
    \item Non ramifié : Pour chaque $i$, $e_i=1$ et $k_{L_i}/k_K$
        est séparable.
    \item Modérément ramifié : pour chaque $i$, $p\nmid e_i$ et 
        $k_{L_i}/k_K$ est séparable.
    \item Sauvagement ramifié : il existe un $i$ tq $p\mid e_i$
        ou $k_{L_i}/k_K$ inséparable.
    \item Totalement ramiifié : $[L:K]=e$ et $\tilde\Or_K=\Or_L$.
        (On a la condition de finitude)
\end{enumerate}

Attention y'a pas toujours l'égalité $\sum e_if_i=[L:K]$, dans la
plupart des cas qui m'intéressent si quand même.

\section{Lien entre liberté dans $k_K,k_L$ et dans $L/K$}
On regarde $L=K(\alpha)$ et $P=\mu_\alpha$ unitaire dans 
$\Or_K[X]$. Si $\overline{Q(\alpha)}=0$ (liberté de $(\alpha^i)$)
on a $Q(\alpha)\in \m_L$ et pas dans $\m_K$. D'où on peut pas
directement comparer les libertés dans ce sens ! À l'inverse,
si $(\bar e_i)_i$ est libre dans $k_L-k_K$ et qu'on a 
$\sum a_ie_i=0$ alors $a_i\in \m_L\cap \Or_K=\m_K$. Si
y sont tous non nuls $0<|(\sum a_ie_i)|$ on a un problème.

\section{Factorisation de $\bar P=F^d$ et $e.f=d\deg F$}
Même contexte, dans le cas complet c'est plus simple : Par Hensel
$\bar P=F^d$ et $\deg(F)\mid f$ parce que $F$ se scinde dans
$k_L$ vu que $P$ se scinde dans $\Or_L$. En particulier on peut
faire descendre la racine. On déduit
\[e.f=\deg(P)=d.\deg(F)\]
d'où $e\mid d$ et $\deg(F)\mid f$.

\begin{rem}
    Comme Vincent m'a fait remarquer pas d'égalité vu que par
    exemple si $K[\alpha]/K$ est non ramifiée et $\alpha$ engendre 
    l'extension résiduelle alors $\pi_L^dP(X/\pi_L)$ annule
    $\pi_L\alpha$ mais $F=X^d$, donc on est dans le pire cas.
\end{rem}

\section{Polynômes d'eisenstein et extensions totalement ramifiées}
\begin{center}\textbf{(1)}\end{center}
Si $P(X)=X^d+\sum a_iX^i$ avec $v_K(a_0)=1$ et $v_K(a_i)\geq 1$
alors $L=K[X]/(P(X))$ est totalement ramifiée et $X$ est une 
uniformisante. Si $\alpha$ est une racine dans $L$ de $P$ :

Y'a deux points, $B=\Or_K[\alpha]$ a un seul idéal maximal car
$a_0$ et $\alpha$ sont dans le même idéal maximal et y contiennent
tous $a_0$ (!) puis $(a_0,\alpha)=\alpha B$ est maximal
(via le quotient!). Ça prouve que $B$ est local et principal donc
un DVR, i.e. $\tilde\Or_K=B$. Pour la valuation $e = d=[L:K]$
directement, d'où le résultat.

\begin{center}\textbf{(2)}\end{center}
Si $L/K$ est totalement ramifiée, alors $\pi_L$ est annulé par
un Eisenstein. L'idée c'est que si $P$ l'annule, alors
si $a_{i_0}\notin \m_K$ alors :
\[\pi_L^{i_0}(a_{i_0}/\pi_L^{i_0}+\sum_{i=i_0}^n a_i\pi_L^{i-i_0})\]
est de valuation $i_0$. Si $v_K(a_j)>0$ pour $j<i_0$ alors la
valuation est strictement plus grande que $e=v_L(\pi_K)$. Sauf
que
\[\sum_{i=0}^{i_0-1}a_i\pi_L^i=\pi_L^{i_0}(a_{i_0}/\pi_L^{i_0}+\sum_{i=i_0}^n a_i\pi_L^{i-i_0})\]
d'où c'est eisenstein. En plus
\[a_0/\pi_L^n=-1+\sum a_i\pi_L^i/\pi_L^n\]
d'où $v_L(a_0/\pi_L^n)=0$ vu que $v_L(a_i)\geq e$ et $n=e$.

\chapter{Cas complet}
\section{Extension totalement modérément ramifiée}
Cette fois on peut trouver $\pi_L$ et $\pi_K$ tels que 
$P(X)=X^e-\pi_K$. Déjà
\[\Or_K/\m_K\to \Or_L/\m_L\]
est un iso et donc si $u\pi_L^e=\pi_K$, on regarde $u=v$ dans
$k_L$ (car c'est là que $u$ vit) avec $v\in\Or_K$. D'où
$u=v+\epsilon$, $\epsilon\in \m_L$ 
(car c'est dans $k_L$ l'égalité). Ensuite $u=v(1+v^{-1}\epsilon)$.
Sauf que $1+v^{-1}\epsilon$ a une racine $e$-ème par Hensel, 
$\zeta$. D'où $(\pi_L\zeta)^e=\pi_K/v$.
\section{Trouver les extensions totalement modérement ramifiées}
En gros dans $L/K$ finie complète telle que $k_K-k_L$
est purement inséparable (c'est juste une généralisation), 
On regarde presque le corps engendré par $\pi_L^{e/e'}$.
On choisit $e'\mid e/p^v_p(e)$, il existe $k_L^{p^r}\subset k_K$
alors $ap^r+be'=1$ et
\[\bar u=(\bar u^{p^r})^a(\bar u^b)^{e'}\mod \m_L\]
et ducoup on relève $u=\lambda^a(\bar u^b)^{e'}(1+\epsilon)$
avec $\lambda\in \Or_K^\times $ puis comme d'hab le truc à droite
a une racine $e'$-ème par hensel, disons $\zeta$. D'où en notant
$\pi_{e'}=u^b\zeta\pi_L^{e/e'}$ c'est une racine $e'$-ème de 
$\pi_K/\lambda^a$. Alors
\[K(\pi_{e'})\]
est totalement ramifiée vu que 
\textbf{engendrée par un eisenstein.}

\subsection{Unicité}
C'est pas très satisfaisant.
\subsubsection{Apparté}
Si on regarde $\Or_K\to\Or_L/\m_L$ ça induit $i\colon k_K\to k_L$.
En particulier dire que $u\in k_L$ est en fait dans $k_K$ 
\textbf{ça veut dire que $u+\m_L=v+\m_L$ avec $v\in \Or_K$.}
\subsubsection{Preuve}
Concrètement, $\lambda=(\pi_1\pi_2^{-1})^{e'}\in \Or_K^\times$ et
en regardant dans $k_L$ ça engendrerait une sous-extension de 
degré premier à $p^r$, i.e $1$. D'où $\bar u=\bar v\in k_K$ et
$(\bar v)^{e'}(1+\epsilon)=\lambda$ sauf que 
$(1+\epsilon)=\lambda/(v)^{e'}$ d'où est dans $\Or_K$ puis
$1+\m_K$ c'est que des puissances $e'$-ème. On obtient que
$(\pi_1\pi_2^{-1})^{e'}=(v')^{e'}$ avec $v'\in \Or_K\times$.
En particulier comme les racines $e'$-ème de l'unité sont dans
$\Or_K$ (pas toute en fait, celle qui nous faut)
$\pi_1=u\pi_2$ avec $u\in\Or_K^\times$ d'où unicité.

\begin{rem}
    En résumé, pour tout $e'\mid e/p^{v_p(e)}$, on a une 
    sous-extension $K-L_{e'}-L$, dans le cas complet sous
    l'hypothèse d'extension résiduelle purement inséparable.
\end{rem}
\section{Trouver les sous-extensions non ramifiée}
En dessous de $K-L$ on regarde $k_K-k-k_L$ avec $k_K-k$ séparable.
On a une correspondance entre $k$ et $K-K_k-L$ où la première est
non ramifiée.
\subsection{Existence}
$k_K^{sep}=k_K(\bar\theta)$. Comme \textbf{tout se passe dans} 
$k_L$ on lift un polynôme $P\in\Or_K[X]$ de même degré qui a une
racine dans $\Or_L$ \textbf{Par hensel},
$\theta$ en est une racine et il est \textbf{scindé séparable} dans
$L$. On regarde $K(\theta)$, c'est séparable vu que $\bar P$ est
séparable \textbf{via la dérivée} $\mod \m_L$ (!). Et c'est non
ramifié vu que $f=\deg(\bar P)=\deg(P)=[K(\theta):K]$.
\subsection{Unicité}
Le détail en fait c'est qu'on prend un lift $P$ dans $\Or_K[X]$ du
pol min de $\theta\in k_{F'}=k=k_F$. Et on a $P\in \Or_{F'}[X]$
donc on applique hensel pour celui là, on obtient 
$\theta'\in F'\subset L$, d'où par l'uncité dans $L$, 
$\theta=\theta'$. Ça dit que $F\subset F'$ puis dimension.

\begin{rem}
    On aurait pu imaginer un lift différent de $P$ pour $F$ et
    $F'$.
\end{rem}

\subsection{Extension non ramifiée maximale}
On prend $k=k_K^{sep}$ et un générateur $\bar \theta$ fournit
$K(\theta)=:K^{un}$ puis pour n'importe quelle sous-extension non
ramifiée de $L$, $F$, on applique la partie d'avant dans $K^{un}/K$
à $k_F$ pour obtenir $F'\subset K^{un}$, sauf que $k_F=k_{F'}$
dans $k_L$ d'où $F=F'$.

\subsection{Remarque sur la fonctorialité de Hensel}
Hensel fournit une convergence de normes de Gauss d'un facteur
$f$ de $p$ en un $F$ de $P$. Et les normes de Gauss s'étendent
naturellement de $K[X]$ à $L[X]$. D'où la compatibilité.

\section{Extensions modérément ramifiée maximale}
On prend $K^{tam}$ qui correspond à $e/p^v_p(e)$ dans $L-K^{un}$.
Pour prouver que c'est maximal on prend une extension modérement
ramifiée $L-F-K$, on peut considérer $F-F^un-L$. Et on a
$k_{F^{un}}=k_K^{sep}$, ducoup on remplace $F$ par $F^{un}$. Et
on a $K^{un}\subset F^{un}$ par le trick de l'unicité non ramifiée.
Ensuite, $K^{un}-F^{un}$ est totalement modérément ramifiée,
d'où par construction dans l'autre section $F^{un}\subset K^{tam}$.

\begin{rem}
    $K-F-F^{un}$ est modérement ramifiée! Et $k_{F^{un}}-k_F-k_K$
    est séparable car les deux intermédiaires sont séparable.
\end{rem}

\section{Résumé}
On a un premier découpage 
\[K\to K^{un}\to K^{tam}\to L\]
dans le cas complet et $L/K$ finie. On peut aussi facilement 
regarder les extensions intérmédiaires des deux premières 
sous-extensions.


\chapter{Cas galoisien}
\section{Un peu de théorie de Galois}
Peut-être la chose la plus importante. Quand on a une extension
normale finie $L/K$, on a 
\[Aut(L/K)=Gal(K^{sep}/K)\]

ÉTant donné $K-L^H=F-L$ galoisienne, $F/K$ est galoisienne si et
seulement si $ H\leq G:=Gal(L/K)$ est distingué. Ou $F$ est stable
par l'action de $Gal(L/K)$.

\section{Groupe de décomposiiton et groupe de Galois résiduel}
Étant donné $L/K$ finie galoisienne, et $\m$ un idéal max.
On regarde $D=D_\m$ et $I=I_\m$. Avec pour rappel
\[D_\m=\{\sigma\in Gal(L/K)|\sigma(\m)=\m\}\]
et $I_\m=\ker(D_\m\to Aut(k_L/k_K))=Gal(k_K^{sep}/k_K)$. 

\subsection{$1\to I_\m\to D_\m\to Aut_k(k_\m)\to 1$ et
$k-k_\m$ est normale}
L'idée c'est de voir que dans $\Or_K[X]$ on a 
$P(X)=\prod (X-\sigma(x))$ et si $\bar x\in k_\m$ de pol min
$p(X)$ alors $\bar P(\bar x)=\overline{P(x)}=0$ d'où
\[p\mid \bar P\]
et le deuxième est scindé dans $L$ donc dans $k_\m$. Ensuite
faut montrer que $D_\m\to Aut_k(k_\m)=Gal(k_\m^s/k)$ est 
surjectif. L'idée c'est de lift un générateur 
\[k(\bar\theta)=k_\m^s\]
alors si $\tau\in Gal(k_\m^s/k)$ on peut trouver $\sigma\in G$
tel que $\overline{\sigma(\theta)}=\tau(\bar\theta)$ parce
que $p\mid P$!!! Ensuite faut juste lift $\theta$ intelligemment
pour que $\sigma\in D_\m$. On lift de
\[\ker(\prod \tilde\Or_K/\m'\to \tilde\Or_K/\m)\]
i.e. $\theta \in \prod_{\m\ne\m'} \m'-\m$
alors $\sigma^{-1}\m=\m'$ force $\sigma(\theta)\in \m$ d'où
\[\tau(\bar\theta)=0\]
puis $k_\m^s=k$. Suffit d'exclure ce cas
(où le résultat est clair).


\section{Résumé bref}
\begin{rem}
    On a des nouvelles formules : $e.f.g=[L:K]$. ET $|G|=
    |Gal(L/K)|=[L:K]$. Faut utiliser ça exhaustivement.
\end{rem}
Si $L/K$ est galoisienne, pas forcément complète, on a une
décomposition similaire, on fixe $\m=\m_L$ et $|.|_D,|.|_I$ les
restriction de $|.|_\m$ on a:
\[K-L^D-L^I-L\]
où $D=D_\m$ et $I=I_\m$. Maintenant $L^D-K$ est inerte en $\m$
($e=1=f$, d'où $\hat K=\hat{L^D}$!) et non ramifiée
($k_{L^D}-k_K$ est séparable). La raison 
\begin{itemize}
    \item $(|G|/|D|)e_{L/K}f_{L/K}=[L:K]$ et $|D|=e_{L/L^D}f_{L/L^D}$.
\end{itemize}
Le dernier argument c'est quon a une unique extension de 
$|.|_D$ à $|.|_L$ vu que le groupe de galois (ici $D$)agit
transitivement. En plus, $L\otimes_{L^D}\hat{L^D}$ est galoisienne
sur $\hat{L^D}$ de même groupe de galois. 
Maintenant
\[L^D-L^I\]
est non ramifiée, $k_D=k_K$ et $k_I=k_\m^s$ la clôture séparable
de $k_K$ dans $k_L=k_\m$. Et $L^I-L$ est totalement ramifiée
(elle a toute la ramification), en plus $k_\m^s=k_I-k_\m$ est
purement inséparable. L'argument pour les deux consiste à utiliser
exhaustivement l'exactitude de
\[1\to I_\m\to D_\m\to Aut_{k}(k_L)\]
et le fait que $k_L-k$ est normale dans le cas où $L/K$ galois 
(ca se montre bien en liftant etc..).




\section{Résumé très bref}
L'extension $K-L^D$ est immédiate car 
$\#\{\m|\m_K\}=|G/D|=[L^D:K]$, et $efg=|G/D||D|=|G/D|e_Df_D$.
Maintenant $k_L/k_I$ est purement inséparable car $Aut(k_L/k_I)=
I/I=1$ par la suite exacte sur $L/L^I$. Enfin, 
\[[k_\m^s:k]\leq [k_I:k]=[k_I:k_D]\leq |D/I|\]
sauf que le truc de gauche c'est $|D/I|$ par la suite exacte car
$L/K$ est galoisienne d'où $k_L/k$ est normale et 
$|Gal(k_\m/k)|=k_\m^s$. Et là $f_{sep}=[L^I:L^D]$. En conclusion
$k_L/k_I$ contient $f_{insep}$, $L/L^I$ contient
$e$, et $L^I/L^D$ contient $f_{sep}$. Enfin, on regarde $L^I-L$,
on peut supposer $L^I$ complet, et $E=(L^I)^{tam}$. D'où la tour
\[L^I-(L^I)^{tam}-L\]
Là $(L^I)^{tam}-L^I$ est totalement modérément ramifiée et
fixée par $I$, d'où galoisienne de groupe de Galois $T=I/P$ avec
$P$ le (car normal) $p$-sylow. Vu que 
$e_{L/(L^I)^{tam}}=p^{v_p(e)}$ et 
$k_L-k_{L^{I^{tam}}}=k_{L^I}$ est purement inséparable. Enfin,
$T$ est cyclique car $L^I$ contient les racines $e/p^{v_p(e)}$-eme
de l'unité. Le groupe $T$ est donné par $T\to \mu_{e/p^{v_p(e)}}$.
Alors on a
\[K-L^D-L^I-L^P-L\]

\section{Groupes de ramification}
On se place dans le cas complet, alors $G=D$.
On regarde les 
\[G_i=\ker(Gal(L/K)\to \Or_L/\m_L^{i+1})\]
qu'on peut traduire en $v_L(\sigma(x)-x)\geq i+1$ pour tout 
$x\in \Or_L$. Si $\Or_L=\Or_K[\alpha]$ on peut juste regarder
sur $\alpha$. On déf \[i_G(\sigma):=v_L(\sigma(\alpha)-\alpha)\]
\begin{rem}
    On pourrait probablement le définir comme un min sinon.
    On a aussi $I=G_0$ et $P=G_1$ (c'est pas trivial le deuxième).
\end{rem}

\subsection{Quotients des groupes de ramifications}
On regarde les $G_i/G_{i+1}$, le cas $i=-1$ c'est $D/I=Gal(k_L/k)$
donc on le saute. Le cas $i=0$ est un peu différent. On note
$U^{(i)}=\ker(\Or_L^\times\to (\Or_L/\m_L^i)^\times)$. On a
$U^{(i)}=1+\m_L^i$ pour $i>0$ et $U^{(0)}=\Or_L^\times$. On a (!)
\[\begin{cases}
    U^{(i)}/U^{(i+1)}\simeq k,~i\ne 0\\
    U^{(0)}/U^{(1)}\simeq k^\times,~i=0
\end{cases}\]
le premier donné par $1+\pi_L^ix\mapsto x$ le deuxième par
$x\mapsto x$ (oui ca marche). Pour le premier, la multiplicativité
modulo $U^{(i+1)}$ on écrit 
$y=1+\pi_L^i(x_1+x_2)+\pi_L^{2i}(x_1x_2)$ et 
$x^{-1}=(1+\pi_L^{2i}(x_1x_2))\in U^{(2i)}$. Alors $x^{-1}y=1+
\pi_L^i(x^{-1}(x_1+x_2)\mapsto x^{-1}(x_1+x_2)=x_2+x_2\mod\m_L$
car $x^{-1}\in 1+\m_L$. L'isomorphisme est une question
de cardinalité. Maintenant on a 
\[G_{i}/G_{i+1}\to U^{(i)}/U^{(i+1)}\]
donné par $\sigma \mapsto \sigma(\pi_L)/\pi_L$. En particulier,
$G_{i}/G_{i+1}$ est cyclique pour tout $i\geq 0$. Et dès que
$i>0$ c'est un $p$-groupe! 
\begin{rem}
    $\sigma\mapsto \sigma(\pi_L)/\pi_L$ c'est l'ami de toujours,
    c'est un super morphisme de groupes injectifs. Injectif c'est
    facile mais morphisme de groupe c'est pas évident.
\end{rem}
Plusieurs idées dans cette partie : L'idée c'est que 
$\sigma(x)-x\in \m_L^{i+1}$ se traduit en 
$\sigma(x)/x-1\m_L^{i+1}/x$. Ducoup avec $x=\pi_L$, 
\[\sigma(\pi_L)/\pi_L\in 1+\m_L^i=U^{(i)}\]
et avec $x=u\in U^{(i)}$ (!)
\[\sigma(u)/u\in 1+\m_L^{i+1}=U^{(i+1)}\]
trop cool. Ducoup en notant $\sigma(\pi_L)=\pi_Lu$ avec 
$u\in U^{(i)}$ on a 
\[\sigma_1(\sigma(\pi_L))/\pi_L=\sigma_1(\pi_L)\sigma_1(u)/\pi_L\]
et $\sigma_1(u)=(\sigma_2(\pi_L)/\pi_L)\sigma_1(u)/u$. Ducoup 
modulo $U^{(i+1)}$ c'est un morphisme de groupe (!).

\begin{rem}
    Pour rappel, on se restreint bien à $i\geq 0$. Quand $i=-1$
    $\sigma\mapsto \sigma(\pi_L)/\pi_L$ est pas à valeur dans
    $U^{(i)}$, y'a pas de $U^{(-1)}$.
\end{rem}

\chapter{Remarques}
\section{Le cas complet où $k_L-k_K$ est purement inséparable.}
On s'y en retrouve fait souvent : $L^I-L$ pour le cas galoisien
complet, $K^{un}-L$, $K^{tam}-L$. 
\section{Descriptions des $G_i$}
On a pour $g\in G_i$ la déf générale : pour tout $x$ : 
$v_L(gx-x)\geq i+1$, pour $\Or_L=\Or_K[\alpha]$, 
$v_L(g\alpha-\alpha)\geq i+1 $, et sinon pour $i\geq 0$ dans
le cas des corps locaux ma préf :
\[v_L(g(\pi_L)/\pi_L-1)\geq i\]
pour la troisième $g$ fixe $L^I$ et $L-L^I$ est totalement ramifiée
(hypothèse corps local, $k_I-k_L$ est purement inséparable donc
triviale) d'où $\Or_L=\Or_{L^I}[\pi_L]$, puis $x=\sum a_i\pi_L^i$
et (!)
\[gx-x=\sum a_i((g\pi_L)^i -\pi_L)\]
et $i_{L/K}(g)=i_{L/L^I}(g)$.

\section{Comparaison avec les $U^{(i)}$}
On regarde $i\geq 0$!!
Donc la troisième description est plus claire là. On regarde 
\[U_K^{(i)}=\ker(\Or_K^\times\to (\Or_K/\m_K^i)\times)\]
on a 
$U_K^{(i)}=\begin{cases} \Or_K^\times,~i=0\\ 1+\m_K^i~sinon\end{cases}$.
on a $g\in G_i\equiv g(\pi_L)/\pi_L\in U^{(i)}$. Et en plus
\[G_i\to U^{(i)}/U^{(i+1)}\]
via $g\mapsto g(\pi_L)/\pi_L$ est un m.g. L'idée clé c'est que
$\sigma(x)/x\in 1+x^{-1}\m_K^{(i+1)}$ d'où $U^{(i+1)}$ si
$x=u\in \Or_K^\times$ et $U^{(i)}$ si $x=\pi_K$ (!).


\section{Les $U^{(i)}$ et $k_K$}
On a $U^{(i)}/U^{(i+1)}\to
\begin{cases}k_K^\times,~i=0\\ k_K,~i>0\end{cases}$. Donnés par
$x\mapsto x$ et $1+\pi_K^ix\mapsto x$. En particulier
$G_0/G_1=I/P\hookrightarrow k_K^\times$ d'où d'ordre $\wedge p=1$
et $G_i/G_{i+1}\hookrightarrow k_K$ d'où un $p$-groupe. En 
particulier 
$|I|=|I/P|.\prod_{i=1}^\infty |G_i/G_{i+1}|=t.p^{v_p(|I|)}$.
Enfin, $L^I-L^P$ a pour groupe de galois $G_0/G_1$ est totalement
ramifiée et cyclique vu que sous-groupe fini multiplicatif d'un
corps. Ou sinon, vu que tame et a les racines de l'unité. 
\begin{rem}
    Penser que pour $g\ne g'$, $g(\pi_L)/\pi_L\ne g'(\pi_L)/\pi_L$
    d'où des racines de l'unité différente quand
    $\pi_L^e=\pi_K$.
\end{rem}


\end{document}


