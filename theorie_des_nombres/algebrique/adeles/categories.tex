\documentclass[a4paper,12pt]{book}
\usepackage{amsmath,  amsthm,enumerate}
\usepackage{csquotes}
\usepackage[provide=*,french]{babel}
\usepackage[dvipsnames]{xcolor}
\usepackage{quiver, tikz}

%symbole caligraphique
\usepackage{mathrsfs}

%hyperliens
\usepackage{hyperref}

%pseudo-code
\usepackage{algorithm}
\usepackage{algpseudocode}

\usepackage{fancyhdr}

\pagestyle{fancy}
\addtolength{\headwidth}{\marginparsep}
\addtolength{\headwidth}{\marginparwidth}
\renewcommand{\chaptermark}[1]{\markboth{#1}{}}
\renewcommand{\sectionmark}[1]{\markright{\thesection\ #1}}
\fancyhf{}
\fancyfoot[C]{\thepage}
\fancyhead[LO]{\textit \leftmark}
\fancyhead[RE]{\textit \rightmark}
\renewcommand{\headrulewidth}{0pt} % and the line
\fancypagestyle{plain}{%
    \fancyhead{} % get rid of headers
}

%bibliographie
\usepackage[
backend=biber,
style=alphabetic,
sorting=ynt
]{biblatex}

\addbibresource{bib.bib}

\usepackage{appendix}
\renewcommand{\appendixpagename}{Annexe}

\definecolor{wgrey}{RGB}{148, 38, 55}

\setlength\parindent{24pt}

\newcommand{\Z}{\mathbb{Z}}
\newcommand{\R}{\mathbb{R}}
\newcommand{\rel}{\omathcal{R}}
\newcommand{\Q}{\mathbb{Q}}
\newcommand{\C}{\mathbb{C}}
\newcommand{\Cat}{\mathcal{C}}
\newcommand{\Dat}{\mathcal{D}}
\newcommand{\Aat}{\mathcal{A}}
\newcommand{\N}{\mathbb{N}}
\newcommand{\K}{\mathbb{K}}
\newcommand{\A}{\mathbb{A}}
\newcommand{\B}{\mathcal{B}}
\newcommand{\Or}{\mathcal{O}}
\newcommand{\F}{\mathscr F}
\newcommand{\Hom}{\textrm{Hom}}
\newcommand{\disc}{\textrm{disc}}
\newcommand{\Pic}{\textrm{Pic}}
\newcommand{\End}{\textrm{End}}
\newcommand{\Spec}{\textrm{Spec}}
\newcommand{\Supp}{\textrm{Supp}}
\newcommand{\Ouv}{\textrm{Ouv}}
\newcommand{\im}{\textrm{im}}
\newcommand{\coker}{\textrm{coker}}
\newcommand{\coim}{\textrm{coim}}


\newcommand{\cL}{\mathscr{L}}
\newcommand{\G}{\mathscr{G}}
\newcommand{\D}{\mathscr{D}}
\newcommand{\E}{\mathscr{E}}
\renewcommand{\P}{\mathscr{P}}
\renewcommand{\H}{\mathscr{H}}

\makeatletter
\newcommand{\colim@}[2]{%
  \vtop{\m@th\ialign{##\cr
    \hfil$#1\operator@font colim$\hfil\cr
    \noalign{\nointerlineskip\kern1.5\ex@}#2\cr
    \noalign{\nointerlineskip\kern-\ex@}\cr}}%
}
\newcommand{\colim}{%
  \mathop{\mathpalette\colim@{\rightarrowfill@\scriptscriptstyle}}\nmlimits@
}
\renewcommand{\varprojlim}{%
  \mathop{\mathpalette\varlim@{\leftarrowfill@\scriptscriptstyle}}\nmlimits@
}
\renewcommand{\varinjlim}{%
  \mathop{\mathpalette\varlim@{\rightarrowfill@\scriptscriptstyle}}\nmlimits@
}
\makeatother

\theoremstyle{plain}
\newtheorem{thm}{Théoreme}
\newtheorem{lem}{Lemme}
\newtheorem{prop}{Proposition}
\newtheorem{cor}{Corollaire}
\newtheorem{heur}{Heuristique}
\newtheorem{rem}{Remarque}
\newtheorem{note}{Note}

\theoremstyle{definition}
\newtheorem{conj}{Conjecture}
\newtheorem{prob}{Problème}
\newtheorem{quest}{Question}
\newtheorem{prot}{Protocole}
\newtheorem{algo}{Algorithme}
\newtheorem{defn}{Définition}
\newtheorem{exmp}{Exemples}
\newtheorem{exo}{Exercices}
\newtheorem{ex}{Exemple}
\newtheorem{exs}{Exemples}

\theoremstyle{remark}

\definecolor{wgrey}{RGB}{148, 38, 55}
\definecolor{wgreen}{RGB}{100, 200,0} 
\hypersetup{
    colorlinks=true,
    linkcolor=wgreen,
    urlcolor=wgrey,
    filecolor=wgrey
}

\title{Adèles et idèles}
\date{2024-2025}

\begin{document}
\maketitle
\tableofcontents

\chapter{Cas des corps locaux}
J'ai trouvé ça : \href{https://math.mit.edu/classes/18.785/2015fa/LectureNotes12.pdf}{lien}.
Apparemment $p^{v_p(.)}$ c'est la valeur absolue naturelle grâce
à la mesure de Haar! Wah en fait c'est joli proposition 12.14. ET
c'est Weil qui a fait la preuve générale sur les groupes localement
compacts de l'existence et unicité de la mesure de Haar.
\section{Mesure de Haar, valuations discrètes et modules}
Ce que Weil définit comme un module, en fait c'est une valeur
absolue lol. Genre d'après le lien 
$\mu(xS)=(\card k_K)^{v(x)}\mu(S)$. Et l'argument clé qui me
paraissait mystique c'est qu'on normalise $\mu$ avec $\mu(\Or_K)=1$
comme c'est compact puis :
\[\mu(\Or_K)=[\Or_K:x\Or_K]\mu(x\Or_K)=(\card k_K)^{v(x)}\mu(x\Or_K)\]
vu que $\Or_K=\sqcup y\Or_K$ sur un système de représentants de
$k_K$ c'est ouf mdr. D'où $\mu(x\Or_K)=(1/\card k_K)^{v(x)}$!!
DOnc y'a qu'une valeur absolue compatible avec la mesure de Haar.


\chapter{Théorie de la mesure et mesures de Haar}
Je fais un petit chap sur ça juste parce que ça me branche.
La source c'est \href{https://terrytao.wordpress.com/wp-content/uploads/2012/12/gsm-126-tao5-measure-book.pdf}{ça}.
Le cours de tao! Je regarde aussi le cours de Villani qui est très
bien.
\section{Résumé après lectures}
Concrètement pour les constructions et théorèmes d'existences 
Villani utilise le théorème de Carathéodory à gogo à partir
de constructions naturelles! Il construit la mesure de Lebesgue
et la mesure produit facilement grâce à ça.

\section{Mesures de Jordan et intégrale de Riemann}
On a les boites $:=\prod I_i$ des produits d'intervalles. 
Les ensembles élémentaires qui en sont des unions finies.
Puis la mesure de Jordan ceux ci défini de manière évidente.
On dit que $E\subset \R^d$ est jordan mesurable si 
\begin{enumerate}
  \item $m_o(E):=inf_{B\supset E \textrm{élémentaire}} m(B)<\infty$.
  \item $m_i(E):=sup_{B\subset E \textrm{élémentaire}} m(B)<\infty$.
  \item $m_o(E)=m_i(E)=: m(E)$.
\end{enumerate}
\subsection{Exemples}
Maintenant deux bons exemples peut-être c'est les rationnels t.q :
\begin{enumerate}
  \item $m_o(\Q)=\infty$. Alros que la mesure de Lebesgue est nulle.
  \item $m_i(\Q)=0$ car on l'approche par unions finies de points.
\end{enumerate}
et le graphe de $f\colon B\subset \R^d\to \R$ continue avec $B$ une boite fermée, alors
\begin{enumerate}
  \item $m_o(E)=0$ parce que $f$ est uniformément continue sur $B$.
  \item La preuve c'est qu' on a un $\delta$ et $\epsilon$ t.q 
    $\Gamma\subset \cup B_\delta\times B_\epsilon$
    un nombre linéaire de copies du truc de droite d'où
    \[m(B_\delta\times B_\epsilon).m(B)/\delta\geq m(\Gamma)\]
    en particulier $\delta\epsilon.m(B)/\delta\geq m(\Gamma)$ d'où
    le résultat. Le point c'est que $B_\delta$ bouge en même temps
    que $B_\epsilon$ vu que c'est pas pointé. Penser aux recouvrements
    de $X\times_Y X$ recouverts diagonalement! Ca fait une dépendance 
    linéaire sur le nombre de multiple de $B_\delta\times B_\epsilon$
    nécessaires.
\end{enumerate}
\subsection{Critère}
Être Jordan mesurable c'est facilement équivalent à ce que
pour tout $\epsilon >0$ on puisse trouver $A\subset E\subset B$
tels que $m(B-A)\leq \epsilon$. Parce que
\[m(A)\leq m_i(E)\leq m_o(E)\leq m(B)\leq m(A)+\epsilon\]

\subsection{Lien avec l'intégrale de Riemann}
Pour le lien avec l'intégrale de Riemann. On prends $f\colon [a,b]\to \R$
et une partition pointée $P=((x_0,\ldots,x_n),(x_1^*,\ldots, x_n^*))$ avec
$x_i^*\subset [x_{i-1},x_i]$. Ensuite on définit
\[R(f,P):=\sum f(x_i^*)(x_i-x_{i-1})\]
et $\Delta(P):=sup_i |x_i-x_{i-1}|$ puis
\[\int_a^b f(x)dx:=\lim_{\Delta(P)\to 0} R(f,P)\]
On peut maintenant définir l'intégrale comme une aire via
la mesure de Jordan. On regarde
\[E^+=\{(x,t)|x\in [a,b],0\leq t\leq f(x)\}\]
la partie au dessus de $0\times \R$ 
et 
\[E^-=\{(x,t)|x\in [a,b],f(x)\leq t\leq 0\}\]
la partie en dessous. Alors $f$ est Riemann intégrable
si et seulement si $E^+$ et $E^-$ sont Jordan mesurables. Alors
$\int_a^b f=E^+-E^-$!
\subsection{Preuve}
En prenant $f$ positive, le point c'est que si $E^+$ est
Jordan mesurable on peut construire $R(f,P)$ comme suit :
On regarde une union disjointe de boites qui tend vers 
$E^+$. C'est des rectangles donc on peut supposer qu'elle
collent $0\times \R$. Ensuite elle contiennent les $(x,f(x))$
en particulier $(x,\sup(f(x')))$ dans une petite boule. Puis
on enchaine...
\subsection{Unicité de l'intégrale de Riemann comme Fonctionnelle}
Une fonctionnelle c'est une application linéaire de $C(X,\R)$ dans
$\R$. Les propriétés c'est la valeur en les indicatrices est la
mesure.


\section{Mesure de Lebesgue sur $\R^d$}
On peut redéfinir l'intégration habituelle. La mesure est 
pas si facile à construire. C'est la complétion de celle donnée
sur les intervalles par la longueur en dimension $1$. Et la mesure
produit en dimension $d$.
\section{Mesure de Lebesgue abstraite}
Page 93, il y parle de choses familières, convergence
d'ensembles! Via la convergence de $1_{E_n}$ en tout
point. Intersections et unions d'ensembles imbriqués
et convergence de mesure (si mesure finie).
\subsection{Catégorie des espaces mesurés}
Une fonction mesurable $f\colon (X,\tau)\to (Y,\theta)$
est telle que $f^{-1}(E)$ est $\theta$-mesurable si
$E$ est $\theta$-mesurable. Et ca se compose! 
\subsection{$f_*\mu$ et changement de variable!}
Étant donné $\varphi\colon X\to Y$ mesurable on peut définir
$\varphi_*\mu(E):=\mu(\varphi^{-1}E)$! En particulier
si on a aussi $f\colon Y\to \R$ on peut montrer que
\[\int_X f\circ \varphi d\mu=\int_Yfd\varphi_*\mu!\]
Apparemment faut un théorème de convergence dominée.
\subsection{$\mu|_E$}
On peut restreindre la mesure à un espace mesurable $E$!
Par exemple $GL_n(K)\subset K^{n^2}$ avec $K$ muni d'une
valeur absolue par exemple d'où une mesure. À l'inverse
si $K$ est muni d'une mesure de Haar, on peut définir une valeur absolue
à la Weil. Ensuite $GL_n(K)$ est ouvert dans $K^{n^2}$
d'où on peut restreindre $\mu$.

\subsection{$\sigma$-algèbres finies}
Si $(X,\tau,\mu)$ est mesuré et $\tau$ est finie, 
$X$ est partitionné en un nombre fini d'atomes
(simplement union disjointe de trucs mesurables $X=\cup A_i$,
via l'indicatrice, l'union,... Ça donne une algèbre booléenne).
Maintenant là dessus une fonction mesurable $f\colon X\to [0,+\infty]$
c'est juste une somme finie d'indicatrices :
\[f:=\sum_{i=1}^n c_i 1_{A_i}\]
On peut définir ensuite l'intégrale via
\[\int_X f:=\sum_i c_i\mu(A_i)\]
Inversement, si $f$ est une somme finie d'indicatrice
\[f=\sum_{i=1}^n c_i 1_{A_i}\]
Alors on peut prendre la $\sigma$-algèbre engendrée par
les $f^{-1}(c_i)$.
\subsection{Engendrer, étendre, comparer des $\sigma$-algèbres}
C'est juste indicatif cette section, en fait on peut prendre
des $\sigma$-algèbres plus fines et ça laisse invariantes
les intégrales de fonctions simples!

\subsection{Théorème de Carathéodory}
Y'a un certain Théorème de Carathéodory (généralisé) qui permet
de construire des mesures à partir d'une fonction 
$\mu\colon \F\to [0,+\infty]$ où $F\subset P(X)$ est une famille de
partie stable par intersection (faut des conditions sur $\F$, $\mu$
et $X$). C'est intéressant. En particulier
on peut prendre les pavés $A\times B$ par exemple.

L'extension $\mu^*$ est complète! Avec complet qui veut dire
$A\subset B$ et $\mu(B)=0$ implique $A$ mesurable.

\subsection{Construction de la mesure de lebesgue et les mesurables}
On utilise le théorème du dessus à plusieurs reprises sur
les pavés pour construire la mesure sur $\R$ puis les mesures
produits.  En particulier, la mesure de lebesgue est une complétion
et un truc mesurable c'est un $E$ tel que $A\subset E\subset B$
avec $A,B$ boréliens tels que $\mu(A)=\mu(B)$.
\subsection{Intégrale de Lebesgue}
Étant donné $f\colon X\to [0,+\infty]$ mesurable.
On peut définir
\[\int_X fd\mu:=\sup_{0\leq g\leq f;g~simple}\int_X gd\mu\]
\begin{rem}
  Si on a une mesure sur $GL_n$, par exemple celle de lebesgue
  sur $GL_n(\R)\subset \R^{n^2}$ la continuité du déterminant
  rend mesurable $GL_n(\R)$ et le det. D'où on peut calculer
  des intégrales là dessus. 
\end{rem}
Pour l'étendre à $f\colon X\to \R$ mesurable on doit avoir 
l'absolue intégrabilité
\[||f||_{L^1(X,f,\mu)}:=\int_X |f|d\mu\]
ensuite on peut définir $f^\pm:=\max(0,\pm f)$. 
Puis si $f$ est absolument intégrable :
\[\int_X f:=||f^+||-||f^-||\]
Enfin, pour $f\colon X\to \C$ on déf via la partie imaginaire
et réelle. Page $100$ y'a pleins de propriétés de l'intégrale.
Notamment les
propriétés vraies $\mu$-presque partout. Vu que les ensembles
de mesures $0$ sont invisible du point de vue de l'intégrale.
Aussi des découpages via $f.1_{E_n}$. 

\section{Théorème de convergence (page 105)}

\section{Ou regarder}
Page $100$-$103$ y'a pleins de choses très cool et intuitives
sur le calcul d'intégrales!

\section{Formes volumes}
Woh, le cours de variété algébrique va me servir.

\chapter{à faire}

\section{Mesures}
Continuité sur un compact métrique implique uniforme continuité.
\subsection{Lien mesure et $f$ mesurable}
Étant donné $X$ et une mesure finie. La mesurabilité de $f$
force des choses sur $f$ ? Étant donné des tribus particulières?

%\printbibliography
\end{document}

