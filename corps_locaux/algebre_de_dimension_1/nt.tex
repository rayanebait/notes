\documentclass[a4paper,12pt]{book}
\usepackage{amsmath,  amsthm,enumerate}
\usepackage{csquotes}
\usepackage[provide=*,french]{babel}
\usepackage[dvipsnames]{xcolor}
\usepackage{quiver, tikz}

%symbole caligraphique
\usepackage{mathrsfs}

%hyperliens
\usepackage{hyperref}

%pseudo-code
\usepackage{algpseudocode}
\usepackage{algorithm}
\makeatletter
  \renewcommand{\ALG@name}{Algorithme}
  \makeatother
\usepackage{fancyhdr}

\pagestyle{fancy}
\addtolength{\headwidth}{\marginparsep}
\addtolength{\headwidth}{\marginparwidth}
\renewcommand{\chaptermark}[1]{\markboth{#1}{}}
\renewcommand{\sectionmark}[1]{\markright{\thesection\ #1}}
\fancyhf{}
\fancyfoot[C]{\thepage}
\fancyhead[LO]{\textit \leftmark}
\fancyhead[RE]{\textit \rightmark}
\renewcommand{\headrulewidth}{0pt} % and the line
\fancypagestyle{plain}{%
    \fancyhead{} % get rid of headers
}

%bibliographie
\usepackage[
backend=biber,
style=alphabetic,
sorting=ynt
]{biblatex}

\addbibresource{bib.bib}

\usepackage{appendix}
\renewcommand{\appendixpagename}{Annexe}

\definecolor{wgrey}{RGB}{148, 38, 55}

\setlength\parindent{24pt}

\newcommand{\Z}{\mathbb{Z}}
\newcommand{\R}{\mathbb{R}}
\newcommand{\rel}{\omathcal{R}}
\newcommand{\Q}{\mathbb{Q}}
\newcommand{\C}{\mathbb{C}}
\newcommand{\N}{\mathbb{N}}
\newcommand{\K}{\mathbb{K}}
\newcommand{\A}{\mathbb{A}}
\newcommand{\B}{\mathcal{B}}
\newcommand{\Or}{\mathcal{O}}
\newcommand{\F}{\mathbb F}
\newcommand{\m}{\mathfrak m}
\renewcommand{\b}{\mathfrak b}
\renewcommand{\a}{\mathfrak a}
\newcommand{\p}{\mathfrak p}
\newcommand{\q}{\mathfrak q}
\newcommand{\I}{\mathfrak I}
\newcommand{\Hom}{\textrm{Hom}}
\newcommand{\disc}{\textrm{disc}}
\newcommand{\Pic}{\textrm{Pic}}
\newcommand{\End}{\textrm{End}}
\newcommand{\Spec}{\textrm{Spec}}
\newcommand{\Frac}{\textrm{Frac}}

\newcommand{\cL}{\mathscr{L}}
\newcommand{\G}{\mathscr{G}}
\newcommand{\D}{\mathscr{D}}
\newcommand{\E}{\mathscr{E}}

\theoremstyle{plain}
\newtheorem{thm}{Théoreme}
\newtheorem{lem}{Lemme}
\newtheorem{prop}{Proposition}
\newtheorem{cor}{Corollaire}
\newtheorem{heur}{Heuristique}
\newtheorem{rem}{Remarque}
\newtheorem{rembis}{Remarque}
\newtheorem{note}{Note}

\theoremstyle{definition}
\newtheorem{conj}{Conjecture}
\newtheorem*{eq}{Équivalences}
\newtheorem{prob}{Problème}
\newtheorem{quest}{Question}
\newtheorem{prot}{Protocole}
\newtheorem{algo}{Algorithme}
\newtheorem{defn}{Définition}
\newtheorem{defnbis}{Définition}
\newtheorem{ex}{Exemple}
\newtheorem{exo}{Exercices}

\theoremstyle{remark}

\definecolor{wgrey}{RGB}{148, 38, 55}
\definecolor{wgreen}{RGB}{100, 200,0} 
\hypersetup{
    colorlinks=true,
    linkcolor=wgreen,
    urlcolor=wgrey,
    filecolor=wgrey
}

\title{Algèbre en dimension $1$}
\date{}

\begin{document}
\maketitle

\chapter{Annulateurs et idéaux quotients}
Je veux parler de l'utilisation des annulateurs à coefficients
dans différents anneaux et quotients d'idéaux dans différents
anneaux. Par exemple, si j'ai $I,J$ deux idéaux fractionnaires
de $A$ avec $K=Frac(A)$. Je peux considérer
\[_A(I:J)\]
ou
\[_K(I:J)\]
ou bien
\[_K(A:I)\]
ou encore
\[_{S^{-1}A}(I:J)\]
à noter que c'est pas clair $_K(I:J)$ vu que $I$ et $J$ sont pas
des $K$-modules. Mais cet ensemble fait sens. En général si
$M,N$ sont des $A$-modules contenus dans un $S^{-1}A$-module $P$,
alors on peut faire sens de 
\[_{S^{-1}A}(M:N)\]
et ici $K$ est un $S^{-1}A$ module pour tout $S$.

\section{Arguments locaux}
Considérer les $(A:I)$ et $I.(A:I)\subseet A$ permet de ramener
$I$ dans $A$. Mais aussi d'avoir une manière propre d'écrire des
arguments communs à plusieurs localisations. Par exemple, 
écrire
\[_{A_\p}(IA_\p:JA_\p)=A_\p _{A_\p}(JA_\p:IA_\p)\]
se traduit en $IA_\p=JA_\p$. Et de cette dernière condition
on peut montrer que $_A(I:(x))\nsubseteq \p$ pour tout $x\in J$.
D'où si c'est vrai pour tout $\p$ alors 
\[_A(I:(x))=A\]
puis $x\in I$ et inversement.
\chapter{Balades dans ComRing}
\section{Localisation et cloture intégrale}
À savoir que la localisation commute avec clôture intégrale.


\subsection{Si $A$ est intégralement clos}
Pour $S$ une partie multiplicative, si $A$ est intégralement clos,
et $x^n+\sum a_i x^i=0$ avec $a_i\in S^{-1}A$. Il existe $s$ tel
que $sx\in A$ puis $x\in S^{-1}A$.

\subsection{Sinon}
On prends $\tilde A$ sa clôture dans $K$ alors $\varphi(S)$ 
est multiplicative et $\varphi(S)^{-1}\tilde A$ est intégralement
clos. Reste à voir que $\tilde{S^{-1}A}$ est intégralement
close et égale. En gros qu'on a
% https://q.uiver.app/#q=WzAsNSxbMCwwLCJBIl0sWzEsMCwiU157LTF9QSJdLFswLDEsIlxcdGlsZGUgQSJdLFsxLDEsIlxcd2lkZXRpbGRle1Neey0xfUF9Il0sWzEsMiwiaShTKV57LTF9XFx0aWxkZSBBIl0sWzAsMSwiIiwwLHsic3R5bGUiOnsiaGVhZCI6eyJuYW1lIjoibm9uZSJ9fX1dLFswLDIsIiIsMix7InN0eWxlIjp7ImhlYWQiOnsibmFtZSI6Im5vbmUifX19XSxbMiwzLCIiLDIseyJzdHlsZSI6eyJoZWFkIjp7Im5hbWUiOiJub25lIn19fV0sWzEsMywiIiwwLHsic3R5bGUiOnsiaGVhZCI6eyJuYW1lIjoibm9uZSJ9fX1dLFsyLDQsIiIsMix7InN0eWxlIjp7ImhlYWQiOnsibmFtZSI6Im5vbmUifX19XSxbMyw0LCI/IiwxLHsic3R5bGUiOnsiYm9keSI6eyJuYW1lIjoibm9uZSJ9LCJoZWFkIjp7Im5hbWUiOiJub25lIn19fV1d
\[\begin{tikzcd}
	A & {S^{-1}A} \\
	{\tilde A} & {\widetilde{S^{-1}A}} \\
	& {i(S)^{-1}\tilde A}
	\arrow[no head, from=1-1, to=1-2]
	\arrow[no head, from=1-1, to=2-1]
	\arrow[no head, from=1-2, to=2-2]
	\arrow[no head, from=2-1, to=2-2]
	\arrow[no head, from=2-1, to=3-2]
	\arrow["{?}"{description}, draw=none, from=2-2, to=3-2]
\end{tikzcd}\]
égalité au niveau du $?$. On a clairement 
\[S^{-1}A\subset i(S)^{-1}\tilde A\] d'où 
\[\widetilde{S^{-1}A}\subset {i(S)^{-1}\tilde A}\]
par minimalité en plus on a 
$\tilde A\subset \widetilde{S^{-1}A}$ et $i(S)$ est
clairement inversible par cette flèche d'où l'égalité par propriété
universelle par exemple.

\section{Extensions entières où transmettre la dimension.}
Si $A\hookrightarrow B$ est fini. Alors
\[\Spec(B)\to \Spec(A)\]
est surjectif et a des fibres finies. En plus $\dim(A)=\dim(B)$.
On peut isoler deux choses déjà
\begin{enumerate}
    \item Si $\m\subset B$ est maximal, $\m\cap A$ aussi. Parce
	que $A/\m\cap A \to B/\m$ est entière.
    \item Si $\q\subset A$ est premier. $S=(A-\q)$ est multiplicative
	dans $B$. Et $A\subset S^{-1}B\ne 0$, d'où un idéal maximal
	$\p$ vérifie $\p\cap A=\q$. Pour le vérifier, 
	\[A_\q \to S^{-1}B\]
	est entière, d'où $\p\cap A_\q$ est maximal puis $\p\cap A_\q= \q A_\q$
	et le résultat. $\p\cap A$ est pas max parce que $A\to S^{-1}B$ est
	pas entière.
\end{enumerate}
ensuite pour montrer que ça préserve la dimension suffit de montrer 
que si $(0)\ne \p\in \Spec(B)$ alors $\p\cap A\ne 0$. Suffit de 
prendre une relation entière minimale $x(\sum a_i x^{i-1})=-a_0$
avec $x\in \p-0$ d'où $a_0\in \p\cap A$ et par minimalité $a_0\ne 0$.

\subsection{Balades}
En fait on est dans le cadre de morphismes finis dominants 
de schémas affines
\[\Spec(B)\to \Spec(A).\]
Vu que c'est dominant c'est surjectif et de fibres finies.
En plus la dimension est toujours préservée.

\subsection{Normalisation}
Si $A$ est intègre, alors $A\subset \tilde A\subset \Frac(A)$
est entier et on obtient une normalisation entière dominante
\[\Spec(\tilde A)\to \Spec(A)\]
mais pas forcément finie ! Si $A$ est une $k$-algèbre de type
fini c'est vrai.




\chapter{Algèbre en dimension $1$}
Je vais quasi-toujours rajouter la noethérianité.


\section{Anneaux de valuation discrète}
Pour le contexte, moi je m'intéresse au cas intègre déjà et
au cas où le DVR est un $A_\p$ pour un anneau noethérien intègre
de dimension $1$. Sa clôture intégrale dans $\Frac(A)$ devient
un anneau de Dedekind.
\subsection{Les 2 définitions.}
Y'a deux manières de les voir :
\begin{enumerate}
    \item Dans un corps $(K,v)$ muni d'une valuation discrète.
	Avec $A=\{x\in K|v(x)=0\}$. 
    \item Comme un anneau principal (donc intègre) ayant un seul
	idéal premier non nul.
\end{enumerate}
L'implication $1.$ implique $2.$ consiste juste à se placer
dans l'espace ambiant $K$. L'autre côté consiste à construire une
valuation par l'absurde,
via $t\in A$ est soit dans $A^\times$ soit dans $\m$. D'où 
on construit $t^{(i)}=\pi t^{(i+1)}$ puis une suite 
\[(t^{(i)})\subset (t^{(i+1)})\]
et ca se déroule en rappelant que $1+\m\subset A^\times$.

\newpage
\subsection{Première caractérisation}
\begin{eq}
    Dans un anneau noethérien, 
    \begin{center}
	DVR $\equiv$ local, noethérien, $\m=(\pi)$ non nilpotent.
    \end{center}
\end{eq}
On veut écrire $x=\pi^n u$ de manière unique. On peut faire
exactement la même chose que la partie d'avant. Dans le cas
intègre c'est vraiment facile. 

\begin{note}
    Serre prouve que $\cap \m^n=(0)$ ce qui est un peu plus fort
    en soi.
\end{note}

\subsection{Deuxième caractérisation}
\begin{eq}
    \begin{center}
	DVR $\equiv$ Noethérien, intégralement clos, un seul idéal
	premier $\ne 0$ (local mais pas un corps ? Non! Tu
	verras pq.).
    \end{center}
\end{eq}
Là c'est un peu plus dur. Le point c'est de montrer que $\m$ est 
inversible, alors $\m$ est principal. On note 
$\m'=\{x\in K|x\m\subset A\}$. On a 
\begin{center}
    $\m\m'\subset A$ et $A\subset \m'$ implique 
    $\m\subset \m\m'$.
\end{center}
d'où $\m\m'=\m$ ou $A$. Maintenant en fait
\begin{center}
    si $\m\m'=A$ alors $\sum x_iy_i=1$ d'où 
    $u=x_{i_0}y_{i_0}\in A-\m=A^{\times}$ 
\end{center}
par l'absurde. En particulier tout $z\in\m$ se réécrit 
\[z=x_{i_0}(u^{-1}y_{i_0}z)\]
parce que $x_{i_0}y_{i_0}u^{-1}=1$ et $y_{i_0}z\in A$!
\subsection{Idées de preuve de la deuxième caractérisation.}
Donc a une remarque :
\[\m\subseteq\m\m'\subseteq A\]
d'où $\m\m'=\m$ ou $\m\m'=A$. On peut montrer que
\begin{enumerate}
    \item Dans le cas intégralement clos $\m\m'=\m$ implique 
	$\m'=A$.
    \item Dans le cas local, $\m'\ne A$.
\end{enumerate}
D'où le résultat. 
\subsection{Premier point}
On veut montrer que si $A$ est intégralement clos alors
$\m'\m=\m$ implique $\m'=A$. On prends $z\in \m'-0$, alors
\[z\m\subset \m\]
d'où $z^i\m\subset\m$ en itérant. Puis $\sum_{i=0}^n z^i A\subset
\m'$ pour tout $n$ et on sait que $\m'$ est noethérien. D'où 
\[\textrm{$A[z]$ est noethérien}\]
d'où $z$ est entier sur $A$ et le résultat.

\subsection{Second point}
On veut montrer que si $A$ a un seul idéal premier non nul alors
$\m'\ne A$. Un argument c'est
\begin{enumerate}
    \item On montre que pour $x\in \m-0$, $A_x=K$ via l'hypothèse.
    \item En faisant varier $x$, $\m^N\subset zA$ pour un $N$
	minimal.
    \item Puis si $y\in \m^{N-1}-zA$ alors $y\m\subset zA$
	puis \[y/z\in \m'-A\]
\end{enumerate}
Plusieurs détails où faut faire attention :
\begin{enumerate}
    \item Il faut prendre $z\in \m-0$, sinon on peut pas prendre
	$y\in \m^{N-1}-zA$. Par exemple si $z\in A^\times$
	alors $zA=A$ et $\m^{N-1}-zA=\emptyset$. Ducoup dans
	tout les autres cas c'est bon.
    \item L'hypothèse c'est un seul idéal premier non nul. 
	Et pour la première étape c'est nécessaire, pas juste 
	local. Parce que si $A_x\ne K$, alors on a $\p\in A_x$
	maximal non nul. Et $\p\cap A\ne \m$ car $x\notin \p$.
	D'où $\p\cap A\ne \m$ est premier non nul.
\end{enumerate}
Voilà ça conclut la preuve !

\subsection{Notes}
\begin{note}
Le troisième point se traduit en $\Spec(K)=D(x)\subset \Spec(A)$ et
    \begin{align*}
    A_x&=\Or_{D(x)}(D(x))\\
       &=\Or_{\Spec(A)}(\Spec(A))|_{(0)}\\
       &=\Or_{\Spec(A),(0)}\\
       &= (A\backslash 0)^{-1}A\\
       &=K
    \end{align*}
\end{note}
Aussi, cette histoire de $y\in \m^{n-1}-zA$ et $y\m\subset zA$
ça fait remarquer de l'arithmétique plus habituelle.

\subsection{But de ces caractérisations}
Celle qui nous intéresse c'est la deuxième qui permet de montrer
que $\m$ est principal. Alors on peut utiliser la première
pour montrer que c'est un DVR.

\section{Idéaux fractionnaires}
On se met dans un anneau noethérien intègre $A$ et $K=Frac(A)$.
Y'a l'équivalence $A$-modules $M$ de type fini dans $K$ et 
$M=y^{-1}I$ avec $I\leq A$ un idéal. Pour gauche à droite
une manière cool c'est que de $x\in (A : M)$ on obtient $M\leq
x^{-1}A$. Faut montrer que c'est non vide, bon bah ça c'est que
$M$ est engendré par des fractions.

\section{Anneaux de Dedekind}
On montre que 
\begin{eq}
    \begin{center}
	$A_\p$ est un dvr pour tout $\p\equiv$ $A$ est
	noethérien intégralement clos de dimension $1$.
    \end{center}
\end{eq}

On montre que 
\begin{eq}
    \begin{center}
	$A_\p$ est un dvr pour tout $\p\equiv$ $A$ est
	noethérien intégralement clos de dimension $1$.
    \end{center}
\end{eq}
On sait que un $DVR\equiv$ noethérien intègre avec un seul
idéal premier non nul. I.e. de dimension $\leq 1$. Donc faut juste
Que intégralement clos équivaut à tout les $A_\p$ sont 
intégralement clos.

\subsection{Tout les $A_\p$ sont des DVR implique de Dedekind. (instructif)}
Si $x\in K$ est entier sur $A$, on le note $x=b/c$. On
a pour tout $\p$ :
\[x\in A_\p\]
d'où $b\in cA_\p$ autrement dit il existe $s\in A-\p$ tel que 
$sb\in cA$. En particulier $(cA:bA)$ est contenu dans aucun
idéal premier d'où $1\in (cA:bA)$!!

\subsection{Relever l'inversibilité pour les premiers}
De l'inversibilité de $\p$ dans $A_\p$ je veux relever dans $A$.
En fait ça vient du fait que $\_\otimes_A A_\p$ est un morphisme
de monoides $Mod_{A,\subset K}\to Mod_{A_\p,\subset K}$ additif
et multiplicatif !

\subsection{$\Spec(A)$ est de dimension $1$, i.e. $V(I)$ est fini
pour $I\ne (0)$.}

Si on prends $x\in A$, psq si $\p_1,\ldots,\p_k,\ldots$ le
contiennent alors on a une chaine 
\[A\subset (A:\p_1)\subset (A:\p_1\cap \p_2)\subset \ldots\subset
(A:\p_1\cap\ldots\cap \p_k)\subset \ldots\subset x^{-1}A\]
d'où ça stationne et c'est fini. 

Comme $I$ est de type fini on obtient direct que $V(I)$ est fini.
\subsection{Valuations et décomposition}
Étant donné $I\subset A$, on a $I_\p=\p^m$ et on peut définir
$v_\p(I)=m$, on l'étend à $I\subset K$ par $v_\p(I.(A_\p:I))=0$.

\subsection{Relever l'inversibilité en général}

Il transforme aussi $(I : J)$ en $(I_\p:J_\p)$.
Concrètement :
\[(\p.(A:\p))\otimes_A A_\p = (\p A_\p.(A_\p:\p A_\p))=A_\p\]
sauf que à gauche $\p.(A:\p)\subseteq A$ est un idéal et y'a
que $A$ qui a pour image $A_\p$.
\begin{rem}
    ATTENTION. On a utilisé que $\p \mapsto \p A_\p$ est injectif !
\end{rem}
\begin{rem}
    Ca se généralise à $I$ un idéal fractionnaire car
    $I\otimes_A A_\p=\p^n$ d'où $(I\p^{-n})_\p=A_\p$. On peut 
    pas conclure que $\p^{-n}=(A:I)$ parce que si $\p'\mid I$ 
    alors $\p'\otimes_A A_\p=A_\p$. 
\end{rem}
Y faut maintenant remarquer que 
$(A:\prod \p^{v_\p(I)})_\p=(A:I)_\p$ pour tout $\p$ et conclure.
On conclut via le fait que $IA_\p=JA_\p$ pour tout $\p$ implique
\[_A(I:J)=A=_A(J:I)\]

\begin{note}
Une première manière de le résoudre c'est de dire $I$ et $J$ sont
des types finis donc suffit de montrer que $(J : (x_i))=A$ pour
chaque générateur de $I$ et inversement. EN FAIT, on peut
montrer que pour tout $x\in I$, $_A(J:(x))=A$ d'où $x\in J$ puis
$I\subset J$! Et inversement.
\end{note}

\begin{note}
    Le foncteur $\_\otimes_A A_\p$ de $Mod_A$ dans $Mod_{A_\p}$
    a surement les mêmes propriétés pour les bonnes définitions
    de produits et sommes.
\end{note}
\subsection{Décomposition en idéaux}
De la même manière que pour l'inversibilité on remarque que 
$\prod \p^{v_\p(I)}$ coincide avec $I$ dans
tout les $A_\p$. Faut montrer que donc il est égal à $I$.

\begin{note}
    Faire une section sur la décomposition en idéaux primaires.
\end{note}



\end{document}

