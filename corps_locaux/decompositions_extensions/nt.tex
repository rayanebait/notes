\documentclass[a4paper,12pt]{book}
\usepackage{amsmath,  amsthm,enumerate}
\usepackage{csquotes}
\usepackage[provide=*,french]{babel}
\usepackage[dvipsnames]{xcolor}
\usepackage{quiver, tikz}

%symbole caligraphique
\usepackage{mathrsfs}

%hyperliens
\usepackage{hyperref}

%pseudo-code
\usepackage{algpseudocode}
\usepackage{algorithm}
\makeatletter
  \renewcommand{\ALG@name}{Algorithme}
  \makeatother
\usepackage{fancyhdr}

\pagestyle{fancy}
\addtolength{\headwidth}{\marginparsep}
\addtolength{\headwidth}{\marginparwidth}
\renewcommand{\chaptermark}[1]{\markboth{#1}{}}
\renewcommand{\sectionmark}[1]{\markright{\thesection\ #1}}
\fancyhf{}
\fancyfoot[C]{\thepage}
\fancyhead[LO]{\textit \leftmark}
\fancyhead[RE]{\textit \rightmark}
\renewcommand{\headrulewidth}{0pt} % and the line
\fancypagestyle{plain}{%
    \fancyhead{} % get rid of headers
}

%bibliographie
\usepackage[
backend=biber,
style=alphabetic,
sorting=ynt
]{biblatex}

\addbibresource{bib.bib}

\usepackage{appendix}
\renewcommand{\appendixpagename}{Annexe}

\definecolor{wgrey}{RGB}{148, 38, 55}

\setlength\parindent{24pt}

\newcommand{\Z}{\mathbb{Z}}
\newcommand{\R}{\mathbb{R}}
\newcommand{\rel}{\omathcal{R}}
\newcommand{\Q}{\mathbb{Q}}
\newcommand{\C}{\mathbb{C}}
\newcommand{\N}{\mathbb{N}}
\newcommand{\K}{\mathbb{K}}
\newcommand{\A}{\mathbb{A}}
\newcommand{\B}{\mathcal{B}}
\newcommand{\Or}{\mathcal{O}}
\newcommand{\F}{\mathbb F}
\newcommand{\m}{\mathfrak m}
\renewcommand{\b}{\mathfrak b}
\renewcommand{\a}{\mathfrak a}
\newcommand{\p}{\mathfrak p}
\newcommand{\I}{\mathfrak I}
\newcommand{\Hom}{\textrm{Hom}}
\newcommand{\disc}{\textrm{disc}}
\newcommand{\Pic}{\textrm{Pic}}
\newcommand{\End}{\textrm{End}}
\newcommand{\Spec}{\textrm{Spec}}

\newcommand{\cL}{\mathscr{L}}
\newcommand{\G}{\mathscr{G}}
\newcommand{\D}{\mathscr{D}}
\newcommand{\E}{\mathscr{E}}

\theoremstyle{plain}
\newtheorem{thm}{Théoreme}
\newtheorem{lem}{Lemme}
\newtheorem{prop}{Proposition}
\newtheorem{cor}{Corollaire}
\newtheorem{heur}{Heuristique}
\newtheorem{rem}{Remarque}
\newtheorem{rembis}{Remarque}
\newtheorem{note}{Note}

\theoremstyle{definition}
\newtheorem{conj}{Conjecture}
\newtheorem*{eq}{Équivalences}
\newtheorem{prob}{Problème}
\newtheorem{quest}{Question}
\newtheorem{prot}{Protocole}
\newtheorem{algo}{Algorithme}
\newtheorem{defn}{Définition}
\newtheorem{defnbis}{Définition}
\newtheorem{ex}{Exemple}
\newtheorem{exo}{Exercices}

\theoremstyle{remark}

\definecolor{wgrey}{RGB}{148, 38, 55}
\definecolor{wgreen}{RGB}{100, 200,0} 
\hypersetup{
    colorlinks=true,
    linkcolor=wgreen,
    urlcolor=wgrey,
    filecolor=wgrey
}

\title{Décompositions d'extensions}
\date{}

\begin{document}
\maketitle
\section{Non ramifiée, modérément, sauvagement, totalement ramifié}
Pour le vocabulaire : Avec $L/K$ extension de corps de valuations
discrètes, i.e. $v_K$ discrète fixée.
\begin{enumerate}
    \item Non ramifié : Pour chaque $i$, $e_i=1$ et $k_{L_i}/k_K$
        est séparable.
    \item Modérément ramifié : pour chaque $i$, $p\nmid e_i$ et 
        $k_{L_i}/k_K$ est séparable.
    \item Sauvagement ramifié : il existe un $i$ tq $p\mid e_i$
        ou $k_{L_i}/k_K$ inséparable.
    \item Totalement ramiifié : $[L:K]=e$ et $\tilde\Or_K=\Or_L$.
        (On a la condition de finitude)
\end{enumerate}

Attention y'a pas toujours l'égalité $\sum e_if_i=[L:K]$, dans la
plupart des cas qui m'intéressent si quand même.

\section{Lien entre liberté dans $k_K,k_L$ et dans $L/K$}
On regarde $L=K(\alpha)$ et $P=\mu_\alpha$ unitaire dans 
$\Or_K[X]$. Si $\overline{Q(\alpha)}=0$ (liberté de $(\alpha^i)$)
on a $Q(\alpha)\in \m_L$ et pas dans $\m_K$. D'où on peut pas
directement comparer les libertés dans ce sens ! À l'inverse,
si $(\bar e_i)_i$ est libre dans $k_L-k_K$ et qu'on a 
$\sum a_ie_i=0\mod \m_L$ alors $a_i\in \m_L\cap \Or_K=\m_K$. Si
y sont tous non nuls $0<|(\sum a_ie_i)|<1$ on a un problème.

\section{Factorisation de $\bar P$ et $e.f$}
Même contexte, dans le cas complet c'est plus simple : Par Hensel
$\bar P=F^d$ et $\deg(F)\mid f$ parce que $F$ se scinde dans
$k_L$ vu que $P$ se scinde dans $\Or_L$. En particulier on peut
faire descendre la racine. On déduit
\[e.f=\deg(P)=d.\deg(F)\]
d'où $e\mid d$ et $\deg(F)\mid f$.

\begin{rem}
    Comme Vincent m'a fait remarquer pas d'égalité vu que par
    exemple si $K[\alpha]/K$ est non ramifiée et $\alpha$ engendre 
    l'extension résiduelle alors $\pi_L^dP(X/\pi_L)$ annule
    $\pi_L\alpha$ mais $F=X^d$, donc on est dans le pire cas.
\end{rem}

\section{Polynômes d'eisenstein et extensions totalement ramifiées}
\begin{center}\textbf{(1)}\end{center}
Si $P(X)=X^d+\sum a_iX^i$ avec $v_K(a_0)=1$ et $v_K(a_i)\geq 1$
alors $L=K[X]/(P(X))$ est totalement ramifiée et $X$ est une 
uniformisante. Si $\alpha$ est une racine dans $L$ de $P$ :

Y'a deux points, $B=\Or_K[\alpha]$ a un seul idéal maximal car
$a_0$ et $\alpha$ sont dans le même idéal maximal et y contiennent
tous $a_0$ (!) puis $(a_0,\alpha)=\alpha B$ est maximal
(via le quotient!). Ça prouve que $B$ est local et principal donc
un DVR, i.e. $\tilde\Or_K=B$. Pour la valuation $e = d=[L:K]$
directement, d'où le résultat.

\begin{center}\textbf{(2)}\end{center}
Si $L/K$ est totalement ramifiée, alors $\pi_L$ est annulé par
un Eisenstein. L'idée c'est que si $P$ l'annule, alors
si $a_{i_0}\notin \m_K$ alors :
\[\pi_L^{i_0}(a_{i_0}/\pi_L^{i_0}+\sum_{i=i_0}^n a_i\pi_L^{i-i_0})\]
est de valuation $i_0$. Si $v_K(a_j)>0$ pour $j<i_0$ alors la
valuation est strictement plus grande que $e=v_L(\pi_K)$. Sauf
que
\[\sum_{i=0}^{i_0-1}a_i\pi_L^i=\pi_L^{i_0}(a_{i_0}/\pi_L^{i_0}+\sum_{i=i_0}^n a_i\pi_L^{i-i_0})\]
d'où c'est eisenstein. En plus
\[a_0/\pi_L^n=-1+\sum a_i\pi_L^i/\pi_L^n\]
d'où $v_L(a_0/\pi_L^n)=0$ vu que $v_L(a_i)\geq e$ et $n=e$.

\chapter{Cas complet}
\section{Extension totalement modérément ramifiée}
Cette fois on peut trouver $\pi_L$ et $\pi_K$ tels que 
$P(X)=X^e-\pi_K$. Déjà
\[\Or_K/\m_K\to \Or_L/\m_L\]
est un iso et donc si $u\pi_L^e=\pi_K$, on regarde $u=v$ dans
$k_L$ (car c'est là que $u$ vit) avec $v\in\Or_K$. D'où
$u=v+\epsilon$, $\epsilon\in \m_L$ 
(car c'est dans $k_L$ l'égalité). Ensuite $u=v(1+v^{-1}\epsilon)$.
Sauf que $1+v^{-1}\epsilon$ a une racine $e$-ème par Hensel, 
$\zeta$. D'où $(\pi_L\zeta)^e=\pi_K/v$.
\section{Trouver les extensions totalement modérement ramifiées}
En gros dans $L/K$ finie complète telle que $k_K-k_L$
est purement inséparable (c'est juste une généralisation), 
On regarde presque le corps engendré par $\pi_L^{e/e'}$.
On choisit $e'\mid e/p^v_p(e)$, il existe $k_L^{p^r}\subset k_K$
alors $ap^r+be'=1$ et
\[\bar u=(\bar u^{p^r})^a(\bar u^b)^{e'}\mod \m_L\]
et ducoup on relève $u=\lambda^a(\bar u^b)^{e'}(1+\epsilon)$
avec $\lambda\in \Or_K^\times $ puis comme d'hab le truc à droite
a une racine $e'$-ème par hensel, disons $\zeta$. D'où en notant
$\pi_{e'}=u^b\zeta\pi_L^{e/e'}$ c'est une racine $e'$-ème de 
$\pi_K/\lambda^a$. Alors
\[K(\pi_{e'})\]
est totalement ramifiée vu que 
\textbf{engendrée par un eisenstein.}

\subsection{Unicité}
C'est pas très satisfaisant.
\subsubsection{Apparté}
Si on regarde $\Or_K\to\Or_L/\m_L$ ça induit $i\colon k_K\to k_L$.
En particulier dire que $u\in k_L$ est en fait dans $k_K$ 
\textbf{ça veut dire que $u+\m_L=v+\m_L$ avec $v\in \Or_K$.}
\subsubsection{Preuve}
Concrètement, $\lambda=(\pi_1\pi_2^{-1})^{e'}\in \Or_K^\times$ et
en regardant dans $k_L$ ça engendrerait une sous-extension de 
degré premier à $p^r$, i.e $1$. D'où $\bar u=\bar v\in k_K$ et
$(\bar v)^{e'}(1+\epsilon)=\lambda$ sauf que 
$(1+\epsilon)=\lambda/(\bar v)^{e'}$ d'où est dans $\Or_K$ puis
$1+\m_K$ c'est que des puissances $e'$-ème. On obtient que
$(\pi_1\pi_2^{-1})^{e'}=(v')^{e'}$ avec $v'\in \Or_K\times$.
En particulier comme les racines $e'$-ème de l'unité sont dans
$\Or_K$ $\pi_1=u\pi_2$ avec $u\in\Or_K^\times$ d'où unicité.


\begin{rem}
    En résumé, pour tout $e'\mid e/p^{v_p(e)}$, on a une 
    sous-extension $K-L_{e'}-L$, dans le cas complet sous
    l'hypothèse d'extension résiduelle purement inséparable.
\end{rem}
\section{Trouver les sous-extensions non ramifiée}
En dessous de $K-L$ on regarde $k_K-k-k_L$ avec $k_K-k$ séparable.
On a une correspondance entre $k$ et $K-K_k-L$ où la première est
non ramifiée.
\subsection{Existence}
$k_K^{sep}=k_K(\bar\theta)$. Comme \textbf{tout se passe dans} 
$k_L$ on lift un polynôme $P\in\Or_L[X]$ de même degré. 
\textbf{Par hensel},
$\theta$ en est une racine et il est \textbf{scindé séparable} dans
$L$. On regarde $K(\theta)$, c'est séparable vu que $\bar P$ est
séparable \textbf{via la dérivée} $\mod \m_L$ (!). Et c'est non
ramifié vu que $f=\deg(\bar P)=\deg(P)=[K(\theta):K]$.
\subsection{Unicité}
Donc le détail encore c'est qu'on plonge $k$ dans $k_L$. I.e.
on peut faire Hensel QUE dans $\Or_L[X]$. D'où si $F$ est
une autre sous-extension de $L$

\chapter{Cas galoisien}




\end{document}

