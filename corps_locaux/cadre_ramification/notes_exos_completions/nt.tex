\documentclass[a4paper,12pt]{book}
\usepackage{amsmath,  amsthm,enumerate}
\usepackage{csquotes}
\usepackage[provide=*,french]{babel}
\usepackage[dvipsnames]{xcolor}
\usepackage{quiver, tikz}

%symbole caligraphique
\usepackage{mathrsfs}

%hyperliens
\usepackage{hyperref}

%pseudo-code
\usepackage{algpseudocode}
\usepackage{algorithm}
\makeatletter
  \renewcommand{\ALG@name}{Algorithme}
  \makeatother
\usepackage{fancyhdr}

\pagestyle{fancy}
\addtolength{\headwidth}{\marginparsep}
\addtolength{\headwidth}{\marginparwidth}
\renewcommand{\chaptermark}[1]{\markboth{#1}{}}
\renewcommand{\sectionmark}[1]{\markright{\thesection\ #1}}
\fancyhf{}
\fancyfoot[C]{\thepage}
\fancyhead[LO]{\textit \leftmark}
\fancyhead[RE]{\textit \rightmark}
\renewcommand{\headrulewidth}{0pt} % and the line
\fancypagestyle{plain}{%
    \fancyhead{} % get rid of headers
}

%bibliographie
\usepackage[
backend=biber,
style=alphabetic,
sorting=ynt
]{biblatex}

\addbibresource{bib.bib}

\usepackage{appendix}
\renewcommand{\appendixpagename}{Annexe}

\definecolor{wgrey}{RGB}{148, 38, 55}

\setlength\parindent{24pt}

\newcommand{\Z}{\mathbb{Z}}
\newcommand{\R}{\mathbb{R}}
\newcommand{\rel}{\omathcal{R}}
\newcommand{\Q}{\mathbb{Q}}
\newcommand{\C}{\mathbb{C}}
\newcommand{\N}{\mathbb{N}}
\newcommand{\K}{\mathbb{K}}
\newcommand{\A}{\mathbb{A}}
\newcommand{\B}{\mathcal{B}}
\newcommand{\Or}{\mathcal{O}}
\newcommand{\F}{\mathbb F}
\newcommand{\m}{\mathfrak m}
\renewcommand{\b}{\mathfrak b}
\renewcommand{\a}{\mathfrak a}
\newcommand{\p}{\mathfrak p}
\newcommand{\I}{\mathfrak I}
\newcommand{\Hom}{\textrm{Hom}}
\newcommand{\disc}{\textrm{disc}}
\newcommand{\Pic}{\textrm{Pic}}
\newcommand{\End}{\textrm{End}}
\newcommand{\Spec}{\textrm{Spec}}
\newcommand{\Frac}{\textrm{Frac}}

\newcommand{\cL}{\mathscr{L}}
\newcommand{\G}{\mathscr{G}}
\newcommand{\D}{\mathscr{D}}
\newcommand{\E}{\mathscr{E}}

\theoremstyle{plain}
\newtheorem{thm}{Théoreme}
\newtheorem{lem}{Lemme}
\newtheorem{prop}{Proposition}
\newtheorem{cor}{Corollaire}
\newtheorem{heur}{Heuristique}
\newtheorem{rem}{Remarque}
\newtheorem{rembis}{Remarque}
\newtheorem{note}{Note}

\theoremstyle{definition}
\newtheorem{conj}{Conjecture}
\newtheorem*{eq}{Équivalences}
\newtheorem{prob}{Problème}
\newtheorem{quest}{Question}
\newtheorem{prot}{Protocole}
\newtheorem{algo}{Algorithme}
\newtheorem{defn}{Définition}
\newtheorem{defnbis}{Définition}
\newtheorem{ex}{Exemple}
\newtheorem{exo}{Exercices}

\theoremstyle{remark}

\definecolor{wgrey}{RGB}{148, 38, 55}
\definecolor{wgreen}{RGB}{100, 200,0} 
\hypersetup{
    colorlinks=true,
    linkcolor=wgreen,
    urlcolor=wgrey,
    filecolor=wgrey
}

\title{Complétions}
\date{}

\begin{document}
\maketitle

\chapter{Topologie}
\section{Complétions}
\subsection{Corps ultramétriques}
Étant donné un corps ultramétrique, montrer
que $|K|=|\hat K|$ et les corps résiduels sont isomorphes.

Si c'est une valuation discrète on peut montrer que $|x_n|$ 
stationne toujours (loin de 0) en prenant une boule proche. Et en
fait c'est toujours le cas via
\[|a-x_n+x_n|=|x_n|\]
quand $x_n$ est assez proche de $a$ et au dessus. 
\begin{rem}
    J'ai galéré ptdr, parce que j'osais pas considérer un élement
    de $\hat K$ en dehors de la suite qui l'approche. (I.e. la 
    suite mais pas sa limite)
\end{rem}

\section{Densité de $\Or_K$ dans $\Or_{\hat K}$ et corps résiduel.}
Simplement parce que $(x_n)$ stationne toujours par
l'inégalité ultramétrique. Ducoup on peut prendre les suites
dans $\Or_K$. Ensuite réduire modulo $\m$ dans $\Or_{\hat K}$
ça veut dire quoi ? L'idéal c'est $B(0,1)$ ouvert. On peut
simplement prendre $x\in \Or_K$ tel que 
\[|a-x|<1\]
par densité. En particulier, $x=a\mod B(0,1)$ par définition.

\subsection{$\Z_p[p^{-1}]=\Q_p$, $\Q_p/\Z_p=\Z[p^{-1}]/\Z$}
On a $\Z_{(p)}\subseteq \Z_p$ et 
\[\Z_{(p)}[p^{-1}]=\Q\subset \Z_p[p^{-1}]\]
maintenant le truc de droite est complet donc fermé. Pour le voir
on remarque que une suite $(x_n)$ de ce truc converge dans $\Q_p$
et sa valuation se fixe. En particulier il existe $N$ tel que 
$(p^Nx_n)\in \Z_p$ et le premier résultat.

Pour le deuxième,
\section{Locale compacité}
Montrer maintenant que dans le cas d'avant, les corps résiduels 
sont finis. En fait on fait descendre une topologie discrète
\[\pi \colon \Or_K\to k_K\]
d'où $k_K$ est fini car compact.


\section{}


\end{document}

