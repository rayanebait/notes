\documentclass[a4paper,12pt]{book}
\usepackage{amsmath,  amsthm,enumerate}
\usepackage{csquotes}
\usepackage[provide=*,french]{babel}
\usepackage[dvipsnames]{xcolor}
\usepackage{quiver, tikz}

%symbole caligraphique
\usepackage{mathrsfs}

%hyperliens
\usepackage{hyperref}

%pseudo-code
\usepackage{algpseudocode}
\usepackage{algorithm}
\makeatletter
  \renewcommand{\ALG@name}{Algorithme}
  \makeatother
\usepackage{fancyhdr}

\pagestyle{fancy}
\addtolength{\headwidth}{\marginparsep}
\addtolength{\headwidth}{\marginparwidth}
\renewcommand{\chaptermark}[1]{\markboth{#1}{}}
\renewcommand{\sectionmark}[1]{\markright{\thesection\ #1}}
\fancyhf{}
\fancyfoot[C]{\thepage}
\fancyhead[LO]{\textit \leftmark}
\fancyhead[RE]{\textit \rightmark}
\renewcommand{\headrulewidth}{0pt} % and the line
\fancypagestyle{plain}{%
    \fancyhead{} % get rid of headers
}

%bibliographie
\usepackage[
backend=biber,
style=alphabetic,
sorting=ynt
]{biblatex}

\addbibresource{bib.bib}

\usepackage{appendix}
\renewcommand{\appendixpagename}{Annexe}

\definecolor{wgrey}{RGB}{148, 38, 55}

\setlength\parindent{24pt}

\newcommand{\Z}{\mathbb{Z}}
\newcommand{\R}{\mathbb{R}}
\newcommand{\rel}{\omathcal{R}}
\newcommand{\Q}{\mathbb{Q}}
\newcommand{\C}{\mathbb{C}}
\newcommand{\N}{\mathbb{N}}
\newcommand{\K}{\mathbb{K}}
\newcommand{\A}{\mathbb{A}}
\newcommand{\B}{\mathcal{B}}
\newcommand{\Or}{\mathcal{O}}
\newcommand{\F}{\mathbb F}
\newcommand{\m}{\mathfrak m}
\renewcommand{\b}{\mathfrak b}
\renewcommand{\a}{\mathfrak a}
\newcommand{\p}{\mathfrak p}
\newcommand{\I}{\mathfrak I}
\newcommand{\Hom}{\textrm{Hom}}
\newcommand{\disc}{\textrm{disc}}
\newcommand{\Pic}{\textrm{Pic}}
\newcommand{\End}{\textrm{End}}
\newcommand{\Spec}{\textrm{Spec}}
\newcommand{\Frac}{\textrm{Frac}}

\newcommand{\cL}{\mathscr{L}}
\newcommand{\G}{\mathscr{G}}
\newcommand{\D}{\mathscr{D}}
\newcommand{\E}{\mathscr{E}}

\theoremstyle{plain}
\newtheorem{thm}{Théoreme}
\newtheorem{lem}{Lemme}
\newtheorem{prop}{Proposition}
\newtheorem{cor}{Corollaire}
\newtheorem{heur}{Heuristique}
\newtheorem{rem}{Remarque}
\newtheorem{rembis}{Remarque}
\newtheorem{note}{Note}

\theoremstyle{definition}
\newtheorem{conj}{Conjecture}
\newtheorem*{eq}{Équivalences}
\newtheorem{prob}{Problème}
\newtheorem{quest}{Question}
\newtheorem{prot}{Protocole}
\newtheorem{algo}{Algorithme}
\newtheorem{defn}{Définition}
\newtheorem{defnbis}{Définition}
\newtheorem{ex}{Exemple}
\newtheorem{exo}{Exercices}

\theoremstyle{remark}

\definecolor{wgrey}{RGB}{148, 38, 55}
\definecolor{wgreen}{RGB}{100, 200,0} 
\hypersetup{
    colorlinks=true,
    linkcolor=wgreen,
    urlcolor=wgrey,
    filecolor=wgrey
}

\title{Extensions de valuation}
\date{}

\begin{document}
\maketitle


Ici, on s'en fout de savoir si $\tilde \Or_K$ est fini
sur $\Or_K$.




\chapter{Extension via les complétions : cas ultramétrique}
\section{Unicité sur $(\hat K)^c$}
Étant donné un corps complet $K$, les extensions
$L$ de $K$ sont des corps complets et les normes
sont équivalentes.
\subsection{Extension pour les corps complets}
Ce truc 
\[x\mapsto |N_{L/K}(x)|^{1/[L:K]}\]
est une valeur absolue qui étend $|.|$ donc l'unique. Pour 
montrer que c'est ultramétrique on peut par multiplicativité
étudier si $|x|_L\leq 1$ force $|1+x|_L\leq 1$ ou la norme de 
$1+x$. Via Le corollaire de Hensel
et le fait que le polynôme minimal de $1+x$ est $\mu(x-1)$ avec
$\mu$ celui de $x$ on peut conclure simplement parce que $\mu(-1)
\in \Or_K$.

\subsection{Détails}
L'équivalence de normes force l'équivalence de valeur
absolues (passer par la topologie par exemple mais c'est pas clair
pourquoi c'est pas bizarre).

\href{https://math.stackexchange.com/questions/4096251/
equivalence-of-norms-on-valued-fields}{Cette réf} 
explique bien la nuance entre valeur absolue et norme. En gros
$|.|_1$ et $|.|_2$ deux valeurs absolues étendant $|.|$ sont
des $K$-normes pour $|.|$. Le petit truc bizarre sur le fait
que l'équivalence de valeur absolue est exponentielle et 
l'équivalence de norme est linéaire c'est que. L'équivalence de
normes sous entend la même valeur absolue pour laquelle c'est des
normes donc forcément l'égalité de valeurs absolues.

\begin{rem}
    I.e. quand y'a pas égalité sur la restriction des v.a y'a pas
    équivalence de normes du tout. Quand y'a égalité sur 
    la restriction des v.a (elles étendent $|.|_K$) automatiquement
    elles sont \textbf{égales} pas seulement équivalentes.
\end{rem}

\subsection{Passer à la clôture algébrique}
D'une v.a sur $K^c$ suffit de restreindre à $L/K$
on garde une v.a donc l'unique!

\section{Extensions en général par plongements}

Y'a deux manières de faire. Soit regarder
$K$ dans $\hat L$. Soit regarder $L$ dans
$(\hat K)^c$. La première à l'avantage de 
sous-entendre la valeur absolue.
\subsection{Deuxième manière}
On regarde $K^c$ dans $(\hat K)^c$.
Alors si $|.|_L$ étend $|.|$ et
\[\tau \colon L\to K^c\] est un plongement.
On peut étendre $\tau$ à $\hat K.\tau(L)$ un
corps complet sur $\hat K$ de dimension finie.
En plus la valeur absolue s'étend aussi 
directement et étend bien celle de $\hat K$
par limite. En particulier, pour $x\in L$
$|\tau(x)|_{\hat K.\tau(L)}=|\tau(x)|_c$ est
uniquement determinée par $\tau$. À l'inverse
n'importe quel plongement $L\to K^c$ fournit
une valeur absolue.

\section{Extensions en général via les idéaux
maximaux de $L\otimes_K\hat K$ (topologie)}
On note $B_L=L\otimes_K \hat K$. En tant qu'espace
vectoriel c'est de dimension 
\[[L:K]\]
le produit est sur $K$. C'est un Banach et on a 
des flèches
\[L\to B_L\]
via $x\mapsto x\otimes 1$ et si on a une extension de $|.|_K$ à
$L$, et qu'on note sa complétion $E$ on a
$\iota \colon\hat K\to E$ donné par $K\to L\to E$ et
$L\to E$ donné par l'injection d'où 
\[B_L\to E\]
via $x\otimes y\mapsto x\iota(y)$. À gauche ça a une structure
d'algèbre et donc le noyau est un idéal maximal si c'est
non nul. C'est clairement non nul par la dimension.

\begin{rem}
    Je pense qu'un truc intéressant c'est d'utiliser une base
    normale d'une clôture galoisienne de sorte à ce que des racines
    soient associées à des valeurs absolues. En fait le cas 
    séparable est clair et on peut pas se ramener au cas galoisien
    dans le cas inséparable :c. Mais pour
     le cas séparable c'est pas
    mal!
\end{rem}
\subsection{À regarder}
La même chose que $\dim_k \tilde\Or_K/\m_K\tilde\Or_K=[L:K]$
mais avec $\dim_K (L\otimes \hat K)/\prod\m_i$. 

D'ailleurs c'est quel schéma qui a les deux en même temps ?
$\Spec(\tilde \Or_K)\to \Spec(\Or_K)$ j'imagine, mais le premier
quotient c'est pas clair ou il apparaît. Fin c'est 
$V(\m_K\tilde\Or_K)$ qui est pas un point. Et le produit tensoriel
c'est la fibre en $(0)$ d'un $\Or_X$-module j'imagine où d'une
fibre bien choisie.

\subsection{Interprétation : pourquoi on a pas $L_i\to B_L$}
La norme $|.|_i$ sur $L_i$ mesure que la partie associée
à $\m_i$ de $L$ sur $K$. Tandis que la norme naturelle sur
$L\otimes_K \hat K$ est vraiment la norme terme à terme
sur $K$. En gros, si par exemple $L/K$ est totalement
décomposée au dessus de $|.|$ et $(x_i)$ est une base
normale. Alors $|.|_i$ force $|x_j|_i=|x_i|_i$ avec $j\ne i$.
Tandis qu'une norme sur $B_L$ ne mesure pas les $x_i$. 
Seulement via $\hat K$.



\section{Calcul par la deuxième version}
Comment savoir combien on a d'extension ? Il faut calculer
$B_L$ explicitement ! Par exemple :
\[\Q_p\otimes_\Q \Q(\sqrt 2)\simeq \Q_p[X]/(X^2-2)\]
et $\begin{pmatrix} 2\\ p\end{pmatrix}=1 $ si et 
seulement si $p^2\equiv 1\mod 8$. Donc $|.|$ a une
ou deux extensions en fonction de $p$ via Hensel.

\subsection{Lien avec la ramification}
Je le mentionne ici, mais pour avoir une extension non ramifiée
en $\p$, faut que $B_L$ soit un produit de corps ducoup! 



\chapter{Cas archimédien}
On se place donc en caractéristique $0$ et sur $K=\Q$. Parce
que un corps archimédien algébriquement clos complet est isométrique
à $\C$.

\section{Cas complet}
On est soit $\C$ soit $\R$. 

\section{En général}
On a $r_1+r_2$ extensions de $|.|_\infty$ de $\Q$ à $L$.
Avec $r_1+2r_2=[L:\Q]$ et les complétés pour $r_1$ c'est $\R$.
Pour $r_2$ c'est $\C$. Le point c'est que $|z|=|\bar z|$ pour
un complexe. Aussi, faut juste montrer que chaque plongement
différent donne une valeur absolue différente. On doit pouvoir
le faire avec $B_L$. Parce que 
\[\dim_{\R=\hat K} L\otimes_K \R=\prod \R[X]/(P_i(X))=\R^{r_1}\times \C^{r_2}\]
Où $\C$ apparaît dès que $\deg P_i(X)=2$.

\chapter{Extension par les anneaux de valuations discrètes}
Étant donné un corps $(K,v)$ de valuation discrète. Une 
extension $(L,w)$ finie de $v$. L'anneau $\tilde\Or_K=\tilde\Or_v$
est semi-local sur $\Or_K=\Or_v$. Le but c'est de dire
que $\Or_w=(\tilde\Or_K)_\m$ pour un idéal max de $\Or_K$
et que chaque idéal maximal de $\tilde\Or_K$ correspond
à une valuation.

Ici, on s'en fout de savoir si $\tilde \Or_K$ est fini
sur $\Or_K$.
\section{Extensions par les idéaux maximaux}
Étant donné seulement $L-K$ finie. un ideal maximal
$\m$ de $\tilde\Or_K$ fournit un dvr $(\tilde\Or_K)_\m$
et une extension de dvrs. Alors
\[w_\m:=v_\m/v(\pi_K)\]
étend $v$ et est une valuation discrète sur $L$. C'est
clair que différents idéaux fournissent des valuations
différentes.
\section{Idéal maximal associé à une valuation}
À l'inverse si $w$ étend $v$. Alors $\Or_w$ contient
$\tilde\Or_K(w)$ en étudiant la valuation d'une relation.
Ensuite faut montrer que 
\[(\tilde\Or_K)_{\m_w\cap \tilde\Or_K}=\Or_w\]
À faire c'est pas dur.


\chapter{Remarques}
Le but c'est de voir les différentes extension $L_i$ 
et $k_{L_i}$ associées aux valuations apparaître.
\section{Cas primitif $L=K[\alpha]$, l'extension}
J'écris $L=K[X]/(P(X))$. On regarde $B_L$
\[L\otimes_K \hat K = \prod \hat K[X]/(P_i(X)^{b_i})\]
où si y'a un $b_i>1$ c'est que l'extension est inséparable,
parce que $P$ est irréductible donc séparable équivaut à
racine simple. Maintenant chaque facteur engendre un corps
qui est associé à une valuation.

\section{Cas primitif $L=K[\alpha]$ et monogène $\tilde \Or_K=\Or_K[\alpha]$, l'extension résiduelle}
Ça devrait plutôt être dans calculramification un truc comme
ça mais bon.
On regarde $\Or_{\hat K}\otimes_{\Or_K}\tilde\Or_K=\prod \Or_{\hat K}[X]/(P_i(X)^{b_i})$.
On étend via $k_K\simeq \Or_{\hat K}/\m_{\hat K}$ pour obtenir
la forme résiduelle avec la ramification.
\[\prod \Or_{\hat K}[X]/(P_i(X)^{b_i})\otimes_{\Or_K}k_K=\prod k_K[X]/(P_i(X)^{e_i})\]

\begin{rem}
    Pour remarque ducoup, ce produit tensoriel y'a pas de pb
    le calcul est identique à celui pour les corps. Ça utilise
    toujours que la prop universelle des anneaux de polynômes
    et l'exactitude à droite.
\end{rem}
Maintenant c'est bien clair que c'est non ramifié ssi la version
résiduelle est un produit de corps.



\end{document}

