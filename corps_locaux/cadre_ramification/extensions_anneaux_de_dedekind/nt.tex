\documentclass[a4paper,12pt]{book}
\usepackage{amsmath,  amsthm,enumerate}
\usepackage{csquotes}
\usepackage[provide=*,french]{babel}
\usepackage[dvipsnames]{xcolor}
\usepackage{quiver, tikz}

%symbole caligraphique
\usepackage{mathrsfs}

%hyperliens
\usepackage{hyperref}

%pseudo-code
\usepackage{algpseudocode}
\usepackage{algorithm}
\makeatletter
  \renewcommand{\ALG@name}{Algorithme}
  \makeatother
\usepackage{fancyhdr}

\pagestyle{fancy}
\addtolength{\headwidth}{\marginparsep}
\addtolength{\headwidth}{\marginparwidth}
\renewcommand{\chaptermark}[1]{\markboth{#1}{}}
\renewcommand{\sectionmark}[1]{\markright{\thesection\ #1}}
\fancyhf{}
\fancyfoot[C]{\thepage}
\fancyhead[LO]{\textit \leftmark}
\fancyhead[RE]{\textit \rightmark}
\renewcommand{\headrulewidth}{0pt} % and the line
\fancypagestyle{plain}{%
    \fancyhead{} % get rid of headers
}

%bibliographie
\usepackage[
backend=biber,
style=alphabetic,
sorting=ynt
]{biblatex}

\addbibresource{bib.bib}

\usepackage{appendix}
\renewcommand{\appendixpagename}{Annexe}

\definecolor{wgrey}{RGB}{148, 38, 55}

\setlength\parindent{24pt}

\newcommand{\Z}{\mathbb{Z}}
\newcommand{\R}{\mathbb{R}}
\newcommand{\rel}{\omathcal{R}}
\newcommand{\Q}{\mathbb{Q}}
\newcommand{\C}{\mathbb{C}}
\newcommand{\N}{\mathbb{N}}
\newcommand{\K}{\mathbb{K}}
\newcommand{\A}{\mathbb{A}}
\newcommand{\B}{\mathcal{B}}
\newcommand{\Or}{\mathcal{O}}
\newcommand{\F}{\mathbb F}
\newcommand{\m}{\mathfrak m}
\renewcommand{\b}{\mathfrak b}
\renewcommand{\a}{\mathfrak a}
\newcommand{\p}{\mathfrak p}
\newcommand{\I}{\mathfrak I}
\newcommand{\Hom}{\textrm{Hom}}
\newcommand{\disc}{\textrm{disc}}
\newcommand{\Pic}{\textrm{Pic}}
\newcommand{\End}{\textrm{End}}
\newcommand{\Spec}{\textrm{Spec}}
\newcommand{\Frac}{\textrm{Frac}}

\newcommand{\cL}{\mathscr{L}}
\newcommand{\G}{\mathscr{G}}
\newcommand{\D}{\mathscr{D}}
\newcommand{\E}{\mathscr{E}}

\theoremstyle{plain}
\newtheorem{thm}{Théoreme}
\newtheorem{lem}{Lemme}
\newtheorem{prop}{Proposition}
\newtheorem{cor}{Corollaire}
\newtheorem{heur}{Heuristique}
\newtheorem{rem}{Remarque}
\newtheorem{rembis}{Remarque}
\newtheorem{note}{Note}

\theoremstyle{definition}
\newtheorem{conj}{Conjecture}
\newtheorem*{eq}{Équivalences}
\newtheorem{prob}{Problème}
\newtheorem{quest}{Question}
\newtheorem{prot}{Protocole}
\newtheorem{algo}{Algorithme}
\newtheorem{defn}{Définition}
\newtheorem{defnbis}{Définition}
\newtheorem{ex}{Exemple}
\newtheorem{exo}{Exercices}

\theoremstyle{remark}

\definecolor{wgrey}{RGB}{148, 38, 55}
\definecolor{wgreen}{RGB}{100, 200,0} 
\hypersetup{
    colorlinks=true,
    linkcolor=wgreen,
    urlcolor=wgrey,
    filecolor=wgrey
}

\title{Anneaux de Dedekind}
\date{}

\begin{document}
\maketitle


\chapter{Manipuler}
Donc là les outils comme d'hab c'est les extensions entières
et les clôtures intégrales. La première on montre comme transmettre
la dimension, la deuxième transmettre le fait d'être intégralement 
clos. Pour la noethérianité, ici on manipule des localisations pour
la cas semi-local (corollaire 2.26) et des DVRS. La preuve d'extensions
de valuations par les idéaux (Prop 2.28). (Le cas séparable c'est
(théorème 2.20)
\subsection{Extension d'anneaux de Dedekind, cas semi-local}
Il existe $k$ t.q. $L^{p^k}\subset K$ puis on déf
\[v_L(\theta):=v_K(\theta^{p^k})/p^k\]
tout marche bien.

\subsection{Extensions de valuation par les idéaux}
On regarde $\tilde \Or_K$ de Dedekind, et d'un idéal on obtient
un DVR, d'une valuation un DVR. 

\chapter{Extensions d'anneaux de Dedekind}
\section{Annexe : Extensions entières où transmettre le
spectre}
J'en parle juste brièvement ici, plus dans le récap
d'algèbre. Mais quand $A\to B$ est fini injectif. La
dimension est préservée et 
\[\Spec(B)\to \Spec(A)\]
est surjective avec des fibres finies.

\section{Le cadre général}
Comme on part d'un anneau de Dedekind $\Or_K$. La première question c'est
Quand est-ce que 
\[\Or_K - \tilde\Or_K\]
est une extension d'anneaux de Dedekind. La question se réduit 
systématiquement à 
\[\textrm{Est-ce que $\tilde\Or_K$ est noethérien ?}\]

\section{Se ramener au cadre}

L'anneau de Dedekind est un anneau noethérien de dimension $1$. 
Autrement dit, on est sur un schéma affine de dimension $1$ :
\[\Spec(\Or_K)\to \Spec(\Z)\]
Mais surtout, tout ses localisés à des premiers sont des anneaux 
de valuation discrète ! Ça ça m'intéresse pour les raisons suivantes :
\begin{enumerate}
    \item On obtient un local-global en localisant.
    \item On peut compléter et obtenir le cadre des corps locaux.
\end{enumerate}
en particulier on a le pont de, si 
\[K-L\]
est une extension de corps finis, on peut en faire une extension de corps
valués via le processus suivant :
\begin{enumerate}
    \item D'un premier de $\Or_K$ on obtient $(\Or_K)_{\m_K}$.
    \item On obtient aussi un corps valué $(K,|.|_{\m_K})$.
    \item De $\m_K\tilde\Or_K=\prod \m_i^{e_i}$ on obtient des premiers 
	$\m_i$.
    \item D'un $\m_i|\m_K$ on obtient une extension de valeur absolue
	$|.|_{\m_i}$ et une extension de DVR 
	\[\Or_K-(\tilde \Or_K)_{\m_i}.\]
\end{enumerate}
Géometriquement c'est l'étude de la fibre en $\m_K\in \Spec(\Or_K)$.
Maintenant, on cherche a obtenir le $e$ tel que $\m_K=\m_i^e$. On 
obtient la cardinalité de la fibre et l'indice de ramification des
uniformisantes.


En fait maintenant le point puissant, c'est le point où on complète 
$K$ en $\m_K$ et $L$ en $\m_i$. Alors
\[\widehat K -\widehat L\]
a dimension $f_ie_i$.
\begin{rem}
    Un point qui semble flou là c'est : mais si je complète $K$
    en $\m_K$, $\widehat L=L.\widehat K$ donc il est où le choix
    de compléter en $\m_i$ pour $L$ ? On dirait que le choix est
    immédiat et les idéaux se contractent en un. En fait écrire
    \[L.\widehat K\]
    équivaut à faire vivre $L$ et $\widehat K$ au même endroit.
    D'où à plonger $L$ dans une clôture de $K$, $K^c$ et la 
    regarder dans $(\widehat K)^c$. Sinon on pourrait regarder
    \[L\otimes_K \widehat K\]
    et là son spectre est non trivial ! Choisir 
    \[L\otimes_K \widehat K\to (\widehat K)^c\]
    équivaut à choisir un idéal maximal. Voir notes.
\end{rem}

On peut maintenant utiliser la théorie des corps complets et
Hensel !

\section{Les points omis}
J'ai direct dit que $\tilde\Or_K$ était de Dedekind. En fait si $L/K$ 
est finie c'est toujours vrai.


\section{Quand est-ce que $\tilde\Or_K$ est noethérien ?}
Comme d'hab on prends 
\[\textrm{$L/K$ finie avec $K=Frac(\Or_K)$ un anneau de valuation
discrète}\]
Ensuite on regarde sa fermeture intégrale dans $L$, $\tilde\Or_K$. On
veut une extension de DVR, pour ça y faut que $\tilde\Or_K$ soit
de Dedekind. Le problème c'est toujours de montrer que c'est noethérien.

\subsection{Cas séparable}
Étant donnée $L/K$ finie séparable, on a tout ce qui nous faut. On
a un discriminant non nul bien défini, d'où si $L=\oplus Ke_i$ alors
\[(Tr(e_ie_j))_{i,j}\]
est non dégénérée. Puis on a une base duale à $e_i$, i.e. $e_i^*$ telle
que $Tr(e_i^*e_j)=\delta_{ij}$. Avec ça on peut
\begin{enumerate}
    \item à partir de $e_i$ une base de $L/K$ dans $\tilde\Or_K$,
	obtenir sa base duale pour la trace $e_i^*$.
    \item montrer que tout élément entier $b=\sum \lambda_i e_i^*$ 
	vérifie $\lambda_i\in \Or_K$, via $Tr(be_i)=\lambda_i$!
    \item D'où $\Or_K$ est un sous $\Or_K$-module d'un module de type
	fini donc noethérien.
\end{enumerate}

\subsection{Cas semi-local}
On sépare en une extension séparable et une purement inséparable.
Ensuite d'une extension purement inséparable d'une valuation discrète
$v$, on obtient directement une valuation discrète $w$.


\[\textrm{}\]
\end{document}

