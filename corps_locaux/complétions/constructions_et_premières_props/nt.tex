\documentclass[a4paper,12pt]{book}
\usepackage{amsmath,  amsthm,enumerate}
\usepackage{csquotes}
\usepackage[provide=*,french]{babel}
\usepackage[dvipsnames]{xcolor}
\usepackage{quiver, tikz}

%symbole caligraphique
\usepackage{mathrsfs}

%hyperliens
\usepackage{hyperref}

%pseudo-code
\usepackage{algpseudocode}
\usepackage{algorithm}
\makeatletter
  \renewcommand{\ALG@name}{Algorithme}
  \makeatother
\usepackage{fancyhdr}

\pagestyle{fancy}
\addtolength{\headwidth}{\marginparsep}
\addtolength{\headwidth}{\marginparwidth}
\renewcommand{\chaptermark}[1]{\markboth{#1}{}}
\renewcommand{\sectionmark}[1]{\markright{\thesection\ #1}}
\fancyhf{}
\fancyfoot[C]{\thepage}
\fancyhead[LO]{\textit \leftmark}
\fancyhead[RE]{\textit \rightmark}
\renewcommand{\headrulewidth}{0pt} % and the line
\fancypagestyle{plain}{%
    \fancyhead{} % get rid of headers
}

%bibliographie
\usepackage[
backend=biber,
style=alphabetic,
sorting=ynt
]{biblatex}

\addbibresource{bib.bib}

\usepackage{appendix}
\renewcommand{\appendixpagename}{Annexe}

\definecolor{wgrey}{RGB}{148, 38, 55}

\setlength\parindent{24pt}

\newcommand{\Z}{\mathbb{Z}}
\newcommand{\R}{\mathbb{R}}
\newcommand{\rel}{\omathcal{R}}
\newcommand{\Q}{\mathbb{Q}}
\newcommand{\C}{\mathbb{C}}
\newcommand{\N}{\mathbb{N}}
\newcommand{\K}{\mathbb{K}}
\newcommand{\A}{\mathbb{A}}
\newcommand{\B}{\mathcal{B}}
\newcommand{\Or}{\mathcal{O}}
\newcommand{\F}{\mathbb F}
\newcommand{\m}{\mathfrak m}
\renewcommand{\b}{\mathfrak b}
\renewcommand{\a}{\mathfrak a}
\newcommand{\p}{\mathfrak p}
\newcommand{\I}{\mathfrak I}
\newcommand{\Hom}{\textrm{Hom}}
\newcommand{\disc}{\textrm{disc}}
\newcommand{\Pic}{\textrm{Pic}}
\newcommand{\End}{\textrm{End}}
\newcommand{\Spec}{\textrm{Spec}}
\newcommand{\Frac}{\textrm{Frac}}

\newcommand{\cL}{\mathscr{L}}
\newcommand{\G}{\mathscr{G}}
\newcommand{\D}{\mathscr{D}}
\newcommand{\E}{\mathscr{E}}

\theoremstyle{plain}
\newtheorem{thm}{Théoreme}
\newtheorem{lem}{Lemme}
\newtheorem{prop}{Proposition}
\newtheorem{cor}{Corollaire}
\newtheorem{heur}{Heuristique}
\newtheorem{rem}{Remarque}
\newtheorem{rembis}{Remarque}
\newtheorem{note}{Note}

\theoremstyle{definition}
\newtheorem{conj}{Conjecture}
\newtheorem*{eq}{Équivalences}
\newtheorem{prob}{Problème}
\newtheorem{quest}{Question}
\newtheorem{prot}{Protocole}
\newtheorem{algo}{Algorithme}
\newtheorem{defn}{Définition}
\newtheorem{defnbis}{Définition}
\newtheorem{ex}{Exemple}
\newtheorem{exo}{Exercices}

\theoremstyle{remark}

\definecolor{wgrey}{RGB}{148, 38, 55}
\definecolor{wgreen}{RGB}{100, 200,0} 
\hypersetup{
    colorlinks=true,
    linkcolor=wgreen,
    urlcolor=wgrey,
    filecolor=wgrey
}

\title{Complétions}
\date{}

\begin{document}
\maketitle

C'est pas mal du chapitre $10$ de Atiyaah-McDonald.

\chapter{Récap}
\chapter{Complétions de corps}
J'prends $K$ un corps.
\section{Complété de $(K,|.|)$}
On suppose $\R$ construit (pas dur).
On peut prendre $C(K)/\m$ où $\m$ définit
la relation d'équivalence
\[(x_n)\sim (y_n)\equiv |x_n-y_n|\to 0\]
alors via 
\[-|x_n-x_m|\leq |x_n|-|x_m|\leq |x_n-x_m|\]
on sait que $(|x_n|)$ est de Cauchy dans $\R$
d'où on déf 
\[|(x_n)|_{\hat K} := \lim |x_n|\]

\begin{rem}
    $\m$ est maximal c'est pas si évident,
    le fait que le quotient est un corps faut
    définir l'ordre et ça se voit bien.
\end{rem}

\begin{rem}
    On peut remarquer que si $|.|$ est ultramétrique et
    $a=(x_n)_n\in \hat K$, alors $|a-x_n+x_n|=|x_n|$. On a pas 
    besoin de faire tendre par "au dessus". Mais on pourrait peut
    être le faire dès qu'il existe 
    une suite qui tend vers 0 dans $K$ je pense. Par exemple
    $(p^n)_n$ dans $\Q$, alors si $(x_n)$ tend vers $a$ on prend
    $(x_n+p^{k_n})$ de sorte que $|x_n+p^{k_n}|>|x|$.
\end{rem}

\section{Structure d'anneau}
C'est un quotient d'un sous-anneaux de
$K^\N$ donc multiplication et addition terme à terme.
\begin{rm}
    Tout se passe bien car valeur absolue.
\end{rm}
\section{Étude sur $\Z_p$ (inversibles et via Cauchy)}
Un élément de $\Z_p$ on peut le voir comme une suite 
$(x_n)_n\in \Z^\N$, parce que si on prends une suite
$(u_np^{k_n})$, avec $v_p(u_n)=0$. Bah le dénominateur
de la fraction irréductible est une unité dans $\Z_p$
au sens de la valuation.
Y suffit donc de remarquer que si $u\in \Z-p\Z$ alors
$(1/u)\in \Z_p$ s'écrit comme une série à coeffs
dans $\Z$. Ça sera la suite.

\begin{rem}
    Ça éclaircit le flou que $\Z$ est bien dense dans
    $\Z_p$, pas besoin de $\Z_{(p)}$.
\end{rem}

\begin{rem}
    C'est le pont primordial aussi pour parler de la limite
    projective comme si c'était la complétion. Parce que
    ducoup ça fait sens de réduire mod $p^n$.
\end{rem}

\subsection{Densité de $\Z$ dans $\Z_p$}
Ducoup plus concrètement. On veut approcher
pour $u\in \Z-p\Z$, la fraction $1/u$. On peut
prendre \[ua+pv=1\]
alors $1/u =\frac{a}{1-pv}$ puis 
\[1/u=a.\sum v^i p^i\]
où ici la somme est vraiment une somme. Pas une
écriture du développement. Donc faut étudier la
convergence de 
\[\sum_{i=0}^n a(vp)^i\]
et dans les $p$-adiques il suffit que 
\[\lim a(vp)^n =0\]
et ça c'est clair.
\begin{rem}
    À LIRE. Ce qui est étonnant c'est que $1/u$ 
    et $\sum a(vp)^i$ ont a priori rien à voir
    dans $\Q$. Genre cette somme tend vers l'infini
    en métrique réel. Mais faut bien se rappeler qu'on
    est en métrique $p$-adique!!
\end{rem}
\begin{rem}
    Un truc un peu frustrant c'est que là j'utilise
    le trick que $1=ua+pv$. En fait le $a$ et
    le $v$ qui apparaissent peuvent provenir
    de manipulation du
    développement $p$-adique de $u$.
\end{rem}
Je pense qu'une approche plus claire c'est que via
ce développement $u=\sum_{i=0}^{\log_p(u)} a_ip^i$,
on multiplie par l'inverse de $u$ modulo $p$, ce sera $a$.
Et alors 
\[au=1-pv!\]
La manière de Borcherds est strictement équivalente.
Ça veut dire quoi diviser modulo $p$ ? Ou plutôt
être divisible ? Prendre l'inverse de $1=\sum p^n$

Bon qu'est-ce qu'il s'est passé ? Le point flou
que j'essaierai de me raisonner c'est de remplacer
$1/u$ par $1/u =\frac{a}{1-pv}$. 

\begin{rem}
    On peut le voir comme juste remplacer $1/u \mod p$ par
    $1 \mod p$.
\end{rem}
Pour rappel si
$S_n=\sum_{i=0}^n q^i$, alors
\[(1-q)S_n=1-q^{n+1}\]

\section{Les $10$-adiques}
La vidéo de Borcherds est super instructive 
\href{https://www.youtube.com/watch?v=VTtBDSWR1Ac}{VIDEO}.
Il mentionne
par exemple que pour $x\in \Z$, $x^n$ quand il converge
peut converger vers $z$ tel que $z^2=z$. Ça montre que la 
série voit ses premiers termes fixé petits à petit.
\begin{rem}
    Attention, le $x$ choisit c'est $5$. Le point clé je
    pense c'est que $10 \mid 5^2-5$. Puis $100\mid 5^3-25$.
\end{rem}

\section{Expansion $t$-adique}

Dans le cas $p$-adique en fait dire qu'il existe une
expansion \[x=\sum a_i p^i\]
où $0\leq a_i<p$ c'est juste trouver une série qui converge
vers $x$. 

\section{Densité de $\Or_K$ dans $\Or_{\hat K}$ et corps résiduel.}
Simplement parce que $(x_n)$ stationne toujours par
l'inégalité ultramétrique. Ducoup on peut prendre les suites
dans $\Or_K$. Ensuite réduire modulo $\m$ dans $\Or_{\hat K}$
ça veut dire quoi ? L'idéal c'est $B(0,1)$ ouvert. On peut
simplement prendre $x\in \Or_K$ tel que 
\[|a-x|<1\]
par densité. En particulier, $x=a\mod B(0,1)$ par définition.


\chapter{Groupes abéliens topologiques}
Les opérations du groupes sont continues de sorte que si
$U$ est un voisinage de $0$, alors $x+U$ un voisinage de $x$.
\section{Séparation}
Étant donné $H=\cap_{0\in U} U$, on a $H=\{0\}$, et si $x\in H$,
alors 
\[0\in U-x\]
pour tout $0\in U$. D'où $-x\in H$. Pareil si $x,y\in H$, alors
pour tout voisinage de $0$ :
\[x\in U-y\]
d'où $x+y\in U$. D'où $H$ est un sous-groupe de $G$.
On déduit rapidement que $x+H$ est fermé pour tout $x$ d'où
\[G/H\]
est séparé. Alors 
\[G\textrm{ est séparé ssi }H=0\]


\chapter{Complétions via les suites de Cauchy}
On se met dans un groupe abélien topologique.
Étant donné une base de voisinage dénombrable de $0$. On peut
construire les suites de Cauchy en disant 
\[\textrm{Pour tout $U\ni 0$ il existe un entier $s(U)$}\]
tel que 
\[x_n-x_m\in U\textrm{ dés que} $n,m\geq s(U)$\]
\begin{rem}
    La limite est définie à l'adhérence près!
\end{rem}
\section{Complété de $G$}
Deux suites de Cauchy sont équivalentes si $x_n-y_n$ tend
vers $0$. Au sens défini d'avant. Maintenant on quotiente comme
d'hab et la somme est bien définie. On obtient 
\[\hat G:=C(G)/\m.\]

Maintenant un point intéressant, on a 
\[i\colon G\to \hat G\]
donné par les suites constantes. Mais 
\[\ker i = H\]
ça c'est rigolo. 
\begin{rem}
    Dans le cas d'un corps ou d'un anneau de valuation discrète
    $H=0$. En fait par le chapitre d'avant, quand $G$ est
    séparé.
\end{rem}

\chapter{Complétions algébriques}
J'aimerai bien aller jusqu'à Artin-Rees.
\section{Cadre}
On suppose qu'on a une base de voisinages de $0$ qui sont des
groupes. Ca évince $\R$ et la topologie habituelle. Par contre
pas $\Q$ et ses topologies $p$-adiques.

\begin{rem}
    Automatiquement, les ouverts de la base sont ouverts
    fermés. Car si $g\in G_n$, comme $g+G_n\subset G_n$ est un
    ouvert $G_n$ est ouvert. À l'inverse, pour $h\notin G_n$,
    $h+G_n\cap G_n=\emptyset$ et 
    $h+G_n$
    est ouvert d'où
    \[\cup{h\in G-G_n} h+G_n = G-G_n\]
    est ouvert puis $G_n$ est fermé.
\end{rem}

\begin{rem}
    Je me suis rendu compte de deux trucs vraiment
    marrant. La condition $G_m\cup G_n=G_m$ c'est la
    condition ultramétrique ! Aussi, le faisceau
    associé au préfaisceau tel que $x+G_i\mapsto G/G_i$
    est flasque! Et sa fibre en $0$ est $\hat G$.
\end{rem}

\section{}




\[\textrm{}\]
\end{document}

