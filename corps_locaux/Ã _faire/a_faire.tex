\documentclass[a4paper,12pt]{article}
\usepackage{amsmath,  amsthm,enumerate}
\usepackage{csquotes}
\usepackage[provide=*,french]{babel}
\usepackage[dvipsnames]{xcolor}
\usepackage{quiver, tikz}

%symbole caligraphique
\usepackage{mathrsfs}

%hyperliens
\usepackage{hyperref}

%pseudo-code
\usepackage{algpseudocode}
\usepackage{algorithm}
\makeatletter
  \renewcommand{\ALG@name}{Algorithme}
  \makeatother
\usepackage{fancyhdr}

%
%\pagestyle{fancy}
%\addtolength{\headwidth}{\marginparsep}
%\addtolength{\headwidth}{\marginparwidth}
%\renewcommand{\chaptermark}[1]{\markboth{#1}{}}
%\renewcommand{\sectionmark}[1]{\markright{\thesection\ #1}}
%\fancyhf{}
%\fancyfoot[C]{\thepage}
%\fancyhead[LO]{\textit \leftmark}
%\fancyhead[RE]{\textit \rightmark}
%\renewcommand{\headrulewidth}{0pt} % and the line
%\fancypagestyle{plain}{%
%\fancyhead{} % get rid of headers
%}
%

%bibliographie
\usepackage[
backend=biber,
style=alphabetic,
sorting=ynt
]{biblatex}

\addbibresource{bib.bib}

\usepackage{appendix}
\renewcommand{\appendixpagename}{Annexe}

\definecolor{wgrey}{RGB}{148, 38, 55}

\setlength\parindent{24pt}

\newcommand{\Z}{\mathbb{Z}}
\newcommand{\R}{\mathbb{R}}
\newcommand{\rel}{\omathcal{R}}
\newcommand{\Q}{\mathbb{Q}}
\newcommand{\C}{\mathbb{C}}
\newcommand{\N}{\mathbb{N}}
\newcommand{\K}{\mathbb{K}}
\newcommand{\A}{\mathbb{A}}
\newcommand{\B}{\mathcal{B}}
\newcommand{\Or}{\mathcal{O}}
\newcommand{\F}{\mathbb F}
\newcommand{\m}{\mathfrak m}
\renewcommand{\b}{\mathfrak b}
\renewcommand{\a}{\mathfrak a}
\newcommand{\p}{\mathfrak p}
\newcommand{\I}{\mathfrak I}
\newcommand{\Hom}{\textrm{Hom}}
\newcommand{\disc}{\textrm{disc}}
\newcommand{\Pic}{\textrm{Pic}}
\newcommand{\End}{\textrm{End}}
\newcommand{\Spec}{\textrm{Spec}}

\newcommand{\cL}{\mathscr{L}}
\newcommand{\G}{\mathscr{G}}
\newcommand{\D}{\mathscr{D}}
\newcommand{\E}{\mathscr{E}}

\theoremstyle{plain}
\newtheorem{thm}{Théoreme}
\newtheorem{lem}{Lemme}
\newtheorem{prop}{Proposition}
\newtheorem{cor}{Corollaire}
\newtheorem{heur}{Heuristique}
\newtheorem{rem}{Remarque}
\newtheorem{rembis}{Remarque}
\newtheorem{note}{Note}

\theoremstyle{definition}
\newtheorem{conj}{Conjecture}
\newtheorem*{eq}{Équivalences}
\newtheorem{prob}{Problème}
\newtheorem{quest}{Question}
\newtheorem{prot}{Protocole}
\newtheorem{algo}{Algorithme}
\newtheorem{defn}{Définition}
\newtheorem{defnbis}{Définition}
\newtheorem{ex}{Exemple}
\newtheorem{exo}{Exercices}

\theoremstyle{remark}

\definecolor{wgrey}{RGB}{148, 38, 55}
\definecolor{wgreen}{RGB}{100, 200,0} 
\hypersetup{
    colorlinks=true,
    linkcolor=wgreen,
    urlcolor=wgrey,
    filecolor=wgrey
}

\title{Trucs à faire}
\date{}

\begin{document}
\maketitle

\section{Ramification totale}
Pq si $f=1$ et $k_K\simeq k_L$ alors $u\in \Or_L^*$ s'écrit
$v(1+\epsilon)$ avec $v\in \Or_K^*$ et $\epsilon\in \m_L$? Y'a des
quotients à débroussailler.

Y'a un nouvel exemple dans le lemme 3.27/3.26.

En fait c'est stupide, on a une flèche surjective $\Or_K\to k_L$
d'où $u=v+\epsilon'$ puis on peut remplacer $\epsilon'$ par
$\epsilon=v^{-1}\epsilon'$ alors $v(1+\epsilon)=u$. Et oui
$v$ est bien une unité en passant au quotient en $k_K$ par exemple.

\section{Norme de Gauss}
Pourquoi sur $K(X)$ muni de la norme de Gauss, on a 
$k_{K(X)}=k_K(X)$. Je buguais comme un con, ptn. En fait le fait
que la localisation commute avec les quotients c'est ça la clé
(ou pas).
Les élements de $k_{K(X)}$ s'écrivent comme des $P/Q$ avec 
$P,Q\in \Or_K[X]$ parce que $\Or_K[X]\subset \Or_{K(X)}$ et y ont
le même corps de fraction. Maintenant on a le diagramme
% https://q.uiver.app/#q=WzAsNixbMCwxLCJcXE9yX0tbWF0iXSxbMCwyLCJrX0tbWF0iXSxbMSwxLCJcXE9yX3tLKFgpfSJdLFsxLDIsImtfe0soWCl9Il0sWzAsMCwiSyhYKSJdLFsxLDAsIksoWCkiXSxbMSwzLCIiLDAseyJzdHlsZSI6eyJ0YWlsIjp7Im5hbWUiOiJob29rIiwic2lkZSI6InRvcCJ9fX1dLFswLDEsIiIsMCx7InN0eWxlIjp7ImhlYWQiOnsibmFtZSI6ImVwaSJ9fX1dLFsyLDMsIiIsMix7InN0eWxlIjp7ImhlYWQiOnsibmFtZSI6ImVwaSJ9fX1dLFswLDIsIiIsMix7InN0eWxlIjp7InRhaWwiOnsibmFtZSI6Imhvb2siLCJzaWRlIjoidG9wIn19fV0sWzAsNF0sWzIsNV0sWzQsNSwiPSIsMSx7InN0eWxlIjp7ImJvZHkiOnsibmFtZSI6Im5vbmUifSwiaGVhZCI6eyJuYW1lIjoibm9uZSJ9fX1dXQ==
\[\begin{tikzcd}
	{K(X)} & {K(X)} \\
	{\Or_K[X]} & {\Or_{K(X)}} \\
	{k_K[X]} & {k_{K(X)}}
	\arrow["{=}"{description}, draw=none, from=1-1, to=1-2]
	\arrow[from=2-1, to=1-1]
	\arrow[hook, from=2-1, to=2-2]
	\arrow[two heads, from=2-1, to=3-1]
	\arrow[from=2-2, to=1-2]
	\arrow[two heads, from=2-2, to=3-2]
	\arrow[hook, from=3-1, to=3-2]
\end{tikzcd}\]
et je pense que maintenant faut tensoriser 
\[0\to \m_K\Or_K[X]\to \Or_K[X]\to k_K[X]\to 0\]
par $K(X)$ et comparer avec
\[0\to \m_{K(X)}\Or_{K(X)}\to k_{K(X)}\to 0\]
sachant que les deux premiers termes deviennent égaux ?Non y'a
un problème.

SOLUTION : Plutôt que ces trucs compliqués : on écrit 
$P/Q=(P/||P||)/(Q/||P||)$ (ca fait pass sens tel que mais on 
s'arrange) ensuite $P/||P||\ne 0\in k_{K(X)}$ et est dans
$k_K[X]$ d'où $k_{K(X)}$ est le corps de fraction. On a aussi
$||Q||/||P||=1$ d'où $Q/||P||\in k_K[X]$.

\section{$\Q_p-\Q_p(\Zeta_{p^r})$}
Cette extension est totalement ramifiée de degré $(p-1)p^{r-1}$.
\begin{enumerate}
  \item $\lambda_{p^r,i}=\zeta_{p^r}^i-1$, on a 
    $\prod_{i=0}^{p-1} \lambda_{p,i}=p$, $\lambda_p/\lambda_{p,i}
    \in \Or_{\Q_p(\zeta_p)}^\times$ car $\sum_{j=0}^{i-1} \zeta_p^j\ne 0\mod \m_{\Q_p(\zeta_p)}$.
  \item D'où $p\in \lambda_p^{p-1}\Or_{\Q_p(\zeta_p)}^\times$
    et $e_{K/\Q_p}=p-1=[\Q_p(\zeta_p):\Q_p]$.
  \item 
    $\prod(\lambda_{p^r}-\lambda_{p^{r-1},i})=\lambda_{p^{r-1}}$.
  \item Comme 
    $v_p(\lambda_{p^r}-\lambda_{p^{r-1},i})\geq 1/p^{r-1}(p-1)$
    où l'inégalité est par dimension de l'extension! Et qu'on a
    $p^{r-1}$ termes $>0$ on peut conclure.
  \item Dans le cas $r=2$ on peut utiliser l'inversibilité de 
    $\lambda_{p}/\lambda_{p,i}$. Ça doit se prouver pareil avec
    $\lambda_{p^r,i}/\lambda_{p^r}=\sum_{j=0}^{i-1} \zeta_{p^r}^i$.
    et si $p\mid i-1$ ? Ces termes apparaissent pas dans la somme
    vue que le pol min de $\zeta_{p^r}$ divisent $\phi_p(X^r)$?
    Un truc comme ça.
\end{enumerate}


\end{document}

