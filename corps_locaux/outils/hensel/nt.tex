\documentclass[a4paper,12pt]{book}
\usepackage{amsmath,  amsthm,enumerate}
\usepackage{csquotes}
\usepackage[provide=*,french]{babel}
\usepackage[dvipsnames]{xcolor}
\usepackage{quiver, tikz}

%symbole caligraphique
\usepackage{mathrsfs}

%hyperliens
\usepackage{hyperref}

%pseudo-code
\usepackage{algpseudocode}
\usepackage{algorithm}
\makeatletter
  \renewcommand{\ALG@name}{Algorithme}
  \makeatother
\usepackage{fancyhdr}

\pagestyle{fancy}
\addtolength{\headwidth}{\marginparsep}
\addtolength{\headwidth}{\marginparwidth}
\renewcommand{\chaptermark}[1]{\markboth{#1}{}}
\renewcommand{\sectionmark}[1]{\markright{\thesection\ #1}}
\fancyhf{}
\fancyfoot[C]{\thepage}
\fancyhead[LO]{\textit \leftmark}
\fancyhead[RE]{\textit \rightmark}
\renewcommand{\headrulewidth}{0pt} % and the line
\fancypagestyle{plain}{%
    \fancyhead{} % get rid of headers
}

%bibliographie
\usepackage[
backend=biber,
style=alphabetic,
sorting=ynt
]{biblatex}

\addbibresource{bib.bib}

\usepackage{appendix}
\renewcommand{\appendixpagename}{Annexe}

\definecolor{wgrey}{RGB}{148, 38, 55}

\setlength\parindent{24pt}

\newcommand{\Z}{\mathbb{Z}}
\newcommand{\R}{\mathbb{R}}
\newcommand{\rel}{\omathcal{R}}
\newcommand{\Q}{\mathbb{Q}}
\newcommand{\C}{\mathbb{C}}
\newcommand{\N}{\mathbb{N}}
\newcommand{\K}{\mathbb{K}}
\newcommand{\A}{\mathbb{A}}
\newcommand{\B}{\mathcal{B}}
\newcommand{\Or}{\mathcal{O}}
\newcommand{\F}{\mathbb F}
\newcommand{\m}{\mathfrak m}
\renewcommand{\b}{\mathfrak b}
\renewcommand{\a}{\mathfrak a}
\newcommand{\p}{\mathfrak p}
\newcommand{\I}{\mathfrak I}
\newcommand{\Hom}{\textrm{Hom}}
\newcommand{\disc}{\textrm{disc}}
\newcommand{\Pic}{\textrm{Pic}}
\newcommand{\End}{\textrm{End}}
\newcommand{\Spec}{\textrm{Spec}}

\newcommand{\cL}{\mathscr{L}}
\newcommand{\G}{\mathscr{G}}
\newcommand{\D}{\mathscr{D}}
\newcommand{\E}{\mathscr{E}}

\theoremstyle{plain}
\newtheorem{thm}{Théoreme}
\newtheorem{lem}{Lemme}
\newtheorem{prop}{Proposition}
\newtheorem{cor}{Corollaire}
\newtheorem{heur}{Heuristique}
\newtheorem{rem}{Remarque}
\newtheorem{rembis}{Remarque}
\newtheorem{note}{Note}

\theoremstyle{definition}
\newtheorem{conj}{Conjecture}
\newtheorem*{eq}{Équivalences}
\newtheorem{prob}{Problème}
\newtheorem{quest}{Question}
\newtheorem{prot}{Protocole}
\newtheorem{algo}{Algorithme}
\newtheorem{defn}{Définition}
\newtheorem{defnbis}{Définition}
\newtheorem{ex}{Exemple}
\newtheorem{exo}{Exercices}

\theoremstyle{remark}

\definecolor{wgrey}{RGB}{148, 38, 55}
\definecolor{wgreen}{RGB}{100, 200,0} 
\hypersetup{
    colorlinks=true,
    linkcolor=wgreen,
    urlcolor=wgrey,
    filecolor=wgrey
}

\title{Approximation $p$-adiques(Hensel)}
\date{}

\begin{document}
\maketitle


\section{Relever des facteurs}
Y'a trois théorèmes, dans $K$ ultramétrique, si 
\[P\in \Or_K[X]\]
et
\[\bar P= f.g\in k_K[X]\]
avec $(f,g)=1$ alors il existe d'uniques (à unités près)
$\bar F=f$, $\bar G=g$ 
avec $\deg(f)=\deg(F)$ (on peut fixer un des deux degrés) et
\[P=F.G\in \Or_K[X]\]
\begin{note}
    On peut relever des facteurs premiers entre eux de manière
    unique.
\end{note}
\begin{rem}
    La preuve consiste à approximer $F$ et $G$ par $(F_n)$
    et $(G_n)$ et dire que $||P-F_nG_n||$ tend vers $0$ en
    norme de Gauss. En particulier on obtient les coefficients
    des facteurs!
\end{rem}
La deuxième version dit que si il existe $\alpha\in \Or_K$
tel que \[|P(\alpha)|<|P'(\alpha)|^2\] alors il existe un unique
zéro $\widetilde\alpha\in \Or_K$ proche de $\alpha$, i.e. 
telque 
\[|\alpha-\tilde\alpha|<|P'(\alpha)|\]
\begin{note}
    Le point c'est de dire que si la pente est suffisamment grande
    et $P(\alpha)$ est suffisamment petit devant la pente. On
    va pouvoir se rapprocher assez de $0$ pour avoir un zéro!
    Penser au cas réel.
\end{note}
\begin{rem}
    En notant $P(\alpha)=\lambda$ le tricks c'est de considérer 
    l'expansion de Taylor $P(X+\alpha)=P(\alpha)+\lambda X+X^2R(X)$
    d'où
    \[\lambda^{-2}P(\lambda X+\alpha)=\lambda^{-2}(P(\alpha)+\lambda^2 X+X^2S(X))\]
    puis en réduisant modulo $\m$ le résultat.
\end{rem}
\section{Critère nécessaire à l'irréducibilité}
On sait que si $P\in \Or_K[X]$, alors 
$\max |a_i|=\max |a_0|,|a_d|$. Le point c'est que sinon si 
$|a_i|>|a_0|,|a_d|$ alors $Q=a_i^{-1}P\in\Or_K[X]$ et 
$\bar Q=X^rQ'\mod \m$ avec $r<d$ (important) et $Q'\wedge X=1$.
D'où on conclut avec Hensel.

\end{document}

