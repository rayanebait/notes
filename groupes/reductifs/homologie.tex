\documentclass[a4paper,12pt]{book}
\usepackage{amsmath,  amsthm,enumerate}
\usepackage{csquotes}
\usepackage[provide=*,french]{babel}
\usepackage[dvipsnames]{xcolor}
\usepackage{quiver, tikz}

%symbole caligraphique
\usepackage{mathrsfs}

%hyperliens
\usepackage{hyperref}

%pseudo-code
\usepackage{algorithm}
\usepackage{algpseudocode}

\usepackage{fancyhdr}

\pagestyle{fancy}
\addtolength{\headwidth}{\marginparsep}
\addtolength{\headwidth}{\marginparwidth}
\renewcommand{\chaptermark}[1]{\markboth{#1}{}}
\renewcommand{\sectionmark}[1]{\markright{\thesection\ #1}}
\fancyhf{}
\fancyfoot[C]{\thepage}
\fancyhead[LO]{\textit \leftmark}
\fancyhead[RE]{\textit \rightmark}
\renewcommand{\headrulewidth}{0pt} % and the line
\fancypagestyle{plain}{%
    \fancyhead{} % get rid of headers
}

%bibliographie
\usepackage[
backend=biber,
style=alphabetic,
sorting=ynt
]{biblatex}

\addbibresource{bib.bib}

\usepackage{appendix}
\renewcommand{\appendixpagename}{Annexe}

\definecolor{wgrey}{RGB}{148, 38, 55}

\setlength\parindent{24pt}

\newcommand{\Z}{\mathbb{Z}}
\newcommand{\R}{\mathbb{R}}
\newcommand{\rel}{\omathcal{R}}
\newcommand{\Q}{\mathbb{Q}}
\newcommand{\C}{\mathbb{C}}
\newcommand{\N}{\mathbb{N}}
\newcommand{\K}{\mathbb{K}}
\newcommand{\A}{\mathbb{A}}
\newcommand{\Gr}{\mathbb{G}}
\newcommand{\B}{\mathcal{B}}
\newcommand{\Or}{\mathcal{O}}
\newcommand{\F}{\mathscr F}
\newcommand{\Hom}{\textrm{Hom}}
\newcommand{\disc}{\textrm{disc}}
\newcommand{\Pic}{\textrm{Pic}}
\newcommand{\End}{\textrm{End}}
\newcommand{\Spec}{\textrm{Spec}}
\newcommand{\Spm}{\textrm{Spm}}
\newcommand{\Supp}{\textrm{Supp}}
\renewcommand{\Im}{\textrm{Im}}


\newcommand{\m}{\mathfrak{m}}
\newcommand{\p}{\mathfrak{p}}
\newcommand{\q}{\mathfrak{p}}


\newcommand{\cL}{\mathscr{L}}
\newcommand{\G}{\mathscr{G}}
\newcommand{\D}{\mathscr{D}}
\newcommand{\E}{\mathscr{E}}
\renewcommand{\Pr}{\mathbb{P}}
\renewcommand{\P}{\mathscr{P}}
\renewcommand{\H}{\mathscr{H}}

\makeatletter
\newcommand{\colim@}[2]{%
  \vtop{\m@th\ialign{##\cr
    \hfil$#1\operator@font colim$\hfil\cr
    \noalign{\nointerlineskip\kern1.5\ex@}#2\cr
    \noalign{\nointerlineskip\kern-\ex@}\cr}}%
}
\newcommand{\colim}{%
  \mathop{\mathpalette\colim@{\rightarrowfill@\scriptscriptstyle}}\nmlimits@
}
\renewcommand{\varprojlim}{%
  \mathop{\mathpalette\varlim@{\leftarrowfill@\scriptscriptstyle}}\nmlimits@
}
\renewcommand{\varinjlim}{%
  \mathop{\mathpalette\varlim@{\rightarrowfill@\scriptscriptstyle}}\nmlimits@
}
\makeatother

\theoremstyle{plain}
\newtheorem{thm}[subsection]{Théoreme}
\newtheorem{lem}[subsection]{Lemme}
\newtheorem{prop}[subsection]{Proposition}
\newtheorem{cor}[subsection]{Corollaire}
\newtheorem{heur}{Heuristique}
\newtheorem{rem}{Remarque}
\newtheorem{note}{Note}
\newtheorem{strat}{Stratégie}

\theoremstyle{definition}
\newtheorem{conj}{Conjecture}
\newtheorem{prob}{Problème}
\newtheorem{quest}{Question}
\newtheorem{prot}{Protocole}
\newtheorem{algo}{Algorithme}
\newtheorem{defn}[subsection]{Définition}
\newtheorem{exmp}[subsection]{Exemples}
\newtheorem{exo}[subsection]{Exercices}
\newtheorem{ex}[subsection]{Exemple}
\newtheorem{exs}[subsection]{Exemples}

\theoremstyle{remark}

\definecolor{wgrey}{RGB}{148, 38, 55}
\definecolor{wgreen}{RGB}{100, 200,0} 
\hypersetup{
    colorlinks=true,
    linkcolor=wgreen,
    urlcolor=wgrey,
    filecolor=wgrey
}

\title{Groupes réductifs}
\date{}

\begin{document}
\maketitle
\tableofcontents
\[\ldots\]
Soit $k$ algébriquement clos. Un groupe algébrique sur $k$ est
une $k$-variété avec une structure de groupe telle que $m\colon
G\times G\to G$ et $i\colon G\to G$ sont algébriques. 
\begin{rem}[Cartier]
  Les groupes algébriques en caractéristique $0$ sont réduits.
  (si on prends des variétés pas forcément réduites)
\end{rem}
\begin{rem}
  La même définition avec des variétés complexes donne les
  groupes de Lie complexe.
\end{rem}
\begin{exs}
  $\Gr_a$ le groupe additif sur $\A^1_k$ et $\Gr_m$ sur $\A^1_k-0$.
  C'est les seuls connexes de dimension $1$.
\end{exs}
Les sous-groupes fermés sont canoniquements des groupes algébriques
pour $n\geq 1$.
\begin{exs}
  $SL_n/K\subset GL_n/k\subset M_n/k$. Et aussi $T_n\subset GL_n$
  les triangulaires supèrieures et $U_n$ telles que $M-I_n$ sont
  nilpotentes dans $T_n\cap SL_n$. $\Gr_m^n\simeq D_n\subset GL_n$.
  Aussi $O(n)\subset GL_n$ tq $^tgg=I_n$. Et $Sp(2n)\subset GL_n/k$
  telles que $^tgJg=J$ avec 
  $J=\begin{pmatrix}0& I_n\\-I_n&0\end{pmatrix}$. Aussi, $A$
  une $k$-algèbre de dimension finie 
  (possiblement non commutative). Alors, $Aut(A)\subset GL(A)$
  est fermé! Pour $A=M_n$, on a $GL_n\to GL(M_n)$ donné par
  la conjugaison et ca passe au quotient en $PGL_n$! D'où
  $PGL_n=Aut(M_n)$ et celui de gauche est un groupe algébrique
  donc, c'est non trivial à priori mdr.
\end{exs}
\begin{thm}[Chevalley]
  Tout groupe algébrique connexe est uniquement une extension
  d'une variété abélienne (lisse connexe propre) par un groupe
  algébrique affine.
\end{thm}
Autrement dit une unique s.e.c
\[1\to H\to G\to A\to 1\]
dans Grp donnée par des morphismees algébriques.

\begin{thm}
  Tout les groupes algébriques affines sont linéaires.
\end{thm}

\begin{thm}
  Soit $G/\C$ un groupe algébrique affine connexe. Alors 
  les props suivantes sont équivalentes :
  \begin{enumerate}
    \item $G$ est isomorphe à un sous-groupe auto-adjoint
      d'un $GL_n(\C)$, i.e. un sous-groupe fermé algébrique
      stable par $g\mapsto ^t\bar g$.
    \item $G$ est linéairement réductif : toute représentation
      linéaire de $G$ se décompose en somme directe de 
      représentations irréductibles. (la catégorie des 
      représentations est semi-simple)
    \item $G$ est réductif : admet pas de sous-groupe fermé
      distingué tels que tout ses éléments nilpotents sont
      unipotents.
    \item $G$ admet une forme compacte réelle.
    \item $G$ admet un sous-groupe compact Zariski-dense.
    \item $\C[X]^G$ est de type fini sur $\C$ pour toute action
      algébrique de $G$ sur une variété affine $X$.
  \end{enumerate}
\end{thm}
\begin{exo}
  Prouver qu'être réductif est stable sur les sous-groupes
  distingués fermés ou quotients.
\end{exo}
\chapter{Groupes algébriques affines réductifs}
On prend tjr $k$ algébriquement clos.
\section{Les bases}
\subsection{La composante neutre}
\begin{prop}
  Soit $G$ un groupe algébrique.
  \begin{enumerate}
    \item L'élément neutre est contenue dans une unique
      composante irréductible de $G$, $G^0$, la composante neutre.
    \item $G^0$ est fermé normal dans $G$ d'indice fini.
    \item Les classes à gauche de $G^0$ dans $G$ sont les
      composantes connexes ainsi que irréductibles de $G$.
    \item Tout sous-groupe d'indice fini fermé contient $G^0$
  \end{enumerate}
\end{prop}
\begin{proof}
  $1.$ il existe un point dans une unique composante puis par
  translation (l'image isomorphe d'un truc irréd est irréd). Sinon
  juste psq c'est lisse. \\

  $2.$ c'est un sous-groupe irréd via l'image de 
  $G^0\times G^0\to G$. Ça contient $G^0$ donc c'est $G^0$,
  l'inversion envoie une c.c
  sur une c.c, et $e$ sur $e$. La conjugaison est un automorphisme
  et envoie $e$ sur $e$.
\end{proof}

\subsection{Noyau et image}
\begin{thm}
  Étant donné $f\colon G\to H$ un morphisme de groupes algébriques.
  \begin{enumerate}
    \item Le noyau est fermé normal.
    \item l'image est fermée.
    \item $f(G^0)=f(G)^0$.
    \item $f$ est un iso ssi c'est une bijection.
  \end{enumerate}
\end{thm}
\begin{proof}
  $1.$ est facile.\\
  \begin{thm}[Chevalley]
    Les images de flèches algébriques contiennent des ouverts
    dense.
  \end{thm}
  \begin{lem}
    Si $U,V$ sont deux ouverts denses distincts de $G$.
  \end{lem}
  \begin{proof}
    On a pour tout $g$, $U\cap g.i(V)\ne \emptyset$ d'où le
    résultat lol. En particulier $U^2=G$ lol.
  \end{proof}
  \begin{prop}
    Si $H$ est un sous-groupe (abstrait) de $G$ alors 
    $\bar H$ est un sous-groupe fermé de $G$. Si en plus $H$
    contenait un ouvert dense de $\bar H$ alors $H$ était fermé.
  \end{prop}
  \begin{proof}
    On a direct que $i(\bar H)=\bar{i(H)}=\bar H$ lol. En plus
    $h.\bar H=\bar{hH}=\bar H$ car $L_h$ est un automorphisme
    d'où $H.\bar H=\bar H$, maintenant pour $h\in \bar H$ on a
    $\bar H.h=\bar{H.h}$ car le truc de gauche est fermé d'où
    $\bar H.h\subset \bar H$. I.e. $\bar H.\bar H\subset\bar H$.
    Si $H$ contient $U$ dense ouvert, $H\supset U^2=\bar H$ par le
    lemme (omg).
  \end{proof}
  \begin{rem}
    Les localement fermés sont fermés.
  \end{rem}
  $2.$ Grâce à Chevalley, $f(G)$ contient un ouvert dense d'où
  $\bar(f(G))=f(G)$ vu que $f(G)$ est un groupe.
\end{proof}

\section{Actions et représentations}
\begin{defn}
  Une action d'un groupe algébrique $G$ sur une variété algébrique
  $X$ est un morphisme $G\times X\to X$ tel que $G\to Aut(X)$
  (à droite les bijections) est un morphisme de groupe.
\end{defn}
\begin{defn}
  Soit $G\curvearrowright X$. Et $Y$ un fermé de $X$. Le 
  normalisateur de $Y$ dans $G$, $N_G(Y)$ est le groupe
  des $g$ qui préservent $Y$. Le centralisateur fixe $Y$, $C_G(Y)$.
  Le noyau de l'action de $N_G(Y)\curvearrowright Y$ est $C_G(Y)$.
  L'action est fidèle si $G\to Aut(X)$ est injective, $C_G(X)=e$.
  $X^G\subset X$ les points fixes par $G$. Quand $Y=\{x\}$,
  $N_G(x)=C_G(x)=G_x$ est la ifbre en $x$ de $G\to X$.
\end{defn}
\begin{prop}[$G\curvearrowright X$]
  On a 
  \begin{enumerate}
    \item Toutes les orbites sont localement fermés et lisses.
    \item Toute les orbites de dimension minimale sont fermées.
      (donc il en existe)
  \end{enumerate}
\end{prop}
\begin{proof}
  Soit $O=G.x=im(G\times\{x\}\to X)$, ça contient un ouvert dense
  de $\bar O$. Comme $O$ est $\bar O$ sont stable par $G$ et
  via $U\subset O\subset \bar O$, pour $o\in O$ on a un $g.o\in U$
  d'où $g^{-1}U\cap O$ est un ouvert qui contient $o$ dans $O$.
  En plus $\bar O- O$ est $G$-stable d'où une union de $G$-orbites
  de même dimension plus petite que $\bar O$ pas forcément un nb
  fini.
  \begin{ex}
    $GL_n\curvearrowright k^n\times k^n$ via l'action diagonale
    a une infinité d'orbites.
  \end{ex}
\end{proof}
\begin{defn}
  Soit $G$ un groupe algébrique. Un $G$-module (rationnel) est un
  $k$-ev $V$ muni d'une action linéaire de $G$, $G\to GL(V)$
  algébrique, tel que tout $v\in V$ est contenu dans un $G$-sous
  module de dimension finie stable par l'action induite.
\end{defn}
\begin{ex}
  Toute action algébrique $G\curvearrowright X$ fournit une
  action de $G$ sur $k[X]$ donné par $(g.f)(x)=f(g^{-1}(x))$
  pour $g\in G$, $f\in k[X]$ et $x\in X$ lisse.
\end{ex}
\begin{prop}
  Soit $G$ un groupe affine agissant sur une variété algébrique 
  $X$ via la définition précédente. Alors $k[X]$ est un $G$-module.
\end{prop}
\begin{proof}
  Soit $f\in k[X]$, on a $a\colon G\times X\to X$ d'où 
  $a^*\colon k[X]\to k[G\times X]=k[G]\otimes_k k[X]$. On a
  $a^*(f)(g,x)=\sum a_i(g)f_i(x)$, par déf c'est
  $f(g^{-1}.x)$. L'espace vectoriel engendré par les $x\mapsto
  f_i(g^{-1}.x)$ est de dimension finie. D'où les $g.f_i$ aussi
  pour tout $g\in G$ qui est le résultat.
\end{proof}
\begin{ex}
  L'action par translation à gauche $G\times G\to G$ si $G$
  est affine donne un $G$-module $k[G]$ qui est appelé la
  représentation régulière.
\end{ex}
\begin{ex}
  On a $\C[\Gr_m]=\C[T,T^{-1}]$ un $\Gr_m$-module qui se décompose
  en modules irreductibles.
\end{ex}

\begin{thm}[Chevalley]
  Les images de morphismes $X\to Y$ sont constructibles,
  i.e. union finie de localement fermés.
\end{thm}
\begin{rem}
  Les constructibles sont stables par complement union finie
  et intersection finie. I.e. algebre booléenne, et la plus 
  petite contenant les ouverts de $X$.
  C'est un corollaire de $\bar{f(X)}$ contient un ouvert.
\end{rem}

\begin{prop}
  Soit $G$ un g.a qui agit transitivement sur $X$ et $Y$.
  Et $f\colon X\to Y$, $G$-équivariante. Si $f$ est bijective
  alors $f$ est un iso (!).
\end{prop}
\begin{cor}
  Si $f\colon G\to H$ est un m.g de g.a. Alors $f$ est un iso
  ssi bijective (char $0$). En plus $f$ est une immersion fermée
  ssi $\ker(f)=e$.
\end{cor}
\begin{rem}
  L'action de $G$ sur $H$ est transitive automatiquement si bij.
\end{rem}


\printbibliography
\end{document}

