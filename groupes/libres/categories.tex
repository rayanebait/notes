\documentclass[a4paper,12pt]{book}
\usepackage{amsmath,  amsthm,enumerate}
\usepackage{csquotes}
\usepackage[provide=*,french]{babel}
\usepackage[dvipsnames]{xcolor}
\usepackage{quiver, tikz}

%symbole caligraphique
\usepackage{mathrsfs}

%hyperliens
\usepackage{hyperref}

%pseudo-code
\usepackage{algorithm}
\usepackage{algpseudocode}

\usepackage{fancyhdr}

\pagestyle{fancy}
\addtolength{\headwidth}{\marginparsep}
\addtolength{\headwidth}{\marginparwidth}
\renewcommand{\chaptermark}[1]{\markboth{#1}{}}
\renewcommand{\sectionmark}[1]{\markright{\thesection\ #1}}
\fancyhf{}
\fancyfoot[C]{\thepage}
\fancyhead[LO]{\textit \leftmark}
\fancyhead[RE]{\textit \rightmark}
\renewcommand{\headrulewidth}{0pt} % and the line
\fancypagestyle{plain}{%
    \fancyhead{} % get rid of headers
}

%bibliographie
\usepackage[
backend=biber,
style=alphabetic,
sorting=ynt
]{biblatex}

\addbibresource{bib.bib}

\usepackage{appendix}
\renewcommand{\appendixpagename}{Annexe}

\definecolor{wgrey}{RGB}{148, 38, 55}

\setlength\parindent{24pt}

\newcommand{\Z}{\mathbb{Z}}
\newcommand{\R}{\mathbb{R}}
\newcommand{\rel}{\omathcal{R}}
\newcommand{\Q}{\mathbb{Q}}
\newcommand{\C}{\mathbb{C}}
\newcommand{\Cat}{\mathcal{C}}
\newcommand{\Dat}{\mathcal{D}}
\newcommand{\Aat}{\mathcal{A}}
\newcommand{\N}{\mathbb{N}}
\newcommand{\K}{\mathbb{K}}
\newcommand{\A}{\mathbb{A}}
\newcommand{\B}{\mathcal{B}}
\newcommand{\Or}{\mathcal{O}}
\newcommand{\F}{\mathscr F}
\newcommand{\Hom}{\textrm{Hom}}
\newcommand{\disc}{\textrm{disc}}
\newcommand{\Pic}{\textrm{Pic}}
\newcommand{\End}{\textrm{End}}
\newcommand{\Spec}{\textrm{Spec}}
\newcommand{\Supp}{\textrm{Supp}}
\newcommand{\Ouv}{\textrm{Ouv}}
\newcommand{\im}{\textrm{im}}
\newcommand{\coker}{\textrm{coker}}
\newcommand{\coim}{\textrm{coim}}


\newcommand{\cL}{\mathscr{L}}
\newcommand{\G}{\mathscr{G}}
\newcommand{\D}{\mathscr{D}}
\newcommand{\E}{\mathscr{E}}
\renewcommand{\P}{\mathscr{P}}
\renewcommand{\H}{\mathscr{H}}

\makeatletter
\newcommand{\colim@}[2]{%
  \vtop{\m@th\ialign{##\cr
    \hfil$#1\operator@font colim$\hfil\cr
    \noalign{\nointerlineskip\kern1.5\ex@}#2\cr
    \noalign{\nointerlineskip\kern-\ex@}\cr}}%
}
\newcommand{\colim}{%
  \mathop{\mathpalette\colim@{\rightarrowfill@\scriptscriptstyle}}\nmlimits@
}
\renewcommand{\varprojlim}{%
  \mathop{\mathpalette\varlim@{\leftarrowfill@\scriptscriptstyle}}\nmlimits@
}
\renewcommand{\varinjlim}{%
  \mathop{\mathpalette\varlim@{\rightarrowfill@\scriptscriptstyle}}\nmlimits@
}
\makeatother

\theoremstyle{plain}
\newtheorem{thm}{Théoreme}
\newtheorem{lem}{Lemme}
\newtheorem{prop}{Proposition}
\newtheorem{cor}{Corollaire}
\newtheorem{heur}{Heuristique}
\newtheorem{rem}{Remarque}
\newtheorem{note}{Note}

\theoremstyle{definition}
\newtheorem{conj}{Conjecture}
\newtheorem{prob}{Problème}
\newtheorem{quest}{Question}
\newtheorem{prot}{Protocole}
\newtheorem{algo}{Algorithme}
\newtheorem{defn}{Définition}
\newtheorem{exmp}{Exemples}
\newtheorem{exo}{Exercices}
\newtheorem{ex}{Exemple}
\newtheorem{exs}{Exemples}

\theoremstyle{remark}

\definecolor{wgrey}{RGB}{148, 38, 55}
\definecolor{wgreen}{RGB}{100, 200,0} 
\hypersetup{
    colorlinks=true,
    linkcolor=wgreen,
    urlcolor=wgrey,
    filecolor=wgrey
}

\title{}
\date{2024-2025}

\begin{document}
\maketitle
\tableofcontents
\chapter{Groupes libres et colimites}
\section{Propriété universelle ou adjonction ?}
Le truc bizarre c'est que les objets et objets sont dans $Set$
et/ou dans $Grp$. Ducoup en notant $i(G)\in Set$ le foncteur
d'oubli on a
\[\Hom_{Grp}(L(S),G)\simeq \Hom_{Set}(S,i(G))\]
Et la propriété universelle du groupe libre sur $S\in Set$
est donnée par 
% https://q.uiver.app/#q=WzAsMyxbMCwwLCJTIl0sWzEsMCwiTChTKSJdLFsxLDEsIkciXSxbMCwxLCJqIl0sWzAsMiwiayIsMl0sWzEsMiwiXFxleGlzdHMgIUwoaykiXV0=
\[\begin{tikzcd}
	S & {L(S)} \\
	& G
	\arrow["j", from=1-1, to=1-2]
	\arrow["k"', from=1-1, to=2-2]
	\arrow["{\exists !L(k)}", from=1-2, to=2-2]
\end{tikzcd}\]
\section{Construction de $L(S)$}
Pour construire le groupes libre on prend les suites finies 
à coefficients dans $S\times \{\pm 1\}$ munies de la concaténation
et l'élement neutre c'est le vide.

\section{Colimites de groupes}
Étant donné $(G_i)_i$ un petit diagramme dans $Grp$. On a
\[f_i\colon L(G_i)\to L(\sqcup G_i)\]
via $G_i\to \sqcup G_i$ dans $Set$
et
\[0\to N_i\to L(G_i)\to G_i\to 0\]
dans $Grp$ via $G_i\to G_i$ l'identité! Ensuite on a
$N:=L(\sqcup N_i)\to L(\sqcup G_i)$
puis $L(\sqcup G_i)/N$ est pas tout à faire universel, à ce
stade c'est le coproduit dans $Grp$. On 
identifie les $G_i$ et leurs image dans $L(\sqcup G_i)$ puis on
quotiente par le sous groupe normal (normalisé ? Oui faut
normaliser) engendré par l'ensemble des
$x\rho_{ij}^{-1}(x)$ pour tout $x\in G_i$ et $f_{ij}\in F_{ij}$
(on prend un diagramme petit quelconque). 

Pour prouver que c'est universel on peut juste obtenir d'abord
$L(\sqcup G_i)\to H$ puis montrer que ça passe au quotient de
manière unique. 

\section{Remarque}
Dans un groupe libre $L=L(S)$, l'ensemble des mots engendrés par
une partie $S'$ est en fait $L(S')$.

\section{Objet libre en général}
L'adjonction du 1.1 se généralise en prenant $Set$ avec
$Mod_R$ par exemple, ou $Ab$. On obtient le produit libre.

\chapter{Sommes amalgamées et arbres}
\section{Somme amalgamée à deux éléments}
C'est le cas particulier
% https://q.uiver.app/#q=WzAsNCxbMCwwLCJDIl0sWzEsMCwiQV8xIl0sWzAsMSwiQV8yIl0sWzEsMSwiQV8xKl9DQV8yIl0sWzAsMiwiaV8yIiwyXSxbMCwxLCJpXzEiXSxbMSwzXSxbMiwzXV0=
\[\begin{tikzcd}
	C & {A_1} \\
	{A_2} & {A_1*_CA_2}
	\arrow["{i_1}", from=1-1, to=1-2]
	\arrow["{i_2}"', from=1-1, to=2-1]
	\arrow[from=1-2, to=2-2]
	\arrow[from=2-1, to=2-2]
\end{tikzcd}\]
et dans ce cas, on peut écrire une présentation de $A_1*_CA_2$ 
à partir de présentations de $A_1$ et $A_2$. On a construit
la limite générale à partir de celle du bas :
% https://q.uiver.app/#q=WzAsOCxbMCwwLCIwIl0sWzEsMCwiUl9pIl0sWzIsMCwiU19pIl0sWzAsMSwiMCJdLFsxLDEsIk5faSJdLFsyLDEsIkwoQV9pKSJdLFszLDEsIkFfaSJdLFs0LDEsIjAiXSxbMCwxXSxbMSwyXSxbMyw0XSxbNCw1XSxbNSw2XSxbNiw3XSxbMiw2XV0=
\[\begin{tikzcd}
	0 & {R_i} & {S_i} \\
	0 & {N_i} & {L(A_i)} & {A_i} & 0
	\arrow[from=1-1, to=1-2]
	\arrow[from=1-2, to=1-3]
	\arrow[from=1-3, to=2-4]
	\arrow[from=2-1, to=2-2]
	\arrow[from=2-2, to=2-3]
	\arrow[from=2-3, to=2-4]
	\arrow[from=2-4, to=2-5]
\end{tikzcd}\]
Pour utiliser celle du haut on remarque qu'on a 
$\pi_i\colon L(S_i)\to A_i$
surjection canonique et $L(R_i)\to L(S_i)$ injective, où plutôt le 
groupe engendré par $R_i$. On regarde dans $L(S_1\sqcup S_2)$
le groupe (libre) engendré par 

\[<R_1\sqcup R_2\sqcup \{a_1a_2^{-1}| (a_1,a_2)\in \pi_1^{-1}(i_1(c))\times \pi_2^{-1}(i_2(c)),~c\in C\}\]
En gros on a quotienté $L(S_1\sqcup S_2)$ plutôt que 
$L(A_1\sqcup A_2)$. On a quand même 
% https://q.uiver.app/#q=WzAsNSxbMSwwLCJMKFJfMVxcc3FjdXAgUl8yKSJdLFswLDAsIjAiXSxbMiwwLCJMKFNfMVxcc3FjdXAgU18yKSJdLFszLDAsIkwoR18xXFxzcWN1cCBHXzIpIl0sWzQsMCwiMCJdLFsxLDBdLFswLDJdLFsyLDNdLFszLDRdXQ==
\[\begin{tikzcd}
	0 & {L(R_1\sqcup R_2)} & {L(S_1\sqcup S_2)} & {L(G_1\sqcup G_2)} & 0
	\arrow[from=1-1, to=1-2]
	\arrow[from=1-2, to=1-3]
	\arrow[from=1-3, to=1-4]
	\arrow[from=1-4, to=1-5]
\end{tikzcd}\]

\section{Exemple}
Si je me suis pas trompé, on a via 
$\Z/20\Z\to \Z/10\Z;1\mapsto 1:=a$ et
$\Z/20\Z\to D_{10};1\mapsto r$ :
\[\Z/10\Z *_{\Z/20\Z} D_{10}=<a,r,s|a^{10}=1,r^5=1,s^2=1,srs=r^{-1}, ar^{-1}=1>\]
d'où $\Z/10\Z *_{\Z/20\Z} D_{10}=D_{10}$ à cause de la relation
$ar^{-1}=1$. J'ai mis que celle là parce que $i_1(i)=i(i_1(1))=a^i$
et $i_2(i)=i(i_2(1))=r^i$.

\section{Somme amalgamée générale}
Si on a une famille $(G_i)_{i\in I}$ et $A\to G_i$. ON peut former
$*_A G_i=: G$. Là le quotient à une forme spéciale, on peut
décrire assez bien ses éléments. On a quelques bijections, si
on note $S_i$ les classes à gauche de $A$ dans $G_i$. On
\[A\times S_i\to G_i\]
via $(a,s)\mapsto a.s$ une bijection.
\subsection{Mots réduits}
À chaque $s\in \sqcup S_i$ on peut associer $i_s$ l'unique
élément de $\{i|s\in S_i\}$. Alors une suite finie de la forme
$(a,s_1,s_2,\ldots,s_n)$
avec $s_j\in S_{i_{s_j}}-\{e\}$ et $i_{s_j}\ne i_{s_{j+1}}$ est
appelé réduit. En gros on peut regarder chaque lettre modulo $A$
et concentrer $A$ à l'avant. Et deux lettres adjacentes sont dans
des $S_i$ différents.

\subsection{Théorème}
On a que pour tout $g\in G$, $g$ a une forme réduite unique, i.e. 
$g=f(a)f_{i_{s_1}}(s_1)\ldots f_{i_{s_n}}(s_n)$
On identifie les $G_i$ à leurs images dans $G$. Si on note
$X$ l'ensemble des mots réduits 
On peut faire agir $G$ dessus. Y suffit de faire agir chaque
$G_i$ tel que l'action de $A$ ne dépende pas de $i$. On peut 
définir $Y_i:=\{(1,s_1,\ldots,s_n)|i_{s_1}\ne i\}$. On a 
\[A\times Y_i\to X\]
$(a,(1,s_1,\ldots,s_n))\mapsto (a,s_1,\ldots,s_n)$ et
\[A\times S_i-\{e\}\times Y_i\to X\]
via $(a,s,(1,s_1,\ldots,s_n))\mapsto (a,s,s_1,\ldots,s_n)$.
Le point c'est que $A\times Y_i$ c'est les mots réduits où
$i_{s_1}\ne i$ et $A\times S_i-\{e\}\times Y_i$ c'est les mots
réduits où $i_{s_1}=i$. Ducoup on a une bijection
\[A\times Y_i\cup A\times S_i-\{e\}\times Y_i\to X\]
et si on identifie $A\times Y_i$ et $A\times \{e\}\times Y_i$
le membre de gauche c'est $G_i\times Y_i$. Donc on a une bijection
\[G_i\times Y_i\to X\]
et on peut agir avec $G_i$ sur le truc de gauche via 
$g((a,s), (m))=(g.a.s,(m))$.
et si $g\in A$ $g(a,s)=(ga,s)$. D'où $G$ agit sur $X$.
\subsubsection{En résumé}
D'un mot de $X$, $m=(a,s_1,\ldots, s_n)$, et de 
$g=g_1\ldots g_n\in G$ on agit successivement sur les deux
premiers termes OU sur le premier terme. En gros si 
$i_{s_1}=i_{g_1}$ alors on calcule $(g_1.a.s_1)=(a',s_1')$ et
on réinjecte. Sinon on calcule $g_1.a=(a',s)$ et on réinjecte!
Vu que dans le deuxième cas $s_1\notin S_{i_{g_1}}=S_{i_s}$
on est bon. En plus, l'action de $G$ sur $X\subset G$ coincide
avec l'action de $G$ sur lui même. 

\subsection{Unicité}
Le point de l'action d'avant c'est de montrer qu'on a une section
$G\to X$. On a une flèche $X\to G$ donnée par 
\[(a,s_1,\ldots,s_n)\mapsto f(a)f_{i_{s_1}}(s_1)\ldots f_{i_{s_n}}(s_n)\]
et une section $G\to X$ donnée par $g\mapsto g((1;)$. 
On peut remarquer que si $m=as_1\ldots s_n$ et 
$g=f(a).f_{i_{s_1}}(s_1)\ldots f_{i_{s_n}}(s_n)$ alors $g(1;)=m$.
Pour le voir c'est pas dur, comme $i_{s_i}\ne i_{s_{i+1}}$
on est toujours dans le deuxième cas de l'action et on a des
représentants directement. 
\subsection{Existence}
D'après Serre come $X\to G$ est injective, on peut l'identifier
à son image. Ensuite suffit de remarquer que $G_i.X\subset X$.
D'où $G.X\subset X$ puis $G\subset X$ car $(1;)=1\in X$.
En fait c'est vraiment immédiat mdr, ptn c'est magique. Vu
autrement on peut remarquer que 
$f_{i_{g_1}}(g_1)\ldots f_{i_{g_n}}(g_n)$ se réecrit
\[f_{i_{g_1}}(g_1)\ldots f_{i_{g_n}}(a_ns_n)\] puis 
\[f_{i_{g_1}}(g_1)\ldots f_{i_{g_{n-1}}}(g_{n-1}a_n) f_{i_{g_n}}(s_n)\]
d'où par récurrence le résultat. En fait c'est la preuve de 
Serre dans le sens inverse un peu. On agit par $g_n=a_ns_n$ sur
$(1;)$ puis par $a_{n-1}s_{n-1}=g_{n-1}$ sur $(a_n,s_n)$, etc.. On
a bien une réecriture à chaque étape, celle de 
$a_{n-1}(s_{n_1}a_n)$.

\section{Description systématique}
Avec des nouvelles notations, $\bar i=(i_1,\ldots,i_n)$ vérifie
$(T)$ si $i_j\ne i_{j+1}$.
Serre propose une description systématique du processus de 
réduction. Si on note $G_i'=G_i-A$ en fait on peut agir sur
\[\prod_{j=1}^n G_{i_j}'\]
avec $A^{n-1}$ via 
\[(a_1,\ldots,a_{n-1})(g_1,\ldots, g_n)=(g_1a_1^{-1},a_1g_2a_2^{-1},\ldots,a_{n-1}g_n)\]
\begin{rem}
Par exemple avec $A^2$ sur $G_1'\times G_2'\times G_3'$ via
$(a_1,a_2)(g_1,g_2,g_3)=(g_1a_1^{-1},a_1g_2a_2^{-1},a_2g_3)$.
\end{rem}
Maintenant pour tout $\bar i$, je note $G_{\bar i}'$ le quotient
par l'action. On a 
\[f_{\bar i}\colon G_{\bar i}'\to G\]
le morphisme produit qui est passé automatiquement au quotient
est injectif vu que dans chaque classe d'équivalence y'a un mot
réduit (si $\bar i$ vérifie (T), c'est ça qu'il faut éclaircir) !
On obtient \[A\sqcup \bigsqcup_{\bar i\textrm{ vérifie (T)}} G_{\bar i}'\to G\]
vu que chaque mot à une longueure bien définie. Pour conclure
on peut comprendre qu'il y'a un mot réduit dans chaque classe
de l'action ca : $g_1\ldots g_n=a_1s_1\ldots a_ns_n$ puis
en notant $a_n=:a_{n-1}'$ 
on réecrit 
\begin{align*}
g_1\ldots g_n&=a_1(s_1a_1'^{-1})(g_2a_2'^{-1})\ldots (a_{n-1}'g_n)\\
	     &=a_1((s_1a_1'^{-1})\ldots (s_{n-1}a_{n-1}'^{-1}))s_n\\
\end{align*}
maintenant on agit sur le sous mot avec $A^{n-1}$ en mettant
$a_{n-1}=1$ puis par récurrence.

\subsection{Nouveau résumé}
Pour voir $G$ comme les mots réduits $X$ on fait agir $G$ sur
$X$ via $G_i$ sur 
\[G_i\times Y_i\]
ou on a deux types d'actions suivant si $i_{s_1}=i$.
Puis on a $G\to X$ donné par $g\mapsto g.(1;)$. D'où $X\subset G$.
Enfin, $G_i.X\subset X$ par définition puis $G.1\subset X$.

En plus la réduction de $g_1\ldots g_n$ via \[g_1\ldots a_n s_n=
g_1\ldots (g_{n-1}a_n)s_n=g_1\ldots(a_{n-1}s_{n-1})s_n=\ldots\]
peut se faire dans
\[G_{\bar i}':=(\prod G_{i_j}')/A^{|\bar i|-1}\]
avec $G_i':=G_i-A$. Donc on a une autre décomposition
\[A\sqcup \bigsqcup_{\bar i\in (T)}G_{\bar i}'\]
et là $G_{\bar i}'$ c'est les mots de type $\bar i$ vérifiant 
$(T)$, i.e. réduits, et $A$ c'est $(a;)$.

\section{Longueurs de mots et applications du théorème}
Nouvelle notation : si $g=as_1\ldots s_n$ alors $i_j(g):=i_{s_j}$.
Pour que ce soit bien déf de $\N\times G\to I$ faut ajouter 
$\{*\}$ à $I$ quand $j>l(g)$.
On remarque que la longueur de $g$ donnée par $l(g)=|\bar i|$ si
$g$ est de type $\bar i$ est bien déf !! Donc récurrence hehe.
Par exemple, $g\in G$ est cycliquement réduit si 
$i_1(g)!=i_{l(g)}(g)$. 
\subsection{Cycliquement réduit et ordre}
Si $g$ est cycliquement réduit alors $l(g^i)=i.l(g)$! D'où
d'ordre infini. Aussi, un $g$ quelconque est soit conjugué à un
élément de $G_i$ soit conjugué à un élément cycliquement réduit. 
Suffit de remarquer que 
$g_1^{-1}gg_1=s_1^{-1}a^{-1}(as_1\ldots s_n)a_1s_1$
est cycliquement réduit où de longueur $l(g)-1$ donc
récurrence. Penser au fait que $G_{\bar i}'$
est stable par réduction, pour voir qu'il suffit de regarder
$i_1(s_na_1)$! Ou $i_1(g_{n-1}g_1)$.

\begin{rem}
 Un élément d'ordre fini est donc conjugué à un élément d'un 
 $G_i$. Aussi, si les $G_i$ sont sans torsion, $G$ aussi par
 contraposée.
\end{rem}
\section{Sous somme amalgamées}
Étant donné une famille de sous-groupes $H_i\leq G_i$. Tels que
$B=H_i\cap A$ ne dépend pas de $i$ dans $G$. Alors, étant
donné un système de représentants $e\in T_i$ du quotient 
$H_i/B$ on peut l'étendre en $S_i$ de $G_i$! D'où les mots
réduits de $*_B H_i$ sont des mots réduits de $G$.
D'où $*_B H_i\to *_A G_i$ est injective.

En particulier si $H_i\cap A = \{1\}$ pour tout $i$,
$*_B H_i=* H_i$ est libre dans $G$. 

\subsection{Conséquence : produit libre et produit abélien libre.}
Étant donné $A*B\to A\times B$, son noyau est le groupe engendré
par les commutateurs $X:=\{[a:b]:=a^{-1}b^{-1}ab\}$ ET $X$ est
libre dans, je note $S$ le groupe. On peut remarque que pour
$a'\in A$ $a'^{-1}[a:b]a'=[aa':b][a':b]^{-1}=[aa':b][b:a']$
(calcul). D'où $S$ est normal dans $A*B$ (on fait pareil avec
$b\in B$). Maintenant $(A*B)/S$ est un groupe qui vérifie la
propriété universelle du produit $A\times B$. La liberté de $X$
est plus compliquée. Je regarderai plus tard. Faut simplement
scruter les mots. 

\subsection{Conséquence marrante}
Le produit libre de deux groupes finis contient un groupe libre
d'indice fini. Wah dit comme ça ca parait compliqué, en fait
$A*B$ contient $S$ qui est libre et $(A*B)/S = A\times B$ qui
est fini lol.




%\printbibliography
\end{document}

