\documentclass[a4paper,12pt]{book}
\usepackage{amsmath,  amsthm,enumerate}
\usepackage{csquotes}
\usepackage[provide=*,french]{babel}
\usepackage[dvipsnames]{xcolor}
\usepackage{quiver, tikz}

%symbole caligraphique
\usepackage{mathrsfs}

%hyperliens
\usepackage{hyperref}

%pseudo-code
\usepackage{algorithm}
\usepackage{algpseudocode}

\usepackage{fancyhdr}

\pagestyle{fancy}
\addtolength{\headwidth}{\marginparsep}
\addtolength{\headwidth}{\marginparwidth}
\renewcommand{\chaptermark}[1]{\markboth{#1}{}}
\renewcommand{\sectionmark}[1]{\markright{\thesection\ #1}}
\fancyhf{}
\fancyfoot[C]{\thepage}
\fancyhead[LO]{\textit \leftmark}
\fancyhead[RE]{\textit \rightmark}
\renewcommand{\headrulewidth}{0pt} % and the line
\fancypagestyle{plain}{%
    \fancyhead{} % get rid of headers
}

%bibliographie
\usepackage[
backend=biber,
style=alphabetic,
sorting=ynt
]{biblatex}

\addbibresource{bib.bib}

\usepackage{appendix}
\renewcommand{\appendixpagename}{Annexe}

\definecolor{wgrey}{RGB}{148, 38, 55}

\setlength\parindent{24pt}

\newcommand{\Z}{\mathbb{Z}}
\newcommand{\R}{\mathbb{R}}
\newcommand{\rel}{\omathcal{R}}
\newcommand{\Q}{\mathbb{Q}}
\newcommand{\C}{\mathbb{C}}
\newcommand{\N}{\mathbb{N}}
\newcommand{\K}{\mathbb{K}}
\newcommand{\A}{\mathbb{A}}
\newcommand{\B}{\mathcal{B}}
\newcommand{\Or}{\mathcal{O}}
\newcommand{\F}{\mathscr F}
\newcommand{\Hom}{\textrm{Hom}}
\newcommand{\disc}{\textrm{disc}}
\newcommand{\Pic}{\textrm{Pic}}
\newcommand{\End}{\textrm{End}}
\newcommand{\Spec}{\textrm{Spec}}
\newcommand{\Supp}{\textrm{Supp}}
\renewcommand{\Im}{\textrm{Im}}
\newcommand{\m}{\mathfrak{m}}
\renewcommand{\P}{\mathbb{P}}
\newcommand{\p}{\mathfrak{p}}


\newcommand{\cL}{\mathscr{L}}
\newcommand{\G}{\mathscr{G}}
\newcommand{\D}{\mathscr{D}}
\newcommand{\E}{\mathscr{E}}
\newcommand{\Po}{\mathscr{P}}
\renewcommand{\H}{\mathscr{H}}

\makeatletter
\newcommand{\colim@}[2]{%
  \vtop{\m@th\ialign{##\cr
    \hfil$#1\operator@font colim$\hfil\cr
    \noalign{\nointerlineskip\kern1.5\ex@}#2\cr
    \noalign{\nointerlineskip\kern-\ex@}\cr}}%
}
\newcommand{\colim}{%
  \mathop{\mathpalette\colim@{\rightarrowfill@\scriptscriptstyle}}\nmlimits@
}
\renewcommand{\varprojlim}{%
  \mathop{\mathpalette\varlim@{\leftarrowfill@\scriptscriptstyle}}\nmlimits@
}
\renewcommand{\varinjlim}{%
  \mathop{\mathpalette\varlim@{\rightarrowfill@\scriptscriptstyle}}\nmlimits@
}
\makeatother

\theoremstyle{plain}
\newtheorem{thm}[subsection]{Théoreme}
\newtheorem{lem}[subsection]{Lemme}
\newtheorem{prop}[subsection]{Proposition}
\newtheorem{cor}[subsection]{Corollaire}
\newtheorem{heur}{Heuristique}
\newtheorem{rem}{Remarque}
\newtheorem{note}{Note}

\theoremstyle{definition}
\newtheorem{conj}{Conjecture}
\newtheorem{prob}{Problème}
\newtheorem{quest}{Question}
\newtheorem{prot}{Protocole}
\newtheorem{algo}{Algorithme}
\newtheorem{defn}[subsection]{Définition}
\newtheorem{exmp}[subsection]{Exemples}
\newtheorem{exo}[subsection]{Exercices}
\newtheorem{ex}[subsection]{Exemple}
\newtheorem{exs}[subsection]{Exemples}
\newtheorem{res}{Résumé}
\newtheorem{rep}{Réponse}
\newtheorem{cons}{Conséquence}

\theoremstyle{remark}

\definecolor{wgrey}{RGB}{148, 38, 55}
\definecolor{wgreen}{RGB}{100, 200,0} 
\hypersetup{
    colorlinks=true,
    linkcolor=wgreen,
    urlcolor=wgrey,
    filecolor=wgrey
}

\title{Théorie de Galois et revêtements}
\date{}

\begin{document}
\maketitle
\tableofcontents
Quelques notes et notes de lecture sur le Douady!

\section{Clôture algébrique}
C'est sombre mdr dans le Douady. Y'a une construction explicite
dans le pdf de Benois par induction sur des gros anneaux de 
polynomes. En gros prendre $K[X_f]$


\section{Bases normales}
Représentations régulière isomorphe à
représentation de $Gal(L/K)$ naturelle,
i.e. dans $GL(L)$. Théorème de Krull-Schmidt
sur les modules indécomposables.

\chapter{Théorie de Galois des extensions de corps finies}
Y'a plusieurs points où j'suis pas au clair. Le
nombre de plongements et la séparabilité. Les
extensions successives et la séparabilité/normalité.

\section{Plongements et séparabilité}
Pour les problèmes de compatibilité juste toujours considérer les
morphismes induits de $s\colon \varphi\colon K\to L$ à
\[K[X]\to L[X]\]
$P\in K[X]$ a une racine dans $E$ veut dire $s(P)$ a une racine
dans $E$. Toute les notions sont alors relatives au surcorps.
Ou à isomorphisme près (pas unique (!)).
\subsection{Cas monogène}
Étant donné $F\to \Omega$ on a 
$|\Hom_F(F(\alpha),\Omega)|=\{\textrm{racines de }\mu_{\alpha,F}\}$
Via $F[X]\to \Omega$, $X\mapsto~\{racines\}$, et on passe au 
quotient.

\begin{rem}
  C'est là qu'on utilise $P$ irréductible. Ca explique que
  si on regarde $\hat L/\hat K$ via $L=K[\alpha]$ et $P=\mu_\alpha$
  sur $K$ alors le nombre de plongements qui étendent $K$,
  $\hat L\to (\hat K)^c$ est pas $[L:K]$, y faut un générateur.
\end{rem}


\subsection{Subtilité, p.q. pas un polynôme pas irréd}
Si $f$ est pas irréductible, ça fait pas sens de regarder le
corps engendré par une racine de $f$. Puisque les corps de facteurs
différents ont pas de liens ! $F[X]\to E$ passe au même quotient
en le même corps pour des racines distinctes du même pol irred!

\subsection{Corps de rupture/décomposition}
Y'a pas de subtilité c'est des corps de ruptures à la chaîne.

\subsection{Nombre de morphismes d'une extension algébrique finie
vers une extension}
On écrit $E=F(\alpha_1,\ldots,\alpha_k)$ et $f$ le produit des
polynômes minimaux sur $F$, étant donné $\Omega/F$ on cherche 
le nombre de plongement $E\to \Omega$. Pour simplifier on peut
supposer que $f$ split dans $\Omega$.

On trouve un premier plongement 
$\varphi_1\colon F[\alpha_1]\to \Omega$, puis le polynôme minimal
de $\alpha_2$ sur $\varphi_1(F[\alpha_1])$ divise $\varphi_1(f)$
donc split dans $\Omega$. Maintenant faut juste utiliser 
le cas monogène des prolongements entre :
\[F[\alpha_1,\alpha_2]\to\varphi_1(F[\alpha_1])[\beta_2]\]
avec $\beta_2$ une racine du pol min de $\alpha_2$ dans
$\varphi_1(F[\alpha_1])[X]$. 

À chaque étape, en notant $F_i=F[\alpha_1,\ldots,\alpha_i]$, on
obtient $\leq [F_{i+1}:F_i]$ nouveaux morphismes. Comme on 
construit le nouveau en fonction du précédent on prend le 
produit $\leq \prod [F_{i+1}:F_i]=[E:F]$. L'égalité dépend à
chaque étape du nombre de racines distinctes de
$\varphi_i(\mu_{\alpha_{i+1},F_i}(X))$ dans $\Omega$.


\begin{rem}
  Donc "l'extension" est cachée dans le morphisme $F_i[X]\to
  \varphi_i(F_i)[X]$. À droite on quotient par la racine et
  à gauche aussi.
\end{rem}
\subsection{En résumé}
Le cadre type c'est $F\to E=F[\alpha_1,\ldots,\alpha_k]$ et 
$F\to \Omega$. On cherche à comprendre $\Hom(E,\Omega)$ en fonction
des polynômes minimaux des $\alpha_i$ dans $F$.

La construction de $E$ peut-être et même est dûe soit à des corps
de ruptures successifs soit à l'adjonction d'éléments d'un 
surcorps (!).

Les morphismes/corps sont construits par passage au quotient de
$F[X]\to E$.

Il y'a une bijection entre les morphismes $F(\alpha)\to \Omega$
et les racines de $\mu_{\alpha,F}$ dans $\Omega$.

Les racines de $P\in F[X]$ dans un corps $L$ étant donné $F\to L$
veut dire étant donné le morphisme $F[X]\to L[X]$. Et c'est ce 
même morphisme qui cache "l'extension" de $F\to L$ à $F(\alpha)
\to L$. Penser 
\[0\to (\mu_{\alpha,F})\to F[X]\to L[X]/(X-\beta)=L\to 0\]
pour chaque racine $\beta$.

\section{$F$-automorphismes et séparabilité}
En combinant tout le reste. Dans $\Omega$ si on prends 
$E$ un corps de décomposition de $f\in F[X]$ séparable on a
$Aut_F(E)=[E:F]$.

Question : C'est si clair ? Lien dimension du splitting field et
nombre de racines ?

\noindent En fait c'est bien $[E:F]$ le nombre de plongements même
en prenant un polynôme de degré $[F(\alpha_1):F]$ le point c'est
vraiment que $\mu_{\alpha_1}/(X-\alpha_1)$ peut rester irréductible
donc on peut itérer.

\begin{rem}
  Y'a un petit détail à éclaircir, pourquoi c'est bien 
  multiplicatif à chaque étape le nombre de plongements ? C'est
  vraiment qu'on choisit d'envoyer $\alpha_i$ sur un élément
  spécifique à chaque étape. Ça définit uniquement le morphisme.
\end{rem}

\section{Extension normales, séparables, galoisiennes.}
Donc $E/F$ est galoisienne si elle est finie séparable et normale.
Y'a le théorème d'Artin qui dit que pour tout $G\subset Aut(E)$
fini
\[[E:E^G]\leq |G|\]
d'où $Aut(E/E^G)=G$ car $G\subset Aut(E/E^G)$. Et 
$Aut(|E/E^G)|)\leq [E:E^G]$. 
\subsection{Corps de décomposition et galois}
Toute extension galoisienne $E/F$ finie est donnée par le corps
de décomposition de $f\in F[X]$ séparable et $E/F$. On peut écrire
$E=F[\alpha_1,\ldots,\alpha_k]$ et $f$ le produit des polynômes
minimaux. Comme $E/F$ est normale, $f$ split, et comme les $f_i$
sont séparables, $f$ aussi. 

À l'inverse $E/E^G$ est galoisienne pour tout $G\subset Aut(E)$.
D'abord séparable, on peut regarder $f$ le polynôme minimal de 
$\alpha\in E$, on a $f=\prod(X-g\alpha)\in F[X]$ en prenant 
juste l'orbite (!) et celui de droite est séparable et split. 
La double divisibilité faut aussi voir que $f(g\alpha)=0$ pour
tout $g$.

\subsection{Théorème d'Artin}
L'extension $E/E^G$ est galoisienne de groupe de galois $G$
et $[E:F]=|G|$. Le fait que c'est galoisien c'est la sous section
d'avant, le groupe de galois c'est le lemme d'Artin, et la 
dimension c'est le nombre de plongements vu que c'est séparable.

\subsection{Clôture galoisienne et transitivité}
Quand on a une extension séparable on peut split tout les polynômes
minimaux, c'est galoisien via avant. 

En plus si $E/M/F$ et $E/F$ est galoisienne, $E/M$ aussi via
$F[X]\subset M[X]$ est les corps de décompositions nécessitent pas
de polynômes irréductibles.

\subsection{Résumé}
Le lemme d'Artin dit que $[E:E^G]=|G|$, elle est galoisienne
en construisant des polynômes irréductibles via $\prod(X-g\alpha)$.
On peut en faire un corps de décomposition en splittant les pols
mins des générateurs de $E=E^G(\alpha_1,\ldots,\alpha_k)$ via
les orbites.

Inversement $F=E^{Aut(E/F)}$, et la propriété galoisienne découle
directement. 

Donc pour résumer quand on a $E/F$ finie, on peut écrire 
$E=F[\alpha_1,\ldots,\alpha_k]$. Si on a suffisamment 
d'automorphismes, on peut split les polynômes minimaux et en faire
une extension normale (orbites suffisamment grande) ET séparable
(automorphismes distincts). Sinon on peut passer au corps de 
décomposition.

Penser à $Aut(E/F)$ comme $\Hom_F(E,E)$ pour compter les 
automorphismes comme des plongements. 

\section{Correspondance de Galois}
Comme $Gal(E/E^G)=G$ la correspondance est bien bijective. Via
le théorème d'Artin, $[E:E^G]=(G:1)$ d'où 
\[[E:E^{H_1}][E^{H_1}:E^{H_2}]=[E:E^{H_2}]=(H_1:1)(H_1:H_2)=(H_2:1)\]
enfin les sous-corps isomorphes de $E$ ont des groupes de Galois
conjugués d'où un corps correspondant à un sous-groupe distingué
est invariant par $G$ d'où $g\mapsto g|_{E^H}$ est bien défini
de noyau $H$.

\subsection{Remarques}
Le foncteur est contravariant pour l'inclusion, d'où le compositum
correspond à l'intersection (!). Si $N=\cap gHg^{-1}$ est le plus
petit sous-groupe normal de $H$ dans $G$, alors $E^N$ est la 
clôture galoisienne de $E^H/E^G$.

\section{Résumé général}
Étant donné $E/F$ finie, on a toujours 
$E=F[\alpha_1,\ldots,\alpha_k]$. On a aussi $F[X]\to E[X]$.




\printbibliography
\end{document}

