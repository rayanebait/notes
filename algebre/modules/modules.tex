\documentclass[a4paper,12pt]{book}
\usepackage{amsmath,  amsthm,enumerate}
\usepackage{csquotes}
\usepackage[provide=*,french]{babel}
\usepackage[dvipsnames]{xcolor}
\usepackage{quiver, tikz}

%symbole caligraphique
\usepackage{mathrsfs}

%hyperliens
\usepackage{hyperref}

%pseudo-code
\usepackage{algorithm}
\usepackage{algpseudocode}

\usepackage{fancyhdr}

\pagestyle{fancy}
\addtolength{\headwidth}{\marginparsep}
\addtolength{\headwidth}{\marginparwidth}
\renewcommand{\chaptermark}[1]{\markboth{#1}{}}
\renewcommand{\sectionmark}[1]{\markright{\thesection\ #1}}
\fancyhf{}
\fancyfoot[C]{\thepage}
\fancyhead[LO]{\textit \leftmark}
\fancyhead[RE]{\textit \rightmark}
\renewcommand{\headrulewidth}{0pt} % and the line
\fancypagestyle{plain}{%
    \fancyhead{} % get rid of headers
}

%bibliographie
\usepackage[
backend=biber,
style=alphabetic,
sorting=ynt
]{biblatex}

\addbibresource{bib.bib}

\usepackage{appendix}
\renewcommand{\appendixpagename}{Annexe}

\definecolor{wgrey}{RGB}{148, 38, 55}

\setlength\parindent{24pt}

\newcommand{\Z}{\mathbb{Z}}
\newcommand{\R}{\mathbb{R}}
\newcommand{\rel}{\omathcal{R}}
\newcommand{\Q}{\mathbb{Q}}
\newcommand{\C}{\mathbb{C}}
\newcommand{\N}{\mathbb{N}}
\newcommand{\K}{\mathbb{K}}
\newcommand{\A}{\mathbb{A}}
\newcommand{\B}{\mathcal{B}}
\newcommand{\Or}{\mathcal{O}}
\newcommand{\F}{\mathscr F}
\newcommand{\Hom}{\textrm{Hom}}
\newcommand{\disc}{\textrm{disc}}
\newcommand{\Pic}{\textrm{Pic}}
\newcommand{\End}{\textrm{End}}
\newcommand{\Spec}{\textrm{Spec}}
\newcommand{\Supp}{\textrm{Supp}}
\renewcommand{\Im}{\textrm{Im}}
\newcommand{\m}{\mathfrak{m}}
\renewcommand{\P}{\mathbb{P}}
\newcommand{\p}{\mathfrak{p}}


\newcommand{\cL}{\mathscr{L}}
\newcommand{\G}{\mathscr{G}}
\newcommand{\D}{\mathscr{D}}
\newcommand{\E}{\mathscr{E}}
\newcommand{\Po}{\mathscr{P}}
\renewcommand{\H}{\mathscr{H}}

\makeatletter
\newcommand{\colim@}[2]{%
  \vtop{\m@th\ialign{##\cr
    \hfil$#1\operator@font colim$\hfil\cr
    \noalign{\nointerlineskip\kern1.5\ex@}#2\cr
    \noalign{\nointerlineskip\kern-\ex@}\cr}}%
}
\newcommand{\colim}{%
  \mathop{\mathpalette\colim@{\rightarrowfill@\scriptscriptstyle}}\nmlimits@
}
\renewcommand{\varprojlim}{%
  \mathop{\mathpalette\varlim@{\leftarrowfill@\scriptscriptstyle}}\nmlimits@
}
\renewcommand{\varinjlim}{%
  \mathop{\mathpalette\varlim@{\rightarrowfill@\scriptscriptstyle}}\nmlimits@
}
\makeatother

\theoremstyle{plain}
\newtheorem{thm}[subsection]{Théoreme}
\newtheorem{lem}[subsection]{Lemme}
\newtheorem{prop}[subsection]{Proposition}
\newtheorem{cor}[subsection]{Corollaire}
\newtheorem{heur}{Heuristique}
\newtheorem{rem}{Remarque}
\newtheorem{note}{Note}

\theoremstyle{definition}
\newtheorem{conj}{Conjecture}
\newtheorem{prob}{Problème}
\newtheorem{quest}{Question}
\newtheorem{prot}{Protocole}
\newtheorem{algo}{Algorithme}
\newtheorem{defn}[subsection]{Définition}
\newtheorem{exmp}[subsection]{Exemples}
\newtheorem{exo}[subsection]{Exercices}
\newtheorem{ex}[subsection]{Exemple}
\newtheorem{exs}[subsection]{Exemples}
\newtheorem{res}{Résumé}
\newtheorem{rep}{Réponse}
\newtheorem{cons}{Conséquence}

\theoremstyle{remark}

\definecolor{wgrey}{RGB}{148, 38, 55}
\definecolor{wgreen}{RGB}{100, 200,0} 
\hypersetup{
    colorlinks=true,
    linkcolor=wgreen,
    urlcolor=wgrey,
    filecolor=wgrey
}

\title{Modules}
\date{}

\begin{document}
\maketitle
\tableofcontents
\chapter{Algèbre linéaire sur les anneaux à divisions, $D$ et $D^{op}$}
\href{https://math.stackexchange.com/questions/45056/linear-algebra-over-a-division-ring-vs-over-a-field}{Une discussion}

sur les différences entre modules sur un corps et un anneau à division non
commutatif.
\section{Endomorphismes et la distinction $D$ et $D^{op}$}
\subsection{Action de $D^{op}$ sur $D$}
De manière générale, $D$ agit sur lui même à gauche via 
\[d\mapsto(\varphi_d\colon a\mapsto da)\]
mais l'action est $D$-linéaire si et seulement si $D$ est commutatif! Parce
que $\phi_d(d'a)=dd'a!=d'\phi_d(a)=d'da$. Par contre, $D^{op}$ agit à droite
$D$-linéairement : deux trucs à décrire,  $D^{op}$ en tant que $D$-module
et traduire la linéarité.

En conséquence,
\begin{thm}
	Si $D$ est une algèbre à division, $\End_D(D)\simeq D^{op}$. D'où
	$\End_D(D^n)\simeq M_n(D^{op})$.
\end{thm}
Ensuite,
\begin{prop}[Il existe $D\neq D^{op}$]
	Il existe des algèbres à division $D$ non-isomorphes à leur opposé.
\end{prop}
\begin{ex}
	Pour un corps $F$, $M_n(F)\simeq M_n(F)^{op}$ via $A\mapsto A^t$. D'après
	ce \href{https://math.stackexchange.com/questions/45085/an-example-of-a-division-ring-d-that-is-not-isomorphic-to-its-opposite-rin?noredirect=1&lq=1}{post}
	en tant qu'algèbre, il suffit de trouver une algèbre d'ordre $2<$ dans
	le groupe de Brauer d'un corps $F$ pour avoir un contre exemple (parce 
	que l'inverse est $D^{op}$).Une algèbre cyclique d'ordre $3$ fait l'affaire.
	Mais d'après une des réponses et cet 
	\href{Division algebras with an anti-automorphism but with no involut
	ion}{article} en tant qu'anneaux c'est moins facile. Mais pas besoin
	d'aller chercher dans des algèbres d'Azumaya. Le groupe de Brauer
	de $\Q$ est suffisamment compliqué, et dans ce cas pour toutes
	les algèbres à division sur $\Q$, les morphismes d'anneaux sont des
	morphismes de $\Q$-algèbres. 
\end{ex}
Ce serait marrant d'investiguer plus en détail l'algèbre opposée.

\section{}


\chapter{Comparaisons de modules sur anneaux commutatifs, non commutatifs}
\section{Anneaux principaux}
En suivant \href{https://kconrad.math.uconn.edu/blurbs/linmultialg/modulesoverPID.pdf}{K. Conrad}

Dans le cas commutatif (pids), on a
\begin{lem}[Le rang est bien défini]
	Si $A$ est un pid commutatif, $A^m\simeq A^n$ en tant que $A$-modules 
	implique $m=n$.
\end{lem}
\begin{proof}
	Suffit de prendre un idéal maximal $\m$ et passer au quotient.
\end{proof}
Dans le cas non commutatif $\m$ doit être maximal à gauche et à droite 
et bilatère pour avoir un anneau quotient.








\printbibliography
\end{document}

