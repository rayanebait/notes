\documentclass[a4paper,12pt]{book}
\usepackage{amsmath,  amsthm,enumerate}
\usepackage{csquotes}
\usepackage[provide=*,french]{babel}
\usepackage[dvipsnames]{xcolor}
\usepackage{quiver, tikz}

%symbole caligraphique
\usepackage{mathrsfs}

%hyperliens
\usepackage{hyperref}

%pseudo-code
\usepackage{algpseudocode}
\usepackage{algorithm}
\makeatletter
  \renewcommand{\ALG@name}{Algorithme}
  \makeatother
\usepackage{fancyhdr}

\pagestyle{fancy}
\addtolength{\headwidth}{\marginparsep}
\addtolength{\headwidth}{\marginparwidth}
\renewcommand{\chaptermark}[1]{\markboth{#1}{}}
\renewcommand{\sectionmark}[1]{\markright{\thesection\ #1}}
\fancyhf{}
\fancyfoot[C]{\thepage}
\fancyhead[LO]{\textit \leftmark}
\fancyhead[RE]{\textit \rightmark}
\renewcommand{\headrulewidth}{0pt} % and the line
\fancypagestyle{plain}{%
    \fancyhead{} % get rid of headers
}

%bibliographie
\usepackage[
backend=biber,
style=alphabetic,
sorting=ynt
]{biblatex}

\addbibresource{bib.bib}

\usepackage{appendix}
\renewcommand{\appendixpagename}{Annexe}

\definecolor{wgrey}{RGB}{148, 38, 55}

\setlength\parindent{24pt}

\newcommand{\Z}{\mathbb{Z}}
\newcommand{\R}{\mathbb{R}}
\newcommand{\rel}{\omathcal{R}}
\newcommand{\Q}{\mathbb{Q}}
\newcommand{\C}{\mathbb{C}}
\newcommand{\N}{\mathbb{N}}
\newcommand{\K}{\mathbb{K}}
\newcommand{\A}{\mathbb{A}}
\newcommand{\B}{\mathcal{B}}
\newcommand{\Or}{\mathcal{O}}
\newcommand{\F}{\mathbb F}
\newcommand{\m}{\mathfrak m}
\renewcommand{\b}{\mathfrak b}
\renewcommand{\a}{\mathfrak a}
\newcommand{\p}{\mathfrak p}
\newcommand{\I}{\mathfrak I}
\newcommand{\Hom}{\textrm{Hom}}
\newcommand{\disc}{\textrm{disc}}
\newcommand{\Pic}{\textrm{Pic}}
\newcommand{\End}{\textrm{End}}
\newcommand{\Spec}{\textrm{Spec}}
\newcommand{\Frac}{\textrm{Frac}}

\newcommand{\cL}{\mathscr{L}}
\newcommand{\G}{\mathscr{G}}
\newcommand{\D}{\mathscr{D}}
\newcommand{\E}{\mathscr{E}}
\newcommand{\U}{\mathscr{U}}

\theoremstyle{plain}
\newtheorem{thm}{Théoreme}
\newtheorem{lem}{Lemme}
\newtheorem{prop}{Proposition}
\newtheorem{cor}{Corollaire}
\newtheorem{heur}{Heuristique}
\newtheorem{rem}{Remarque}
\newtheorem{rembis}{Remarque}
\newtheorem{note}{Note}

\theoremstyle{definition}
\newtheorem{conj}{Conjecture}
\newtheorem*{eq}{Équivalences}
\newtheorem{prob}{Problème}
\newtheorem{quest}{Question}
\newtheorem{prot}{Protocole}
\newtheorem{algo}{Algorithme}
\newtheorem{defn}{Définition}
\newtheorem{defnbis}{Définition}
\newtheorem{ex}{Exemple}
\newtheorem{exo}{Exercices}

\theoremstyle{remark}

\definecolor{wgrey}{RGB}{148, 38, 55}
\definecolor{wgreen}{RGB}{100, 200,0} 
\hypersetup{
    colorlinks=true,
    linkcolor=wgreen,
    urlcolor=wgrey,
    filecolor=wgrey
}

\title{Surfaces de Riemann}
\date{}

\begin{document}
\maketitle

\section{Motivation topologique}
Pour expliquer l'idée des revêtements ramifiés. Un
super exemple en fait assez général c'est les surfaces
de Riemann. On peut construire les revêtements via des
fonctions à valeurs multiples obtenues via des fonctions
inversibles sur des branches, genre logarithme ou
exponentielle ou polynômes. 

\subsection{Polynômes, logarithmes et exponentielles.}
La fonction $z\mapsto z^2$ est un super exemple. On 
peut étendre la fonction à $S^2=\C\cup \infty$ séparer 
$S^2$ en deux selon l'axe imaginaire. Dans ce cas
si $C_1$ est la partie à gauche et $C^2$ la partie
à droite. Les mettre au carré recouvre $S^2$. 

L'axe de symétrie étant donné par les points fixes de
$z\mapsto -z$, il va falloir recoller sur cet axe.
En fait on a : Faire dessin.


Le truc maintenant c'est que à gauche on peut prendre la
racine carrée comme étant $-z^{1/2}$ et à droite comme
$z^{1/2}$ en posant $z^{1/2}$ la branche principale.

À noter que chaque demi-sphère est pas stable par la
racine carrée, mais sur la surface de Riemann c'est
bon.

AUSSI, la surface de Riemann si on prendre un lacet
autour de $\infty$ sur $S^2$, celui correspond sur
la surface de riemann est que la "moitié" de ce qu'il
devrait être, ça fait un lemniscate en fait faut
faire 2 fois le tour.

Enfin, on peut voir la surface via le recollement de
$ab^{-1}$ et $ab^{-1}$. En recollant selon $b^{-1}$
on obtient $aa^{-1}$. L'orientation est renversée,
voir pq.
\end{document}
