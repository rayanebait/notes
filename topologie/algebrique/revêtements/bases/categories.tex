\documentclass[a4paper,12pt]{article}
\usepackage{amsmath,  amsthm,enumerate}
\usepackage{csquotes}
\usepackage[provide=*,french]{babel}
\usepackage[dvipsnames]{xcolor}
\usepackage{quiver, tikz}

%symbole caligraphique
\usepackage{mathrsfs}

%hyperliens
\usepackage{hyperref}

%pseudo-code
\usepackage{algorithm}
\usepackage{algpseudocode}


%bibliographie
\usepackage[
backend=biber,
style=alphabetic,
sorting=ynt
]{biblatex}

\addbibresource{bib.bib}


\definecolor{wgrey}{RGB}{148, 38, 55}

\setlength\parindent{24pt}

\newcommand{\Z}{\mathbb{Z}}
\newcommand{\R}{\mathbb{R}}
\newcommand{\rel}{\omathcal{R}}
\newcommand{\Q}{\mathbb{Q}}
\newcommand{\C}{\mathbb{C}}
\newcommand{\Cat}{\mathcal{C}}
\newcommand{\Dat}{\mathcal{D}}
\newcommand{\Aat}{\mathcal{A}}
\newcommand{\N}{\mathbb{N}}
\newcommand{\K}{\mathbb{K}}
\newcommand{\A}{\mathbb{A}}
\newcommand{\B}{\mathcal{B}}
\newcommand{\Or}{\mathcal{O}}
\newcommand{\F}{\mathscr F}
\newcommand{\Hom}{\textrm{Hom}}
\newcommand{\disc}{\textrm{disc}}
\newcommand{\Pic}{\textrm{Pic}}
\newcommand{\End}{\textrm{End}}
\newcommand{\Spec}{\textrm{Spec}}
\newcommand{\Supp}{\textrm{Supp}}
\newcommand{\Ouv}{\textrm{Ouv}}
\newcommand{\im}{\textrm{im}}
\newcommand{\coker}{\textrm{coker}}
\newcommand{\coim}{\textrm{coim}}


\newcommand{\cL}{\mathscr{L}}
\newcommand{\G}{\mathscr{G}}
\newcommand{\D}{\mathscr{D}}
\newcommand{\E}{\mathscr{E}}
\renewcommand{\P}{\mathscr{P}}
\renewcommand{\H}{\mathscr{H}}

\makeatletter
\newcommand{\colim@}[2]{%
  \vtop{\m@th\ialign{##\cr
    \hfil$#1\operator@font colim$\hfil\cr
    \noalign{\nointerlineskip\kern1.5\ex@}#2\cr
    \noalign{\nointerlineskip\kern-\ex@}\cr}}%
}
\newcommand{\colim}{%
  \mathop{\mathpalette\colim@{\rightarrowfill@\scriptscriptstyle}}\nmlimits@
}
\renewcommand{\varprojlim}{%
  \mathop{\mathpalette\varlim@{\leftarrowfill@\scriptscriptstyle}}\nmlimits@
}
\renewcommand{\varinjlim}{%
  \mathop{\mathpalette\varlim@{\rightarrowfill@\scriptscriptstyle}}\nmlimits@
}
\makeatother

\theoremstyle{plain}
\newtheorem{thm}{Théoreme}
\newtheorem{lem}{Lemme}
\newtheorem{prop}{Proposition}
\newtheorem{cor}{Corollaire}
\newtheorem{heur}{Heuristique}
\newtheorem{rem}{Remarque}
\newtheorem{note}{Note}

\theoremstyle{definition}
\newtheorem{conj}{Conjecture}
\newtheorem{prob}{Problème}
\newtheorem{quest}{Question}
\newtheorem{prot}{Protocole}
\newtheorem{algo}{Algorithme}
\newtheorem{defn}{Définition}
\newtheorem{exmp}{Exemples}
\newtheorem{exo}{Exercices}
\newtheorem{ex}{Exemple}
\newtheorem{exs}{Exemples}

\theoremstyle{remark}

\definecolor{wgrey}{RGB}{148, 38, 55}
\definecolor{wgreen}{RGB}{100, 200,0} 
\hypersetup{
    colorlinks=true,
    linkcolor=wgreen,
    urlcolor=wgrey,
    filecolor=wgrey
}

\title{Propriétés de base}
\date{2024-2025}

\begin{document}
\maketitle
\tableofcontents

\section{Revêtements}
Un revêtement de $X$ est la donnée d'un ensemble $F$, 
et pour chaque $x\in X$ de l'existence d'un ouvert 
$x\in U\subset X$ d'un diagramme commutatif 
% https://q.uiver.app/#q=WzAsNSxbMywwLCJcXHdpZGV0aWxkZSBYIl0sWzMsMSwiWCJdLFsyLDAsInBeey0xfVUiXSxbMiwxLCJVIl0sWzAsMCwiVVxcdGltZXMgRiJdLFswLDEsInAiXSxbMiwwLCIiLDAseyJzdHlsZSI6eyJ0YWlsIjp7Im5hbWUiOiJob29rIiwic2lkZSI6InRvcCJ9fX1dLFsyLDMsInB8X3twXnstMX1VfSIsMl0sWzQsMiwiXFxzaW1lcSIsMSx7InN0eWxlIjp7ImJvZHkiOnsibmFtZSI6Im5vbmUifSwiaGVhZCI6eyJuYW1lIjoibm9uZSJ9fX1dLFs0LDNdLFszLDEsIiIsMCx7InN0eWxlIjp7InRhaWwiOnsibmFtZSI6Imhvb2siLCJzaWRlIjoidG9wIn19fV1d
\[\begin{tikzcd}
	{U\times F} && {p^{-1}U} & {\widetilde X} \\
	&& U & X
	\arrow["\simeq"{description}, draw=none, from=1-1, to=1-3]
	\arrow[from=1-1, to=2-3]
	\arrow[hook, from=1-3, to=1-4]
	\arrow["{p|_{p^{-1}U}}"', from=1-3, to=2-3]
	\arrow["p", from=1-4, to=2-4]
	\arrow[hook, from=2-3, to=2-4]
\end{tikzcd}\]
tel que $x\in U$ dans $Top$ où $F$ est discret. Ducoup
$U$ \textbf{trivialise} le revêtement.

Je note $U_x$ un ouvert trivialisant qui contient $x$.

\subsection{Degré}
Le degré est donné par 
\[\deg\colon X\to \N\cup\infty\]
via $\deg(x):=\#p^{-1}x$. Avec la topologie discrète sur
$\N$ le degré est continu. En général c'est localement
constant ducoup et si $X$ est connexe le degré
est constant.

\begin{rem}
  La continuité c'est que pour $x$ tel que $\deg(x)=n$
  alors $U_x\subset \deg^{-1}n$. En fait 
  $\deg^{-1}n=\cup_{\deg(y)=n}U_y$.
\end{rem}


\section{Relèvement des homotopies}
Étant donné une homotopie $F\colon Y\times I\to X$ et
un relèvement $\widetilde F_0\colon Y\times\{0\}\to \widetilde X$
y'a un unique relèvement 
% https://q.uiver.app/#q=WzAsNCxbMiwwLCJcXHdpZGV0aWxkZSBYIl0sWzIsMSwiWCJdLFswLDEsIllcXHRpbWVzIEkiXSxbMCwwLCJZXFx0aW1lcyBcXHswXFx9Il0sWzAsMSwicCJdLFsyLDEsIkYiLDJdLFszLDIsIiIsMix7InN0eWxlIjp7InRhaWwiOnsibmFtZSI6Imhvb2siLCJzaWRlIjoidG9wIn19fV0sWzMsMCwiXFx3aWRldGlsZGUgRnxfe1lcXHRpbWVzXFx7MFxcfX0iXSxbMiwwLCIiLDEseyJzdHlsZSI6eyJib2R5Ijp7Im5hbWUiOiJkYXNoZWQifX19XV0=
\[\begin{tikzcd}
	{Y\times \{0\}} && {\widetilde X} \\
	{Y\times I} && X
	\arrow["{\widetilde F|_{Y\times\{0\}}}", from=1-1, to=1-3]
	\arrow[hook, from=1-1, to=2-1]
	\arrow["p", from=1-3, to=2-3]
	\arrow[dashed, from=2-1, to=1-3]
	\arrow["F"', from=2-1, to=2-3]
\end{tikzcd}\]
Je fais un outline de la preuve et un peu de détail
après.
\subsection{Outline de la preuve}
Ça se fait comme ça :
\begin{proof}
  Y'a $3$ étapes, pour la première : à chaque $y$ on peut
  attacher un voisinage $N$ et une partition
  $0=t_0< \ldots< t_n=1$ tels que pour 
  chaque $i$, $F(N\times [t_i,t_{i+1}])$ est dans
  un ouvert trivialisant avec 
  $p\colon (\tilde U_i)\simeq U_i$
  et $\tilde F(y_0,t_i)\in \tilde U_i$ qui fait le
  lien (je parle après de la condition initiale)
  maintenant le lift est calculé via 
  \[F|_{N\times [t_i,t_{i+1}]}\circ p^{-1}\colon N\times [t_i,t_{i+1}]\to U_i\to \tilde U_i\]
  
  Pour la deuxième étape (unicité sur $N\times I$), on
  regarde juste le lift en chaque $\{y\}\times I$ et le
  fait que $\tilde F(\{y\}\times [t_i,t_{i+1}])$ 
  est connexe donc dans un seul $\tilde U_i$. En 
  particulier la dessus $p$ est injective un 
  monomorphisme d'où unicité.

  Pour la dernière, les lifts locaux sont uniques sur
  chaque $\{y\}\times I$ donc partout. La continuité
  est claire.
\end{proof}
\subsection{Détail}
Pour lifter sur un $\{y\}\times I$, il faut un choisir
un point dans $\tilde x \in p^{-1}(F(y,t))$, pour
pouvoir choisir la copie de $\tilde U_i$.
Par compacité de $\{y\}\times I$, on peut
partitionner $I$ en $\cup_{i=0}^n [t_i,t_{i+1}]$ de 
telle sorte que si $U$ est trivialisant pour $y$ et
en restreignant $p$ à $\tilde U\ni \tilde x$ alors 
$F\circ p^{-1}$ est un lift unique qui se recoupe aux
$t_i$.

\subsection{Relèvement des chemins et homotopies de chemins}
Y'a le joli résultat maintenant : étant donné un point 
base $x_0\in X$
et un lacet $\gamma$ dans $\pi_1(X,x_0)$, pour tout 
$\tilde x_0\in p^{-1}x_0$,
\begin{itemize}
  \item Il existe un unique relèvement de $\gamma$, $\tilde \gamma$
    t.q $\tilde\gamma(0)=\tilde x_0$.
\end{itemize}
Ensuite pour toute homotopie de chemins ($f_t(1)=f_t(0)$
pour tout $t$), $f_t\colon I\to Y$ de $\gamma$ à
$\gamma'$ on peut les relever uniquement en 
$\tilde \gamma$ et $\tilde \gamma'$ commençant en 
$\tilde x_0$. Et maintenant 
\begin{itemize}
  \item Il existe un unique relèvement de $f_t$, 
    $\tilde f_t$ entre $\tilde\gamma$ et $\tilde\gamma'$.
\end{itemize}
La preuve est un corollaire direct de la section d'avant
ducoup.

\begin{rem}
  Le point clé à remarquer c'est le choix du point
  $\tilde x_0\in p^{-1}(x_0)$.
\end{rem}


%\printbibliography
\end{document}

