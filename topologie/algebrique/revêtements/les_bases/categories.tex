\documentclass[a4paper,12pt]{article}
\usepackage{amsmath,  amsthm,enumerate}
\usepackage{csquotes}
\usepackage[provide=*,french]{babel}
\usepackage[dvipsnames]{xcolor}
\usepackage{quiver, tikz}

%symbole caligraphique
\usepackage{mathrsfs}

%hyperliens
\usepackage{hyperref}

%pseudo-code
\usepackage{algorithm}
\usepackage{algpseudocode}


%bibliographie
\usepackage[
backend=biber,
style=alphabetic,
sorting=ynt
]{biblatex}

\addbibresource{bib.bib}


\definecolor{wgrey}{RGB}{148, 38, 55}

\setlength\parindent{24pt}

\newcommand{\Z}{\mathbb{Z}}
\newcommand{\R}{\mathbb{R}}
\newcommand{\rel}{\omathcal{R}}
\newcommand{\Q}{\mathbb{Q}}
\newcommand{\C}{\mathbb{C}}
\newcommand{\Cat}{\mathcal{C}}
\newcommand{\Dat}{\mathcal{D}}
\newcommand{\Aat}{\mathcal{A}}
\newcommand{\N}{\mathbb{N}}
\newcommand{\K}{\mathbb{K}}
\newcommand{\A}{\mathbb{A}}
\newcommand{\B}{\mathcal{B}}
\newcommand{\Or}{\mathcal{O}}
\newcommand{\F}{\mathscr F}
\newcommand{\Hom}{\textrm{Hom}}
\newcommand{\disc}{\textrm{disc}}
\newcommand{\Pic}{\textrm{Pic}}
\newcommand{\End}{\textrm{End}}
\newcommand{\Spec}{\textrm{Spec}}
\newcommand{\Supp}{\textrm{Supp}}
\newcommand{\Ouv}{\textrm{Ouv}}
\newcommand{\im}{\textrm{im}}
\newcommand{\coker}{\textrm{coker}}
\newcommand{\coim}{\textrm{coim}}


\newcommand{\cL}{\mathscr{L}}
\newcommand{\G}{\mathscr{G}}
\newcommand{\D}{\mathscr{D}}
\newcommand{\E}{\mathscr{E}}
\renewcommand{\P}{\mathscr{P}}
\renewcommand{\H}{\mathscr{H}}

\makeatletter
\newcommand{\colim@}[2]{%
  \vtop{\m@th\ialign{##\cr
    \hfil$#1\operator@font colim$\hfil\cr
    \noalign{\nointerlineskip\kern1.5\ex@}#2\cr
    \noalign{\nointerlineskip\kern-\ex@}\cr}}%
}
\newcommand{\colim}{%
  \mathop{\mathpalette\colim@{\rightarrowfill@\scriptscriptstyle}}\nmlimits@
}
\renewcommand{\varprojlim}{%
  \mathop{\mathpalette\varlim@{\leftarrowfill@\scriptscriptstyle}}\nmlimits@
}
\renewcommand{\varinjlim}{%
  \mathop{\mathpalette\varlim@{\rightarrowfill@\scriptscriptstyle}}\nmlimits@
}
\makeatother

\theoremstyle{plain}
\newtheorem{thm}{Théoreme}
\newtheorem{lem}{Lemme}
\newtheorem{prop}{Proposition}
\newtheorem{cor}{Corollaire}
\newtheorem{heur}{Heuristique}
\newtheorem{rem}{Remarque}
\newtheorem{note}{Note}

\theoremstyle{definition}
\newtheorem{conj}{Conjecture}
\newtheorem{prob}{Problème}
\newtheorem{quest}{Question}
\newtheorem{prot}{Protocole}
\newtheorem{algo}{Algorithme}
\newtheorem{defn}{Définition}
\newtheorem{exmp}{Exemples}
\newtheorem{exo}{Exercices}
\newtheorem{ex}{Exemple}
\newtheorem{exs}{Exemples}

\theoremstyle{remark}

\definecolor{wgrey}{RGB}{148, 38, 55}
\definecolor{wgreen}{RGB}{100, 200,0} 
\hypersetup{
    colorlinks=true,
    linkcolor=wgreen,
    urlcolor=wgrey,
    filecolor=wgrey
}

\title{Propriétés de base}
\date{2024-2025}

\begin{document}
\maketitle
\tableofcontents

\section{Revêtements}
Un revêtement de $X$ est la donnée d'un ensemble $F$, 
et pour chaque $x\in X$ de l'existence d'un ouvert 
$x\in U\subset X$ d'un diagramme commutatif 
% https://q.uiver.app/#q=WzAsNSxbMywwLCJcXHdpZGV0aWxkZSBYIl0sWzMsMSwiWCJdLFsyLDAsInBeey0xfVUiXSxbMiwxLCJVIl0sWzAsMCwiVVxcdGltZXMgRiJdLFswLDEsInAiXSxbMiwwLCIiLDAseyJzdHlsZSI6eyJ0YWlsIjp7Im5hbWUiOiJob29rIiwic2lkZSI6InRvcCJ9fX1dLFsyLDMsInB8X3twXnstMX1VfSIsMl0sWzQsMiwiXFxzaW1lcSIsMSx7InN0eWxlIjp7ImJvZHkiOnsibmFtZSI6Im5vbmUifSwiaGVhZCI6eyJuYW1lIjoibm9uZSJ9fX1dLFs0LDNdLFszLDEsIiIsMCx7InN0eWxlIjp7InRhaWwiOnsibmFtZSI6Imhvb2siLCJzaWRlIjoidG9wIn19fV1d
\[\begin{tikzcd}
	{U\times F} && {p^{-1}U} & {\widetilde X} \\
	&& U & X
	\arrow["\simeq"{description}, draw=none, from=1-1, to=1-3]
	\arrow[from=1-1, to=2-3]
	\arrow[hook, from=1-3, to=1-4]
	\arrow["{p|_{p^{-1}U}}"', from=1-3, to=2-3]
	\arrow["p", from=1-4, to=2-4]
	\arrow[hook, from=2-3, to=2-4]
\end{tikzcd}\]
tel que $x\in U$ dans $Top$ où $F$ est discret. Ducoup
$U$ \textbf{trivialise} le revêtement.

Je note $U_x$ un ouvert trivialisant qui contient $x$.

\subsection{Degré}
Le degré est donné par 
\[\deg\colon X\to \N\cup\infty\]
via $\deg(x):=\#p^{-1}x$. Avec la topologie discrète sur
$\N$ le degré est continu. En général c'est localement
constant ducoup et si $X$ est connexe le degré
est constant.

\begin{rem}
  La continuité c'est que pour $x$ tel que $\deg(x)=n$
  alors $U_x\subset \deg^{-1}n$. En fait 
  $\deg^{-1}n=\cup_{\deg(y)=n}U_y$.
\end{rem}



\section{Relèvement de chemins}
Étant donné $p\colon \tilde X\to X$ et $\gamma\colon I\to X$.
Pour tout $t\in I$ on associe $U_t\subset X$ trivialisant
avec $\gamma(t)\in U_t$. Ensuite $\cup_{t\in I}\gamma^{-1}U_t=I$
et par compacité et décomposition en composantes connexes,
on a un recouvrement minimal 
\[I=\cup_{k=0}^{n-1} I_k\]
où les $I_k$ sont ouverts connexes par arcs. On pose 
$t_0=0$ et $t_n=1$, puis $t_k\in I_{k-1}\cap I_{k}$ pour 
$1\leq k\leq n-1$. Puis 
\[\gamma_k:=\gamma|_{I_k}\]
ensuite par déf $\gamma(I_k)\subset V_k$ est dans un 
ouvert
trivialisant. Maintenant on peut tout construire : si 
$\gamma_0(0)=x_0$, à un choix $\tilde x_0\in p^{-1}(x_0)$
on pose
\[p_0:=p^{-1}\colon V_0\to \tilde V_0\]
c'est un homéomorphisme, puis 
$\tilde\gamma_0:=p_0(\gamma_0)$, ça lift uniquement 
$\gamma$ sur $[0,t_1]$. Maintenant, en supposant que
les $\gamma_j$ sont liftés en $\tilde\gamma_j$ pour 
$0\leq j\leq k-1$ on réitère avec 
$x_k=\gamma_{k-1}(t_k)$ et 
$\tilde x_k=p_{k-1}(\gamma_{k-1}(t_k))$.
Alors $\gamma_{k-1}$ et $\gamma_{k}$ sont toujours
composables dans le groupoide et le lift de $\gamma$ est
unique immédiatemment via les lifts locaux.

\section{Relèvement des homotopies}
On regarde une homotopie $f_t\colon Y\to X$ ou
$F\colon Y\times I\to X$, on suppose qu'on a un lift
$\tilde F\colon Y\times \{0\}\to \tilde X$. On construit
pour tout $y\in Y$ un lift local unique 
$N\times I\to \tilde X$ de $F$ avec $y\in N$. Pour ça :
pour chaque $t\in I$, on pose $U_t$ tel que 
$f_t(y)\in U_t$ est trivialisant, ensuite si on écrit
\[F^{-1}(U_t)=N_t\times J_t=\cup_{j\in J_t}N_t\times I_j\]
où $I_j$ est un intervalle alors
\[I=\bigcup_{t\in I} F^{-1}(U_t)=\bigcup_{t\in I}\bigcup_{j\in J_t}N_t\times I_j\]
puis par compacité de $I$, 
\[I=p_I(\bigcup_{i=1}^nN_{t_i}\times I_{t_i})\]
et même 
\[I=p_I(\cup_{i=1}^nN\times I_{t_i})\]
avec $N=\cap_i N_{t_i}$. En plus 
$y\in p_Y(N\times I_{t_i})$ et
\[F(N\times I_{t_i})\subset U_{t_i}\]
est trivialisant d'où les conditions qu'on voulait.


Maintenant le lift se construit exactement de la même
manière que pour les chemins, avec la conditions
initiales étant le lift $N\times \{0\}$ qu'on a déjà,
et la condition de récurrence étant le lift de
$N\times \{t_i\}$ qui est fournit via
$p_i^{-1}F|_{N\times [t_{i-1},t_i]}(N\times \{t_i\}$
qui est supposée construite. Ici, 
$p_i\colon \tilde U_{t_i}\to U_{t_i}$ est choisie via
$f_{t_{i-1}}(y)\in U_{t_i}$ et $p_{i-1}^{-1}(f_{t_{i-1}}(y)$





\section{En pratique}
Étant donné un revêtement pointé 
$p\colon(\tilde X, \tilde x_0)\to (X,x_0)$,
la condition initiale pour relever des chemins c'est le point base $\tilde x_0$
et pour relever des homotopies c'est un chemin base $p_*\gamma=f_0$.

L'unicité du chemin force le relèvement pointé. L'unicité
d'homotopies force le chemin d'arrivée dès le point
choisi.

C'est TRÈS rigide.

%\printbibliography
\end{document}

