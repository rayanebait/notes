\documentclass[a4paper,12pt]{book}
\usepackage{amsmath,  amsthm,enumerate}
\usepackage{csquotes}
\usepackage[provide=*,french]{babel}
\usepackage[dvipsnames]{xcolor}
\usepackage{quiver, tikz}

%symbole caligraphique
\usepackage{mathrsfs}

%hyperliens
\usepackage{hyperref}

%pseudo-code
\usepackage{algorithm}
\usepackage{algpseudocode}

\usepackage{fancyhdr}

\pagestyle{fancy}
\addtolength{\headwidth}{\marginparsep}
\addtolength{\headwidth}{\marginparwidth}
\renewcommand{\chaptermark}[1]{\markboth{#1}{}}
\renewcommand{\sectionmark}[1]{\markright{\thesection\ #1}}
\fancyhf{}
\fancyfoot[C]{\thepage}
\fancyhead[LO]{\textit \leftmark}
\fancyhead[RE]{\textit \rightmark}
\renewcommand{\headrulewidth}{0pt} % and the line
\fancypagestyle{plain}{%
    \fancyhead{} % get rid of headers
}

%bibliographie
\usepackage[
backend=biber,
style=alphabetic,
sorting=ynt
]{biblatex}

\addbibresource{bib.bib}

\usepackage{appendix}
\renewcommand{\appendixpagename}{Annexe}

\definecolor{wgrey}{RGB}{148, 38, 55}

\setlength\parindent{24pt}

\newcommand{\Z}{\mathbb{Z}}
\newcommand{\R}{\mathbb{R}}
\newcommand{\rel}{\omathcal{R}}
\newcommand{\Q}{\mathbb{Q}}
\newcommand{\C}{\mathbb{C}}
\newcommand{\Cat}{\mathcal{C}}
\newcommand{\Dat}{\mathcal{D}}
\newcommand{\Aat}{\mathcal{A}}
\newcommand{\N}{\mathbb{N}}
\newcommand{\K}{\mathbb{K}}
\newcommand{\A}{\mathbb{A}}
\newcommand{\B}{\mathcal{B}}
\newcommand{\Or}{\mathcal{O}}
\newcommand{\F}{\mathscr F}
\newcommand{\Hom}{\textrm{Hom}}
\newcommand{\disc}{\textrm{disc}}
\newcommand{\Pic}{\textrm{Pic}}
\newcommand{\End}{\textrm{End}}
\newcommand{\Spec}{\textrm{Spec}}
\newcommand{\Supp}{\textrm{Supp}}
\newcommand{\Ouv}{\textrm{Ouv}}
\newcommand{\im}{\textrm{im}}
\newcommand{\coker}{\textrm{coker}}
\newcommand{\coim}{\textrm{coim}}


\newcommand{\cL}{\mathscr{L}}
\newcommand{\G}{\mathscr{G}}
\newcommand{\D}{\mathscr{D}}
\newcommand{\E}{\mathscr{E}}
\renewcommand{\P}{\mathscr{P}}
\renewcommand{\H}{\mathscr{H}}

\makeatletter
\newcommand{\colim@}[2]{%
  \vtop{\m@th\ialign{##\cr
    \hfil$#1\operator@font colim$\hfil\cr
    \noalign{\nointerlineskip\kern1.5\ex@}#2\cr
    \noalign{\nointerlineskip\kern-\ex@}\cr}}%
}
\newcommand{\colim}{%
  \mathop{\mathpalette\colim@{\rightarrowfill@\scriptscriptstyle}}\nmlimits@
}
\renewcommand{\varprojlim}{%
  \mathop{\mathpalette\varlim@{\leftarrowfill@\scriptscriptstyle}}\nmlimits@
}
\renewcommand{\varinjlim}{%
  \mathop{\mathpalette\varlim@{\rightarrowfill@\scriptscriptstyle}}\nmlimits@
}
\makeatother

\theoremstyle{plain}
\newtheorem{thm}{Théoreme}
\newtheorem{lem}{Lemme}
\newtheorem{prop}{Proposition}
\newtheorem{cor}{Corollaire}
\newtheorem{heur}{Heuristique}
\newtheorem{rem}{Remarque}
\newtheorem{note}{Note}

\theoremstyle{definition}
\newtheorem{conj}{Conjecture}
\newtheorem{prob}{Problème}
\newtheorem{quest}{Question}
\newtheorem{prot}{Protocole}
\newtheorem{algo}{Algorithme}
\newtheorem{defn}{Définition}
\newtheorem{exmp}{Exemples}
\newtheorem{exo}{Exercices}
\newtheorem{ex}{Exemple}
\newtheorem{exs}{Exemples}

\theoremstyle{remark}

\definecolor{wgrey}{RGB}{148, 38, 55}
\definecolor{wgreen}{RGB}{100, 200,0} 
\hypersetup{
    colorlinks=true,
    linkcolor=wgreen,
    urlcolor=wgrey,
    filecolor=wgrey
}

\title{Homotopie et homologies}
\date{2024-2025}

\begin{document}
\maketitle
\tableofcontents

\section{Homologie singulière et cohomologie des faisceaux}
\href{Cette ref}{https://math.stackexchange.com/questions/1794725/detail-in-the-proof-that-sheaf-cohomology-singular-cohomology}
dit qu'en gros sur des espaces localement contractile (ceux qui
m'intéressent) le complexe d'homologie singulière à coefficients
entiers est acyclique et une résolution de $\Z$ d'où calcule
\[H^i(X,\Z)=H^i(X,\Z_X)\]
l'homologie du faisceau constant.

\section{Homotopie, connexité et faisceau constant.}
Petite étude du faisceau constant $A_X$ sur un $X$ tel que les
composantes connexes soient ouvertes.


\section{Cadre de base de l'homotopie}
On se place sur $C(X,Y)$ les fonctions continues munies
de la topologie compacte-ouverte. 
\begin{rem}
  C'est donné par la base d'ouverts $V(K,U)$ t.q
  $f\in V(K,U):=\{f(K)\subset U\}$.
\end{rem}
Alors une homotopie
c'est un $H\colon [0,1]\to C(X,Y)$. Parce que
\[h\colon C([0,1]\times X,Y)\simeq C([0,1],C(X,Y))\]
canonique via 
\[H\mapsto (a\mapsto H(a,\_))\]
C'est injectif et continu mais pour que ce soit
un surjectif faut demander que $X$ soit localement
compact et séparé. Ce qui est pas évident
à montrer c'est que $H$ est continue ssi
$h(H)$ est continue, on utilise la locale compacité
pour comparer la continuité de $h(H)$ à la continuité
de $h$ (ça revient à montrer la surjectivité aussi!
Tu peux regarder que c'est pas clair! Je suis
soulagé d'avoir compris mdr.

Une fois qu'on sait ça c'est clair que 
\[h(V(K_1\times K_2,U))=V(K_1,V(K_2,U))\]
et inversement comme c'est bijectif.
Y'a une preuve
\href{https://www.youtube.com/watch?v=N1shTZiP_pA&list=PLKnx70LRf21eSXaawNhiCCPrzsbLdT5do&index=25&t=83s}{là}.

\chapter{CW-complexes}

\chapter{Groupes d'homotopies}
\section{Homotopies}
Faudra toujours penser à une homotopie de plusieurs points de vues.
De $H\colon [0,1]\times X\to Y$ on remarque que 
$h^x:=(t\mapsto H(t,x))$ est un chemin entre $f(x)$ et $g(x)$. Et
si $X$ est un segment. Alors $h_t:=s\mapsto H(t,s)$ est aussi
un chemin. On peut se balader sur $[0,1]^2$.
\subsection{Défs}
\subsubsection{$\sim$ sur $C(X,Y)$}
On déf $\sim$ sur $C(X,Y)$ par
$f\sim g$ si y sont homotopes. C'est facilement
une relation d'équivalences!
\begin{rem}
  Montrer que la composée d'homotopies est
  continue. C'est facile quand on prend la déf
  $[0,1]\to C(X,Y)$!! En fait c'est juste 
  $H_3^{-1}(U)=H_2^{-1}U\cup H_1^{-1}U$ vraiment.
\end{rem}
\subsubsection{Par rapport à une partie}
Si $f,g$ coincident sur $A$, on peut demander à
ce que l'homotopie fixe $A$. I.e. $H(\_,A)$ un point et on note
$f\sim_A g$.
\subsubsection{Contractile}
On est contractile si $id_X\sim (x\mapsto x_0)$ (plus non vide).

\subsubsection{Équivalence d'homotopie}
C'est $f\colon X\rightleftarrows Y\colon g$ t.q. 
$f\circ g\sim id_Y$ et $g\circ f\sim id_X$.

\subsubsection{Rétract}
$i\colon Z\subset X$ est un rétract si on a $r\colon X\to Z$
t.q $r\circ i= id_Z$. 

\subsubsection{Rétract par déformation}
En plus $i\circ r\sim id_X$. En particulier 
$X$ et $Z$ sont homotopiquement équivalents.
\subsubsection{Rétract par déformation forte}
En plus $i\circ r\sim_Z id_X$.


\subsection{Étendre $S^1$ en $B^2$, topologie quotient et finale.}
Là d'un coup y'a plusieurs notions. La simple connexité c'est
que tout les lacets s'étendent en $B^2$. Donc y'a pas de trous.
La contractilité implique la simple connexité via le fait que
si $id_X\sim x_0$ par $H$ et $S^1\to X$ est un lacet alors on
obtient 
\[[0,1]\times S^1\to [0,1]\times X\to X\]
et que $\{1\}\times S^1\mapsto x_0$. D'où ça passe au quotient
en $B_2\to X$.
\subsection{La continuité}
Donc la propriété universelle de $X/\sim_A$ ca doit être que
pour tout $X\to Y$ continue t.q. $f(A)=\{*\}$ alors on a une
unique $X/\sim_A\to Y$ continue pour la topologie quotient à
gauche qui fait commuter. En particulier, 
\[[0,1]\times S^1\to X\]
est continue. Plus en détail, si on a 
% https://q.uiver.app/#q=WzAsMyxbMCwwLCJYIl0sWzAsMSwiWC9BIl0sWzEsMCwiWSJdLFswLDEsIlxccGkiXSxbMCwyLCJmIl0sWzEsMiwiXFxiYXIgZiIsMix7InN0eWxlIjp7ImJvZHkiOnsibmFtZSI6ImRhc2hlZCJ9fX1dXQ==
\[\begin{tikzcd}
	X & Y \\
	{X/A}
	\arrow["f", from=1-1, to=1-2]
	\arrow["\pi", from=1-1, to=2-1]
	\arrow["{\bar f}"', dashed, from=2-1, to=1-2]
\end{tikzcd}\]
alors $\bar f$ est continue vu que $\bar f^{-1}V$ est ouvert
ssi $\pi^{-1}\bar f^{-1} V=f^{-1}V$ l'est.

\subsubsection{Commentaire sur la topologie quotient}
C'est le petit twist sur l'incertitude de $\pi^{-1}\pi U=U$ ou
$U\cup A$ et $A$ par forcément ouvert. Là, $f^{-1} V$ est ouvert
et contient $A$ OU ne contient pas $A$. Alors que $U\cup A$
ca peut déborder.


\section{Groupe fondamental}
Le lemme 2.1 donne toutes les homotopies pour que $\pi_1(X,x)$ 
soit un groupe (composition, associativité, inverse,...). Y'a 
plusieurs points de vues à retenir.

\subsection{Homotopies de chemins}
On est dans le cas où une homotopie c'est un chemin dans 
\[C([0,1],X)\]
On peut écrire les homotopies de chemins/lacets : 
\[H\colon [0,1]\times [0,1]\to X\]
comme pour l'associativité, où par déf $(\alpha.\beta).\gamma$
on va $4$ fois plus vite sur les deux premiers et $2$ fois
plus vite sur $\gamma$. Et faut aller vers $\alpha.(\beta.\gamma)$
où ça va $4$ fois plus vite sur les deux derniers.
Conseil :
\[\textrm{Faire ce qui marche au niveau des bords mdr}\]
\subsection{$\pi_1$ est un groupe}
Faut juste vérifier que tout est vrai à homotopie près
\subsection{Propriétés du foncteur}
Dans une composante connexe par arc on peut changer à isomorphisme
près le point base par 
\[\phi_l\colon \pi_1(X,x)\to \pi_1(X,y);[c]\mapsto [\bar l.c.l]\] avec $l(0)=x$
et $l(1)=y$. C'est un isomorphisme unique à homotopie relative de
$l$ près. Et si $l'$ est un autre chemin,
alors $\phi_l$ et $\phi_l'$ sont conjugués via $[l.\bar l']$ (dans
ce sens).
\begin{rem}
  Ça veut dire quoi que le $\pi_1$ est abélien ?! Genre le tore.
\end{rem}
Étant donné $f\colon X\to Y$, $[c]\mapsto [f\circ c]$
est bien déf, i.e. préserve les homotopies par $H\mapsto f\circ H$,
et préserve les compositions (par définition des compos! écrire)
donc est un m.d.g. En plus
$\pi_1$ est additif via $c\times c'$ (c'est clairement un lacet,
pas de pb de continuité ni de définition). On obtient
un foncteur covariant additif
\[Top_*\to Grp\]
donné par $(X,x)\mapsto \pi_1(X,x)$. Et pour $(X,x)\to (Y,y)$ on
a
\[\pi_1(f)\colon \pi_1(X,x)\to \pi_1(Y,f(x)=y)\]

\section{Homotopie et $\pi_1$}
Si $f,g\colon X\to Y$ sont homotopes, on obtient un chemin
$f(x)\to g(x)$ via $c(t):=h_x(t):=H(t,x)$. D'où un iso
$u\colon \pi_1(Y,g(x))\to \pi_1(Y,f(x))$. Et en fait
\[f_*=ug_*\]
faut montrer que $[f\circ a]=[\bar c.(g\circ a).c]$. Et pour ça,
par déf de $c$, on peut redéfinir $c=c_1$ et $c_s(t):=h_x(st)$.
En particulier $(s,t)\mapsto \bar c_s.(h_s\circ a).c_s(t)$
c'est une homotopie entre les deux classes. En détail
$c_s$ c'est un chemin entre $f(x)$ et $H(s,x)$, puis $h_s\circ a$
c'est un lacet en $h_s(x)$ puis $\bar c_s$ c'est un chemin entre
$h_s(x)$ et $f(x)$. Ça fixe bien les extremités. 
\begin{rem}
  Pour chaque $x$, $t\mapsto h_t(x)=H(t,x)$
  est un chemin entre $f(x)$ et $g(x)$.
\end{rem}
\section{Invariance de $\pi_1$}
Ducoup on en déduit direct que $\pi_1$ est invariant par 
équivalence d'homotopie.
\section{Indice à Van Kampen}
\subsection{Découper des chemins}
Étant donné $X=U\cup V$, on peut découper un chemin $a=\prod a_i$
de telle sorte à ce que $a_i$ soit dans $U$ ou $V$. On peut
regarder $a^{-1}U\cup a^{-1}V=[0,1]$, le piège c'est que c'est
des ouverts mais pas connexes nécessairemment. Donc c'est une 
union infinie d'intervalles. On peut écrire 
\[\cup ]u_{i1},u_{i2}[\cup ]v_{i1},v_{i2}[\]
et par compacité on a une union finie de trucs soit dans $a^{-1}U$
soit dans $a^{-1}V$. Ensuite 
$a(]u_{i1},u_{i2}[)\subset a(a^{-1}(U))\subset U$. Maintenant on
peut raffiner dans $[0,1]$ en une union type $\cup ]i/n,(i+1)/n[$.

\subsection{Lemme}
Étant donné $X=U\cup V$ avec $U,V,U\cap V$ connexes par arcs, alors
pour tout $x\in U\cap V$, 
\[<\pi_1(U,x),\pi_1(V,x)>=\pi_1(X,x)\]
En particulier si $U$ et $V$ sont simplements connexes, $X$ aussi.
La différence avec Van Kampen c'est la précision de l'énoncé. 
Donc concrètement, on découpe $a$ en $\prod a_i$ avec 
\[a_i(t)=a((i+t)/n)\]
ensuite on prends $c_i$ un chemin entre $x$ et $a_i(0)$ contenu
dans $U,V,U\cap V$ en fonction de $a_i(0)$. Alors
\[\bar c_i.a_i.c_{i+1}\in \pi_1(W,x)\]
(!) avec $W=U,V$ ou $U\cap V$ en fonction de $a_i(0),a_i(1)$.
Enfin
\[a=\prod_{i=0}^n \bar c_i.a_i.c_{i+1}\]
D'où \[<\pi_1(U,x),\pi_1(V,x)>=\pi_1(X,x)\]


\chapter{Exemples}
\section{Exemples de contractions/rétractions}
\subsection{Convexe métrique}
Dans un espace vectoriel topologique $V$ si on 
a un convexe non vide $C$ :
\[(s,x)\mapsto (1-s)x+sx_0\]
contracte $C$ sur $x_0$. C'est vraiment agréable!
\subsection{$n-1$-sphère et $R^n-0$}
On regarde $(s,x)\mapsto (1-s)x+sx/||x||$. En 
particulier $x\mapsto x/||x||$ est
un rétract par déformation forte!
\subsection{Tore privé d'un point et cercles}
Je peux pas faire de desseins, mais l'idée c'est de regarde le
tore comme le recollement de côtés opposés dans le même sens 
% https://q.uiver.app/#q=WzAsNCxbMCwxLCJcXGJ1bGxldCJdLFswLDAsIlxcYnVsbGV0Il0sWzEsMCwiXFxidWxsZXQiXSxbMSwxLCJcXGJ1bGxldCJdLFswLDFdLFsxLDJdLFswLDNdLFszLDJdXQ==
\[\begin{tikzcd}
	\bullet & \bullet \\
	\bullet & \bullet
	\arrow[from=1-1, to=1-2]
	\arrow[from=2-1, to=1-1]
	\arrow[from=2-1, to=2-2]
	\arrow[from=2-2, to=1-2]
\end{tikzcd}\]
et si au lieu de regarder le carré on écrit un cercle à $4$-points,
on a un rétract par déformation forte via $x/||x||$. Ensuite, 
on obtient juste les bords du carrés avec les côtés opposés 
identifiés. Ça c'est pareil que

% https://q.uiver.app/#q=WzAsMyxbMCwxLCJcXGJ1bGxldCJdLFswLDAsIlxcYnVsbGV0Il0sWzEsMCwiXFxidWxsZXQiXSxbMCwxXSxbMSwyXV0=
\[\begin{tikzcd}
	\bullet & \bullet \\
	\bullet
	\arrow[from=1-1, to=1-2]
	\arrow[from=2-1, to=1-1]
\end{tikzcd}\]
où on identifie les $3$ points. D'où le bouquet de deux cercles.

\subsection{Ruban de moebius et $S^1$}
On définit $[0,1]^2\to ([0,1]/(1\sim 0))\times \{1/2\}$, ça passe
au quotient pour $(0,s)\sim (1,1-s)$. Plus précisément
% https://q.uiver.app/#q=WzAsNCxbMCwxLCJbMCwxXV4yLygoMCxzKVxcc2ltKDEsMS1zKSkiXSxbMCwwLCJbMCwxXV4yIl0sWzEsMCwiWzAsMV1cXHRpbWVzXFx7MS8yXFx9Il0sWzEsMSwiWzAsMV0vKDFcXHNpbSAwKVxcdGltZXNcXHsxLzJcXH0iXSxbMSwwXSxbMSwyXSxbMCwzLCIiLDIseyJzdHlsZSI6eyJib2R5Ijp7Im5hbWUiOiJkYXNoZWQifX19XSxbMiwzXV0=
\[\begin{tikzcd}
	{[0,1]^2} & {[0,1]\times\{1/2\}} \\
	{[0,1]^2/((0,s)\sim(1,1-s))} & {[0,1]/(1\sim 0)\times\{1/2\}}
	\arrow[from=1-1, to=1-2]
	\arrow[from=1-1, to=2-1]
	\arrow[from=1-2, to=2-2]
	\arrow[dashed, from=2-1, to=2-2]
\end{tikzcd}\]
Ensuite, en haut c'est un rétract par déformation forte et 
l'homotopie descend en bas, on le voit par propriété universelle.
En particulier ça reste un rétract par déformation forte. D'où
un isomorphisme de $\pi_1$.

\section{Peigne, pas de rétract par déformation forte}
Si on regarde 
\[P=[0,1]\times 0\bigcup 0\times [0,1]\bigcup\cup_i 2^{-i}\times [0,1]\]
dans $\R^2$ avec la \textbf{topologie induite} (c'est tout le pb)
c'est contractile parce que y'a un rétract par déformation
forte sur $[0,1]\times 0$. Par contre $(0,1)$ est pas un
rétract par déformation forte. Et ça je pense que c'est parce que
pour converger de manière connexe par arc vers $(0,1)$ on peut pas
facilement vu que $B(x_0,1/2)$ est une union disjointe infinie.
Mais j'arrive pas à le prouver
\section{Bouquet de $n$-Sphère et plan moins $n$-points}
On peut prendre $x_i=(i,0)$. Et on sépare le plan en $n$-bandes
verticales. À homotopie près on peut couper le reste. Ensuite
en gros on définit 
\[(s,x)\mapsto (1-s)(x-x_i)+s(x-x_i)/||x-x_i||+x_i\]
si $x$ est dans la $i$-ème bande. Le fait de faire $-x_i$ ça
pousse dans la première bande. Ensuite, ça met sur le cercle,
comme la bande est convexe et contient pas $0$ c'est bon. Ensuite
on renvoie sur la $i$-ème sphère. Pour que ce soit continu faut
que les bords des sphères soient collées intuitivement. D'où
$n$-sphères collées, puis ça c'est homotope au bouquet.

\begin{rem}
  Pour le faire formellement dans $\R^n$ ça a l'air faisable. Faut
  juste revoir la définition avec les suites convergentes je pense.
\end{rem}

\subsection{Groupe fondamentaux de groupes topologiques}
Là y se passe un truc vraiment drôle. Si on déf sur $G$ un
groupe topologique $fg\colon t\mapsto f(t)g(t)$ ça fait
un lacet pour $f,g\in \pi_1(G,e)$. Et maintenant on peut montrer
que 
\[fg\sim f.g\]
sauf que ducoup on peut en déduire que le $\pi_1$ est abélien!!
En fait l'idée c'est que
\[[fg]=[(f.c_e)(c_e.g)]=[f.g]\]
et
\[[fg]=[(c_e.f)(g.c_e)]=[g.f]\]
et ça c'est drôle (et vrai par définition en fait). Attention
$f(g.c_e)\ne fg.f$ à cause des
vitesses! On aurait que $f$ de $0$ à $1/2$ sur la première partie
en fait.

\section{Variétés topologiques et locale contractilité}
En gros tout ce qu'on regarde et l'endroit où la cohomologie
des faisceaux est la cohomologie singulière c'est sur les
espaces localements contractiles.

\chapter{Revêtements}

%\printbibliography
\end{document}

