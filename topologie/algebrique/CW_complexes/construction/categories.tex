\documentclass[a4paper,12pt]{book}
\usepackage{amsmath,  amsthm,enumerate}
\usepackage{csquotes}
\usepackage[provide=*,french]{babel}
\usepackage[dvipsnames]{xcolor}
\usepackage{quiver, tikz}

%symbole caligraphique
\usepackage{mathrsfs}

%hyperliens
\usepackage{hyperref}

%pseudo-code
\usepackage{algorithm}
\usepackage{algpseudocode}

\usepackage{fancyhdr}

\pagestyle{fancy}
\addtolength{\headwidth}{\marginparsep}
\addtolength{\headwidth}{\marginparwidth}
\renewcommand{\chaptermark}[1]{\markboth{#1}{}}
\renewcommand{\sectionmark}[1]{\markright{\thesection\ #1}}
\fancyhf{}
\fancyfoot[C]{\thepage}
\fancyhead[LO]{\textit \leftmark}
\fancyhead[RE]{\textit \rightmark}
\renewcommand{\headrulewidth}{0pt} % and the line
\fancypagestyle{plain}{%
    \fancyhead{} % get rid of headers
}

%bibliographie
\usepackage[
backend=biber,
style=alphabetic,
sorting=ynt
]{biblatex}

\addbibresource{bib.bib}

\usepackage{appendix}
\renewcommand{\appendixpagename}{Annexe}

\definecolor{wgrey}{RGB}{148, 38, 55}

\setlength\parindent{24pt}

\newcommand{\Z}{\mathbb{Z}}
\newcommand{\R}{\mathbb{R}}
\newcommand{\rel}{\omathcal{R}}
\newcommand{\Q}{\mathbb{Q}}
\newcommand{\C}{\mathbb{C}}
\newcommand{\Cat}{\mathcal{C}}
\newcommand{\Dat}{\mathcal{D}}
\newcommand{\Aat}{\mathcal{A}}
\newcommand{\N}{\mathbb{N}}
\newcommand{\K}{\mathbb{K}}
\newcommand{\A}{\mathbb{A}}
\newcommand{\B}{\mathcal{B}}
\newcommand{\Or}{\mathcal{O}}
\newcommand{\F}{\mathscr F}
\newcommand{\Hom}{\textrm{Hom}}
\newcommand{\disc}{\textrm{disc}}
\newcommand{\Pic}{\textrm{Pic}}
\newcommand{\End}{\textrm{End}}
\newcommand{\Spec}{\textrm{Spec}}
\newcommand{\Supp}{\textrm{Supp}}
\newcommand{\Ouv}{\textrm{Ouv}}
\newcommand{\im}{\textrm{im}}
\newcommand{\coker}{\textrm{coker}}
\newcommand{\coim}{\textrm{coim}}


\newcommand{\cL}{\mathscr{L}}
\newcommand{\G}{\mathscr{G}}
\newcommand{\D}{\mathscr{D}}
\newcommand{\E}{\mathscr{E}}
\renewcommand{\P}{\mathscr{P}}
\renewcommand{\H}{\mathscr{H}}

\makeatletter
\newcommand{\colim@}[2]{%
  \vtop{\m@th\ialign{##\cr
    \hfil$#1\operator@font colim$\hfil\cr
    \noalign{\nointerlineskip\kern1.5\ex@}#2\cr
    \noalign{\nointerlineskip\kern-\ex@}\cr}}%
}
\newcommand{\colim}{%
  \mathop{\mathpalette\colim@{\rightarrowfill@\scriptscriptstyle}}\nmlimits@
}
\renewcommand{\varprojlim}{%
  \mathop{\mathpalette\varlim@{\leftarrowfill@\scriptscriptstyle}}\nmlimits@
}
\renewcommand{\varinjlim}{%
  \mathop{\mathpalette\varlim@{\rightarrowfill@\scriptscriptstyle}}\nmlimits@
}
\makeatother

\theoremstyle{plain}
\newtheorem{thm}{Théoreme}
\newtheorem{lem}{Lemme}
\newtheorem{prop}{Proposition}
\newtheorem{cor}{Corollaire}
\newtheorem{heur}{Heuristique}
\newtheorem{rem}{Remarque}
\newtheorem{note}{Note}

\theoremstyle{definition}
\newtheorem{conj}{Conjecture}
\newtheorem{prob}{Problème}
\newtheorem{quest}{Question}
\newtheorem{prot}{Protocole}
\newtheorem{algo}{Algorithme}
\newtheorem{defn}{Définition}
\newtheorem{exmp}{Exemples}
\newtheorem{exo}{Exercices}
\newtheorem{ex}{Exemple}
\newtheorem{exs}{Exemples}

\theoremstyle{remark}

\definecolor{wgrey}{RGB}{148, 38, 55}
\definecolor{wgreen}{RGB}{100, 200,0} 
\hypersetup{
    colorlinks=true,
    linkcolor=wgreen,
    urlcolor=wgrey,
    filecolor=wgrey
}

\title{CW-complexes}
\date{2024-2025}

\begin{document}
\maketitle
\tableofcontents
\chapter{CW-complexes}
\section{Construction relative}
On recolle des $n$-cellules selon $X^{(n-1)}$ en fait.
Autrement dit $X=\cup X^{(n)}$ et 
\[\sqcup_\alpha e_{\alpha,n}\bigsqcup X^{(n-1)}\simeq X^{(n)}\]
\subsection{Vocabulaire}
Étant donnés $A\to (f\colon X\to Y)$ on note 
\[Z:=X\cup_f Y:=(X\sqcup Y)/<x\sim f(x), x\in A>\]
muni de la topologie quotient. Si $A$ est fermé
(ce qui nous intéresse)
alors $Y\hookrightarrow \bar Y$ est un homéo sur un fermé. Vu
que $\pi^{-1}Y=Y\sqcup A$.
En tant qu'ensemble si on regarde $\bar x\in Z$, on a 
\[\pi^{-1} \bar x=\begin{cases} \{x\}~si~x\in (X-A)\sqcup (Y-f(A))
\\ f^{-1}f(x)\sqcup \{f(x)\}~sinon\end{cases}\]
Ducoup la séparation dépend de plusieurs facteurs, je le fais 
pour les recollements.

\subsection{Décomposition cellulaire relative}
Ducoup, $e_\alpha$ c'est la notation pour une $n$-cellule ($B^n$)
(j'identifie $\mathring B^n$ et son image dans $X$).
et $\partial e_\alpha = \partial B^n=S^{n-1}$. Pas oublier que
c'est compact. 

Maintenant étant donnés $g_\alpha\colon \partial e_\alpha\to Y$.
On a $\sqcup g_\alpha=:g\colon \sqcup \partial e_\alpha \to Y$ et
on peut recoller $(\sqcup e_\alpha)\cup_g Y)=C$. Si $X\simeq C$
alors $C$ est une décomposition cellulaire de $X$ relative à $Y$.

\begin{rem}
  On peut recoller successivement, $(e_1\sqcup e_2)\cup_{g} Y=
  e_1\cup_{g_1}(e_2\cup_{g_2}Y)$. Ça doit se prouver directement
  au niveau des relations d'équivalences.
\end{rem}

\subsection{Détail}
Les cellules sont compactes, si $Y$ est séparé, $im(g)$ est compact
donc fermé dans $Y$. Ensuite, est-ce que 
$X:=\sqcup_\alpha e_\alpha\cup_g Y$ est séparé ? Faut séparer
les trois types de points :
\begin{enumerate}
  \item Quand $x\in X-Y$, i.e. $x\in \mathring e_\alpha$ c'est
    toujours facile vu que on peut prendre
    $Y\cup \cup \partial e_\alpha\times ]1-\epsilon,1]$.
  \item Quand $x\in X\cap Y$ et $y\in X\cap Y$, si y ont même
    image y sont égaux, donc on sépare les images dans $Y$.
  \item Quand $x\in X\cap Y$ et $y\in Y$, on sépare $f(x)$ et $y$.
\end{enumerate}


\subsection{Se ramener au bouquet}
Si on quotiente $C$ par $Y$, on a 
$\sqcup e_\alpha/\sqcup \partial e_\alpha$.
Si les $e_\alpha$ sont tous des $n$-cellules, on obtient un bouquet
de
sphères ! Ou au moins un bouquet de cellules. Ensuite, si $X=C$
alors $X-Y$ est une union disjointe de boules ouvertes.


\section{Construction par le squelette}
Maintenant, on dit que $X$ est obtenu par recollement de 
$n$-sphères sur $Y$ si on a une famille $(e_\alpha)$ de dimension
$n$ (potentiellement vide !) etc etc..

Maintenant $X$ est un CW-complexe si $X=\cup_n X^{(n)}$ avec
la topologie faible et 
$X^{(n)}$ est obtenu par recollement de $n$-cellules sur 
$X^{(n-1)}$. Un sous-CW-complexe c'est un $Y$ tel que sa 
décomposition est donnée par $Y^{(n)}=Y\cap X^{(n)}$.

\begin{rem}
Ducoup $X^{(0)}$ c'est juste des points. Et
on peut avoir $X^{(i)}=X^{(i-1)}$ en recollant rien. En
particulier on peut obtenir la sphère via $X^{(0)}=X^{(1)}=\{*\}$
et $X^{(2)}=B^2\cup_{\{*\}} X^{(1)}$.
\end{rem}

\subsection{Se ramener au bouquet, partie 2}
Si on quotiente $X^{(n)}$ par $X^{n-1}$ on obtient un bouquet
de $n$-sphères! J'imagine que ça écrase tout ce qu'y a au dessus
de $X^{(n)}$ ? Non du tout en fait.

\subsection{Séparation et locale contractilité.}
Si c'est de dimension finie, $X^{(0)}$ est séparé donc par
récurrence $X^{(n)}$ aussi pour tout $n$. Pour généraliser, d'abord
la locale contractilité. On veut m.q que tout point $x$ et tout 
voisinage $V$ de $x$, il existe $x\in V_\infty\subset V$ 
contractile. Y suffit d'avoir $x\in V_n\subset X^{(n)}\cap V$
contractile via $h_n$ et $V_{n+1}\cap X^{(n)}=V_n$ pareil 
$h_{n+1}|_{X^{(n)}\times [0,1]}=h_n$. En fait y'a un $n_0$
et une $n_0$-cellule $e$ telle que $x\in \mathring e$ et on
peut prendre $V_{n_0}$ une petite boule autour dans $V$ qui
est clairement contractile. Ensuite
si on a construit jusqu'à $n$, pour chaque $n+1$-cellule on regarde
$f_i\colon e_i\to X$ et on déf 
\[V_{n+1}=V_n\cup \bigcup_i \partial f_i^{-1}(V_n)\times [0\times \epsilon_i]\]
par la topologie faible $V_{n+1}$ est ouvert et on peut déf
$h_{n+1}(f_i(x,t),s):=f_i(h_n(\partial f_i(x),st))$ dans 
l'intérieur des cellules c'est malin et $h_{n+1}=h_n$ dans 
$X^{(n)}$.

\begin{rem}
  Le seul détail bizarre c'est que $f_i^{-1} V$ contient pas
  forcément $\partial f_i^{-1} V_n \times [0,\epsilon_i[$ si ?
  J'imagine qu'on peut réduire $V_n$ pour que ce soit ok ?
\end{rem}

Maintenant $h_\infty$ et $V_\infty$ se définissent bien. Et pour
la séparation, on peut séparer dans $X^{(n)}$ d'abord puis
prendre des $V_\infty$, par construction y seront disjoints.



\subsection{Compacité des $CW$-complexes finis}

Par déf on recolle qu'un nombre fini de cellules à chaque étapes.
En particulier, $X=X^{(n)}$ est l'image de $\sqcup e_{\alpha,n}
\sqcup X^{(n-1)}$ qui est compact si $X^{(n-1)}$ est compact
ce qui est le cas par récurrence et l'hypothèse.

\chapter{Graphes}
En fait on aura que 
\[\pi_1(X,x)=\pi_1(X^{(2)},x)=N((r_\alpha)_\alpha)\backslash \pi_1(X^{(1)},x)\]
avec $x\in X^{(0)}$ (on suppose $X$ connexe) et ou on note
$\partial g_\alpha\colon \partial e_\alpha\to X^{(1)}$
chaque recollement de $2$-cellules et $r_\alpha=[cg_\alpha\bar c]$
la classe de $cg_\alpha\bar c$ dans $\pi_1(X^{(1)},x)$ pour $c$
un chemin entre $x$ et $g_\alpha(1)$. Ensuite $N(\ldots)$ est
le normalisateur. Parce que $r_\alpha$ est défini qu'à conjugaison
et inverse près (voir dans homotopie 1 pourquoi).

\section{Définitions}
Un graphe topologique c'est un CW-complexe de dimension $\leq 1$.
Un cycle c'est un plongement de $S^1$ (pas exactement un lacet). Un
chemin c'est un chemin topologique.
\section{Groupe fondamental d'un graphe}
\subsection{Limites de groupes libres}
Si on prendre $(S_i)_i$ une famille de partie de $S$ un
ensemble stable par union telles que $S=\cup S_i$. Alors
\[\varinjlim_i L(S_i)=L(\sqcup S_i)/<s_is_j^{-1},~s_i,s_j\in S_i\cap S_j>\]
d'où la flèche canonique $\varinjlim_i L(S_i)\to L(S)$ est
injective. La surjectivité c'est que $\cup S_i=S$.

\subsection{Arbres}
Y'a quelques conditions équivalentes à être un arbre $X$ :
\begin{enumerate}
  \item Pas de plongements de $S^1$.
  \item $X-\mathring e$ est pas connexe pour toute $1$-cellule $e$.
  \item $X$ est simplement connexe.
  \item $X$ est contractile.
\end{enumerate}
La connexité implique la connexité par arc ici. Si $X-\mathring e$
est connexe on prend un chemin injectif entre les extremités de 
$e$. Ça fait un plongement de $S^1$. Son existence on peut le
faire de proche en proche, le chemin est de taille finie par
définition, et entre deux $0$-cellules y'a un chemin injectif,
cqfd. À l'inverse, si y'a un plongement de $S^1$ bah 
$S^1-\mathring e$ est connexe. Pour $2.$ implique $3.$, si 
c'est pas simplement connexe on peut trouver un lacet injectif.
$3.$ implique $4.$ c'est parce que de dimension $\leq 1$. $4.$
implique $3.$ c'est vrai en général.

\subsection{Groupe fondamental d'un graphe}
On peut utiliser la famille des sous-graphes finis et $2.2.1$
pour se ramener à un graphe fini. Ensuite par la dernière 
sous-section soit $X$ est un arbre donc contractile et on est
bon soit $X$ contient $S^1=C$. Puis $X=X-\mathring e\cup C$
d'intersections $C-\mathring e$ connexe par arc. D'où
\[\pi_1(X,x)=\pi_1(X-\mathring e,x)*\Z\]
\begin{note}
  Wouah.
\end{note}

\subsection{$c(X)=1-\chi(X)$, $rk(\pi_1(X,x))=c(X)$}
La connectivité de $X$ : plus grand $n$ tel que 
$X-\cup \mathring e$ est connexe. Et $\chi(X)=n_0-n_1$ le
nombre de sommets moins le nombre d'arrête (caractéristique
d'Euler). En enlevant une arrête, si c'est plus connexe alors
on a deux sous-graphes et $c(X)=c(X_1)+c(X_2)$. En plus 
$n_{01}+n_{02}=n_0$ et $n_{11}+n_{12}=n_1-1$. Si $c(X_i)=0$ alors
$X_i$ est un point d'où $n_{11}=n_1-1$ et $n_{01}=n_0-1$. Et on
peut faire une double récurrence sur le nombre d'arrête et de
sommets. Sinon on fait une récurrence sur la connectivité?


\section{Application des graphes aux groupes fondamentaux de
CW-complexes}
Donc maintenant on peut prouver que $\pi_1(X,x)=\pi_1(X^{(2)},x)$
et $\pi_1(X,x)=N(r_\alpha,\alpha)\backslash\pi_1(X^{(1)},x)$.

\section{Applications en théorie des groupes}
\subsection{Sous groupes des groupes libres}
Étant donné un groupe libre on a un graphe $X$. Et un sous-groupe
est donné par un revêtement $B\to X$. Sauf qu'un revêtement d'un
graphe est un graphe et $\pi_1(B,b)$ est le sous-groupe.










%\printbibliography
\end{document}

