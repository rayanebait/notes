\documentclass[a4paper,12pt]{article}
\usepackage{amsmath,  amsthm,enumerate}
\usepackage{csquotes}
\usepackage[provide=*,french]{babel}
\usepackage[dvipsnames]{xcolor}
\usepackage{quiver, tikz}

%symbole caligraphique
\usepackage{mathrsfs}

%hyperliens
\usepackage{hyperref}

%pseudo-code
\usepackage{algorithm}
\usepackage{algpseudocode}

%bibliographie
\usepackage[
backend=biber,
style=alphabetic,
sorting=ynt
]{biblatex}

\addbibresource{bib.bib}


\definecolor{wgrey}{RGB}{148, 38, 55}

\setlength\parindent{24pt}

\newcommand{\Z}{\mathbb{Z}}
\newcommand{\R}{\mathbb{R}}
\newcommand{\rel}{\omathcal{R}}
\newcommand{\Q}{\mathbb{Q}}
\newcommand{\C}{\mathbb{C}}
\newcommand{\N}{\mathbb{N}}
\newcommand{\K}{\mathbb{K}}
\newcommand{\A}{\mathbb{A}}
\newcommand{\B}{\mathcal{B}}
\newcommand{\Or}{\mathcal{O}}
\newcommand{\F}{\mathscr F}
\newcommand{\Hom}{\textrm{Hom}}
\newcommand{\disc}{\textrm{disc}}
\newcommand{\Pic}{\textrm{Pic}}
\newcommand{\End}{\textrm{End}}
\newcommand{\Spec}{\textrm{Spec}}
\newcommand{\Supp}{\textrm{Supp}}
\renewcommand{\Im}{\textrm{Im}}
\newcommand{\m}{\mathfrak{m}}
\renewcommand{\P}{\mathbb{P}}
\newcommand{\p}{\mathfrak{p}}


\newcommand{\cL}{\mathscr{L}}
\newcommand{\G}{\mathscr{G}}
\newcommand{\D}{\mathscr{D}}
\newcommand{\E}{\mathscr{E}}
\newcommand{\Po}{\mathscr{P}}
\renewcommand{\H}{\mathscr{H}}

\makeatletter
\newcommand{\colim@}[2]{%
  \vtop{\m@th\ialign{##\cr
    \hfil$#1\operator@font colim$\hfil\cr
    \noalign{\nointerlineskip\kern1.5\ex@}#2\cr
    \noalign{\nointerlineskip\kern-\ex@}\cr}}%
}
\newcommand{\colim}{%
  \mathop{\mathpalette\colim@{\rightarrowfill@\scriptscriptstyle}}\nmlimits@
}
\renewcommand{\varprojlim}{%
  \mathop{\mathpalette\varlim@{\leftarrowfill@\scriptscriptstyle}}\nmlimits@
}
\renewcommand{\varinjlim}{%
  \mathop{\mathpalette\varlim@{\rightarrowfill@\scriptscriptstyle}}\nmlimits@
}
\makeatother

\theoremstyle{plain}
\newtheorem{thm}[subsection]{Théoreme}
\newtheorem{lem}[subsection]{Lemme}
\newtheorem{prop}[subsection]{Proposition}
\newtheorem{cor}[subsection]{Corollaire}
\newtheorem{heur}{Heuristique}
\newtheorem{rem}{Remarque}
\newtheorem{note}{Note}

\theoremstyle{definition}
\newtheorem{conj}{Conjecture}
\newtheorem{prob}{Problème}
\newtheorem{quest}{Question}
\newtheorem{prot}{Protocole}
\newtheorem{algo}{Algorithme}
\newtheorem{defn}[subsection]{Définition}
\newtheorem{exmp}[subsection]{Exemples}
\newtheorem{exo}[subsection]{Exercices}
\newtheorem{ex}[subsection]{Exemple}
\newtheorem{exs}[subsection]{Exemples}
\newtheorem{res}{Résumé}
\newtheorem{rep}{Réponse}
\newtheorem{cons}{Conséquence}

\theoremstyle{remark}

\definecolor{wgrey}{RGB}{148, 38, 55}
\definecolor{wgreen}{RGB}{100, 200,0} 
\hypersetup{
    colorlinks=true,
    linkcolor=wgreen,
    urlcolor=wgrey,
    filecolor=wgrey
}

\title{Théorie de Galois infinie}
\date{}

\begin{document}
\maketitle

\section{Extensions galoisiennes}
\subsection{Déf}
On étend la définition en disant que $L/K$ est 
galoisienne si algébrique, séparable et normale. En
particulier si $K\subset E\subset L$ alors $E/L$ est
galoisienne.

\subsection{Paradigme/Rappel}
Y s'agit toujours de remarquer que si $K^{sep}$ est
une clôture séparable et $L/K$ est finie :
\begin{enumerate}
  \item Alors $\Hom_K(L,K^{sep})$ est fini de cardinal
    $[L:K]$.
  \item $\Hom_K(K^{sep},K^{sep})\to \Hom_K(L, K^{sep})$
    est surjectif (Zorn).
\end{enumerate}


\section{Groupes de Galois}
Milne prends une notation particulière : 
$G=Aut_K(K^{sep})$, pour $S\subset K^{sep}$ fini ;
$G(S)=\{g\in G| gs=s,~\forall s\in S\}$. Ensuite pour
une extension galoisienne $E/K$ quelconque on définit
\[Gal(E/K):=Aut_K(E)\]
\subsection{Constructions}
En fait $G(S)=Gal(K^{sep}/K(S))$.
Par $1.2.1$, $\bar S:=\cup_{g\in G} gS$ est fini et
$g\bar S=\bar S$ pour tout $g\in G$, alors $G(\bar S)$
est normal dans $G$ (mini calcul). Ça correspond au 
groupe de galois $Gal(K^{sep}/E)$ où $E$ est la
clôture galoisienne de $K(S)/K$ dans $K^{sep}$.


\subsection{Topologie de Krull}
En fait 
\begin{itemize}
  \item $G(S)$ définit une base de voisinage ouverts de
    $1$ et une topologie unique sur $G$.
\end{itemize}
On montre (1.) que la topologie déterminée par les $G(S)$
coincide avec celle donnée par
\[G\simeq \varprojlim_{[L:K]<+\infty,galois} Gal(L/K)\]

et (2.) que les groupes de galois sont compacts et
totalement déconnectés.
\subsection{Preuve de 1.}
Vu que $G(S)=Gal(K^{sep}/K(S))$ y suffit de montrer
que 
\begin{enumerate}
  \item $Gal(K^{sep}/K)\to Gal(E/K)$ sont continues
    pour tout $E/K$ galoisiennes finies.
  \item On a bien tout les $G(S)$.
\end{enumerate}
Pour le $2.$ c'est juste que si $K(S)\subset E$ avec
$E/K$ la clôture galoisienne de $K(S)$ alors 
\[\bigcup_{g\in Gal(E/K(S))}\tilde{g}G_E=G_{K(S)}\]
pour des lifts des $g$.

Pour la continuité bah c'est juste que si $\tilde g$
lift un $g$ alors $\tilde gG_E$ est ouvert par
définition.

\subsection{Preuve de 2.}
La compacité on la prouve via 
\begin{itemize}
  \item $G\simeq \varprojlim_{[L:K]<+\infty,galois} Gal(L/K)$ est fermé dans
    $ \prod_{[L:K]<+\infty,galois}G/G_L$
    qui est compact.
\end{itemize}
Milne le prouve joliment en disant que l'égalisateur
de \[pr_1\colon \prod_S G/G(S)\to G/G(S_1)\] et
\[q\circ pr_2\colon \prod_S G/G(S)\to G/G(S_2)\to G/G(S_1)\]
est fermé (intersection) pour $S_1\subset S_2$ et que
$G$ est l'intersection de tout ces égalisateurs (!).
La connectivité c'est simplement que 
\[\bigcap_{S\subset K^{sep}~fini} G(S)=\{id_{K^{sep}}\}\]

\subsection{$(K^{sep})^{Aut(K^{sep}/K(S))}=G(S)$}
Ça découle du théorème de galois fini !

\section{Théorème principal}
Étant donné un sous-groupe $H\leq G_K$ :
\[Gal(\overline{K}/\overline{K}^H)=\overline{H}\]
et étant donné un sous-corps $K\subset M\subset \overline K$ :
\[(\overline{K})^{G_M}=M\]
et $G_M$ est fermé dans $G$.

\subsection{$G_M$ est fermé}
Pour montrer que $G_M$ est
fermé l'idée c'est que pour tout $S\subset \overline K$
fini, $G(S)$ est ouvert et $G_K/G(S)$ est fini d'où 
\[G(S)=G_K-\bigsqcup_{\tilde g\ne 1}\tilde gG(S)\]
est fermé. Puis $G_M=\bigcap_{S\subset M}G(S)$ est fermé.
\subsection{Clôture de $H$}
Méga instructif. Clairement 
\[\bar H\subset Gal(\bar K/\bar K^H)\]
maintenant si $g\in G_K-\bar H$, on a $E/K$ galoisienne
t.q. $\bar K^H\subset E$ et (par déf)
\[g.G_E\cap H=\emptyset\]
en particulier si $\phi\colon G_K\to G_K/G_E=G_{E/K}$
alors $\phi(g)\notin \phi(H)\leq G_{E/K}$ (!!) d'où
$\phi(g)$ ne fixe pas $E^{\phi(H)}\subset \bar K^H$.


\printbibliography
\end{document}

